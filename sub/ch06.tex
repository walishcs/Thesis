\chapter{Issues in Proto-Seediq lexical reconstruction} \label{ch6}

In this chapter, external evidence for the reconstruction of some \acl{psed} words will also be first discussed in this section.

Additionally, I will introduce a unique doublet phenomenon found in \acl{psed} and Seediq dialects, along with its potential connection to the ``Gender Register System'' (a.k.a ``men's and women's speech'') in the Atayal language (see \cite{li1980gender,li1982gender,li1983gender,goderich2020phd}). 


\section{External evidence for lexical reconstructions} \label{sec:external_lexical_evidence}

I will provide examples of employing the top-down reconstruction method discussed in Section \ref{sec:top-down_recon} for the reconstruction of some \acl{psed} lexical items. The main external evidence includes the reconstructions mainly from \acl{pata}, with a few instances from \acl{pan}. 14 relevant lexical reconstructions are \acl{psed} *ali `day; time', *buwax `husked rice', *cəhəmu `urine', *gəlabaŋ `broad; wide', *maaw `Litsea cubeba', *məhuaw `thirsty', *nanak `alone; only; self', *papak `foot', *ramil `shoes', *ruluŋ `cloud', *sə[p]uŋ `to measure', *sumay `bear', *təhəya `far', and *uyah `ant', as shown in Table \ref{tab:lex_recon}. The bolded \acs{pan} forms in the table are revised. I will explain the revised \acs{pan} reconstructions along with the reconstructed \acl{psed} lexical items case-by-case later in this section.

\begingroup
\setstretch{1}
\renewcommand\arraystretch{1.5}
\begin{table}[!htbp]
\centering
\caption{Some \acl{psed} lexical reconstructions with their external evidence}
\label{tab:lex_recon}
\begin{tabular}{l>{\raggedright\arraybackslash}p{2cm}>{\raggedright\arraybackslash}p{2cm}>{\raggedright\arraybackslash}p{2cm}>{\raggedright\arraybackslash}p{2cm}>{\raggedright\arraybackslash}p{2cm}}
\hline
\acs{pan}  & *daqaNi       & *bəRas              &                &                 & *maqaw                  \\
\acs{pata} & *riʔax        & *buwax              & *həmuq         & *gVlabaŋ        & *maqaw                  \\ 
\acs{psed} & *ali          & *buwax              & *cəhəmu        & *gəlabaŋ        & *maaw                   \\ \hdashline
\acs{tg}   & ali           & beras               & rebu           &                 & mao                     \\
\acs{cto}  & diyax         & buwax               & cəhəmu         & ləlabaŋ         & məqəri                  \\
\acs{ctr}  & ɟiyax         & buwax               & səhəmu         & ləlabaŋ         & məqərig                 \\
\acs{etr}  & ɟiyax         & buwax               & səhəmu         & ləlabaŋ         & məqərig                 \\
Gloss      & `day; time'   & `husked rice'       & `urine'        & `broad; wide'   & `Litsea cubeba' \\ \hline\hline
\acs{pan}  & *maquSaw      &                     & (*qaqay)       &                 &                         \\
\acs{pata} &               & *nanak              & *kakay         &                 & *ɹamil                  \\ 
\acs{psed} & *məhuaw       & *nanak              & *papak         &                 & *ramil                  \\ \hdashline
\acs{tg}   & muhuqeni      & nanaq               & papak          & papak           & sapic                   \\
\acs{cto}  & məhuaw        & nanak; nanaq        & qaqay          & papak           & ramil                   \\
\acs{ctr}  & məhuaw        & nanaq               & qaqay          & papak           & ramil                   \\
\acs{etr}  & məhuaw        & nanak               & qaqay          & papak           & ramil                   \\
Gloss      & `thirsty'     & `alone; only; self' & `foot (human)' & `foot (animal)' & `shoes'                 \\ \hline\hline
\acs{pan}  & (*luNuŋ)       & \textbf{*Cəpəŋ}              & *Cumay         &                 & \textbf{*q<aR>ujah}              \\ 
\acs{pata} & *ruluŋ        & *cəpuŋ              &                & *tuhiyaq        & *kVtahiʔ                \\ 
\acs{psed} & *ruluŋ        & *səuŋ            & *sumay         & *təhəya         & *uyah                   \\ \hdashline
\acs{tg}   & pulabu        & sumeuŋ              & sume           & teheya          & qutahi                  \\
\acs{cto}  & ruluŋ         & səmuŋ               & kumay          & təhiyaq         & uyah                    \\
\acs{ctr}  & ruluŋ         & səməpuŋ             & kumay          & dəhiyaq         & qətahi                  \\
\acs{etr}  & ruluŋ         & səməpuŋ             & kumay          & təhiyaq         & qətahi                  \\
Gloss      & `clouds; fog' & `to measure'        & `bear'         & `far'           & `ant'                   \\ \hline
\end{tabular}
\end{table}
\endgroup

The first lexical item is `day; time'. The forms of this term in different dialects are as follows: \acl{tg} \textit{ali}, \acl{cto} \textit{diyax}, \acl{ctr}, \acl{etr} \textit{ɟiyax}. I have chosen to reconstruct *ali as the \acl{psed} form, rather than *diyax. Although *diyax aligns relatively consistently with \acl{pata} *riʔax `day; time', \acl{tg} \textit{ali} better reflects the \acl{pan} etymology *daqaNi `day'\footnote{\acs{pan} *q > \acs{psed} ∅. See \textcite{song2024sedq} for details.}. My approach is to prioritize using \acl{pan} when it provides obvious external evidence.

The second lexical is `husked rice.' Similarly, \acl{tg} \textit{beras} differs from \textit{buwax} in other dialects. However, this time I have chosen *buwax, which appears less related to the \acs{pan} etymology *bəRas, as the \acl{psed} reconstruction form. The reason is that the expected reflex of \acs{pan} *bəRas in \acl{psed} would be **bəgah or **bərah\footnote{For possible different reflexes of \acs{pan} *R in Seediq, please see \textcite{song2024Aicg}.}. The sound change \acs{pan} *s > Seediq \textit{s} is irregular, suggesting that the \acl{tg} form may be a borrowing. When \acs{pan} evidence is not applicable, I adopt evidence from \acl{pata}. \acl{pata} also has the corresponding form *buwax `raw rice; rice seeds'.

The reconstruction of the third term *cəhəmu `urine' relies on evidence from \acl{pata}. This word appears in a different form, \textit{rebu}, only in \acl{tg}. \acl{pata} has the word *həmuq `urine', which partially corresponds to \acl{psed} *(cə)həmu. Since \acl{pan} or \acl{paic} *q > \acl{psed} ∅ is a regular sound change, reconstructing this word as *cəhəmu is quite persuasive. However, I am still uncertain about the meaning represented by *cə-; it could be a fossilized prefix, or the syllable might sporadically drop in \acl{pata} (**cVhəmuq > **\sout{cV}həmuq > *həmuq, V stands for an uncertain vowel).

The fourth lexical item, \acl{psed} *gəlabaŋ `broad; wide', is reconstructed based on evidence from \acl{pata} *gVlabaŋ 'broad; wide' and another \acs{psed} term with the same meaning, *gəlahaŋ. In \acl{cto}, \acl{ctr}, and \acl{etr}, the reflexes of \acl{psed} *gəlabaŋ appear as \textit{\textbf{l}əlabaŋ}. However, considering that this word should form a pair with *gəlahaŋ at the \acl{psed} level, both internal and external evidence support reconstructing its initial as *g-. The sporadic assimilation observed in the mentioned dialects could potentially serve as a piece of evidence for subgrouping. For the phenomenon of special doublets like *gəlabaŋ \~{} *gəlahaŋ in Seediq, please refer to Section \ref{sec:sepical_doublet}.

The fifth reconstruction is \acl{psed} *maaw for `Litsea cubeba', a spice commonly used among many indigenous peoples in Taiwan, also known as ``mountain pepper (山胡椒)'' or ``makao (馬告)''. Different dialects of Seediq exhibit various forms. In \acl{tg}, it appears as \textit{mao}; in \acl{cto} as \textit{məqəri}; and in \acl{ctr} and \acl{etr} as \textit{məqərig}. Both \acl{pan} and \acl{pata} with *maqaw provide external evidence supporting the reconstruction back to \acl{psed} *maaw. This suggests that only the \acl{tg} form represents a regular reflex, while other dialects reflect a hypothesized form **məqərig. **məqərig might have originated from a borrowing of Atayal \textit{maqaw} followed by syllable substitution (see Section \ref{sec:syl_sub}). The possible development path might be **maqaw > **maq-rig (syllable substitution) > **maqərig (schwa insertion) > \textit{məqərig}, but verifying such a hypothesis is challenging.

%第六個詞是 \acl{psed} *məhuaw `thirsty' 的重建。這個詞在 \acl{tg} 中為 \textit{muhuqeni},而在其他方言中則皆為 \textit{məhuaw}。 選擇 *məhuaw 作為\acl{psed}的重建可被 \acl{pan} *maquSaw 所支持。只是,在發展過程中產生了一次換位,如 *maquSaw > **maSuqaw > *məhuaw。這樣的換位在 Amis \textit{soʡaw} (< **Suqaw) `thirsty' 中也可以見。另外,\acl{tg} \textit{muhuqeni} 的發展歷程可能與上個例子中的 \textit{məqərig} 類似。即借用了 Atayal 的形式後又產生了 syllable substitution。 \acl{pata} 的形式是 *mVkVhiyaʔ,換位現象也可見於 Squliq 方言的 \textit{məhəqyaʔ},\acl{tg} 可能借用了類似的形式後,以\textit{-ni}取代了Atayal的詞尾\textit{yaʔ}(**məhəqyaʔ > **məhəq-ni (syllable substitution) > **məhəqəni (schwa insertion) > \textit{muhuqeni})。同樣地,這僅只是最可能的假設,實際情形較難百分之百地保證。

The sixth lexical item is the reconstruction of \acl{psed} *məhuaw `thirsty'. In \acl{tg}, this word appears as \textit{muhuqeni}, while in other dialects, it is \textit{məhuaw}. The choice of *məhuaw as the reconstruction for \acl{psed} is supported by \acl{pan} *maquSaw. However, a metathesis occurred during development, such as *maquSaw > **maSuqaw > *məhuaw. This type of metathesis is also seen in Amis \textit{soʡaw} (< **Suqaw) `thirsty'. 

Additionally, the development of \acl{tg} \textit{muhuqeni} may follow a similar path to \textit{məqərig} in the previous example, involving borrowing from Atayal followed by syllable substitution. The \acl{pata} form is *mVkVhiyaʔ `thirsty', and metathesis is also observed in the Squliq Atayal \textit{məhəqyaʔ}. \acl{tg} may have borrowed a similar form, replacing the Atayal ending \textit{yaʔ} with its own \textit{-ni} (**məhəqyaʔ > **məhəq-ni (syllable substitution) > **məhəqəni (schwa insertion) > \textit{muhuqeni}). Again, this is just the most plausible hypothesis, and the actual circumstances are hard to guarantee with absolute certainty.

%接著,第七個 lexical item 則是 *nanak `alone; only; self'。在 Seediq 的不同方言中,可能有 \textit{nanak} \~{} \textit{nanaq} 的變體。Sporadic 的 \textit{k} > \textit{q} 偶爾可在 \acl{psed} 或 Seediq dialects 中見到,尤其在詞尾可能進一步地有聽覺上的混淆可能。不過基於來自 \acl{pata} 的外部證據 *nanak `alone; only; self',可以將 \acl{psed} 的形式重建為 *nanak。

Continuing on, the seventh lexical item is *nanak `alone; only; self'. In various Seediq dialects, there may be variants such as \textit{nanak} \~{} \textit{nanaq}. Sporadic \textit{k} > \textit{q} are occasionally observed in \acl{psed} or Seediq dialects, especially at the end of words, where there might be further auditory confusion. However, based on external evidence from \acl{pata} *nanak `alone; only; self', the form *nanak can be reconstructed for \acl{psed}.

%第八個 lexical item 是 *papak `foot (human, animal)'。首先,\textit{papak} 在所有 dialects under studied 皆有 `foot (animal)' 之意,其中 \acl{tg} 可同時表示 `foot (human)' 以及 `foot (animal)' (i.e. `foot (general)')。不過在 \acl{cto}, \acl{ctr}, and \acl{etr} 中,表示 `foot (human)' 的形式為 \textit{qaqay},與 \textit{papak} `foot (animal)' 產生了分工。我懷疑這些方言借用了 Atayal 的 \textit{kakay},同時有 sporadic \textit{k} > \textit{q}。而在 \acl{psed} 中,僅有一詞表達 general 的 `foot'。

The eighth lexical item is *papak 'foot (human, animal)'. Initially, \textit{papak} is used in all studied dialects to mean 'foot (animal)', and in \acl{tg} it can also denote 'foot (human)' as well as 'foot (animal)' (i.e., 'foot (general)'). However, in \acl{cto}, \acl{ctr}, and \acl{etr}, the form \textit{qaqay} is used to signify 'foot (human)', creating a division of labor with \textit{papak} 'foot (animal)'. I suspect these dialects borrowed the Atayal \textit{kakay}, with sporadic \textit{k} > \textit{q} transformations occurring. In \acl{psed}, there is only one term to express 'foot' in general.

%當然,有人可能會認為這個詞與 \acl{pan} *qaqay `foot; leg' 脫不了關係。但是,(1) \acs{pan} *qaqay 到 \acl{psed} 中的預期反映可能為 **aay(?), (2) \acs{pan} *q > \acs{psed} *q 的例子大多借自 Atayal(\textcite{song2024sedq})。所以應該不是直接反映\acs{pan}的詞源。而 Atayal 的對應詞項為 *kakay,所以也只能說這可能是從 Atayal 借進到 Seediq 中產生了零星的音變。

Certainly, some might suggest that this term is related to \acl{pan} *qaqay `foot; leg'. However, (1) the expected reflection of \acs{pan} *qaqay in \acl{psed} might be **aay(?), and (2) most instances of \acs{pan} *q > \acs{psed} *q are borrowings from Atayal (according to \textcite{song2024sedq}). Thus, it's unlikely to be a direct reflex of the \acs{pan} etymon. The corresponding form in Atayal is *kakay, indicating that this might simply be a borrowing from Atayal into Seediq with sporadic sound changes.

%第九個重建是 \acl{psed} *ramil `shoes' (or `footwear')。在\acl{tg}以外的方言中都是 \textit{ramil} 這個形式,而 \acl{tg} 則顯示 \textit{sapic}。而來自 \acl{pata} 的外部證據顯示 *ɹamil `shoes' ,有非常完美的對應。因此,可以把\acl{psed}的 `shoes' 重建為 *ramil。 Goderich (p.c.) 說:「Plngawan Atayal 也有 \textit{sapit} `footwear' 這個形式,可能是借自 the so-called ``Hoanya'' \textit{sapit} 或 Taokas \textit{sapit}(cf. \cite[80]{tsuchida1982wp})」。而 \acl{tg} Seediq 以及 Plngawan Atayal 之間的借用順序則不明。

The ninth reconstruction is \acl{psed} *ramil `shoes' (or `footwear'). In dialects other than \acl{tg}, the form is consistently \textit{ramil}, whereas \acl{tg} shows \textit{sapic}. External evidence from \acl{pata} shows *ɹamil `shoes'. Thus, `shoes' in \acl{psed} can be reconstructed as *ramil. Andre Goderich (p.c.) mentioned, ``Plngawan Atayal also has the form \textit{sapit} 'footwear', which might be borrowed from the so-called ``Hoanya'' \textit{sapit} or Taokas \textit{sapit} (cf. \cite[80]{tsuchida1982wp}).'' The direction of borrowing between \acl{tg} Seediq and Plngawan Atayal remains unclear.

%第十個詞項是 `cloud; fog'。在 \acl{psed} 中應被重建為 *ruluŋ。Seediq 的方言中,除了 \acl{tg} 中的 \textit{pulabu} 之外,其餘都顯示為 \textit{ruluŋ}。\textit{ruluŋ} 與 \acl{pata} 的 *ɹuluŋ 有一致的 correspondences。而 \textcite{ACD} 所重建的 *luNuŋ `cloud' 事實上可能具有再審視的空間,這個重建僅被 Pazeh \textit{ruluŋ} `cloud' 及 \acl{paic} *ruluŋ `cloud'所支持。然而,這可能僅是兩個語言之間的借用而已。另外,\acl{tg}的\textit{pulabu}可能與動詞 \textit{labu} `to wrap' 有關。

The tenth lexical item is `clouds; fog', which should be reconstructed as *ruluŋ in \acl{psed}. In Seediq dialects, aside from \acl{tg} \textit{pulabu}, the rest display as \textit{ruluŋ}. Seediq \textit{ruluŋ} and \acl{pata} *ɹuluŋ show consistent correspondences. The reconstruction *luNuŋ `cloud' by \textcite{ACD} may warrant reevaluation, as it is only supported by Pazeh \textit{ruluŋ} `cloud' and \acl{paic} *ruluŋ `cloud'. However, this could simply be a case of borrowing between two (proto-)languages. Additionally, \acl{tg} \textit{pulabu} might be related to the verb \textit{labu} `to wrap'.

% 第十一個重建為 \acl{psed} *səuŋ `to measure'。在 \acl{tg} \textit{sumeuŋ} 及\acl{cto} \textit{səmuŋ} 的詞根為 *səuŋ 的反映。然而,\acl{ctr}及\acl{etr} \textit{səməpuŋ} 則反映 *səpuŋ。雖然,\acl{pan}及\acl{pata}都顯示詞中有一個 \textit{p} 的存在,但我選擇將 \acl{psed} 重建為 ∅ 的原因是:雖尚未討論到分群問題,但 \acl{ctr} 及 \acl{etr} 在其口傳歷史上有非常緊密的連結,他們應共享一段發展的歷史。因此,他們有可能向 Atayal 借了 \textit{səpuŋ} 的這個形式,取代掉了原本的 **səuŋ。另外,在Mayrinax (Matu'uwal) Atayal 的 data 中顯示,詞中的 \textit{p} 音被 \textit{ʔ} 取代是有可能的男性形式發展(\textcite[12]{li1983gender})。Seediq 中的這個詞可能經歷過類似的發展(請注意Atayal方言中的 `to measure' 這個詞並沒有 \textit{p} 被刪除或被 \textit{ʔ} 取代的現象)。

% 此外,我將\textcite{ACD}所重建的 *təpəŋ 'to measure quantities, as amounts of grain' revised 為 *Cəpeŋ。原先 *təpəŋ 在 Formosan languages 當中只有被 Amis \textit{tepeŋ} 'to measure grain with a large measure' 及 Kavalan \textit{tepeŋ} 'to measure (quantity)',而這兩個語言都有經過 \acs{pan} *C/*t > \textit{t} 的 merger(\cite[44]{blust1999subgrouping})。因此,重建為 *C 或 *t 都是有可能的。而我在 Saisiyat 中發現了 \textit{səpəŋ} `to measure (sth. abstract)'一詞。 Saisiyat 的 \textit{s} 來自 \acs{pan} *C,因此可更加確定這個重建應是 *Cəpeŋ。

The eleventh lexical reconstruction is \acl{psed} *səuŋ `to measure'. In \acl{tg} \textit{sumeuŋ} and \acl{cto} \textit{səmuŋ}, the root of this verb reflects *səuŋ. However, \acl{ctr} and \acl{etr} \textit{səməpuŋ} reflect *səpuŋ. Although \acl{pan} and \acl{pata} both indicate the presence of a \textit{p} in the word, I choose to reconstruct \acl{psed} as having -∅- for the following reasons: Data from Mayrinax (Matu'uwal) Atayal shows that the replacement of the \textit{p} sound by \textit{ʔ} could be a possible male form development (\cite[12]{li1983gender}). The Seediq form might have undergone a similar development that the \textit{p} got deleted in this form (but note that in the Atayal dialects, `to measure' does not show the deletion of \textit{p} or its replacement by \textit{ʔ} in any dialects). Additionally, although subgrouping issues have not yet been discussed, \acl{ctr} and \acl{etr} indeed have a very close oral/recorded history connection and should share a segment of developmental history. Thus, they might have borrowed the form \textit{səpuŋ} from Atayal, replacing the original **səuŋ.

Moreover, I revised the reconstructed *təpəŋ `to measure quantities, as amounts of grain' from the \ac{acd} (\cite{ACD}) to *Cəpəŋ. Originally, *təpəŋ is attested in two Formosan languages such as Amis \textit{tepeŋ} `to measure grain with a large measure' and Kavalan \textit{tepeŋ} `to measure (quantity)', both of which have undergone the \acs{pan} *C/*t > \textit{t} merger (\cite[44]{blust1999subgrouping}). Thus, reconstructing it as either *C or *t is plausible. Additionally, I found a cognate \textit{səpəŋ} `to measure (sth. abstract)' in Saisiyat and *cəpuŋ `to measure' in \acl{pata}. Saisiyat \textit{s} and \acl{pata} *c both originate from \acs{pan} *C, further confirming that the reconstruction should be *Cəpəŋ.

% 第十二個 lexical item 的重建是 *sume `bear'。在非\acl{tg}的方言中,都顯示為 \textit{kumay},只有 \acl{tg} 顯示 \textit{sume}。這個詞的重建被 \acs{pan} *Cumay 所支持。\acs{pan} 的 *C 在 Seediq 中可能反映為 \textit{s},如\acs{pan} *Caqis > \acs{psed} *sais, \acs{pan} *Capuh `sweep' > \acs{psed} *sapuh `to heal (shamanistic)' 等等。因此,重建為 *sumay 相當合理。\acl{cto}, \acl{ctr}, \acl{etr} 的 \textit{k-}initial 可能暗示著其(共同的)sporadic sound change。

The twelfth lexical item to be reconstructed is *sume `bear'. In dialects other than \acl{tg}, it is represented as \textit{kumay}, while only \acl{tg} shows \textit{sume}. This reconstruction is supported by \acs{pan} *Cumay `bear'. The \acs{pan} *C may be reflected as \textit{s} in Seediq, as seen in \acs{pan} *CaqiS > \acs{psed} *sais `to sew', \acs{pan} *Capuh `sweep' > \acs{psed} *sapuh `to heal (shamanistic)'\footnote{The semantic shift may have progressed through the following stages: `to sweep' > `to sweep undesirable things from the body' > `to heal through shamanistic practices'.}, and so on. Therefore, reconstructing `bear' as *sumay is quite reasonable. The \textit{k-}initial in \acl{cto}, \acl{ctr}, and \acl{etr} may indicate a (shared) sporadic sound change.

% 第十三個`far'在\acl{psed}中的重建是*təhəya。在\acl{tg}中的 \textit{teheya} 規律地反映了 \acl{psed} 的 *təhəya。而 \acl{cto},\acl{etr} \textit{təhiyaq} 及 \acl{ctr} \textit{dəhiyaq} 則應借自 Atayal,如 \acl{pata} *tuhiyaq `far'。\acl{tg} 中也有應是借自 Atayal 的 \textit{teheyaq} 一詞,意義為 `to play; to visit',與規律的 \textit{teheya} 形成了 doublet ,有著語意上的不同。

The thirteenth item `far' is reconstructed as *təhəya in \acl{psed}. In \acl{tg}, \textit{teheya} regularly reflects \acl{psed} *təhəya. Meanwhile, \acl{cto} and \acl{etr} \textit{təhiyaq} as well as \acl{ctr} \textit{dəhiyaq} (with sporadic voicing) are likely borrowed from Atayal, as evidenced by \acl{pata} *tuhiyaq `far'. In \acl{tg}, there is also \textit{teheyaq}, which seems to be borrowed from Atayal with semantic shift from `far' to `to play; to visit', forming a doublet with the regular \textit{teheya} `far' and carrying different meanings.

% 最後,第十四個則是 `ant',即使只有 \acl{cto} 有 \textit{uyah} 這個形式,我還是將其重建為 \acl{psed} *uyah。首先,我們必須先討論 revised 的 \acs{pan} 形式 *q<aR>ujah,而\textcite{acd}的重建原先為 *alujah。這個重建在 \textcite[146--47]{shibata2020pbun} 曾經被討論過,Shibata 以 Amis 中的 epiglottal stop initial [ʡ] 以及 Proto-Bunun 的 *q- 作為這個\acs{pan}重建 *q-initial 的依據。同時,原本的 *-l- 也依照 Proto-Bunun 的反映修正為 *R。故,Shibata 將 `ant' 重新重建成 *qaRujah。

% 然而,筆者在 \acl{cto} Seediq 找到的 \textit{uyah} 只有部分反映出 \acs{pan} *qaRujah。首先,*q > ∅, *j > *y 都是 \acs{pan} 到 \acl{psed} 的規律音變(\cite{song2023Aicgprime,song2024sedq})。不過,\acl{cto} Seediq \textit{uyah} 並沒有將 *-aR- 這個音段序列反映出來。我懷疑這可能跟\acl{pan} 的 fossilized infixes 如 *<aR>, *<al>, *<aN> 有關,如\textcite{li_tsuchida2009_AnInfixes}所討論的,可能與聲音有關。然而,我發現在\acl{pan}某些動物、植物的重建詞彙中也可以發現這些 infixes,如 *qaReNu `a plant, Phragmites spp.', *qaRidaŋ `bean, pea (generic)', *qalupaR `persimmon', *dalayap `kind of citrus tree and fruit', *balalaCuk `a bird: the woodpecker', *kaNasay `a fish: the mullet', *kaNawaS `a small tree bearing round, green fruit: Ehretia spp.', *qaNuNaŋ - 'a tree: Cordia dichotoma', *taNiud `mulberry tree and fruit: Morus formosensis (Hotta)' 等等都可以發現這些音段序列的蹤跡。可能 *qaRujah 的 *<aR> 部分也有相同的來源。

% 另外,Seediq 內部似乎有些時候某些方言會把這類型的化石化中綴給丟掉。如 `corn' 在 \acl{tg} 中是 \textit{suruqemu},在 \acl{ctr}, \acl{etr} 中則為 \textit{səqəmu},兩者差了一個中綴 <\textit{ur}>/<\textit{ər}> (< ?**<al>)。因此,\acl{cto}中的 \textit{uyah} `ant',應是丟掉化石化中綴 *<aR> 後的結果。

% 另外三個方言——\acl{tg} \textit{qutahi}, \acl{ctr}, \acl{etr} \textit{qətahi} 則應是來自泰雅語的借詞。請注意在賽德克語中,歷史的 *k...h 會被同化成 *q...h(\cite[247-48]{li1981paic})。

Finally, the fourteenth lexical item, `ant', is reconstructed as \acl{psed} *uyah, despite \acl{cto} is the only dialect retained this form. First, we must discuss the revised \acs{pan} form *q<aR>ujah, previously reconstructed by \textcite{acd} as *alujah. This reconstruction has been discussed by \textcite[146--47]{shibata2020pbun}, who supported the *q-initial based on the epiglottal stop initial [ʡ] in Amis and Proto-Bunun *q-. Additionally, the original *-l- was corrected to *-R- based on its reflex in Proto-Bunun. Thus, Shibata reconstructed `ant' as *qaRujah.

However, the form \textit{uyah} found in \acl{cto} Seediq only partially reflects \acs{pan} *qaRujah. First, *q > ∅ and *j > *y are regular sound changes from \acs{pan} to \acl{psed} (\cite{song2023Aicgprime,song2024sedq}). Yet, \acl{cto} Seediq \textit{uyah} does not reflect the *-aR- sequence. I suspect this may relate to the fossilized infixes like *<aR>, *<al>, *<aN> in \acl{pan}, which, as discussed by \textcite{li_tsuchida2009_AnInfixes}, may have the meaning `having the sound or quality of'. These infixes are also found in \acl{pan} reconstructions for certain animals and plants\footnote{Relavant examples: *qaReNu `a plant, Phragmites spp.', *qaRidaŋ `bean, pea (generic)', *qalupaR `persimmon', *dalayap `kind of citrus tree and fruit', *balalaCuk `a bird: the woodpecker', *kaNasay `a fish: the mullet', *kaNawaS `a small tree bearing round, green fruit: Ehretia spp.', *qaNuNaŋ - 'a tree: Cordia dichotoma', *taNiud `mulberry tree and fruit: Morus formosensis (Hotta)', and so on.}, suggesting a similar origin for the *<aR> part of *qaRujah. 

Additionally, it seems that some Seediq dialects occasionally drop such fossilized infixes. For instance, `corn' in \acl{tg} is \textit{suruqemu}, while in \acl{ctr} and \acl{etr}, it is \textit{səqəmu}, differing by an infix <\textit{ur}>/<\textit{ər}> (< **<ar> < \acs{pan} *<al>?). Thus, the \textit{uyah} `ant' in \acl{cto} as well as \acl{psed} *uyah likely result from dropping the fossilized infix *<aR>.

The other three dialects—\acl{tg} \textit{qutahi}, \acl{ctr}, and \acl{etr} \textit{qətahi}—are likely borrowings from Atayal (cf. \acl{pata} *kVtahiʔ `k.o. ant'). Note that in Seediq, historical *k...h assimilates to *q...h (\cite[247-48]{li1981paic}).

\section[Special doublet phenomenon in Proto-Seediq lexicon]{Special doublet phenomenon in Proto-Seediq\\ lexicon} \label{sec:sepical_doublet}

Across the modern dialects of the Seediq language, there are words with similar, but not identical forms, where their meanings are the same or related. Often the difference is just a few segments. Sometimes both of the two similar forms appear in the same dialect, and sometimes they appear separately in different dialects without any conditions or rules. 

In historical linguistics, a ``doublet'' generally refers to a pair of words in a language that have the same etymology but differ in form. Typically, one of the words is a regular reflex of the original language, while the other has been borrowed (See \textcite[99]{campbell2007HLglossary} for the term ``learned loan''.). A famous example is the English doublet \textit{shirt} and \textit{skirt}, where \textit{shirt} (< Old English \textit{scyrte} `skirt, tunic') is the regular reflex (*sk > \textit{ʃ}), and \textit{skirt} was borrowed from Old Norse \textit{skyrta} `shirt, a kind of kirtle'. Both words ultimately derive from Proto-Germanic *skurtjon `a short garment' (\cite[70]{campbell2021histling}).

In this section, the term ``(special) doublet'' is used in a broader sense specific to the Seediq language, as defined by me. Here, a ``special doublet'' refers to a pair of words that have the same etymology and have developed independently within the language itself.

In \acl{psed} or Seediq dialects, there are various types of doublets. I will discuss each category and their patterns in detail.

Please note that some reconstructed forms are enclosed in parentheses. These forms may be difficult to reconstruct to the \acl{psed} stage based on the current evidence, and are listed here only for the purposes of comparison. These lexical items could be innovations in a single dialect or within a specific subgroup. Some forms can be supported by external evidence in \acl{pan}, \acl{pata} or Atayal dialects, and therefore can be reconstructed back to \acl{psed}. Relevant external evidence will be presented in the following sections.

\subsection{Proto-Seediq special doublets with fossilized infixes *<ra> or *<na>}

The first category involves fossilized infixes *<ra> or *<na> (their meanings are unknown or ∅). In some (Proto-)Seediq forms, we can observe the appearance of <\textit{ra}> or <\textit{na}> before the final consonant. At the same time, forms without these infixes can sometimes be found in different dialects or within the same dialect. In cases where <\textit{ra}> or <\textit{na}> are inserted, it seems that the original leftmost (i.e., antepenultimate) syllable is deleted. I have reconstructed these examples across various dialects back to \acl{psed}, as shown in *babaw `on top of; after'/*baraw (< **baba<ra>w) `above; on', *rahut/*hunat (< **rahu<na>t) `below', *ahiŋ/*hiraŋ (< **ahi<ra>ŋ) `shoulder', *ŋaŋut/(*ŋərat (< **ŋaŋə<ra>t) ) `outside', and *hapuy/(*həpuray (< **hapu<ra>y) ) `to cook'/*puniq (< **hapu-niq) `fire' in Table \ref{tab:doublet_infix}.

\begin{table}[!htbp]
\centering
\caption{\acl{psed} special doublets with fossilized infixes *<ra> or *<na>}
\label{tab:doublet_infix}
\begin{tabular}{lllll}
\hline
\acs{psed} & *babaw             & *baraw        & *rahut    & *hunat        \\ \hdashline
\acs{tg}   & bobo               & baro          & rahuc     & hunac         \\
\acs{cto}  & babaw              & baraw         & tərahuc   & hunat `south' \\
\acs{ctr}  & babaw              & baraw         &           &               \\
\acs{etr}  & babaw              & baraw         &           & hunat `south' \\
Gloss      & `on top of; after' & `above; on'   & `below'   & `below'       \\ \hline\hline
\acs{psed} & *ahiŋ              & *hiraŋ        & *ŋaŋut    & (*ŋərat)      \\ \hdashline
\acs{tg}   & ahiŋ               &               & ŋaŋuc     & ŋerac         \\
\acs{cto}  &                    & hiraŋ         & ŋaŋuc     &               \\
\acs{ctr}  &                    & hiraŋ         & ŋaŋuc     &               \\
\acs{etr}  &                    & hiraŋ         & ŋaŋut     &               \\
Gloss      & `shoulder'         & `shoulder'    & `outside' & `outside'     \\\hline\hline
\acs{psed} & *hapuy             & (*həpuray)    & *puniq\shb{\dag}&               \\ \hdashline
\acs{tg}   &                    & hupure        & puniq     &               \\
\acs{cto}  & məhapuy            &               & puniq     &               \\
\acs{ctr}  & məhapuy            &               & puniq     &               \\
\acs{etr}  & məhapuy            &               &           &               \\
Gloss      & `to cook (AV)'     & `to cook (AV) & `fire'    &               \\ \hline
\multicolumn{5}{l}{\makecell[l]{\footnotesize \shb{\dag}This word does not contain the fossilized infix *<ra> or *<na> mentioned in this subsection,\\\footnotesize  but it is semantically and etymologically related to `to cook', hence it is included here.}}\\
\end{tabular}
\end{table}

We can observe that when a dialect has two forms, sometimes they appear to be synonymous, while at other times, there seems to be a slight difference in meaning. Occasionally, we can already see the difference in meaning in \acl{psed}.

Take the doublet \textit{bobo}/\textit{babaw}; \textit{baro}/\textit{baraw} for example. In Seediq dialects, the former expresses the basic meaning `on top of' with actual contact between objects, as seen in examples like \textit{bobo butunux} `on top of the stone' and \textit{bobo qusiya} `on the water' in the \acl{tg} dialect like in (\ref{ex:bobo1}-\ref{ex:bobo2}). On the other hand, the latter indicates a more general sense of being above without necessarily requiring physical contact, as seen in \textit{karac baro} `in the sky' and \textit{quhuni baro} `in the tree' like in example (\ref{ex:baro1}-\ref{ex:baro2}). However, sometimes there appears to be mixed usage, which needs further investigation in the future.

\begin{exe}
    \ex 
    \begin{xlist}
        \ex \textcite{ILRDFEdict} \label{ex:bobo1}
        \gll pu-hido bobo butunux lukus=su ga mu-huriq. \\
        \textsc{pu}-sun on.top.of stone clothes=\textsc{2sg.gen} \textsc{prog} \textsc{av}-wet\\
        \glt `Dry your wet clothes on the rock in the sun.'
        \ex \textcite{ILRDFEdict} \label{ex:bobo2}
        \gll osa subuli bobo qusiya yayung ga pawan. \\
        go\textsc{.av.imp} float\textsc{.av} on.top.of water river that \textsc{\acs{pn}} \\
        \glt `Pawan, go float on the river!'
    \end{xlist}
    \ex 
    \begin{xlist}
        \ex \textcite[221]{Sung2018Sedgrammar} \label{ex:baro1}
        \gll wada mugu-puŋerah karac baro ka kiya di \\
        \textsc{\acs{perf}} become-star sky above \textsc{\acs{nom}} that \textsc{\acs{csm}}\\
        \glt `that became star(s) in the sky.'
        \ex \textcite{ILRDFEdict} \label{ex:baro2}
        \gll ga quhuni baro ka burihuc. \\
        be.at tree above \textsc{\acs{nom}} squirrel \\
        \glt `A squirrel is in the tree.'
    \end{xlist}
\end{exe}

We can also notice that in fact, \textit{bobo} functions as a preposition-like unit, while \textit{baro} behaves more like a locative noun or directional noun indicating a specific location or direction. Furthermore, \textit{bobo} (\textit{na}) has further grammaticalized into expressing `after; later' in a temporal sense as in (\ref{ex:bobo3}-\ref{ex:bobo4})\footnote{Atayal has the same usage of the cognate \textit{babaw}. See \textcite{weng2012babaw} for details.}. Although this doublet likely originated from the same etymology, language change and development have resulted in more distinct usage and functions.

\begin{exe}
    \ex 
    \begin{xlist}
        \ex \textcite{ILRDFEdict} \label{ex:bobo3}
        \gll so ka bobo s<un>u-qusiya paru peni, ... \\
        like \textsc{ka} after to.be<\textsc{\acs{perf}}>to.be-water big \textsc{peni}\\
        \glt `Like, right after the big storm, ...'
        \ex \textcite[132]{Sung2018Sedgrammar} \label{ex:bobo4}
        \gll mu-huwa=namu bobo na di \\
        \textsc{av}-how=2\textsc{pl.nom} later \textsc{na} \textsc{\acs{csm}} \\
        \glt `What are you going to do later?'
    \end{xlist}
\end{exe}

In contrast, for the doublet \textit{ŋaŋuc} and \textit{ŋerac} `outside' in the \acl{tg} dialect, it is currently difficult to find such significant differences. However, subtle semantic distinctions can still be identified. According to my investigation results, both \textit{ŋaŋuc} and \textit{ŋerac} can be used to mean `outside' or `outdoors', and are generally interchangeable. However, regarding the specific context of `going to the bathroom', there is a metaphorical expression in Seediq, either \textit{mosa ŋaŋuc} or \textit{pusuŋaŋuc} (literally ``go outside''), in which case \textit{ŋerac} cannot be used.

\subsection{Proto-Seediq special doublets with segment substitution}

The second category consists of doublets where the two forms differ by only one segment. Similar to the previous category, some pairs in this group may not have significant semantic differences, while others have clearly divided semantic meanings. Examples of this category include *gəlabaŋ/*gəlahaŋ `broad; wide', *daliŋ/*dalih `close; near', *məridil/*məridih `crooked', *cəka `to split; middle'/*dəka `half; to halve', as shown in Table \ref{tab:doublet_seg_sub}.

\begin{table}[!htbp]
\centering
\caption{Proto-Seediq special doublets with segment substitution}
\label{tab:doublet_seg_sub}
\begin{tabular}{lllll}
\hline
\acs{psed} & *gəlabaŋ & *gəlahaŋ   & (*daliŋ) & (*dalih) \\ \hdashline
\acs{tg}   &          & gulahaŋ    & daliŋ    &          \\
\acs{cto}  & ləlabaŋ  &            &          & dalih    \\
\acs{ctr}  & ləlabaŋ  &            &          & dalih    \\
\acs{etr}  & ləlabaŋ  &            &          & dalih    \\
Gloss & `broad; wide'  & `broad; wide'       & `close; near'              & `close; near' \\ \hline\hline
\acs{psed} & *məridil & (*məridih) & *cəka    & *dəka    \\ \hdashline
\acs{tg}   & muridin  & muridih    & ceka     & deka     \\
\acs{cto}  & məridil  &            & cəka     & dəka     \\
\acs{ctr}  & məriɟil  &            & səka     & dəka     \\
\acs{etr}  & məriɟil  &            & səka     & dəka     \\
Gloss & `crooked' & `crooked' & `half; middle' & `half; to halve'     \\ \hline
\end{tabular}
\end{table}

Once again, as mentioned in Section \ref{sec:external_lexical_evidence}, the reconstruction of *gəlabaŋ `broad; wide' relies on external evidence from the the initial of \acl{pata} *gVlabaŋ `borad; wide' and the other form from the pair of the doublet, \acl{psed} *gəlahaŋ `borad; wide'. In \acl{cto}, \acl{ctr}, and \acl{etr}, there are sporadic cases of regressive assimilation from *gəlabaŋ to \textit{ləlabaŋ}.

In this category, the pair *cəka `half; middle' and *dəka `half; to halve' have more clearly semantic differences. Taking the \acl{etr} reflexes \textit{səka} and \textit{dəka} as examples, \textit{səka} can act as a preposition-like unit expressing `middle' in terms of time or space, such as in \textit{səka rabi} `midnight' and \textit{səka qəsiya} `in the water'; when used as a verb, it implies `to split or to split in half'. On the other hand, \textit{dəka} conveys meanings such as `half, to halve (with cutting)', etc. Despite the small differences, this pair of words share the core meaning of `half'.

\subsection{Proto-Seediq special doublets with syllable substitution} \label{sec:syl_sub}

The last category involves doublets where the two forms differ by an entire last syllable. Examples include *daŋi/*dawin `friend', *qudas/*qudi `gray hair', *raŋaw/*rəŋədi `k.o. fly', *igi/*iril `left', *uka/*uŋat `not exist', *dəgə\cvc `narrow', and *təhə\cvc `to migrate; to move' as shown in Table \ref{tab:doublet_syl_sub}. 

Please note that the notation *\cvc indicates an uncertain but hypothesized -CVC syllable in \acl{psed}. Due to varying forms across dialects, it is difficult to determine the actual \acl{psed} form, thus it is temporarily reconstructed as *\cvc. Taking *dəgə\cvc `narrow' as an example, the reasons for not reconstructing it as *dəgə- or *dəgə are: (1) it is a free morpheme rather than a bound morpheme, and (2) the form *dəgə violates the phonotactics of \acl{psed} (see Section \ref{sec:psed_phonotactics}). Therefore, a compromise reconstruction as *dəgə\cvc is a relatively reasonable approach.

\begin{table}[!htbp]
\centering
\caption{Proto-Seediq special doublets with syllable substitution}
\label{tab:doublet_syl_sub}
\begin{tabular}{lllll}
\hline
\acs{psed} & *daŋi        & *dawin                                           & *qudas             & *qudi                \\ \hdashline
\acs{tg}   & daŋi         &                                                  &                    & qudi                 \\
\acs{cto}  & daŋi         &                                                  & qudas              &                      \\
\acs{ctr}  & daŋi         &                                                  &                    &                      \\
\acs{etr}  & daŋi `lover' & \makecell[l]{dawin `partner;\\darling (to man)'} &                    & quɟi                 \\
Gloss      & `friend'     & `friend?'                                        & `gray hair'        & `gray hair'          \\ \hline\hline
\acs{psed} & *raŋaw       & *rəŋədi                                          & *igi               & *iril                \\ \hdashline
\acs{tg}   & ruŋedi raŋo  & ruŋedi                                           &                    & irin                 \\
\acs{cto}  & raŋaw        & rəŋədi                                           & iwi                & iril                 \\
\acs{ctr}  &              & rəŋəɟi                                           & igi                & iril                 \\
\acs{etr}  &              & rəŋəɟi                                           &                    & iril                 \\
Gloss      & `blowfly'    & `housefly'                                       & `left'             & `left'               \\ \hline\hline
\acs{psed} & *uka         & *uŋat                                            & *dəgə\cvc        & *təhə\cvc          \\ \hdashline
\acs{tg}   & uka          &                                                  & dugehiŋ            & teheruy              \\
\acs{cto}  & uka          &                                                  & dəwəliŋ \~{} duril & təhədil              \\
\acs{ctr}  & uka          & uŋac                                             &                    & təhəɟil              \\
\acs{etr}  &              & uŋat                                             & dəgəril            & təhəɟil              \\
Gloss      & `not exist'  & `not exist'                                      & `narrow'           & `to migrate; to move' \\ \hline
\end{tabular}
\end{table}

In Table \ref{tab:doublet_syl_sub}, we can observe many doublet combinations where one source comes from \acl{pan}, accompanied by another more innovative form. For example, \acs{pan} *qudaS > \acl{psed} *qudas and *qudi `gray hair'; \acs{pan} *laŋaw `housefly' > \acl{psed} *raŋaw `blowfly' and *rəŋədi `housefly'; \acs{pan} *(w)iRi > \acl{psed} *igi and *iril `left'; \acs{pan} *uka > \acl{psed} *uka and *uŋat `not exist'.

So far, my discussion has mainly focused on the form or meaning of these special doublets. But why does the phenomenon of special doublets occur, and what does it signify? I have noticed that this phenomenon bears a striking resemblance to the Gender Register System observed in the Atayal language. A comparison with Atayal will be presented in Section \ref{sec:sed_ata_grs}.

\subsection{The possible relationship between Seediq's special doublet phenomenon and Atayal's Gender Register System} \label{sec:sed_ata_grs}

Seediq is the sister language of Atayal, so I naturally thought of the relevance of the Gender Register System in Atayal to this phenomenon in Seediq.

According to \textcite{li1980gender,li1982gender,li1983gender} and \textcite{goderich2020phd}, Atayal has a Gender Register System in its lexicon. Nonetheless, among modern dialects, only Matu'uwal has fully retained the system, and only in some elderly speakers' speech. Meanwhile, the Gender Register System collapsed in other dialects, with only occasional remnants remaining.

Generally speaking, the female register is considered to be original, and the corresponding male form is an ``adjusted'' one. There are several strategies to derive words, such as affixation, deletion, substitution, etc. Note that the derivation from the female form to the male form is irregular, and the strategy of derivation is also not fully predictable.

\textcite[157]{goderich2020phd} states that there are relatively few traces of the register system in Seediq. \textcite{li1982gender} examined each Atayal dialect and Seediq with the data in Mayrinax (a.k.a Matu'uwal) Atayal to see which one adopted either the female form or the male form as its sole form of usage. Li used a single Atayal dialect as the benchmark, which might overlook the internal differences within Seediq and its dialects. I found that there are probably more cases than scholars have assumed so far. 

As mentioned earlier in this section (Section \ref{sec:sepical_doublet}), two or more similar forms of the same or similar meaning can be found among Seediq dialects. Sometimes, the forms in some doublets found in Seediq do not have cognates in Atayal. This thus implies that at least in the (Pre-)Proto-Seediq stage, the derivation methods was still productive and independent from Atayal.

There is a main difference between Seediq and Atayal. None of Seediq modern dialect has retained the distinction between male and female forms; therefore, sometimes we cannot ensure which one is the ``original'' word and which one is the derived one without external evidence. Hence, I treat it like the ``legacy'' of the Gender Register System. 

In other words, the current evidence cannot definitively prove that \acl{psed} still retained a clear distinction between male and female vocabulary. They may have retained the morphological strategies in some sense, but whether it has played a certain societal function remains uncertain.

%Similar strategies are implemented in Seediq as well. Following tables show the example of each strategy, see Table \ref{tab:gender1} (Affixation), \ref{tab:gender2} (Segment substitution), and \ref{tab:gender3} (Syllable substitution) for details. 

Most common infixes are *<ra> and *<na>, which are inserted before the final consonant of a word, as shown in Table \ref{tab:doublet_infix} previously. Also, after the infixation, the prepenultimate syllables tends to be deleted (e.g. *babaw `on top of; after' vs. *baraw `above; on' < **baba<ra>w).

The medial *<ɹa> (corresponding to \acl{psed} *<ra>) and *<na> can also be seen in Proto-Atayal or its dialects' reflexes. For example, in the Matu'uwal dialect, we have \textit{sumayug} (female, f) and \textit{sumayu<na>g} (male, m) `to change (AV)'.  In Proto-Atayal, we find *raqis (f) and *raqi<ɹa>s (m) `face' and so on. 

There is also a possible pair, \acl{psed} *risaw `young man' and *rəs<ən>aw `man; husband', but the infix <ən> is inserted in a different position in comparison with other cases (although their meaning is related).

In Table \ref{tab:doublet_seg_sub}, most of the examples are related to *h, probably indicating that the original consonants may have been replaced by *h. \textcite{li1983gender} also mentioned that in Mayrinax (Matu'uwal) Atayal, \textit{h} can replace other consonants such as \textit{l} or \textit{b}.

Regarding the majority of forms seen in Table \ref{tab:doublet_syl_sub}, as mentioned earlier, they mostly consist of very basic words with clear Proto-Austronesian origins, like \acs{pan} *qudaS > \acs{psed} *qudas/*qudi `gray hair'. Moreover, these doublet forms correspond to some female and male forms in \acl{pata}. Take \acl{psed} the pair of *raŋaw `blowfly; robber fly' and *rəŋədi `housefly' as example, the two forms correspond to \acl{pata} *ɹaŋaw (f) and *ɹaŋəriʔ (m).  Table \ref{tab:ata_sed_grs_compare} shows a comparison between \acl{psed} and \acl{pata} forms.

\begin{table}[!htbp]
\centering
\caption{A comparison of some doublet pairs between \acl{psed} and \acl{pata}}
\label{tab:ata_sed_grs_compare}
\begin{tabular}{llll}
\hline
\acs{pan}   & \acl{psed}  & \acl{pata}       & Gloss                 \\ \hline
*Sapuy & *hapuy `to cook' & *hapuy (f)  & `fire'        \\
       & *puniq  & *hapuniq (m) & `fire'                \\
*qudaS & *qudas  & *quras (f)   & `gray hair'           \\
       & *qudi   & *quriʔ (m)   & `gray hair'           \\
*laŋaw & *raŋaw  & *ɹaŋaw (f)   & `blowfly; robber fly' \\
       & *rəŋədi & *ɹaŋəriʔ (m) & `housefly'            \\
*(w)iRi& *igi    & ---          & `left'                \\
       & *iril   & *ʔiɹil       & `left'                \\
*uka   & *uka    & *ʔukas       & `not exist'           \\
       & *uŋat   & *ʔuŋat       & `not exist'           \\ 
---    & *daŋi   & *raŋiʔ (f)   & `friend'              \\
       & *dawin  & *rawin (m)   & `friend'              \\ \hline
\end{tabular}
\end{table}

In addition, some more conservative forms can only be found in Seediq based on the current study, such as \acl{psed} *igi `left'. This word is clearly from \acs{pan} *(w)iRi, and the other derived form in \acl{psed} is *iril, which has a corresponding form *ʔiɹil in Proto-Atayal. However, Proto-Atayal does not keep the expected female form **ʔigiʔ. 

Based on the evidence in Seediq, some strategies of deriving innovative words probably had been fully developed before the split of \acl{psed} and \acl{pata}. Indeed, after the split of the two languages, the development of this system in \acl{pata} and Atayal dialects appear to be even more diverse and rich compared to the system in Seediq. 

However, we still cannot rule out the following possibilities: (1) the development of this system is a result of long-term cultural interaction between Seediq and Atayal at a later stage, and (2) \acl{psed} may not have retained differences in gender-specific vocabulary, but it retained only some strategies of word formation. 

Evidence supporting borrowing includes terms containing *q. \textcite{song2024sedq} suggests that the *q /q/ sound in (Pre-)\acl{psed} was borrowed from an early stage of Atayal, hence in Table \ref{tab:ata_sed_grs_compare}, terms like *puniq `fire' and *qudas/*qudi `gray hair' were likely from Atayal. Whether this implies that the entire system, including these words, was borrowed from Atayal requires further research. Because, similar developments are found in Seediq words that do not exist in Atayal, complicating the situation even further.

Nevertheless, we can establish at least some degree of relationship between the system in the two languages through the comparison of vocabulary and word formation strategies, which can also be related to their common development. Unfortunately, due to the lack of extensive evidence from modern languages, the strength of these inferences remains limited.

What have found in \acl{psed} provides some new evidence for the reconstruction of Proto-Atayalic lexicon and prehistory. This topic requires further future research to support its completeness.

\section{Summary}
\lipsum[1-4]

