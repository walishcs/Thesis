\chapter[Phonological sketches of Seediq dialects]{Phonological sketches of \\ Seediq dialects}\label{ch3}

This chapter provides the synchronic phonological sketches of five Seediq dialects: \acl{tg}, \acl{cto}, \acl{eto}, \acl{ctr}, and \acl{etr}. Section \ref{sec:3.1} discusses the phoneme inventories and phonotactics of all the five dialects. In section \ref{sec:3.2}, different kinds of alternations are discussed, especially those related to morphophonology.

\section{Phoneme inventories and phonotactics} \label{sec:3.1}

This section introduces the phoneme inventories and phonotactics of various Seediq dialects. While the consonant inventories of Seediq dialects are similar, there are still differences; however, the vowels, particularly in \acl{tg}, exhibit more distinctiveness.

Regarding phonotactics, there is a high level of consistency among dialects in terms of constraints on the occurrence of some sounds in certain positions. Stress placement is non-phonemic, with the majority of cases regularly falling on the penultimate syllable, while Central Toda exhibits a stress shift when the penultimate vowel is schwa /ə/ (See Section \ref{sec:cto_phonotactics}).

\subsection{Tgdaya phonology}

\subsubsection{Tgdaya consonant inventory} \label{sec:tg_C}

\acl{tg} has the largest number of consonants among all dialects. The \acl{tg} dialect has eighteen consonants, including seven stops \textit{p, t, k, q, b, d, g}, one affricate \textit{c} [ʦ], three fricatives \textit{s, x, h} [ħ], three nasals \textit{m, n, ŋ}, one lateral \textit{l}, one tap \textit{r} [ɾ], and two glides \textit{w, y} [j]. Table \ref{tab:tgC} shows the consonant inventory of \acl{tg}.

\begin{table}[!htbp]
\centering
\caption{Tgdaya consonant inventory\\}
\label{tab:tgC}
\begin{tabular}{l|ll|ll|ll|ll|ll|ll}
\hline
                    & \multicolumn{2}{c|}{Bilabial} & \multicolumn{2}{c|}{Dental/Alveolar} & \multicolumn{2}{c|}{Palatal} & \multicolumn{2}{c|}{Velar} & \multicolumn{2}{c|}{Uvular} & \multicolumn{2}{c}{Pharyngeal} \\ \hline
Stop                & p            & b           & t               & d               &             &               & k          & g          & q            &            &                 &              \\
Fricative           &               &              & s               &                  &             &               & x          &             &               &            & h [ħ]             &              \\
Affricate           &               &              & c [ʦ]              &                  &             &               &             &             &               &            &                 &              \\
Nasal               &               & m           &                  & n               &             &               &             & ŋ          &               &            &                 &              \\
Lateral &               &              &                  & l               &             &               &             &             &               &            &                 &              \\
Tap                 &               &              &                  & r [ɾ]              &             &               &             &             &               &            &                 &              \\
Glide               &               & w           &                  &                  &             & y [j]           &             &             &               &            &                 &              \\ \hline
\end{tabular}
\end{table}

<h> is realized as a pharyngeal [ħ] by more conservative speakers; on the contrary, other speakers pronounce it as a glottal [h] (also by some elder speakers). Further investigation is needed to determine whether this is influenced by age or other factors (such as societal ones). Additionally, careful phonetic studies are required. But phonologically, the realizations of this phoneme do not change the categorical properties, and it is clearly distinct from velar fricative /x/.

<c> /ʦ/ and <s> always palatalize to [ʨ] and [ɕ] before the high front vowel /i/, and whether it occurs in other environments depends on the speaker. Common environments include: (1) \textit{i}\_\#, and (2) \_\textit{e}. For example, \textit{madis} [madiɕ] `to bring (\acs{av})', \textit{cebu} [ʨebu] `to shoot'. 

<l> is a lateral sound, but in the realization of some speakers, it may exhibit slight frication. However, due to the lack of noise, it does not reach the level of [ɮ].

<r> is usually pronounced as an alveolar (or retracted) tap [ɾ\~{}ɾ̠] or a retroflex flap [ɽ]. However, as the only rhotic sound, there may be even more variations, such as approximants [ɹ\~{}ɻ] or even (rarely) a trill [r].

Regarding the glottal stop [ʔ], \textcite{yang1976sedpho} argues that it lacks phonemic status and that the phonetic realization of [ʔ] and ∅ can be in free variation. Conversely, \textcite{holmer1996parametric} classifies the glottal stop as the phoneme /ʔ/, but without further explanation. \textcite{Sung2018Sedgrammar} does not consider it as a phoneme and also does not provide an explanation. Since the occurrence of the glottal stop [ʔ] can be predicted and there is no phonemic distinction from ∅, I adopt Yang's view and do not consider the glottal stop as a phoneme in this thesis. This also applies to other dialects, and I will not elaborate further on this point in this thesis.

\subsubsection{Tgdaya vowel inventory}

\acl{tg} has five monophthongs: \textit{a, i, e, u, o}, as shown in Table \ref{tab:tgV}. In addition to monophthongs, \acl{tg} has a diphthong -\textit{uy} [uj]. 

\begin{table}[!htbp]
\centering
\caption{Tgdaya vowel inventory}
\label{tab:tgV}
\begin{tabular}{llll}
\hline
     & Front & Central & Back \\ \hline
High &  i    &         &  u   \\
Mid  &  e    &         &  o   \\
Low  &       &  a      &      \\ \hline
\end{tabular}
\end{table}

In Tgdaya, vowel length is not distinguished. If a sequence of two identical vowels is seen, the two vowels actually belong to two separate syllables, as in \textit{haan} [ħa.an] `to go (\acs{lv})'.

\subsubsection{Tgdaya phonotactics} \label{sec:tg_phonotactics}

Table \ref{tab:sy_ty_tg} displays the syllable types of \acl{tg}, listing the syllable types covered in slower and deliberate speech. In everyday and faster speech, other types may occur, which will be briefly introduced later on. Please note that the symbol C in the coda position includes all consonants except for glides, while G represents the two glides \textit{w} and \textit{y}. In the onset position, there is no distinction, and all non-vowel sounds are unified under C. This convention applies to other parts of the thesis as well.

\begin{table}[!htbp]
\centering
\caption{Syllable types in Tgdaya}
\label{tab:sy_ty_tg}
\begin{tabular}{lllllll}
\hline
Type & \multicolumn{2}{l}{Prepenultimate} & \multicolumn{2}{l}{Penultimate} & \multicolumn{2}{l}{Final}                \\ \hline
V             & ---              &                 & \textbf{i}.ma            & `who'         & sa.\textbf{i}        & `to go.\acs{imp} (\acs{pv})' \\
VC            & ---              &                 & ---             &               & sa.\textbf{un}       & `to go (\acs{pv})'           \\
VG            & ---              &                 & ---             &               & (not found) &                            \\
CV            & \textbf{qu}.si.ya         & `water'         & qu.\textbf{si}.ya        & `water'       & qu.si.\textbf{ya}    & `water'                    \\
CVC           & ---              &                 & ---             &               & ba.\textbf{raq}      & `lungs'                    \\
CVG           & ---              &                 & ---             &               & lu.\textbf{buy}      & `bag'                     \\ \hline
\end{tabular}
\end{table}

CV is the only syllable type allowed to appear in the prepenultimate position. Onsetless V will be deleted by a universal phonological rule, as seen in \textit{eyah} /ejaħ/ `to come' and \textit{yah-an} /ejaħ/ + /an/ `to come (\acs{lv})'. In the penultimate position, only V and CV structures are allowed to appear.

In the final position, \acl{tg} can at least have five structures: V, VC, CV, CVC, and CVG, with fewer restrictions. While the VC structure is speculated to be possible, examples may be relatively rare and harder to find. In other dialects, there is \textit{dəmuuy} `to hold in fist (\acs{av})', but irregularly corresponds to \textit{dumoi} in \acl{tg}.

In faster speech, two other syllable structures that are commonly found and are both related to nasals can occur: CVN and N̩. When the nasal consonant with an antepenultimate vowel is followed by a homorganic consonant, syncope occurs between the nasal and the following consonant. If there is a preceding syllable before the nasal consonant, the nasal consonant becomes the coda of the preceding syllable. If there is no preceding syllable, the nasal sound becomes syllabic. The occurrences of these syllable structures can be considered surface phenomena. For example, /tunusapaħ/ → [\textbf{tun}.sa.paħ] `family', and /mupitu/ → [\textbf{m̩}.pi.tu] `seven'. These variations may differ depending on individuals or subdialects.

Some sounds in \acl{tg} also have positional restrictions, some of them are categorical. For instance, voiced stops /b/, /d/, /g/ do not occur in word-final position. Labial consonants /p/ (including /b/) and /m/ are also not found in such position. 

The phoneme /x/ only appears at the word-initial position in the word \textit{\textbf{x}iluy} `iron', with around ten more examples found word-medially, but it is more common in word-final position.

In a significant portion of speakers, both liquids /l/ and <r> /ɾ/ merge with /n/ in word-final position, respectively. More conservative speakers maintain the distinctions, but generally, the merger of /l/ and /n/ in word-final position is more common than that of <r> /ɾ/ and /n/. Speakers of \acl{tg} who retain the word-final <r> often realize it as an approximant [ɹ\~{}ɻ], unlike the tap/flap in word-initial and word-medial positions. Of course, there may be more variants because, as mentioned earlier, this is the only rhotic sound in the language.

The word-final <y> /j/ can only occur with /u/, as in \textit{bab\textbf{uy}} `pig', while in other positions, there are no restrictions. As for /w/, it is totally banned in word-final position. However, according to \textcite{yang1976sedpho}, there is a distinction between -\textit{uw} and -\textit{u}. The absence of this distinction can be due to an ongoing merger and generational differences.

The mid vowels /e/ and /o/ primarily serve as the nucleus of open syllables due to historical developments (see Section \ref{sec:psedV}). However, with language development or analogical changes, there may be extremely rare exceptions, such as the occurrence of \textit{muuyas} \~{} \textit{muu\textbf{wes}} /muujas/ `to sing; to study (\acs{av})' in casual speech.

In \acl{tg}, there is a rule of neutralization of prepenultimate vowels, with the neutral vowel being /u/. This means that, in general, vowel distinctions only occur in penultimate and final syllables, while the vowels in prepenultimate syllables are neutralized to [u]. However, when the onset of the penultimate syllable is <h> /ħ/ or has no onset (i.e., the onset is ∅), the neutralized vowel of the antepenultimate syllable is assimilated to be the same as the penultimate syllable, as in \textit{m\textbf{e}h\textbf{e}diq} [m\textbf{e}.ħ\textbf{e}.diq] `to give way (\acs{av})' and \textit{s\textbf{ii}mah} [s\textbf{i}.\textbf{i}.maħ] `to drink (\acs{cv})'. However, there are exceptions, such as \textit{m\textbf{u}h\textbf{u}hediq} [m\textbf{u}.ħ\textbf{u}.ħe.diq] `to give way to each other' instead of *[m\textbf{e}.ħ\textbf{e}.ħe.diq], possibly indicating that vowel features tend to spread only once. Further synchronic phonological analysis is needed to explore this phenomenon.


\subsection{Central Toda phonology}

\subsubsection{Central Toda consonant inventory}

The \acl{cto} dialect has seventeen consonant, including six stops \textit{p, t, k, q, b, d}, one affricate \textit{c} [ʦ], three fricatives \textit{s, x, h} [ħ], three nasals \textit{m, n, ŋ}, one lateral \textit{l}, one tap \textit{r} [ɾ], and two glides \textit{w, y} [j]; the inventory is as given in Table \ref{tab:ctoC}.

\begin{table}[!htbp]
\centering
\caption{Central Toda consonant inventory}
\label{tab:ctoC}
\begin{tabular}{l|ll|ll|ll|ll|ll|ll}
\hline
                    & \multicolumn{2}{c|}{Bilabial} & \multicolumn{2}{c|}{Dental/Alveolar} & \multicolumn{2}{c|}{Palatal} & \multicolumn{2}{c|}{Velar} & \multicolumn{2}{c|}{Uvular} & \multicolumn{2}{c}{Pharyngeal} \\ \hline
Stop                & p            & b           & t               & d               &             &               & k          &            & q            &            &                 &              \\
Fricative           &               &              & s               &                  &             &               & x          &             &               &            & h [ħ]             &              \\
Affricate           &               &              & c [ʦ]              &                  &             &               &             &             &               &            &                 &              \\
Nasal               &               & m           &                  & n               &             &               &             & ŋ          &               &            &                 &              \\
Lateral &               &              &                  & l               &             &               &             &             &               &            &                 &              \\
Tap                 &               &              &                  & r [ɾ]              &             &               &             &             &               &            &                 &              \\
Glide               &               & w           &                  &                  &             & y [j]           &             &             &               &            &                 &              \\ \hline
\end{tabular}
\end{table}

\acl{cto} lacks the consonant <g> (/g/ or /ɣ/). Apart from this, the consonant inventory is the same as the one in \acl{tg}. 

<s> and <c> /ʦ/ also palatalize to [ɕ] and [ʨ] before the high front vowel /i/, but do not occur in other environments. The fricative <h> can be either pharyngeal [ħ] or glottal [h]. 

The lateral <l> also exhibits slight frication, similar to Tgdaya. The rhotic <r> can also be realized as an alveolar (or retracted) tap [ɾ\~{}ɾ̠] or a retroflex flap [ɽ], and in word-final position, it is also a tap/flap with a release, not an approximant.

\subsubsection{Central Toda vowel inventory}

\acl{cto} has four monophthongs: \textit{a, i, u, ə}, as shown in Table \ref{tab:ctoV}.

\begin{table}[!htbp]
\centering
\caption{Central Toda vowel inventory}
\label{tab:ctoV}
\begin{tabular}{llll}
\hline
     & Front & Central & Back \\ \hline
High &  i    &         &  u   \\
Mid  &       &  ə      &      \\
Low  &       &  a      &      \\ \hline
\end{tabular}
\end{table}

Apart from monophthongs, \acl{cto} has three phonemic diphthongs: /aw/, /aj/, and /uj/. The diphthong /aw/ has two allophones: [ow] and [aw]. [ow] appears in word-medial positions, as in \textit{dowriq} `eye', while [aw] occurs in word-final positions, as in \textit{babaw} `above; on top of'. The two allophones are in complementary distribution. The same situation also occurs with <ey> [ej] and <ay> [aj], where the phoneme /aj/ is the allophone in word-medial and word-final positions, as in \textit{meytaq} `to stab (\acs{av})' and \textit{balay} `true; correct'. However, <uy> /uj/ can only occur in word-final position.

\subsubsection{Central Toda phonotactics} \label{sec:cto_phonotactics}

The syllable types of \acl{cto} are shown in Table \ref{tab:sy_ty_cto}. 

\begin{table}[!htbp]
\centering
\caption{Syllable types in Central Toda}
\label{tab:sy_ty_cto}
\begin{tabular}{lllllll}
\hline
Type & \multicolumn{2}{l}{Prepenultimate} & \multicolumn{2}{l}{Penultimate} & \multicolumn{2}{l}{Final}                \\ \hline
V             & \textbf{u}.ci.luŋ & `lake; sea'    & \textbf{i}.ma            & `who'         & sa.\textbf{i}        & `to go.\acs{imp} (\acs{pv})' \\
VC            & ---              &                 & ---             &               & sa.\textbf{un}       & `to go (\acs{pv})'           \\
VG            & ---              &                 & \textbf{ow}.raw   & `k.o. bamboo' & də.mu.\textbf{uy} &  `to hold (\acs{av})'        \\
CV            & \textbf{qə}.si.ya         & `water'         & qə.\textbf{si}.ya        & `water'       & qə.si.\textbf{ya}    & `water'                    \\
CVC           & ---              &                 & ---             &               & ba.\textbf{raq}      & `lungs'                    \\
CVG           & ---              &                 & \textbf{dow}.riq     & `eye'    & lu.\textbf{buy}      & `bag'                     \\ \hline
\end{tabular}
\end{table}

\acl{cto} stands out in that, in the prepenultimate position, besides CV, it can also accept the presence of a single V. However, the only vowel that can appear in this position is /u/. The cause of this phenomenon stems from a historical change, detailed in Section \ref{sec:psedV}.

As for the penultimate position, it also allows the appearance of types such as VG, CVG, as in \textit{\textbf{ow}.raw} `k.o. bamboo' and \textit{\textbf{dow}.riq} `eye'. The final syllable permits all six types to occur.

The syncope phenomenon of antepenultimate vowels after a nasal mentioned earlier is also quite common in \acl{cto}, occurring more extensively and possibly even being obligatory for some speakers (requiring further extensive investigation). This phenomenon occurs not only when a nasal is followed by a homorganic consonant in the next syllable but also in other situations, but it seems to be limited to situations involving the infix <ən> `\acs{pst}; \acs{nmlz}'. For example: /qənəpaħan/ → [\textbf{qəm}.pa.ħan] `field (lit. the place for working)'. 

In the \acl{cto} dialect, the restrictions on the occurrence of certain phonemes are very similar to what we see in \acl{tg}. For example, voiced stops /b/, /d/, labial consonants /p/ (/b/), /m/ are not allowed in word-final positions. The position and frequency of /x/ appearing in the \acl{cto} dialect are almost identical to those in \acl{tg}. /x/ in word-initial position is only observed in the word \textit{xiluy} `iron'.

The neutralization of prepenultimate vowels can also be observed in \acl{cto}; however, in this dialect, the neutralized vowel is schwa /ə/. Therefore, any vowel occurring before the penultimate syllable, regardless of its underlying form (or diachronically speaking, its historical source), uniformly changes to /ə/. However, there is one unique exception, where the vowel /u/ occasionally appears in that position. For example: \textit{t\textbf{u}daya} `the north side'. This phenomenon is related to a series of historical changes, similar to the case of prepenultimate vowels being /u/, for the same reason (see the discussion in Section \ref{sec:psedV}).

In \acl{cto}, there is another notable restriction that we do not see a VV (or əV) sequence across the antepenultimate and penultimate syllables. If the penultimate syllable lacks an onset (i.e., it starts with a vowel), the neutralized vowel in the antepenultimate position does not appear on the surface. For example, /səədiq/ → [sə.diq] `human; person', /usaani/ (/usa/ + -/ani/) → [sa.ni] `to go.\acs{imp} (\acs{cv})'. This phenomenon can be explained in several ways: (1) deletion of the antepenultimate schwa /ə/ in this environment, (2) vowel coalescence after assimilation of the antepenultimate schwa /ə/ with the following vowel. Currently, only a few exceptions have been found, such as \textit{ciida} `at that time'.

As mentioned earlier, \acl{cto} has a stress shift rule. Generally, the stress falls on the penultimate syllable, but in \acl{cto}, if the vowel in the penultimate syllable is schwa /ə/, the stress shifts to the final syllable. However, during my fieldwork, I found that sometimes the stress still falls on the penultimate syllable in the same word, so this rule of stress shift should be considered more of a tendency rather than mandatory.

\subsection{Eastern Toda phonology}

\subsubsection{Eastern Toda consonant inventory}

The \acl{eto} dialect has sixteen consonant, including six stops \textit{p, t, k, q, b, d}, three fricatives \textit{s, x, h} [ħ], three nasals \textit{m, n, ŋ}, one lateral \textit{l}, one tap \textit{r} [ɾ], and two glides \textit{w, y} [j] (\cite{lee2015tawsa}); the inventory is as given in Table \ref{tab:etoC}.\footnote{The glottal stop /ʔ/ in \acl{eto} is treated as a phoneme in \textcite{lee2015tawsa}, but it is omitted in this thesis based on the reason mentioned earlier in Section \ref{sec:tg_C}.}

\begin{table}[!htbp]
\centering
\caption{Eastern Toda consonant inventory}
\label{tab:etoC}
\begin{tabular}{l|ll|ll|ll|ll|ll|ll}
\hline
                    & \multicolumn{2}{c|}{Bilabial} & \multicolumn{2}{c|}{Dental/Alveolar} & \multicolumn{2}{c|}{Palatal} & \multicolumn{2}{c|}{Velar} & \multicolumn{2}{c|}{Uvular} & \multicolumn{2}{c}{Pharyngeal} \\ \hline
Stop                & p            & b           & t \quad\quad\quad          & d               &             &               & k          &            & q            &            &                 &              \\
Fricative           &               &              & s               &                  &             &               & x          &             &               &            & h [ħ]             &              \\
Nasal               &               & m           &                  & n               &             &               &             & ŋ          &               &            &                 &              \\
Lateral &               &              &                  & l [ɮ]             &             &               &             &             &               &            &                 &              \\
Tap                 &               &              &                  & r [ɾ]              &             &               &             &             &               &            &                 &              \\
Glide               &               & w           &                  &                  &             & y [j]           &             &             &               &            &                 &              \\ \hline
\end{tabular}
\end{table}

The consonant inventory of \acl{eto} is the smallest among all Seediq dialects. Besides its own mergers, some further mergers are influenced by \acl{eto}. For example, \acl{eto} does not have a phonemic <c> /ʦ/ (merged into /s/), which is considered as an influence from \acl{etr} (\cite{lee2015tawsa}). 

\acl{eto} still has two allophonic symbols: <c> [ʨ] and <ɟ> [ɟ\~{ }ʥ]. These represent the occurrence of /t/ and /d/ as allophones before the high front vowel /i/ (or homorganic glide <y> /j/), such as \textit{cimu} [ʨimu] `salt' and \textit{səɟiq} [səɟiq] `person'. Listing these two allophonic sounds with different symbols aims to illustrate the dialectal differences while also adhering to the conventions of past research.

The lateral <l> is a lateral fricative [ɮ] (\cite{lee2012segment,lee2015tawsa}). Apart from the above-mentioned differences, all other consonants in \acl{eto} are the same as those in \acl{cto}.

\subsubsection{Eastern Toda vowel inventory}

\acl{eto} has four monophthongs: \textit{a, i, u, ə} (\cite{lee2012segment,lee2015tawsa}), as shown in Table \ref{tab:etoV}.

\begin{table}[!htbp]
\centering
\caption{Eastern Toda vowel inventory}
\label{tab:etoV}
\begin{tabular}{llll}
\hline
     & Front & Central & Back \\ \hline
High &  i    &         &  u   \\
Mid  &       &  ə      &      \\
Low  &       &  a      &      \\ \hline
\end{tabular}
\end{table}

\acl{eto} has three phonemic diphthongs: /aw/, spelled <ay> /aj/, and <uj> /uj/. The most significant difference from \acl{cto} (and the Truku dialects soon to be mentioned) is that /aw/ and <ay> /aj/ in \acl{eto} do not undergo raising in word-medial positions (as \textit{ow} and \textit{ey} [ej]). In other words, whether in word-medial or word-final positions, they are realized as \textit{aw} and \textit{ay} [aj], as in \textit{d\textbf{aw}riq} `eye', \textit{sən\textbf{aw}} `man; male', \textit{m\textbf{ay}taq} `to stab (\acs{av})', and \textit{har\textbf{ay}} `be not' (data obtained from \cite{lee2015tawsa}).

\subsubsection{Eastern Toda phonotactics} \label{sec:eto_phonotactics}


The syllable types of \acl{eto} are shown in Table \ref{tab:sy_ty_eto}. 

\begin{table}[!htbp]
\centering
\caption{Syllable types in Eastern Toda}
\label{tab:sy_ty_eto}
\begin{tabular}{lllllll}
\hline
Type & \multicolumn{2}{l}{Prepenultimate} & \multicolumn{2}{l}{Penultimate} & \multicolumn{2}{l}{Final}                \\ \hline
V             & \textbf{u}.si.luŋ & `lake; sea'    & \textbf{i}.ya            & `do not'         & ni.\textbf{i}        & `that' \\
VC            & ---              &                 & ---             &               & hi.ya.\textbf{an}       & `\textsc{3sg.obl}'           \\
VG            & ---              &                 & \textbf{aw}.yas   & `song' & də.mu.\textbf{uy} &  `to hold (\acs{av})'        \\
CV            & \textbf{qə}.si.ya         & `water'         & qə.\textbf{si}.ya        & `water'       & qə.si.\textbf{ya}    & `water'                    \\
CVC           & ---              &                 & ---             &               & a.\textbf{laŋ}      & `village'                    \\
CVG           & ---              &                 & \textbf{daw}.riq     & `eye'    & bu.\textbf{kuy}      & `back'                     \\ \hline
\end{tabular}
\end{table}

\acl{eto} accepts word-medial -\textit{aw}- and -\textit{ay}-, as mentioned earlier. Besides this, the constraints on other sounds in various positions are basically consistent with those in \acl{cto}, such as restrictions on voiced stops and labials at the word-final position. Interestingly, there remains a significant constraint on initial /x/. For instance, one of the negators, \textit{haray}, likely developed from Proto-Seediq *uxay with the addition of the infix <\textit{ra}> (see Section \ref{sec:sepical_doublet} for this infix). The development of this word is as follows: *uxay > **uxa<ra>y > **xaray > \textit{haray}. After the original *uxay received the infix *<ra>, the initial *u was deleted, moving *x to the initial position. However, the phonotactics of Seediq generally avoid having /x/ at the start of words, thus it was changed to \textit{h} /ħ/.

The stress shift observed in \acl{cto} (when /ə/ appears in the penultimate syllable) is not found in \acl{eto} (\cite[96--97]{lee2015tawsa}). \acl{eto} retains the stress on the penultimate syllable when /ə/ appears in that position, without shifting it to the final syllable. \textcite[97]{lee2015tawsa} suggests that the stress shift in \acl{cto} is a later development, possibly to avoid stress falling on schwa /ə/. Such a phenomenon can be also found in many languages.

Further analysis of other phonotactic issues in \acl{eto} will require more data.

\subsection{Central and Eastern Truku phonologies}
Both the Central one and the Eastern Truku dialects have the same phoneme inventory. Thus, they will be discussed together here. 

\subsubsection{Central and Eastern Truku consonant inventory}

The two Truku dialects have 17 consonants, including six stops \textit{p, t, k, q, b, d}, four fricatives \textit{s, x, g} [ɣ], \textit{h} [ħ], three nasals \textit{m, n, ŋ}, one lateral fricative \textit{l} [ɮ], one tap \textit{r} [ɾ], and two glides \textit{w, y} [j]. Table \ref{tab:trC} shows the inventory of consonants.

\begin{table}[!htbp]
\centering
\caption{Central and Eastern Truku consonant inventory}
\label{tab:trC}
\begin{tabular}{l|ll|ll|ll|ll|ll|ll}
\hline
                    & \multicolumn{2}{c|}{Bilabial} & \multicolumn{2}{c|}{Dental/Alveolar} & \multicolumn{2}{c|}{Palatal} & \multicolumn{2}{c|}{Velar} & \multicolumn{2}{c|}{Uvular} & \multicolumn{2}{c}{Pharyngeal} \\ \hline
Stop                & p            & b           & t \quad\quad\quad              & d               &             &               & k          &            & q            &            &                 &              \\
Affricate           &               &              &                &                  &             &               &             &             &               &            &                 &              \\
Fricative           &               &              & s               &                  &             &               & x          &   g [ɣ]          &               &            & h [ħ]             &              \\
Nasal               &               & m           &                  & n               &             &               &             & ŋ          &               &            &                 &              \\
Lateral fric. &               &              &                  & l [ɮ]              &             &               &             &             &               &            &                 &              \\
Tap                 &               &              &                  & r [ɾ]              &             &               &             &             &               &            &                 &              \\
Glide               &               & w           &                  &                  &             & y [j]           &             &             &               &            &                 &              \\ \hline
\end{tabular}
\end{table}

Truku lacks the phonemic affricate \textit{c}, but the phonetic \textit{c} can occur as an allophone of /t/ in word-final position. Table \ref{tab:trC} only displays the consonant phonemes of \acl{ctr} and \acl{etr}. However, there are two allophones that are distinguished in writing: <c> [ʦ\~{ }ʨ] and <ɟ> [ɟ\~{ }ʥ]. In \acl{ctr}, <c> at word-final position phonetically appears as [ʦ], functioning as an allophone of /t/; whereas <c> and <ɟ> occur before the high front vowel /i/ in other word-initial and word-medial positions in \acl{ctr} and \acl{etr}, respectively, serving as palatalized/affricated allophones of /t/ and /d/, as seen in examples like \textit{cimu} [ʨimu] `salt' and \textit{səəɟiq} [səəɟiq] `person'.

<g> is pronounced as an fricative [ɣ] in Truku, such as \textit{gaya} [ˈɣaja] `law, tradition', \textit{gaga} [ˈɣaɣa] `that'.

The phoneme <l> is a lateral fricative [ɮ], some \acl{etr} speakers pronounced it as a devoiced [ɬ] (\cite[16]{Lee2018Trugrammar}). In word-final position, it may merge with [n] (\cite[16,23]{Lee2018Trugrammar}). 

The phoneme <r> has more variants. The \acl{ctr} speakers tend to pronounce it as a tap [ɾ], [ɾ̠], or an approximant [ɹ̠], while some of the \acl{etr} speakers pronounce it more like a lateral [l] (e.g. \textcite{Chengetal2017Truku} transcribe this phoneme as [l], and \textcite{ogawaandasai1935} as \textit{ɭ} ). Based on my personal observation, the <r> sound in \acl{etr} does indeed sound relatively longer than other dialects in duration. However, whether it carries the [+lateral] feature phonologically requires further consideration. This is merely a phonetic realization and does not currently affect the overall phonological system; therefore, I still transcribe it as <r>.

The other consonants are the same to the ones in other dialects. 

\subsubsection{Central and Eastern Truku vowel inventory}

Like, \acl{cto} and \acl{eto}, the two Truku dialects also have four monophthongs: \textit{a, i, u, ə}, as shown in Table \ref{tab:trV}. 

\begin{table}[!htbp]
\centering
\caption{Truku vowel inventory}
\label{tab:trV}
\begin{tabular}{lccc}
\hline
     & Front & Central & Back \\ \hline
High &  i    &         &  u   \\
Mid  &       &  ə      &      \\
Low  &       &  a      &      \\ \hline
\end{tabular}
\end{table}

\textcite[12]{Lee2018Trugrammar} suggests that the phoneme /u/ also has an allophone [o]. This pure [o] sound (not diphthong [ow]) is found in a very limited number of words, mostly function words, such as the topic marker \textit{o} and the conjunction \textit{do}. The former corresponds to \textit{u} in Toda dialects, with a possible sporadic change in Truku dialects; while the latter may be a contraction of \textit{da} `\acs{csm}' and \textit{u}\~{}\textit{o} for `\acs{top}'. Therefore, I currently consider it as a marginal phoneme and temporarily exclude it from the table.

\acl{ctr} and \acl{etr} also have three phonemic diphthongs: /aw/, /aj/, and /uj/. The allophones of /aw/ and /aj/ word-medially are [ow] and <ey> [ej], respectively. Meanwhile, /uj/ also does not occur in word-medial positions. These distributions are the same as those of \acl{cto}.

\subsubsection{Central and Eastern Truku phonotactics}

\acl{ctr} and \acl{etr} allow VC (which is actually VN) structures to occur in prepenultimate positions, while nasal codas must be homorganic with the following vowel. The other syllable structure types are the same as those in \acl{cto}. However, VG structures in the penultimate position are only found in \acl{ctr}, with \textbf{\textit{ow}} in \textit{\textbf{ow}raw} `k.o. bamboo' possibly being the only example representing this structure, while no corresponding form is found in \acl{etr}. The syllable structures of \acl{ctr} and \acl{etr} are shown in Table \ref{tab:sy_ty_tr}.

\begin{table}[!htbp]
\centering
\caption{Syllable types in Central and Eastern Truku}
\label{tab:sy_ty_tr}
\begin{tabular}{lllllll}
\hline
Type & \multicolumn{2}{l}{Prepenultimate} & \multicolumn{2}{l}{Penultimate} & \multicolumn{2}{l}{Final}                \\ \hline
V             & \textbf{ə}.u.da   & `thing'         & \textbf{i}.ma              & `who'         & sa.\textbf{i}        & `to go.\acs{imp} (\acs{pv})' \\
VC            & \textbf{əm}.pi.tu & `seven'        & ---                         &               & sa.\textbf{un}       & `to go (\acs{pv})'           \\
VG            & ---              &                 & \textbf{ow}.raw (\acs{ctr})   & `k.o. bamboo' & də.mu.\textbf{uy} &  `to hold (\acs{av})'        \\
CV            & \textbf{qə}.si.ya  & `water'       & qə.\textbf{si}.ya          & `water'       & qə.si.\textbf{ya}    & `water'                    \\
CVC           & ---              &                 & ---                        &               & ba.\textbf{raq}      & `lungs'                    \\
CVG           & ---              &                 & \textbf{dow}.riq           & `eye'    & lu.\textbf{buy}      & `bag'                     \\ \hline
\end{tabular}
\end{table}

Some subdialects of \acl{etr} may exhibit further syncope, and the pattern is not observed in other dialects, such as /mətəɣəsa/ → [mə.təɣ.sa] `to teach (\acs{av})'. However, there is currently no comprehensive synchronic phonological research on this issue. This thesis also excludes these innovative features of individual subdialects from the table because they are not present in the majority or all of the Truku dialects.

As for the phonological constraints in \acl{ctr} and \acl{etr}, they are generally similar to those mentioned in several other dialects, including the prohibition of voice stops and labial sounds in the word-final position. However, <g> [ɣ] is allowed to appear in word-final positions, but only when combined with the high vowels /i/ and /u/, as in \textit{barig} `to buy' and \textit{elug} `road; path'. However, among relatively younger speakers, the endings \textit{-ig} and \textit{-ug} have respectively shifted to -\textit{iy} ([-ij] or [-iː]) and -\textit{uw} ([-uw] or [-uː]). This phenomenon is also observed in most Atayal dialects (\cite{li1982aicage,goderich2020phd}), and there are varying degrees of changes in other Seediq dialects (see Section \ref{sec:psedC}).

Vowels in prepenultimate positions still neutralize to schwa /ə/. They assimilate to homorganic [i] before the glide /y/ (less often to [u] before /w/). However, there are still a few exceptions, especially in the numerals `six' \textit{m\textbf{a}taru} and `eight' \textit{məsəpac} \~{} \textit{m\textbf{a}səpac} (\acs{ctr}), \textit{m\textbf{a}səpat} (\acs{etr}). These two numerals may be borrowed from Atayal.

\section{Synchronic morphophonological alternations} \label{sec:3.2}

In each Seediq dialect, there are some phonological alternations, often resulting from the interaction between position of the sounds and morphology. Interestingly, the patterns of alternations among dialects exhibit a considerable degree of consistency. Most alternating patterns can be traced back to \acl{psed}, implying that synchronic alternations can be viewed as remnants of the Proto-Seediq (See Section \ref{sec:psed_alter} for the alternations in \acl{psed}).

This section attempts to explore all alternations as comprehensively as possible. However, there may still be areas that are not fully developed, such as specific alternations in subdialects of certain villages. In this section, segments participating in alternations and currently under discussion will be marked with bold font, while alternations that are not the immediate focus will not be highlighted prominently. 

In this section, if there is no difference between the phenomena in the Central and Eastern varieties of Toda and Truku, only one example will be chosen from each, typically Central Toda and Eastern Truku, to represent them.

\subsection{Vowel alternations in Seediq dialects}

\subsubsection{Prepenultimate vowel neutralizing}

As discussed in Section \ref{sec:3.1}, all Seediq dialects exhibit the phenomenon of prepenultimate vowel neutralization. This phenomenon can also be referred to as weakening (\cite[22--23]{Lee2018Trugrammar}). The neutralizing vowel in Tgdaya is /u/, while in other dialects, it is /ə/. 

This rule applies in morphological processes as well. For example, when adding monosyllabic suffixes such as -\textit{un} `\textsc{pv}' or -\textit{an} `\textsc{lv}', the original penultimate syllable becomes antepenultimate and thus neutralizes. Some suffixes consist of two syllables, such as -\textit{ani} `\acs{imp}.\textsc{cv}'. If the root originally has two syllables, then both vowels of the two syllables undergo neutralization. Refer to Table \ref{tab:Vneu} for relevant examples in \acl{cto}, \acl{etr}, and \acl{tg}.

\begin{table}[!htbp]
\centering
\caption{Vowel neutralization in Seediq dialects}
\label{tab:Vneu}
\begin{tabular}{lllll}
\hline
Dialect   & Bare form & Suffix & Suffixed form & Gloss      \\ \hline
\acl{cto} & c\textbf{i}kul     & -un    & c\textbf{ə}kulun       & `to push'  \\
\acl{etr} & b\textbf{ey}taq    & -an    & b\textbf{ə}taqan       & `to stab'  \\
\acl{tg}  & p\textbf{a}t\textbf{i}s     & -ani   & p\textbf{u}t\textbf{u}sani      & `to write' \\ \hline
\end{tabular}
\end{table}

In native words, a few exceptions have been discussed in the previous section. However, in modern loanwords borrowed from Japanese and Taiwanese, most do not follow this rule. Table \ref{tab:Vneu_loan} presents those examples where the rule applies. Particularly noteworthy is the word for `television', which in the \acl{tg} dialect is \textit{terebi}, where the rule does not apply, but in \acl{cto} \textit{təreybi} and \acl{ctr} \textit{təribi}, it does.

\begin{table}[!htbp]
\centering
\caption{Vowel neutralization in loanwords}
\label{tab:Vneu_loan}
\begin{tabular}{llll}
\hline
Dialect   & Example & Source                   & Gloss                    \\ \hline
\acl{tg}  & c\textbf{u}bukiŋ & ts\textbf{a}-bóo-king (Taiwanese) & `brothel' > 'prostitute' \\
\acl{cto} & t\textbf{ə}reybi & t\textbf{e}rebi (Japanese)        & `television'             \\
\acl{ctr} & t\textbf{ə}ribi  & t\textbf{e}rebi (Japanese)        & `television'                \\ \hline
\end{tabular}
\end{table}

\subsubsection{Alternations related to /ə/ or /e/} \label{sec:eu_alt}

In \acl{cto}, \acl{eto}, \acl{ctr}, and \acl{etr}, schwa [ə] is not allowed to occur in the final syllable, which is a quite stable phonetic restriction. In some words where the surface form ends in -\textit{u}(C)\# and a monosyllabic suffix is added, the [u] sound changes to [ə], which is related to the aforementioned restriction. Therefore, from a synchronic phonological perspective, [u] is an allophone of the phoneme /ə/ in the final syllable. 

In the phonological system of \acl{tg}, the sound [ə] is lacking, but we can observe a similar phenomenon, namely the alternation between [e] and [u]. The /e/ sound in Tgdaya originates partly from the historical diphthong *ay and partly from *ə. The alternation observed here is related to the historical schwa *ə. Section \ref{sec:psedV} will discuss the particular reflex of \acl{psed} *ə in \acl{tg}.

Many past studies have explored this phenomenon as well, such as \textcite{yang1976sedpho,Sung2018Sedgrammar,Lee2018Trugrammar}. This alternation is certainly a remnant of Proto-Atayalic, since \acl{pan} *ə and *u merged as \acl{paic} *u in the final syllable (see \cite{li1981paic}). Table \ref{tab:eu_alt} displays relevant examples from different dialects.

\begin{table}[!htbp]
\centering
\caption{Alternations related to /ə/ or /e/ in \acl{tg}, \acl{cto}, and \acl{etr}}
\label{tab:eu_alt}
\begin{tabular}{llll}
\hline
Dialect   & Bare form & Suffixed form & Gloss        \\ \hline
\acl{tg}  & deŋ\textbf{u}      & duŋ\textbf{e}un        & `(to) dry'   \\
\acl{cto} & kər\textbf{u}t     & kər\textbf{ə}tun       & `to cut'     \\
\acl{etr} & rəq\textbf{u}n     & rəq\textbf{ə}nun       & `to swallow' \\ \hline
\end{tabular}
\end{table}

However, in the Toda and Truku dialects, there is a phenomenon involving opacity, namely the surface alternation between [u] and [a/i/(u)]. Such a phenomenon is also related to the (historical) schwa, but it actually involves vowel assimilation, so it will be discussed in Section \ref{sec:vassim}.

\subsubsection{Alternations related to vowel hiatus -\textit{a.i}- and -\textit{a.u}-} \label{sec:sed_ai_au}

The structure such as C\textit{a}.\textit{i}C or C\textit{a}.\textit{u}C in the stem generates a special alternation when a monosyllabic suffix is attached. Taking the suffix -\textit{un} `\textsc{pv}' as an example, in \acl{tg}, C\textit{a.i}C-un and C\textit{a.u}C-un respectively become C\textit{e}.C\textit{un} and C\textit{o}.C\textit{un}, in \acl{cto}, \acl{ctr}, and \acl{etr} dialects they become C\textit{ey}.C\textit{un} and C\textit{ow}.C\textit{un}. In the \acl{eto} dialect they 
are supposedly to become C\textit{ay}.C\textit{un} and C\textit{aw}.C\textit{un}. Based on the available data, we can only confirm the alternation between -\textit{a.u}- and -\textit{aw}-. However, since -\textit{ay}- and -\textit{aw}- usually act in parallel, it is reasonable to hypothesize that \acl{eto} also has an alternation between -\textit{a.i}- and -\textit{ay}-. This means that the two vowels originally belonging to different syllables merge when adding a monosyllabic suffix, leading to the formation of a diphthong or monophthongization in different dialects. Table \ref{tab:aiau_alt} illustrates examples from \acl{tg}, \acl{etr}, and \acl{eto}.

\begin{table}[!htbp]
\centering
\caption{Alternations related to vowel hiatus -\textit{a.i}- and -\textit{a.u}-}
\label{tab:aiau_alt}
\begin{tabular}{llll}
\hline
Dialect   & Bare form   & Suffixed form & Gloss                           \\ \hline
\acl{tg}  & tu.nu.n\textbf{a.i}s & tu.nu.n\textbf{e}.sun  & `rapidly'      \\
\acl{etr} & pə.n\textbf{a.i}s    & pə.n\textbf{ey}.sun    & `rapidly'                       \\
\acl{eto} & (-a.i-)   & (-ay-)    &                                \\ 
\hdashline
\acl{tg}  & t\textbf{a.u}s       & t\textbf{o}.sun        & `to wave hand' \\
\acl{etr} & t\textbf{a.u}s       & t\textbf{ow}.sun       & `to wave hand' \\
\acl{eto} & p\textbf{a.u}x       & p\textbf{aw}.xi        & `to turn over' (\cite[98]{lee2015tawsa})\\ \hline
\end{tabular}
\end{table}

However, there is one exception: the word ``to sew'' in \acl{tg} and \acl{etr} has a bare form of \textit{sais}, but when the suffix -\textit{un} `\textsc{pv}' is attached, instead of the expected **sesun and **seysun, it becomes \textit{si.i.sun} and \textit{sə.i.sun}, respectively. This may be related to the historical development of Seediq and cannot be simply viewed as a synchronic phenomenon. The special case of this word will be revisited from a diachronic perspective in Section \ref{sec:psed_alter}.

\subsubsection{Vowel coalescence} \label{sec:V_coal}

The environment for vowel coalescence in Seediq dialects is quite consistent across all dialects, with the exception of \acl{cto} and \acl{eto} having an additional environment compared to other dialects. In Sections \ref{sec:cto_phonotactics} and \ref{sec:eto_phonotactics}, I have already discussed the occurrence of vowel sequences in the antepenultimate and penultimate syllables in both Toda dialects. 

The common environment in other dialects is when a stem ends in a CV.V.CV(C) structure, and when a suffix is added, the original first two vowels, which belong to separate syllables, merge into one and are neutralized. Examples from each dialect and a comparison with \acl{cto} are shown in Table \ref{tab:v_coalescence}.

\begin{table}[!htbp]
\centering
\caption{Vowel coalescence in Seediq dialects}
\label{tab:v_coalescence}
\begin{tabular}{llll}
\hline
Dialect   & Bare form & Suffixed form & Gloss                          \\ \hline
\acl{tg}  & pul\textbf{aa}le   & pul\textbf{u}layun     & \multirow{3}{*}{`to do first'} \\
\acl{etr} & pəl\textbf{əa}lay  & pəl\textbf{ə}layun     &                                \\ \cdashline{1-3}
\acl{cto} & pəlalay   & pələlayun     &                                \\ \hline
\end{tabular}
\end{table}

\subsubsection{Apharesis} %首音刪除

The term ``apharesis'' refers to ``A sound change in which a word-initial vowel is lost, as in the occasional pronunciation of \textit{American} as `\textit{merican}'.'' (\cite[13]{campbell2007HLglossary}).

In Seediq, this phenomenon occurs extensively in all dialects. When there is a disyllabic stem and its penultimate syllable is onsetless, this phenomenon occurs. The aforementioned stem (VCV(C)) undergoes deletion when adding a suffix because the vowel has shifted to a non-stressed syllable position and lacks onset ``protection''. Table \ref{tab:apharesis} shows relevant examples from \acl{tg}, \acl{cto}, and \acl{etr}. In the table, the deleted vowels are represented by an underscore ``\_''.

\begin{table}[!htbp]
\centering
\caption{Apharesis in Seediq dialects}
\label{tab:apharesis}
\begin{tabular}{lllll}
\hline
Dialect   & Bare form & \textsc{stem}-un & \textsc{stem}-ani & Gloss              \\ \hline
\acl{tg}  & \textbf{e}yah      & \_yahun          & \_yahani          & `to come'          \\
\acl{cto} & \textbf{i}mah      & \_mahun          & \_məhani          & `to drink'         \\
\acl{etr} & \textbf{a}pa       & \_paun           & \_pəani           & `to carry on back' \\ \hline
\end{tabular}
\end{table}

From the table above, it can be observed that regardless of whether monosyllabic or disyllabic suffixes are added, when a vowel originally located in the penultimate syllable shifts to any prepenultimate syllable, it is deleted. However, there is one exception to this rule, which occurs during reduplication and may not apply uniformly across all dialects. For example, in the case of the \acl{tg} word \textit{alaŋ} `village', its reduplicated form can be either \textit{ula\~{}alaŋ} or \textit{la\~{}alaŋ}, while in \acl{etr} it must be \textit{ələ\~{}alaŋ}.


\subsubsection{Vowel assimilation} \label{sec:vassim}

The different dialects of Seediq all exhibit the phenomenon of vowel assimilation, but \acl{tg} and non-\acl{tg} are almost entirely separable in terms of the patterns they involve. 

In \acl{tg}, vowel assimilation typically occurs on the left side of the word. More precisely, this assimilation often takes place between the antepenultimate and penultimate syllables, as mentioned in Section \ref{sec:tg_phonotactics}. When the penultimate syllable is onsetless or has an onset of \textit{h}, the vowel of the antepenultimate syllable assimilates to that of the penultimate syllable. This applies regardless of whether the antepenultimate vowel is originally in the word or when a bisyllabic stem is prefixed with a CV- prefix. See \textcite[666-673]{yang1976sedpho} for more details about this issue. Table \ref{tab:tg_vassim} illustrates the vowel assimilation patterns in \acl{tg}. For ease of presentation, examples involving prefixation (\textit{mu-} `\textsc{av}' and \textit{su}- `\textsc{cv}') are used in the table to show the differences.

\begin{table}[!htbp]
\centering
\caption{Vowel assimilation patterns in \acl{tg}}
\label{tab:tg_vassim}
\begin{tabular}{llll}
\hline
Dialect  & Bare form & Prefixed form & Gloss                       \\ \hline
\acl{tg} & apa       & s\textbf{a}apa         & `to carry on back (\acs{cv})' \\
\acl{tg} & imah      & s\textbf{i}imah        & `to drink (\acs{cv})'         \\
\acl{tg} & hulis     & s\textbf{u}hulis       & `to laugh (\acs{cv})'         \\
\acl{tg} & hediq     & m\textbf{e}hediq       & `to give way (\acs{av})'      \\
\acl{tg} & oda       & m\textbf{o}oda         & `to go through (\acs{av})'          \\ \hline
\end{tabular}
\end{table}

In non-\acl{tg} dialects, as mentioned earlier in Section \ref{sec:eu_alt}, there is an alternation involving opacity. This phenomenon is related to both historical schwa and vowel assimilation. In these dialects, if the underlying /ə/ appears in the penultimate syllable and the final syllable is onsetless, the penultimate schwa is assimilated to final vowel following it. In other words, when suffixes are added to stems containing final /ə/ (all suffixes in Seediq begin with a vowel), the schwa is assimilated to the following vowel, leading to a certain degree of opacity. Table \ref{tab:eu_alt_special} compares \acl{etr} with \acl{tg}.

\begin{table}[!htbp]
\centering
\caption{Alternations of [u] and other vowels in \acl{etr} and a comparison with \acl{tg}}
\label{tab:eu_alt_special}
\begin{tabular}{llllll}
\hline
Dialect   & Bare form & \textsc{stem}-i & \textsc{stem}-un & \textsc{stem}-an & Gloss    \\ \hline
\acl{etr} & dəŋ\textbf{u} /dəŋə/      & dəŋ\textbf{i}i   & dəŋ\textbf{u}un   & dəŋ\textbf{a}an   & `to dry' \\
\acl{tg}  & deŋ\textbf{u} /deŋe/      & duŋ\textbf{e}i   & duŋ\textbf{e}un   & duŋ\textbf{e}an   & `to dry' \\ \hline
\end{tabular}
\end{table}

\subsubsection{Alternation between monophthongs and diphthongs (``VG'' sequences)}

This phenomenon is only observed in \acl{tg}. Stems containing -\textit{o} (partially) and -\textit{e} exhibit alternations with -\textit{a.w}- and -\textit{a.y}- after adding suffixes. The alternation between -\textit{e} and -\textit{u}- has been discussed earlier in Section \ref{sec:eu_alt}, while the alternation involving -\textit{o} and -\textit{a.g}- will be introduced in Section \ref{sec:Vg_alt}. Relevant examples can be found in Table \ref{tab:oeaway_alt}.

\begin{table}[!htbp]
\centering
\caption{Alternation between monophthongs and diphthongs in \acl{tg}}
\label{tab:oeaway_alt}
\begin{tabular}{llll}
\hline
Dialect  & Bare form & Suffixed form & Gloss      \\ \hline
\acl{tg} & su.lu.h\textbf{e}  & su.lu.h\textbf{a.y}un  & `to learn' \\
\acl{tg} & ku.ka.r\textbf{o}  & ku.ku.r\textbf{a.w}an  & `to climb' \\ \hline
\end{tabular}
\end{table}

\subsection{Consonant alternations in Seediq dialects}

Consonantal alternations are highly correlated with the word-final position of the sound (with a few exceptions), often occurring when attaching suffixes due to no longer being in such position. However, sometimes a set of alternations may be related to more than one phonological rule, making classification challenging. Therefore, this section will provide separate explanations through direct examples and combine similar cases for discussion.

\subsubsection{Alternations between \textit{c} (or \textit{t}) and \textit{d}/\textit{t}} \label{sec:dtc_alt}

This alternation occurs in \acl{tg}, \acl{cto}, \acl{eto}, and \acl{ctr} because /d/ and /t/ are not allowed in word-final positions and both are affricated (and devoiced) as \textit{c} [ʦ]. Roots containing final /d/ or /t/ in non-suffixed forms meet these conditions, resulting in the alternation of word-final /d/ or /t/ to \textit{c} [ʦ]. However, when suffixes are attached, [d] and [t] are allowed to appear since they are no longer in word-final positions, leading to this alternation. Please refer to Table \ref{tab:dtc_alt} for relevant examples.

\begin{table}[!htbp]
\centering
\caption{\textit{d}/\textit{t} and \textit{c} alternations in Seediq dialects}
\label{tab:dtc_alt}
\begin{tabular}{llll}
\hline
Dialect   & Bare form & Suffixed form & Gloss                \\ \hline
\acl{tg}  & haŋu\textbf{c}     & huŋe\textbf{d}un       & `to cook'            \\
\acl{cto} & qiyu\textbf{c}     & qiyu\textbf{t}un       & `to bite'            \\
\acl{ctr} & lutu\textbf{c}     & lətu\textbf{d}un       & `to join two things' \\ \hline
\end{tabular}
\end{table}

In \acl{etr}, word-final /d/ and /t/ merge into -\textit{t} (though phonetically, both -\textit{t} and -\textit{c} may appear, but the orthography keeps the more conservative form). Thus, we only observe the alternation between -\textit{d}- and -\textit{t}. For example, \textit{haŋu\textbf{t}} and \textit{həŋə\textbf{d}un} `to cook'.

\subsubsection{Alternations between labial sounds and velar sounds}

This alternation also arises due to phonological constraints on word-final positions of the Seediq language, namely, (1) bilabial sounds are not allowed in word-final positions, and (2) voiced stops are not allowed in word-final positions. Consequently, the underlying forms of /b/ and /p/ become and merge as the velar [k]. Similarly, when suffixes are attached to roots ending in /b/ or /p/, the aforementioned environment disappears, and thus the original sound remains unchanged, resulting in alternation between different forms. This alternation occurs in all Seediq dialects. Table \ref{tab:bpk_alt} shows the examples from different dialects. 

\begin{table}[!htbp]
\centering
\caption{\textit{k} and \textit{b}/\textit{p} alternations in Seediq dialects}
\label{tab:bpk_alt}
\begin{tabular}{llll}
\hline
Dialect   & Bare form & Suffixed form & Gloss                  \\ \hline
\acl{tg}  & elu\textbf{k}      & le\textbf{b}un         & `to close'             \\
\acl{cto} & adu\textbf{k}      & du\textbf{p}an         & `to hunt with dogs'   \\
\acl{etr} & qari\textbf{k}     & qəri\textbf{b}un       & `to cut (w/ scissors)' \\ \hline
\end{tabular}
\end{table}

The following alternation, similar to the previous one, arises from the phonological constraint that labial sounds cannot appear in word-final positions and are replaced by their velar counterparts. However, in this case, it involves nasal sounds. This alternation occurs in all Seediq dialects as well. See Table \ref{tab:mng_alt} for the examples. 

\begin{table}[!htbp]
\centering
\caption{\textit{m} and \textit{ŋ} alternations in Seediq dialects}
\label{tab:mng_alt}
\begin{tabular}{llll}
\hline
Dialect   & Bare form & Suffixed form & Gloss          \\ \hline
\acl{tg}  & geru\textbf{ŋ}     & gure\textbf{m}un       & `to break'     \\
\acl{cto} & səməsu\textbf{ŋ}   & səməsə\textbf{m}an     & `to celebrate' \\
\acl{etr} & lau\textbf{ŋ}      & low\textbf{m}un        & `to burn'      \\ \hline
\end{tabular}
\end{table}

\subsubsection{Alternations related to -V\textit{g}- sequences} \label{sec:Vg_alt}

This type of alternation does not apply to the \acl{cto} and \acl{eto} dialects because they do not have the phoneme /g/ or /ɣ/ in their phoneme inventories. In the remaining dialects, the changes involving the sound \textit{g} are quite complex, especially in the \acl{tg} dialect. \textcite[653-57]{yang1976sedpho} discusses the phenomenon using rule-based phonology, and \textcite{songandpan2024g} analyze this set of alternations from the perspective of Optimality Theory. Here, I will only examine the surface alternation, as the goal is solely to understand its patterns. And because the \acl{tg} dialect differs significantly from the two Truku dialects (with \acl{etr} as a representative here, since they behave identically), they will be discussed separately.

The sound /g/ is not permitted to appear at the end of words in the \acl{tg} dialect. However, unlike other voiced stops which directly undergo devoicing, /g/ tends to undergo an alternation with the preceding vowel. In \acl{tg}, we observe the following four types of alternations: \textit{ag} \~{ } \textit{o}, \textit{ig} \~{ } \textit{uy}, \textit{ug} \~{ } \textit{u}, and \textit{eg} \~{ } \textit{u}. The vowel alternations in the last two cases are related to the phenomenon mentioned in Section \ref{sec:eu_alt}.

\begin{table}[!htbp]
\centering
\caption{Alternations related to -V\textit{g}- in Tgdaya}
\label{tab:Vg_alt_tg}
\begin{tabular}{llll}
\hline
Dialect                   & Bare form      & Suffixed form    & Gloss      \\ \hline
\multirow{4}{*}{\acl{tg}} & reŋ\textbf{o}  & ruŋ\textbf{ag}un & `to speak'          \\
                          & bar\textbf{uy} & bur\textbf{ig}un & `to buy'            \\
                          & tab\textbf{u} & tub\textbf{ug}un & `to raise (animal)' \\
                          & sep\textbf{u} & sup\textbf{eg}un & `to count'          \\ \hline
\end{tabular}
\end{table}

In \acl{etr}, the situation is much simpler. In fact, we only observe the alternation between \textit{aw} and \textit{ag} [aɣ], as seen in examples like \textit{rəŋ\textbf{aw}} \~{} \textit{rəŋ\textbf{ag}un} `to speak'. However, in more recent variaties of the Truku dialects, -\textit{ig} and -\textit{ug} has changed to -\textit{iy} and -\textit{uw}, respectively, leading to the emergence of more alternation patterns. Table \ref{tab:Vg_alt_etr} illustrates these alternations in the newer variety of the \acl{etr} dialect.

\begin{table}[!htbp]
\centering
\caption{Alternations related to -V\textit{g}- in \acl{etr}}
\label{tab:Vg_alt_etr}
\begin{tabular}{llll}
\hline
Dialect                   & Bare form      & Suffixed form    & Gloss      \\ \hline
                          & rəŋ\textbf{aw}  & rəŋ\textbf{ag}un & `to speak'          \\
        \acl{etr}         & bar\textbf{iy} & bər\textbf{ig}un & `to buy'            \\
     (Newer variety)      & tab\textbf{uw} & təb\textbf{ug}un & `to raise (animal)' \\
                          & səp\textbf{uw} & səp\textbf{əg}un & `to count'          \\ \hline
\end{tabular}
\end{table}

\subsubsection{Alternations between -\textit{n} and -\textit{l}/\textit{r}- }

This alternation is exemplified by \acl{tg}, as it represents the most complete set of changes associated with this phenomenon. As mentioned in Section \ref{sec:tg_phonotactics}, in the \acl{tg} dialect, most native speakers have merged the word-final -\textit{l} and -\textit{r} into -\textit{n}. However, if stems containing final -\textit{l}/\textit{r} are attached with a suffix, the target segment is no longer in the word-final position, causing the original \textit{l} or \textit{r} to reappear. Examples related to this phenomenon are shown in Table \ref{tab:lrn_alt}.

\begin{table}[!htbp]
\centering
\caption{-\textit{n} and -\textit{l}/\textit{r}- \~{} alternations in Tgdaya}
\label{tab:lrn_alt}
\begin{tabular}{llll}
\hline
Dialect  & Bare form & Suffixed form & Gloss        \\ \hline
\acl{tg} & ciku\textbf{n}     & cuku\textbf{l}un      & `to push'    \\
\acl{tg} & aŋa\textbf{n}      & ŋa\textbf{l}un        & `to take'    \\
\acl{tg} & taku\textbf{n}     & tunuku\textbf{r}an    & `to fall'    \\
\acl{tg} & sudaha\textbf{n}   & suduha\textbf{r}an    & `to lean on' \\\hline
\end{tabular}
\end{table}

\subsubsection{Alternations between initial \textit{p}/\textit{b} and \textit{m}} \label{sec:pbm_alter}

This alternation is one of the few regular dissimilation phenomena in Seediq. It occurs when a stem beginning with \textit{p} or \textit{b} takes the infix <um> (\acs{tg})/<əm> (other dialects) `\textsc{av}'. For example, in the \acl{etr} dialect, the verb root \textit{patas} `to write', when attached with the suffix <əm>, would be expected to form **pəmatas. However, due to both \textit{p} and \textit{m} being bilabial consonants, this leads to a dissimilation resulting in the deletion of the entire syllable ``\textit{pə}-''. Then the surface form of \textit{patas} + <\textit{əm}> is thus \textit{matas}, so, it appears as if the initial \textit{p}-/\textit{b}- is directly replaced by \textit{m}- by looking from the surface. Examples from \acl{tg}, \acl{cto}, and \acl{etr} showing this phenomenon are presented in Table \ref{tab:bpm_alt}.

In fact, the phenomenon described above is exactly what \textcite{blust1998thao} refers to as ``pseudo nasal substitution'' (PNS), i.e., the avoidance of dissimilar labials as the onsets of successive syllables in Austronesian languages. 

\begin{table}[!htbp]
\centering
\caption{Initial \textit{p}/\textit{b} \~{} \textit{m} alternations in Seediq dialects}
\label{tab:bpm_alt}
\begin{tabular}{llll}
\hline
Dialect   & Bare form & Infixed form & Gloss      \\ \hline
\acl{tg}  & \textbf{b}ekuy     & \textbf{m}ekuy        & `to tie'   \\ 
\acl{cto} & \textbf{b}ari      & \textbf{m}ari         & `to buy'   \\
\acl{etr} & \textbf{p}atas     & \textbf{m}atas        & `to write' \\ \hline
\end{tabular}
\end{table}


\subsubsection{Palatalization and affrication of dental/alveolar stops before \textit{i} and \textit{y}}

This type of alternation can be observed in \acl{eto}, \acl{ctr}, and \acl{etr} due to the characteristic of palatalization of the sounds /t/ and /d/ before the high front vowel /i/ in these dialects, often accompanied by affrication. For convenience, this phenomenon is referred to as palatalization throughout this thesis. We can further classify these alternations into three types based on their manifestations: (1) palatalization disappears after adding infixes, (2) palatalization appears at the left side of the word after adding a suffix, and (3) palatalization appears at the right side of the word after adding a suffix -\textit{i} `\textsc{imp.pv}'.

In the first type, the verb root needs to be in the form of /tiCV(C)/ or /diCV(C)/, which become \textit{ci}CVC and \textit{ɟi}CVC on the surface. When we add the suffix <əm>, the environment that caused palatalization originally disappears, so /t/ and /d/ remain unchanged. For example, in \acl{etr}, \textit{cinun} `to weave' + <\textit{əm}> `\textsc{av}' becomes \textit{təminun} instead of **cəminun. Table \ref{tab:cjtd_alt_1} shows relevant examples in \acl{eto}, \acl{ctr}, and \acl{etr} dialects. 

\begin{table}[!htbp]
\centering
\caption{Alternations related to palatalization of dental/alveolar stops in Seediq dialects: Pattern 1}
\label{tab:cjtd_alt_1}
\begin{tabular}{llll}
\hline
Dialect   & Bare form & Infixed form & Gloss      \\ \hline
\acl{eto} & \textbf{ɟ}iɟil     & \textbf{d}əmiɟil      & `to carry' \\
\acl{ctr} & \textbf{c}iyu      & \textbf{t}əmiyu       & `to point' \\
\acl{etr} & \textbf{c}inun     & \textbf{t}əminun      & `to weave' \\ \hline
\end{tabular}
\end{table}

The second type involves verb roots in the form of /tVjV(C)/ or /dVjV(C)/, where the first V cannot be /i/ (as it would palatalize in any case). When such stems take a suffix, the first V undergoes neutralization and is assimilated to a homorganic \textit{i} due to the following \textit{y} /j/, resulting in palatalization of the preceding /t/ or /d/. However, such examples are extremely rare, with only one instance found so far, namely \textit{dayaw} `to help'. Table \ref{tab:cjtd_alt_2} illustrates examples from each dialect.

\begin{table}[!htbp]
\centering
\caption{Alternations related to palatalization of alveolar stops in Seediq dialects: Pattern 2}
\label{tab:cjtd_alt_2}
\begin{tabular}{llll}
\hline
Dialect   & Bare form & Suffixed form & Gloss     \\ \hline
\acl{eto} & \textbf{d}ayaw     & \textbf{j}iyawun      & `to help' \\
\acl{ctr} & \textbf{d}ayaw     & \textbf{j}iyagun      & `to help' \\
\acl{etr} & \textbf{d}ayaw     & \textbf{j}iyagun      & `to help' \\ \hline
\end{tabular}
\end{table}

The third type is much more common compared to the second type. Whenever a root ends in /t/ or /d/ in their underlying representations, palatalization occurs when the suffix -\textit{i} `\textsc{imp.pv}' is attached. In the bare form of such roots or in other forms without suffixation, /t/ and /d/ may change to \textit{c} or \textit{t} in the word-final position depending on the dialect (see Section \ref{sec:dtc_alt}). Relevant examples are shown in Table \ref{tab:cjtd_alt_3}.

\begin{table}[!htbp]
\centering
\caption{Alternations related to palatalization of alveolar stops in Seediq dialects: Pattern 3}
\label{tab:cjtd_alt_3}
\begin{tabular}{llll}
\hline
Dialect   & Bare form & Suffixed form & Gloss                 \\ \hline
\acl{eto} & haŋu\textbf{c} /d/    & həŋə\textbf{ɟ}i       & `to cook'             \\
\acl{ctr} & qiyu\textbf{c} /t/    & qiyu\textbf{c}i       & `to bite'             \\
\acl{etr} & həmu\textbf{t} /t/    & həmə\textbf{c}i       & `to do sth. casually' \\ \hline
\end{tabular}
\end{table}

\section{Summary}

In this chapter, I conducted a phonological sketch of five Seediq dialects, introducing their consonant and vowel inventories, syllable structures, and phonotactic constraints in Section \ref{sec:3.1}. 

Section \ref{sec:3.2} was inspired by \citeauthor{li1977morphophonemic}'s (\citeyear{li1977morphophonemic}) investigation into synchronic alternations in various Formosan languages and \citeauthor{goderich2020phd}'s (\citeyear{goderich2020phd}) study of synchronic alternations in different Atayal dialects. In this section, I compared the morphophonological alternation patterns present in each Seediq dialect. \textcite{li1977morphophonemic} proposed that Atayalic languages exhibit the richest alternations among Formosan languages, with Paran (= Tgdaya) Seediq showing the highest diversity, a conclusion consistent with the findings of this thesis.

Through the comparison of synchronic phonological differences, I believe that a significant portion can be attributed to historical changes. Discrepancies in phoneme inventory, for instance, likely stem from historical mergers or splits. The relationship between phonotactic constraints and synchronic alternations is closely intertwined, as phonotactic constraints, especially those at the word-final position, often give rise to alternations. Consistent alternation patterns across most dialects may reflect those in the proto-language, while variations among dialects are the results of historical sound changes.

Based on the above materials, I will reconstruct the phonological system of \acl{psed} in Chapter \ref{ch4}.


