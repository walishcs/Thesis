\chapter{Phonologies of Seediq dialects}\label{ch3}
This chapter provides the synchronic phonological sketches of five Seediq dialects: \acl{tg}, \acl{cto}, \acl{eto}, \acl{ctr}, and \acl{etr}. Section \ref{sec:3.1} discusses the phoneme inventories and phonotactics of all the five dialects. In section \ref{sec:3.2}, different kinds of alternations are discussed, especially those related to morphophonology.

\section{Phoneme inventories and phonotactics} \label{sec:3.1}

This section introduces the phoneme inventories and phonotactics of various Seediq dialects. While the consonant inventories of Seediq dialects are similar, there are still differences; however, the vowels, particularly in \acl{tg}, exhibit more distinctiveness.

Regarding phonotactics, there is a high level of consistency among dialects in terms of constraints on the occurrence of some sounds in certain positions. Stress placement is non-phonemic, with the majority of cases regularly falling on the penultimate syllable, while Central Toda exhibits a stress shift when the penultimate vowel is schwa /ə/ (See Section \ref{sec:cto_phonotactics}).

\subsection{Tgdaya phonology}

\subsubsection{Tgdaya consonant inventory}

\acl{tg} has the largest number of consonants among all dialects. The \acl{tg} dialect has 18 consonants, including seven stops \textit{p, t, k, q, b, d, g}, one affricate \textit{c} [ʦ], three fricatives \textit{s, x, h} [ħ], three nasals \textit{m, n, ŋ}, one lateral \textit{l}, one tap \textit{r} [ɾ], and two glides \textit{w, y} [j] (\cite{Sung2018Sedgrammar}). Table \ref{tab:tgC} shows the consonant inventory of \acl{tg}.

\begin{table}[!htbp]
\centering
\caption{Tgdaya consonant inventory\\}
\label{tab:tgC}
\begin{tabular}{l|ll|ll|ll|ll|ll|ll}
\hline
                    & \multicolumn{2}{c|}{Bilabial} & \multicolumn{2}{c|}{Dental/Alveolar} & \multicolumn{2}{c|}{Palatal} & \multicolumn{2}{c|}{Velar} & \multicolumn{2}{c|}{Uvular} & \multicolumn{2}{c}{Pharyngeal} \\ \hline
Stop                & p            & b           & t               & d               &             &               & k          & g          & q            &            &                 &              \\
Affricate           &               &              & c [ʦ]              &                  &             &               &             &             &               &            &                 &              \\
Fricative           &               &              & s               &                  &             &               & x          &             &               &            & h [ħ]             &              \\
Nasal               &               & m           &                  & n               &             &               &             & ŋ          &               &            &                 &              \\
Lateral &               &              &                  & l               &             &               &             &             &               &            &                 &              \\
Tap                 &               &              &                  & r [ɾ]              &             &               &             &             &               &            &                 &              \\
Glide               &               & w           &                  &                  &             & y [j]           &             &             &               &            &                 &              \\ \hline
\end{tabular}
\end{table}

<h> is realized as a pharyngeal [ħ] by more conservative speakers; on the contrary, other speakers pronounce it as a glottal [h] (also by some elder speakers). Further investigation is needed to determine whether this is influenced by age or other factors (such as societal ones). Additionally, careful phonetic studies are required. But phonologically, the realizations of this phoneme do not change the categorical properties, and it is clearly distinct from velar fricative /x/.

<c> /ʦ/ and <s> always palatalize to [ʨ] and [ɕ] before the high front vowel /i/, and whether it occurs in other environments depends on the speaker. Common environments include: (1) \textit{i}\_\#, and (2) \_\textit{e}. For example, \textit{madis} [madiɕ] `to bring \acs{av}', \textit{cebu} [ʨebu] `to shoot'. 

<l> is a lateral sound, but in the realization of some speakers, it may exhibit slight frication. However, due to the lack of noise, it does not reach the level of [ɮ].

<r> is usually pronounced as an alveolar (or retracted) tap [ɾ\~{}ɾ̠] or a retroflex flap [ɽ]. However, as the only rhotic sound, there may be even more variations, such as approximants [ɹ\~{}ɻ] or even (rarely) a trill [r].

Regarding the glottal stop [ʔ], \textcite{yang1976sedpho} argues that it lacks phonemic status and that the phonetic realization of [ʔ] and ∅ can be in free variation. Conversely, \textcite{holmer1996parametric} classifies the glottal stop as the phoneme /ʔ/, but without further explanation. \textcite{Sung2018Sedgrammar} does not consider it as a phoneme and also does not provide an explanation. Since the occurrence of the glottal stop [ʔ] can be predicted and there is no phonemic distinction from ∅, I adopt Yang's view and do not consider the glottal stop as a phoneme in this thesis.

\subsubsection{Tgdaya vowel inventory}

\acl{tg} has five monophthongs: \textit{a, i, e, u, o}, as shown in Table \ref{tab:tgV}. \acl{tg} also has a diphthong -\textit{uy} [uj]. 

\begin{table}[!htbp]
\centering
\caption{Tgdaya vowel inventory}
\label{tab:tgV}
\begin{tabular}{llll}
\hline
     & Front & Central & Back \\ \hline
High &  i    &         &  u   \\
Mid  &  e    &         &  o   \\
Low  &       &  a      &      \\ \hline
\end{tabular}
\end{table}

In Tgdaya, vowel length is not distinguished. If a sequence of two identical vowels is seen, the two vowels actually belong to two separate syllables, as in \textit{haan} [ħa.an] `to go \acs{lv}'.

\subsubsection{Tgdaya phonotactics}

Table \ref{tab:sy_ty_tg} displays the syllable types of \acl{tg}, listing the syllable types covered in slower and deliberate speech. In everyday and faster speech, other types may occur, which will be briefly introduced later on.

\begin{table}[!htbp]
\centering
\caption{Syllable types in Tgdaya}
\label{tab:sy_ty_tg}
\begin{tabular}{lllllll}
\hline
Syllable type & \multicolumn{2}{l}{Prepenultimate} & \multicolumn{2}{l}{Penultimate} & \multicolumn{2}{l}{Final}                \\ \hline
V             & ---              &                 & \textbf{i}.ma            & `who'         & sa.\textbf{i}        & `to go.\acs{imp} \acs{pv}' \\
VC            & ---              &                 & ---             &               & sa.\textbf{un}       & `to go \acs{pv}'           \\
VG            & ---              &                 & ---             &               & (not found) &                            \\
CV            & \textbf{qu}.si.ya         & `water'         & qu.\textbf{si}.ya        & `water'       & qu.si.\textbf{ya}    & `water'                    \\
CVC           & ---              &                 & ---             &               & ba.\textbf{raq}      & `lungs'                    \\
CVG           & ---              &                 & ---             &               & lu.\textbf{buy}      & `bag'                     \\ \hline
\end{tabular}
\end{table}

CV is the only syllable type allowed to appear in the prepenultimate position. Onsetless V will be deleted by a universal phonological rule, as seen in \textit{eyah} /ejaħ/ `to come' and \textit{yah-an} /ejaħ/ + /an/ `to come \acs{lv}'. In the penultimate position, only V and CV structures are allowed to appear.

In the final position, \acl{tg} can at least have five structures: V, VC, CV, CVC, and CVG, with fewer restrictions. While the VC structure is speculated to be possible, examples may be relatively rare and harder to find. In other dialects, there is \textit{dəmuuy} `to hold in fist \acs{av}', but irregularly corresponds to \textit{dumoi} in \acl{tg}.

In faster speech, two other syllable structures that are commonly found and are both related to nasals can occur: CVN and N̩. When the nasal consonant is followed by a homorganic consonant, syncope occurs between the nasal and the following consonant. If there is a preceding syllable before the nasal consonant, the nasal consonant becomes the coda of the preceding syllable. If there is no preceding syllable, the nasal sound becomes syllabic. These syllable structures' occurrences can be considered surface phenomena. For example, /tunusapaħ/ → [\textbf{tun}.sa.paħ] `family', and /mupitu/ → [\textbf{m̩}.pi.tu] `seven'. These variations may differ depending on individuals or subdialects.

Some sounds in \acl{tg} also have positional restrictions, some of them are categorical. For instance, voiced stops /b/, /d/, /g/ do not occur in word-final position. Labial consonants /p/ (including /b/) and /m/ are also not found in such position. 

The phoneme /x/ only appears at the word-initial position in the word \textit{\textbf{x}iluy} `iron', with around ten more examples found word-medially, but it is more common in word-final position.

In a significant portion of speakers, both liquids /l/ and <r> /ɾ/ merge with /n/ in word-final position, respectively. More conservative speakers maintain the distinctions, but generally, the merger of /l/ and /n/ in word-final position is more common than that of <r> /ɾ/ and /n/. Speakers of \acl{tg} who retain the word-final <r> often realize it as an approximant [ɹ\~{}ɻ], unlike the tap/flap in word-initial and word-medial positions. Of course, there may be more variants because, as mentioned earlier, this is the only rhotic sound in the language.

The word-final glide /w/ and <y> /j/ can only occur with /u/, as in \textit{el\textbf{uw}} `road' and \textit{bab\textbf{uy}} `pig', while in other positions, there are no restrictions.

The mid vowels /e/ and /o/ primarily serve as the nucleus of open syllables due to the historical developments (see Section \ref{sec:psedV}). However, with language development or analogical changes, there may be extremely rare exceptions, such as the occurrence of \textit{muuyas} \~{} \textit{muu\textbf{wes}} /muujas/ `to sing; to study \acs{av}' in casual speech.

In \acl{tg}, there is a rule of neutralization of prepenultimate vowels, with the neutral vowel being /u/. This means that, in general, vowel distinctions only occur in penultimate and final syllables, while the vowels in prepenultimate syllables are neutralized to [u]. However, when the onset of the penultimate syllable is <h> /ħ/ or has no onset (i.e., the onset is ∅), the neutralized vowel of the antipenultimate syllable is assimilated to be the same as the penultimate syllable, as in \textit{mehediq} [me.ħe.diq] `to give way \acl{av}' and \textit{siimah} [si.i.maħ] `to drink \acl{cv}'. However, there are exceptions, such as \textit{m\textbf{u}h\textbf{u}hediq} [m\textbf{u}.ħ\textbf{u}.ħe.diq] `to give way to each other' instead of *[m\textbf{e}.h\textbf{e}.he.diq], possibly indicating that vowel features tend to spread only once. Further synchronic phonological analysis is needed to explore this phenomenon.


\subsection{Central Toda phonology}
\lipsum[1]

\subsubsection{Central Toda consonant inventory}
\lipsum[1-3]

\subsubsection{Central Toda vowel inventory}
\lipsum[1-2]

\subsubsection{Central Toda phonotactics} \label{sec:cto_phonotactics}
\lipsum[1-5]

\subsection{Eastern Toda phonology}
\lipsum[1]

\subsubsection{Eastern Toda consonant inventory}
\lipsum[1-3]

\subsubsection{Eastern Toda vowel inventory}
\lipsum[1-3]

\subsubsection{Eastern Toda phonotactics}
\lipsum[1-3]

\subsection{Central and Eastern Truku phonologies}
\lipsum[1]

\subsubsection{Truku consonant inventory}
\lipsum[1-3]
\subsubsection{Truku vowel inventory}
\lipsum[1-3]
\subsubsection{Truku phonotactics}
\lipsum[1-3]

\section{Synchronic alternations} \label{sec:3.2}

In each Seediq dialect, there are some phonological alternations, often resulting from the interaction between position of the sounds and morphology. Interestingly, the patterns of alternations among dialects exhibit a considerable degree of consistency. Most alternating patterns can be traced back to \acl{psed}, implying that synchronic alternations can be viewed as remnants of the Proto-Seediq (See Section \ref{sec:psed_alter} for the alternations in \acl{psed}).

This section attempts to explore all alternations as comprehensively as possible. However, there may still be areas that are not fully developed, such as specific alternations in subdialects of certain villages. In this section, segments participating in alternations and currently under discussion will be bolded, while alternations that are not the immediate focus will not be highlighted prominently. 

\subsection{Vowel alternations}

\subsubsection{Prepenultimate vowel neutralizing}
\lipsum[1-3]

\subsubsection{Alternations of [ə]\~{}[u] or [e]\~{}[u]}
\lipsum[1-3]

\subsubsection{Alternations related to vowel hiatus}
\lipsum[1-3]

\subsubsection{Vowel coalescence}
\lipsum[1-3]

\subsubsection{Apharesis} %首音刪除
\lipsum[1-3]

\subsubsection{Syncope}
\lipsum[1-3]

\subsubsection{Vowel harmony}
\lipsum[1-3]

\subsubsection{Alternation between monophthongs and diphthongs}
\lipsum[1-3]

\subsection{Consonant alternations}

Consonantal alternations are highly correlated with the word-final position of the sound (with a few exceptions), often occurring when attaching suffixes due to no longer being in such position. However, sometimes a set of alternations may be related to more than one phonological rule, making classification challenging. Therefore, this section will provide separate explanations through direct examples and combine similar cases for discussion.

\subsubsection{Alternations between -\textit{d}/\textit{t}- and -\textit{c}}
\lipsum[1-3]

\subsubsection{Alternations between -\textit{b}/\textit{p}- and -\textit{k}}
\lipsum[1-3]

\subsubsection{Alternation between -\textit{m}- and -\textit{ŋ}}
\lipsum[1-3]

\subsubsection{Alternations related to -V\textit{g}- sequences}
\lipsum[1-3]

\subsubsection{Alternations between -\textit{l}/\textit{r}- and -\textit{n}}
\lipsum[1-3]

\subsubsection{Palatalization and affrication of alveolar stops before \textit{i} and \textit{y}}
\lipsum[1-3]






\section{Interim summary}
\lipsum[1-2]