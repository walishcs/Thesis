\chapter{Phonologies of Seediq dialects}\label{ch3}
This chapter provides the synchronic phonological sketches of five Seediq dialects: \acl{tg}, \acl{cto}, \acl{eto}, \acl{ctr}, and \acl{etr}. Section \ref{sec:3.1} discusses the phoneme inventories and phonotactics of all the five dialects. In section \ref{sec:3.2}, different kinds of alternations are discussed, especially those related to morphophonology.

\section{Phoneme inventories and phonotactics} \label{sec:3.1}

\lipsum[1]

\subsection{Tgdaya phonology}
\lipsum[1]

\subsubsection{Tgdaya consonant inventory}
\lipsum[1-3]

\subsubsection{Tgdaya vowel inventory}
\lipsum[1-2]

\subsubsection{Tgdaya phonotactics}
\lipsum[1-5]

\subsection{Central Toda phonology}
\lipsum[1]

\subsubsection{Central Toda consonant inventory}
\lipsum[1-3]

\subsubsection{Central Toda vowel inventory}
\lipsum[1-2]

\subsubsection{Central Toda phonotactics}
\lipsum[1-5]

\subsection{Eastern Toda phonology}
\lipsum[1]

\subsubsection{Eastern Toda consonant inventory}
\lipsum[1-3]

\subsubsection{Eastern Toda vowel inventory}
\lipsum[1-3]

\subsubsection{Eastern Toda phonotactics}
\lipsum[1-3]

\subsection{Central and Eastern Truku phonologies}
\lipsum[1]

\subsubsection{Truku consonant inventory}
\lipsum[1-3]
\subsubsection{Truku vowel inventory}
\lipsum[1-3]
\subsubsection{Truku phonotactics}
\lipsum[1-3]

\section{Synchronic alternations} \label{sec:3.2}

In each Seediq dialect, there are some phonological alternations, often resulting from the interaction between position of the sounds and morphology. Interestingly, the patterns of alternations among dialects exhibit a considerable degree of consistency. Most alternating patterns can be traced back to \acl{psed}, implying that synchronic alternations can be viewed as remnants of the Proto-Seediq (See Section \ref{sec:psed_alter} for the alternations in \acl{psed}).

This section attempts to explore all alternations as comprehensively as possible. However, there may still be areas that are not fully developed, such as specific alternations in subdialects of certain villages. In this section, segments participating in alternations and currently under discussion will be bolded, while alternations that are not the immediate focus will not be highlighted prominently. 

\subsection{Vowel alternations}

\subsubsection{Prepenultimate vowel neutralizing}
\lipsum[1-3]

\subsubsection{Alternations of [ə]\~{}[u] or [e]\~{}[u]}
\lipsum[1-3]

\subsubsection{Alternations related to vowel hiatus}
\lipsum[1-3]

\subsubsection{Vowel coalescence}
\lipsum[1-3]

\subsubsection{Apharesis} %首音刪除
\lipsum[1-3]

\subsubsection{Syncope}
\lipsum[1-3]

\subsubsection{Vowel harmony}
\lipsum[1-3]

\subsubsection{Alternation between monophthongs and diphthongs}
\lipsum[1-3]

\subsection{Consonant alternations}

Consonantal alternations are highly correlated with the word-final position of the sound (with a few exceptions), often occurring when attaching suffixes due to no longer being in such position. However, sometimes a set of alternations may be related to more than one phonological rule, making classification challenging. Therefore, this section will provide separate explanations through direct examples and combine similar cases for discussion.

\subsubsection{Alternations between -\textit{d}/\textit{t}- and -\textit{c}}
\lipsum[1-3]

\subsubsection{Alternations between -\textit{b}/\textit{p}- and -\textit{k}}
\lipsum[1-3]

\subsubsection{Alternation between -\textit{m}- and -\textit{ŋ}}
\lipsum[1-3]

\subsubsection{Alternations related to -V\textit{g}- sequences}
\lipsum[1-3]

\subsubsection{Alternations between -\textit{l}/\textit{r}- and -\textit{n}}
\lipsum[1-3]

\subsubsection{Palatalization and affrication of alveolar stops before \textit{i} and \textit{y}}
\lipsum[1-3]






\section{Interim summary}
\lipsum[1-2]