\chapter{Literature review}\label{ch2}

This chapter discusses previous studies related to the historical development of Seediq. Section \ref{sec:reconstruction_lit} provides critiques of previous reconstructions related to \acl{psed} and \acl{paic}, such as: \textcite{li1981paic} and \textcite{ochiai2016buhwan}. Section \ref{sec:subgrouping_lit} concentrates on reviewing relevant literature on dialectal differences and internal relationships of Seediq, like: \textcites{ogawaandasai1935}{Chengetal2017Truku}{Chengetal2019Tgdaya}{Sung2018Sedgrammar}. Section \ref{sec:anth_lit} discusses anthropological amd sociology research related to Seediq, seeking additional historical materials, such as migration patterns (\cite{liao1977Sedtheruy,liao1978Sedtheruy,yap2023}). Finally, Section \ref{sec:ch2_sum} offers a summary of this chapter, including insights for the thesis.

\section{Reconstruction} \label{sec:reconstruction_lit}

In the past, there has been almost no focus on the reconstruction of \acl{psed}. Relevant literature mainly includes \textcite{li1981paic}, who reconstructed \acl{paic}, and \textcite{ochiai2016buhwan}, who reconstructed some \acl{psed} lexical items while interpreting the \textcite{bullock1874formosan} data related to the Bu-hwan dialect.

\subsection{Li (1981)}

\textcite{li1981paic} compared Atayal and Seediq dialects to reconstruct the phonology of \acl{paic} as well as approximately 300 lexical items. The phoneme inventory of Li's \acl{paic} is shown in Table \ref{tab:lipaic}

\begingroup
\setstretch{1}
\renewcommand\arraystretch{1.5}
\begin{table}[H]
\centering
\caption{Li's (1981) \acl{paic} phoneme inventory (adapted from \cite[272]{li1981paic})}
\label{tab:lipaic}
\begin{tabular}{lllllllllllll}
\hline
p & t &    & k & q &   & ʔ &  &  &          &          &          &  \\
b & d & g' & g &   &   &   &  &  & i        &          & u        &  \\
  &   & c  &   &   &   &   &  &  &          & ə        &          &  \\
  &   & s  & x &   & h &   &  &  &          & a        &          &  \\
m & n &    & ŋ &   &   &   &  &  &          &          &          &  \\
  & l & r  &   &   &   &   &  &  & \multicolumn{3}{l}{aw, ay, uy} &  \\
w &   & y  &   &   &   &   &  &  &          &          &          &  \\ \hline
\end{tabular}
\end{table}
\endgroup

The paper does not delve much into Seediq itself, but Li does mention a couple of points where Seediq is considered more conservative than Atayal: (1) non-final /b/, /d/, /g/ in Seediq remain voiced stops, whereas in most Atayal dialects, they lenited into fricatives /β/ (< *b), /ɣ/ (< *g), and rhotic /r/ (< *d), (2) Seediq /r/ is more conservative (< *r), as most Atayal dialects (except for Palŋawan (Plngawan)) typically change it into other sounds (see \cite[264--265]{li1981paic} for details).

Another point worth noting is the distinction between \acl{paic} word-final *-aw and *-ag, which contrast in some Atayal dialects. However, in all four Seediq dialects, their correspondences are as follows: Toŋan -\textit{o} : Toda -\textit{aw} : Truwan -\textit{aw} : Inago -\textit{ag}.

Li's argument is that Inago retained \ac{paic} *-ag as -\textit{ag} , while the sound change of \ac{paic} *-aw to -\textit{ag} is an innovation in this dialect (actually a merger) (\cite[257--258,271]{li1981paic}). However, this can not necessarily guarantee that the contrast between *-ag and *-aw persisted at the \acl{psed} stage, since all dialects show a merger of the two.

Li's reconstruction marked the beginning of comparative research on Atayalic languages and dialects, providing essential reference material. However, as mentioned earlier and according to the arguments presented by \textcite{song2022Aicd,song2023Aicgprime}, the reconstruction of the lower level Proto-languages, remains crucial. Just as \citeauthor{goderich2020phd}'s (\citeyear{goderich2020phd}) reconstruction of \acl{pata} has addressed some issues, the reconstruction of \acl{psed} is the primary goal of this thesis.

\subsection{Ochiai (2016)}

The main target of Ochiai's (\citeyear{ochiai2016buhwan}) paper is to analyze the Bu-hwan vocabulary recorded by \textcite{bullock1874formosan} in comparison with Paran (equivalent to Tgdaya) Seediq and Taroko (equivalent to Eastern Truku) Seediq to elucidate the historical and linguistic relationships with other Seediq dialects.

Even though the main focus of this paper was not the reconstruction of \acl{psed}, she still reconstruct nearly 200 lexical items based on the lexical items form the three dialects.

However, the reconstructions in her paper were not done following the basic Comparative Method (see Section \ref{sec:methodology}). No sound correspondences were proposed; instead, the reconstructions were based on individual lexical items one by one. Ochiai also noted that these reconstructed forms are still tentative and some require further revision. Therefore, the appropriateness of the forms reconstructed in her paper is not discussed further here.

An interesting point is her differentiation between *ð and *y in the reconstruction of \acl{psed}. I have organized relevant correspondences based on her reconstructions as shown in Table \ref{tab:ochipsedzy}.

\begin{table}[!htbp]
\centering
\caption{Ochiai's \acl{psed} *ð and *y}
\label{tab:ochipsedzy}
\begin{tabular}{llll}
\hline
Ochiai's \acl{psed} & Bu-hwan    & Paran (\ac{tg}) & Taroko (\ac{etr}) \\ \hline
*ð                  & z, dz, th  & y               & y                 \\ \hdashline
\multirow{2}{*}{*y} & i, ∅ (V\_V) & y               & y                 \\
                    & N/A        & y               & y                 \\ \hline
\end{tabular}
\end{table}

When a set of cognates has Bu-hwan \textit{z}, \textit{dz}, or \textit{th} corresponding to \textit{y} in both modern dialects, Ochiai reconstructs \acl{psed} *ð. If Bu-hwan \textit{i} or ∅ (V\_V) corresponds to \textit{y} in modern dialects, it is reconstructed as *y. Additionally, when there is no cognate found in Bu-hwan, it is also reconstructed as *y. This correspondence may seem plausible, but it can also lead to redundancy. Therefore, we can examine whether there is a necessity for reconstructing two protophonemes through external evidence. 

I examined the words containing \textit{z}, \textit{dz}, \textit{th} mentioned by Ochiai in \textcite{bullock1874formosan} and those containing \textit{ð} recorded by \textcite{asai1953sedik} (his data was collected in 1927), comparing them with \acl{pata} from \textcite{goderich2020phd} and modern Atayal dialects. I found that out of 28 examples, only 5 were recorded as \textit{y} (Bullock) or \textit{j} (Asai), while the rest were \textit{z/dz/th} (Bullock) or \textit{ð} (Asai). Additionally, in the 21 cases where cognates in Atayal were found, Bu-hwan (1874)/Paran (1927) \textit{z/dz/th} or \textit{ð} corresponded to \acl{pata} *y, *w (u\_a), *g, *ʔ, *s. Examples are shown in Table \ref{tab:bhandprz} (the ``palatal hooks'' in Asai's data are transcribed as <ʲ>).

\begingroup
\setstretch{1}
\renewcommand\arraystretch{1.25}
\begin{longtable}[c]{>{\raggedright}p{2.7cm}>{\raggedright}p{2.7cm}>{\raggedright}p{2.7cm}ll}
\caption{Correspondences among Bu-hwan, Paran, \acl{pata}, and \acl{pan}}
\label{tab:bhandprz}\\
\hline
Bu-hwan                  & Paran                            & \ac{pata}                                     & \ac{pan} & Gloss             \\  \hline
kasia                    & kasʲija                          & *qusiyaʔ                                      &          & `water'           \\
                         & bsijaq                           & *bVsiyaq                                      &          & `long time'       \\
                         & qojoqaja                         & *qayqayaʔ                                     &          & `tool'            \\
daia `east'              & dajja                            & *rayaʔ                                        & *daya    & `upslope; uphill' \\ 
măhoyesh                 & mɯwes                            & *maquwas                                      & *quyaS   & `to sing'         \\ \hdashline
                         & nɯbɯðas                          & *nabuwas                                      &          & `belly'           \\
maidzah                  & meðaħ                            & *muwah                                        &          & `to come'         \\
dagizak paru             & degiðaq                          & *ragiyax                                      &          & `mountain'        \\
sinuzuk                  &                                  & *sinuyug                                      &          & `cord, line'      \\
hazi kăhoni `fruit'      & heði `fruit'                     & *hiʔiʔ                                        &          & `flesh, meat'     \\
ŭduthiŋ `throat'        &                                  & *wariyuŋ                                      &          & `neck; nape'      \\
                         & raði `a match-maker; a relative' & sa-ɹeʔ `to propose marriage' (Plngawan)       &          &                   \\
                         & teheðaq `to visit'               & *tuwahiyaq `far'                              &          & `far'             \\
                         & hiða                             & *hiyaʔ                                        & *si ia   & `3\acs{sg}'       \\ \hdashline
                         & piða                             & *pisaʔ                                        & *pija(x) & `how many'        \\
swadzu makaidil `sister' & sɯwaði                           & *suwaʔiʔ; *suwagiʔ `younger sibling's spouse' & *Suaji   & `younger sibling' \\
                         & ŋaðan                            &                                               & *ŋajan   & `name'            \\
                         & kiða `that; later'               & *kisaʔ `today; soon'                          &          & `soon; later'     \\
                         & dahiða                           & ləhəgaʔ (Squliq)                              &          & `3\acs{pl}'       \\
kuzu                     &                                  & *məquʔ (<**məquʔuʔ ?)                         &          & `snake'           \\
                         & gaða `customary laws'            & *gagaʔ `culture; tradition; law; religion'    &          & `law'             \\ \hdashline
dzadzuŋ                  & ðaðɯŋ                            & (*gaʔuŋ)                                      &          & `river'           \\
kuzuch                   & kɯðɯħ                            & (*quwalax)                                    & *quzaN   & `rain'            \\ \hline
\end{longtable}  
\endgroup

The correspondences in \acl{pata} may seem messy, but they can actually be divided into two groups. *y and *w (u\_a) can be integrated because the Atayal *-uwa- sequence appears to have originated from an earlier stage **-uya-, for example: \ac{pan} *quyaS > \ac{pata} *quwas `song'. On the other hand, *g/*ʔ/*s all derive from Pre-\acl{pata} **g and represent the merger of \acl{paic} *g/*j (see \cite{song2023Aicgprime} for details). Related to the examples here is the reflexes of \acl{paic} *j.

In other words, Bu-hwan \textit{d/dz/th} and Paran \textit{ð} in 1927 correspond to the reflexes of \acl{paic} *y and *j in \acl{pata}. Furthermore, \textcite{song2023Aicgprime} proposed that \acl{paic} *y and *j merged in \acl{psed} as *y, consistently reflected as \textit{y} in all modern Seediq dialects with only few exceptions.

As mentioned earlier, in the data recorded by \textcite{asai1953sedik} for Paran, the majority is \textit{ð}, which correspond to Pre-\acl{pata} **y (< \ac{paic} *y) and **g (< \ac{paic} *j/g). Based on this, it can be concluded that Paran also underwent the \acl{paic} *y/*j merger, which is the same as other modern Seediq dialects. 

In summary, it may not be necessary to reconstruct *y and *ð as two protophonemes at the \acl{psed} level. I propose using *y [j] as the \acl{psed} protophoneme because [j] > [ð] is a more common sound change compared to [ð] > [j]. For the reflexes of \acl{psed} *y in modern dialects, please see Section \ref{sec:psedC}. Regarding the five examples in Bu-hwan/Paran showing \textit{y/j} [j], since they are relatively rare, they could potentially be borrowed from neighboring dialects (at the time, the Tgdaya group had a maximum of 12 communities, as mentioned in Section \ref{sec:tgintro}). However, this issue may require further investigation and research.

\section{Internal relationships of Seediq dialects} \label{sec:subgrouping_lit}

In previous studies, only few of them discussed the subgrouping of the Seediq language. Most of the works take Atayalic as a whole, without a deeper look into Seediq. For example. \textcite{li1981paic} mentions, ``The Sediq dialects, Toŋan, Toda, Truwan and Inago, which are not completely uniform, but also not very divergent[.]'' (p. 235), but also describes: ``As there are significant dialectal differences in Sediq, I have included four dialects in this study'' (p. 236).

In this section, I will discuss the primary literature that has addressed subgrouping, or simply dialectal differences in the past, including \textcite{ogawaandasai1935}, \textcite{Chengetal2017Truku, Chengetal2019Tgdaya, Sung2018Sedgrammar}.

\subsection{Ogawa and Asai (1935)}

\textcite{ogawaandasai1935} is a book introduces the indigenous languages (referred to as ``蕃語 (`savage's language')'' at that time) during the Japanese colonial period in Taiwan, which includes 12 Austronesian languages that were spoken in Japanese colony. Its content includes overviews, grammatical sketches of each language, as well as myths and stories.

\textcite{ogawaandasai1935} point out that Seediq is the language of the Seediq ethnic group that has been living in \textit{Taichūshū} (臺中州) and \textit{Karenkōchō} (花蓮港廳) for a long time.

They argue that Seediq can be roughly divided into two dialects: 霧社方言 (Musha dialect, belongs to 霧社蕃 (Musha)) and タロコ方言 (Taroko dialect, including トロコ蕃 (\textit{Toroko}, (Central) Truku), タウツア蕃 (\textit{Tautsa}, Tawca), タウサイ蕃 (\textit{Tausai}, Tawsay), タロコ蕃 (\textit{Taroko}, (Eastern Truku)), バトラン蕃 (\textit{Batoran}, Btulan)). Table \ref{tab:ona1935} shows the phonological differences between two dialects (including their subdialects).

\begin{table}[!htbp]
\centering
\caption{Phonological relationship between Musha and Taroko dialects in \textcite{ogawaandasai1935}}
\label{tab:ona1935}
\begin{tabular}{cc|c}
\hline
\multicolumn{2}{c|}{Musha}                                                               & Taroko       \\
Paran                                                       & Gungu & Ibuh \\ \hline
ð                                                                     & y [j]              & y [j]           \\
ts (non-final) & ts (non-final)    & s (non-final)  \\
g?                                                                   & q/w     & ǥ [ɣ]       \\
g                                                                     & g              & g           \\ \hline
\end{tabular}
\end{table}

Based on their description, the subgrouping tree can be drawn as is Figure \ref{fig:onatree} (the spelling of each group/village is transcribed based on modern orthography). However, Ogawa and Asai did not explain the criteria they used for classification; they only provided simple descriptions of the differences among dialects.

\begin{figure}[H]
    \centering
    \begin{forest}
           for tree={l sep=4em, s sep=2em, inner sep=0, anchor=north, parent anchor=south, child anchor=north}
            [Seediq
                [Musha
                    [Paran, tier=word]
                    [Gungu, tier=word]
                ]
                [Taroko
                    [C.Truku, tier=word]
                    [Tawca, tier=word]
                    [Tawsay, tier=word]
                    [E.Truku, tier=word]
                    [Btulan, tier=word]
                ]
            ]
            \end{forest}
    \caption{Seediq subgrouping tree based on Ogawa and Asai's (1935) description}
    \label{fig:onatree}
\end{figure}

\subsection{Cheng, Nanu Ruwiq, et al. (2017) and Cheng, Dakis Pawan, et al. (2019)}

\textcite{Chengetal2017Truku} and \textcite{Chengetal2019Tgdaya} respectively refer to the vocabulary handbooks of (Eastern) Truku and Tgdaya. As both of them share exactly the same subgrouping hypothesis, they are discussed together in this context.

\textcite{Chengetal2017Truku} and \textcite{Chengetal2019Tgdaya} mention that Seediq has three ethnic subgroups: Tgdaya, Toda, and Truku. However, they indicate that Seediq can be roughly divided into two linguistic dialects: Tgdaya (德固達雅方言) and Truku-Toda (太魯閣/都達方言). Truku-Toda can be further divided into 太魯閣土語 (Eastern Truku), 德路固土語 (Central Truku), 都達土語 (Toda), and 見晴土語 (Miharasi). See Figure \ref{fig:chengtree} for their subgrouping tree. 

\begin{figure}[H]
    \centering
    \begin{forest}
           for tree={l sep=4em, s sep=3em, inner sep=0, anchor=north, parent anchor=south, child anchor=north}
            [Seediq
                [Tgdaya, tier=word] 
                [Truku-Toda
                    [Eastern Truku, tier=word]
                    [Central Truku, tier=word]
                    [Toda, tier=word]
                    [Miharasi, tier=word]
                ]
            ]
     \end{forest}
    \caption{Seediq subgrouping tree based on description of \textcite{Chengetal2017Truku} and \textcite{Chengetal2019Tgdaya}}
    \label{fig:chengtree}
\end{figure}

Eastern Truku in Hualien and Central Truku in Nantou are slightly different due to the geographic isolation, but they can still communicate with each other without problems. Toda is different from Truku, but not a lot; they has mutual mutual intelligibility. There are also differences between Miharasi and \acl{etr}. (\cites[4]{Chengetal2017Truku}[5]{Chengetal2019Tgdaya})

In \textcite{Chengetal2017Truku} and \textcite{Chengetal2019Tgdaya}, there is no evidence presented for subgrouping of Seediq dialects. Personally, I speculate that they might have used mutual intelligibility as a criterion for the subgrouping (or the ``division of dialects''). However, the main issue with using this criterion for subgrouping is that the causes of mutual intelligibility could be due to shared retentions, convergence resulting from language contact, and so forth, which are not the best indicators for subgrouping. As discussed in Section \ref{sec:methodology}, shared innovations are the preferred criterion for subgrouping, not shared retentions. Therefore, the model provided by them needs to be revised.

\subsection{Sung (2018)}

\textcite{Sung2018Sedgrammar} is a reference grammar book for Tgdaya Seediq, providing clear descriptions of Seediq grammar. However, only certain paragraphs mention phonological differences among dialects, summarized in Table \ref{tab:sung2018}.

\begin{table}[!htbp]
\centering
\caption{Seediq dialectal differences (based on \cite[21--22]{Sung2018Sedgrammar})}
\label{tab:sung2018}
\begin{tabular}{llll}
\hline
Tgdaya & Toda   & Truku                       & Note        \\ \hline
e      & ə      & ə                           & penultimate \\
-e     & -ay    & -ay                         & final       \\
-o     & -aw    & -aw                         & final       \\
t, d   & t, d   & c, ɟ                        & /\_i        \\
g-/-g- & w-/-w- & g-/-g-                      & non-final   \\
-w     & -g     & -g                          & final       \\
c      & c      & s                           & non-final   \\
c      & c      & c (\ac{ctr}) / t (\ac{etr}) & final       \\ \hline
\end{tabular}
\end{table}

\textcite[22]{Sung2018Sedgrammar} mention that the occurrence of -\textit{g} at the word-final position in Toda. This is likely a typo, as there is no /g/ in the native phoneme inventory of Toda (it only appears in loanwords). 

Other than the aforementioned point, this section in the book provides the main differences among Seediq dialects. However, without reconstructing the Proto-Seediq phonology, we cannot determine whether these dialects can form any subgroup, and this is not the focus of \citeauthor{Sung2018Sedgrammar}'s book, so I only examine her description of dialectal differences.

\section{Relevant anthropological and sociology studies} \label{sec:anth_lit}

Understanding the history of ethnic groups is also crucial for understanding the changes and development of languages. This section will examine past research in anthropology, sociology, and related fields on oral history and migration (whether voluntary or forced), to explore the relationships between groups or the background of language contact. 

The primary references include \textcites{utsurikawaetal1935}{liao1977Sedtheruy}{liao1978Sedtheruy}{yap2023}. Most of the history and migration of Seediq(-Truku) groups have been extensively covered in Chapter \ref{ch1}. Therefore, this section will focus on reflecting on the aspects where oral history and migration history may have an impact on language.

\subsection{Utsurikawa et al. (1935)}

\textcite{utsurikawaetal1935} is a book that documents the history of various indigenous nations through field works and interviews to study their respective systems. At that time, Seediq was classified as a branch of Atayal, but the authors also conducted interviews with each Seediq group. However, due to the outbreak of the Musha Incident in 1930, information about the Seediq is quite limited.

According to the historical accounts transmitted by the Paran community, Paran, Suku, Gungu, Mhebu, Tongan, Sipo, Tkanan, and Qacuq successively migrated from the oldest Tgdaya Truwan community. Additionally, Drodux branched off from Gungu. Furthermore, two other communities belonging to the Tgdaya group, Bwarung and Bkasan, originally part of the Toda group, joined the Tgdaya group around the 1900s (\cite[146--157]{TengChian2023musha}), but there are currently no records found regarding the languages of these two communities.

The Truku group migrated upstream along the Jhuoshuei River (濁水溪, Seediq: \textit{Yayung Mtudu}) from low-altitude areas, eventually settling in the current location of Truku Truwan. Later, the Sadu community branched off from Truku, and from Sadu, two more communities, Busig Ska and Busig Daya, emerged. Subsequently, Brayaw community split off from Busig Daya. When Sadu separated from Truku Truwan initially, part of the population migrated eastward, which is the origin of today's Truku Nation (太魯閣族).

The Toda (Tawda) group has ancestral legends suggesting an initial division between the Bkasan and Wahar communities. Eventually, most of the people from Wahar were persuaded to settle in Bkasan, while another group migrated eastward, becoming the Tawsay. Another group from Wahar migrated all the way to Tnbarah, where they later split into five communities: Cka, Bngbung, Rucaw, Ruku Daya, and the original Tnbarah. Regarding the eastward migration of the Tawsay group, some accounts suggest they initially traveled north through the territory of Sqawraw (Atayal: Sqoyaw) before continuing their journey eastward.


\subsection{Liao (1977, 1978)}

\textcite{liao1977Sedtheruy,liao1978Sedtheruy}  delve into the ancestry and origins of the migrating Seediq people. Overall, he provide much more detailed information compared to \textcite{utsurikawaetal1935}.

\textcite{liao1977Sedtheruy} also mentioned that the migrating Tawsay group (\acl{eto}) passed through the territories or hunting grounds of Slamaw and Sqoyaw during their migration process before continuing eastward to the present-day Hualien area in the north. Subsequently, they were invaded by the eastward-migrating Truku people, leading them to reside with the Klesan Atayal Group in the north. According to \textcite[65]{liao1977Sedtheruy}, elderly individuals aged 50-60 (estimated to be born around the 1920s) could still speak their language, but younger generations had lost the ability. Around 1930, the Japanese authorities forcibly relocated the remaining Tawsay group members to live with the Truku people. The last group of Tawsay people migrated southward to the present-day Tawsa community location, which is the last surviving community mainly composed of the Tawsay people (and their language, see \cite{lee2015tawsa}).

The eastward-migrating Tgdaya group (referred to as Pribaw) settled close to the territory of the Amis villages, serving as intermediaries for trade between the eastern and western regions. Specifically, the Eastern Tgdaya people purchased goods produced by the Amis and brought them back to Puli (埔里) for trading what Amis people want to buy, until the Eastern Truku grew stronger and occupied their territory (\cite[70]{liao1977Sedtheruy}).

Some Truku groups migrated eastward through the saddle between the eastern peak of Mount Hehuan (合歡山東峰) and the northern peak of Mount Chilai (奇萊主山北峰), establishing the first Tpuqu community. They then rapidly expanded, creating dozens of subsidiary communities, forming the strongest branch of Seediq in the east and compelling the Tawsay and Pribaw to migrate elsewhere. Their migration path was not far from the territory of the Atayal tribe to the west, suggesting possible historical interactions.

\subsection{Yap (2023)}

\textcite{yap2023}, leveraging his expertise in sociology and geography, explores the fundamental changes in the ``group resettlement policy (集團移住政策 \textit{shūdanijūseisaku})'' within Japan's so-called ``savage-managing policy (理蕃政策 \textit{ribanseisaku})'' before and after the 1930 Musha Incident. What sets Yap's work apart from previous ones is that he does not simply investigate migration in a linear manner. Instead, he delves into the relational study of network changes between communities, examining the impact of group resettlement policies on such network relationships.

\textcite[137,153]{yap2023} mentions that the Tausa (Tawsay) group and the Truku (Eastern) group were enemies and were in a disadvantaged position. Their connection with the central Toda group was also severed by the Truku people. As a result, they turned to establish friendly relations with the Atayal (Klesan group) in Nan'ao, Yilan, forming a collective group in terms of geopolitics. However, later on, the three Tausa communities were relocated southward to the present-day Tawsa community (see Section \ref{sec:etointro}), interrupting their close contact with the Atayal. Nevertheless, we can still discern from this that the relationship between the Eastern Toda (Tawsay, Tausa) and the Atayal was once quite close.

\section{Summary} \label{sec:ch2_sum}

Based on the discussion in this chapter, we have conducted a preliminary review of the historical development of Seediq. In terms of language, we have examined \citeauthor{li1981paic}'s (\citeyear{li1981paic}) \acl{paic} and \citeauthor{ochiai2016buhwan}'s (\citeyear{ochiai2016buhwan}) \acl{psed} reconstruction, discussing research gaps and parts that warrant revision. 

Regarding dialect relations, a basic understanding has been established through a review of literature from the past century. At the same time, it has been observed that there is currently no subgrouping hypothesis proposed based on solid linguistic evidence.

Literature from other relevant disciplines provides insights into internal community relations and potential external contacts of the Seediq, and this thesis will also reference relevant historical backgrounds to explore the internal and external relationships of the Seediq language.
