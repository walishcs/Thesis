\chapter{Conclusion}\label{ch8}
\section{Summary}
% \lipsum[1-10]

% 這篇 thesis 提供 \acl{psed} 的 phonology, morphosyntax, lexical 基於 employing the Comparative Method and a top-down reconstruction framework 的重建。此外,亦根據上述重建以及方言間的 shared innovation,提出了 Seediq modern dialects 的 subgrouping hypothesis。此外,附錄收錄了約800詞左右的 Proto-Seediq etyma,以及方言之間的 reflexes。

This thesis provides a reconstruction of \acl{psed} phonology, morphosyntax, and lexical items by employing the Comparative Method and a top-down reconstruction framework and based on the data of the modern dialects including Tgdaya, Central Toda, Eastern Toda, Central Truku and Eastern Truku. Additionally, it proposes a subgrouping hypothesis for modern Seediq dialects based on the aforementioned reconstructions and shared innovations among the dialects. Moreover, the appendix includes about 800 Proto-Seediq etyma along with their reflexes in Seediq dialects.

% Chapter \ref{ch3} 描述 Seediq dialects 的音韻。分別就每個方言的 phoneme inventories, phonotactics 進行討論。此外,也討論 Seediq dialects 中的 morphophonological alternations。這些 alternations 有些是在現在方言中單獨產生的,然而,有很大一部分是歷史的遺留。

% 基於上述描述及證據,以及 Seediq dialects 的同源詞,我在 Chapter \ref{ch4} 重建 Proto-Seediq 的 phonological system。根據我的重建,Proto-Seediq 有 18 個輔音 *p, *t, *k, *q, *b, *d, *c, *s, *x, *h, *g, *m, *n, *ŋ, *l, *r, *w, *y,4 個單元音 *a, *i, *u, *ə,還有 3 個雙元音 *ay, *aw, *uy。其中有一些 inconsistent sound correspondences,我分別以現階段最有可能解方處理那些 correspondence sets 的重建問題。我也依照現代方言的證據,重建了 Proto-Seediq 的 phonotactics,其中較為顯著的特色是 Proto-Seediq 在詞尾輔音如 *p, *b, *d, *m 等或末音節 *ə 的出現有限制,詞首的 *x 也相當 marked。因此,許多 Proto-Seediq 的 morphophonological alternations 因為前述限制而出現,而我也將其重建至 Proto-Seediq 的 phonological system。

% Chapter \ref{ch5} 則進行 Proto-Seediq morphosyntactic system 的初探。我根據現代方言的語料重建了 nominal 及 verbal morphology,以及對 syntax 部分也有粗淺的描述。(Proto-)Seediq 保留了相當完整的 voice ``focus'' system,只是在 Patient Voice 及 Locative Voice 的分別逐漸消失。格位標記(或名詞組標記)部分僅有 nominative marker *ka 及 genitive marker *na 兩個,不區分名詞類型。

% Chapter \ref{ch6} 針對 Proto-Seediq 的 lexical reconstruction 中的一些議題進行討論。其中部分 Proto-Seediq 的 etyma,因為在現代方言間的分布不一,我採用外部證據作為重建的依據。而我也發現 Proto-Seediq 或 Seediq dialects 有時含有特殊的 ``doublets'',而這個現象與 Atayal 的 Gender Register System 呈現的特徵高度相關。只是,目前的證據顯示至少有一部份的 doublets 是借自 (Proto-)Atayal,這個系統在 Seediq 中的歷時發展還有待更多研究。

% 最後,Chapter \ref{ch7} 則依照 Chapter \ref{ch4}--\ref{ch6} 的證據,提出 Seediq subgrouping hypothesis。這個 hypothesis 是基於方言之間的 shared innovation,與過去文獻中所提到的方言區分方法不同。在我的假說中,Tgdaya 是距離其他方言最遠的方言,它與其他方言並未共享 any significant change。Toda-Truku subgroup 則包含 \acl{cto}, \acl{eto}, \acl{ctr}, and \acl{etr}。這些方言在音韻上共享 historical *ə 的變化、幾個較罕見的 sporadic sound changes,而 morphosyntax 上也同時經歷了 genitive marker *na 的丟失,最後,他們也共享一些少數的詞彙創新,包含詞形或語意層面的皆有。

Chapter \ref{ch3} describes the phonology of the Seediq dialects, discussing the phoneme inventories and phonotactics of each dialect individually. Additionally, it explores the morphophonological alternations present in the Seediq dialects, some of which have arisen independently in the modern dialects, while many are historical remnants.

Based on the phonological descriptions and evidence provided, as well as cognates among the Seediq dialects, I reconstruct the phonological system of Proto-Seediq in Chapter \ref{ch4}. According to my reconstruction, Proto-Seediq had 18 consonants: *p, *t, *k, *q, *b, *d, *c, *s, *x, *h, *g, *m, *n, *ŋ, *l, *r, *w, *y; 4 monophthongs: *a, *i, *u, *ə; and 3 diphthongs: *ay, *aw, *uy. There are several inconsistent sound correspondences that I address with the most plausible solutions based on current evidence. Additionally, I reconstruct the phonotactics of Proto-Seediq, which notably includes restrictions on the occurrence of final consonants such as *p, *b, *d, *c, *m, and the final syllable *ə, as well as the marked nature of initial *x. Many morphophonological alternations in Proto-Seediq arise due to these constraints, and I incorporate them into the reconstructed Proto-Seediq phonological system.

Chapter \ref{ch5} undertakes a preliminary exploration of the Proto-Seediq morphosyntactic system. Based on modern dialectal data, I reconstruct the nominal and verbal morphology, and provide a rudimentary description of the syntax. Proto-Seediq retains a fairly complete voice ``focus'' system, although the distinctions between Patient Voice and Locative Voice are gradually disappearing. The case marking (or noun phrase marking) system is minimal, consisting only of a nominative marker *ka and a genitive marker *na, without differentiating between types of nouns.

Chapter \ref{ch6} discusses some issues in the lexical reconstruction of Proto-Seediq. Some Proto-Seediq etyma, due to their varied distribution among modern dialects, are reconstructed based on external evidence from \acl{pata} and/or evidence from \acl{pan}. I also find that Proto-Seediq or the Seediq dialects sometimes contain special ``doublets'', highly related to the features presented by the Atayal Gender Register System. However, current evidence suggests that at least some of these doublets are borrowed from (Proto-)Atayal, and the diachronic development of this system within Seediq requires further study.

Finally, Chapter \ref{ch7} proposes a Seediq subgrouping hypothesis based on the evidence from Chapters 4 to 6. This hypothesis is based on shared innovations among the dialects, differing from the dialect, which differs from those previously mentioned in the literature. In my hypothesis, Tgdaya is the most distant dialect from the others, sharing no significant changes with them. The Toda-Truku subgroup includes \acl{cto}, \acl{eto}, \acl{ctr}, and \acl{etr}, which share phonological changes associated with historical *ə, several sporadic sound changes, and also experience the loss of the genitive marker *na in morphosyntax. Furthermore, they share a few lexical innovations, involving both forms and meanings.

\section{Remaining issues and future research directions}

In this thesis, I have not mentioned extensively the historical development from \acl{pan} to \acl{psed}. Based on my preliminary examination, several \ac{pan} phonemes have more than one set of reflexes in (Proto-)Seediq. Some examples are listed below, such as \ac{pan} *C > \ac{psed} *c/*s/*l, \ac{pan} *k > \ac{psed} *k/*q, \ac{pan} *q > \ac{psed} *q/∅, \ac{pan} *d > \ac{psed} *d/*c, \ac{pan} *N > \ac{psed} *l/*d, \ac{pan} *S > \ac{psed} *s/*h/*l, and \ac{pan} *R > \ac{psed} *g/*r, among others. The easiest to explain is the *q > *q/∅ set, as it relates to the historical layers of Seediq (\cite{song2024sedq}). However, the real reasons behind other splits remain elusive; some seem related to phonetic or perceptual factors, and some are also seen in Atayal (also with unknown reasons). This topic can be further explored in the future, and clarifying the evolution of each phoneme can help us understand the developmental history of Seediq.

The relationship between Seediq and Atayal is also worth investigating in depth. For a long time, Seediq and Atayal, as the only members of the Atayalic subgroup, have hardly been questioned, even though the initial evidence for their classification was lexical (\cite{ferrell1969}) and described as a ``self-evident'' subgroup (\cite{blust1999subgrouping}). However, based on preliminary investigations in this thesis and my other works, Seediq appears to have borrowed a significant number of loanwords and even a phoneme (*q /q/) from Atayal, with varying degrees of borrowing across different dialects (or subgroups). Given this level of borrowing, the relationship between Seediq and Atayal needs further clarification. I also plan to compare more data from (Proto-)Seediq and (Proto-)Atayal to first ascertain Atayal's influence on Seediq (or possibly vice versa) and then examine whether they should be classified under the same subgroup.

Finally, regarding Seediq itself, many dialects may have undergone language shift (to other Seediq dialects or languages) or lost their distinctive features due to forced relocations by colonial regimes. For instance, the Tgdaya dialect in Hualien (``Pribaw'') shifted to using the Eastern Truku dialect, the Paran community dialect of Tgdaya (the only Seediq dialect with a \textit{ð} sound) which is relocated to the Nakahara community (中原部落), and the Tawsay dialect in Yilan, which shifted to using Atayal or Yilan Creole. Some of this linguistic data can be found in old records (\cite{tashiro1986easterntw, bullock1874formosan}). Some have not been fully published, such as the field notes of Japanese scholar Tsuchida Shigeru (土田 滋), which should contain data on some of the aforementioned dialects (\cite{li2022tsuchida}). If these materials can be interpreted or accessed, they may enhance the reconstruction of \acl{psed} or the subgrouping of its dialects. Additionally, given the broad distribution of Eastern Truku, did the people undergo further dispersal or differentiation during their eastward migration to Hualien? These questions are promising directions for future research.