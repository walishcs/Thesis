\vspace{-1.15cm}

Patis qnquran nii we, pnseengan mu rima dndulan kari (Tgdaya, Toda, Tuda, Truku Nanto, ma Truku Karengko) na kari Seediq, ma bnrnahan mu smmalu ka Proto-kari Seediq. Ptuura ku rmengo pnluban rima dndulan kari nii. Mnugaya we, niqan 18 hengak laqi (*p, *t, *k, *q, *b, *d, *c, *s, *x, *h, *g, *m, *n, *ŋ, *l, *r, *w, *y), 4 hengak bubu mtburux (*a, *i, *u, *ə), ma 3 hengak bubu pnuggaluk (*ay, *aw, *uy). Tikuh bale ka pnggluban hnghengak ga qene mghiti ba na kingal kari, maanu so kiya de, kpryuxun riyung na kari Seediq ka hnghengak so nii de pgaluk kari tgbukuy.

Quri so ttinun kari ma hnyegan kari, ``*ka'' ka pnstiyu nominal case, ma ``*na'' ka pnstiyu genitive case. Ani si, wano ka Tgdaya ida mesa \textit{na}, dndulan kari icil we ini kesa di. Rima qpuruh ka kari tmiyu sseediq so pusu bale, niqan teru qpuruh kari tmiyu sseediq pnuggaluk uri. Sepac ka ``focus'' na kari mlgelu, niqan teru kari suumal pnkingal focus. Meniq ka pnyahan kari puto we, asi ta ka spooda Proto-kari Austronesian uxe de Proto-kari Atayal qmita. Slayan na patis qnquran nii ka ``kari mtrurul'' uri. Niqan daha kari puto, mntena ba ka dungus daha; hnyegan daha we, mntena hari, ani si ini pggaluk bbale. 

Mghiti we, quri so pnsnakan dndulan kari: Daha ka cida pneeyah pusu na. Kingal we Tgdaya, mtburux ka dndulan kari nii. Icil we qpuruh Toda-Truku. Pnskryaan qpuruh nii we, knpryuxan hengak bubu ckceka *ə, nlaxan daha ka ``*na (pnstiyu genetive case)'', ma niqan duma kari puto bnrahan daha. Niqan daha qpuruh biciq ka truma qpuruh Toda-Truku uri: Toda (pnskingal *g daha ka *w), ma Truku (pnskingal *c daha ka *s, ma wada pnskrangan *t ma *d bobo na).