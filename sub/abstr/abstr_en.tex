\vspace{-1.15cm}
This thesis reconstructs the phonology, morphosyntax, and lexicon of Proto-Seediq by applying the Comparative Method to data from five Seediq dialects (Tgdaya, Central Toda, Eastern Toda, Central Truku, and Eastern Truku) and proposes a refined subgrouping hypothesis for these varieties. The reconstructed Proto-Seediq phonemic inventory comprises 18 consonants (*p, *t, *k, *q, *b, *d, *c, *s, *x, *h, *g, *m, *n, *ŋ, *l, *r, *w, *y), 4 monophthongs (*a, *i, *u, *ə), and 3 diphthongs (*ay, *aw, *uy). Proto-Seediq had strict phonotactic constraints at word-final positions, which led to a rich array of morphophonological alternations.

In the morphosyntactic domain, Proto-Seediq distinguishes between a nominative marker *ka and a genitive marker *na. The language originally had five sets of personal pronouns plus an additional clitic set that combined the nominative and genitive functions. Moreover, A detailed reconstruction of the Proto-Seediq voice system reveals four voices and three distinct sets of voice affixes.

Lexical reconstruction identifies several items that require support from Proto-Austronesian or Proto-Atayal. A significant discovery is the occurrence of ``special doublets''—synonymous forms with similar phonological shapes—that appear to be associated with the Atayal Gender Register System. 

The Seediq dialects are classified into two main branches. The Tgdaya dialect constitutes an independent branch, while the remaining dialects form a Toda-Truku subgroup characterized by shared innovations involving historical *ə, sporadic sound changes, the loss of the genitive marker, and specific lexical shifts. Within Toda-Truku, the two Toda dialects merge *g and *w, while the two Truku dialects merge *c and *s and palatalize *t and *d.






% This thesis presents the reconstruction of \acl{psed} phonology, morphosyntax, and lexicon by applying the Comparative Method and based on data from five Seediq dialects: \acl{tg}, \acl{cto}, \acl{eto}, \acl{ctr}, and \acl{etr}. It also proposes a subgrouping hypothesis for these dialects.

% Proto-Seediq phonology can be reconstructed with 18 consonants (*p, *t, *k, *q, *b, *d, *c, *s, *x, *h, *g, *m, *n, *ŋ, *l, *r, *w, *y), 4 monophthongs (*a, *i, *u, *ə), and 3 diphthongs (*ay, *aw, *uy). Phonologically, the Tgdaya dialect is the most conservative, preserving all Proto-Seediq consonants. Other dialects show a reduction in the number of phonemes, mainly due to mergers. In terms of vowels, the Eastern Toda dialect is the most conservative. In addition to its unchanged four monophthongs, the word-medial diphthongs *ay and *aw did not undergo raising or further monophthongization seen in other dialects. The Tgdaya dialect, however, has evolved a five-vowel system (/a, i, u, e, o/). Additionally, both Proto-Seediq and its modern dialects have strict constraints on vowel presence in the final syllable or consonant presence at the word-final position, leading to a rich array of morphophonological alternations.

% In terms of morphosyntax, Proto-Seediq distinguishes between a nominative marker *ka and a genitive marker *na, with the latter lost in all dialects except Tgdaya. Proto-Seediq had five sets of personal pronouns: nominative clitic, genitive clitic, nominative free, oblique free, and possessive free, plus a set of portmanteau nominative + genitive clitic pronouns. The modern dialects have seen different degrees of loss or functional shift among these personal pronouns. I also reconstructs the Proto-Seediq voice system (the ``focus" system), which includes three different affix sets. Additionally, there is a preliminary discussion on the interaction of voice affixes with other aspectual/modal and causative affixes.

% The thesis also discusses some inconsistent lexical items among the Seediq dialects and their possible reconstructions. Some require external evidence from Proto-Austronesian or Proto-Atayalic, and some lexical forms appear to be borrowed from Atayal. Another significant finding is the presence of ``special doublets'' in Seediq, a phenomenon highly related to the Atayal Gender Register System. The ``doublets'' are synonyms share similar word shapes across Seediq dialects. However, it remains challenging to fully determine the development history of this system within Seediq.

% Lastly, I classifies the Seediq dialects into two main branches: the Tgdaya dialect as a distinct branch and the remaining dialects as the Toda-Truku subgroup, including Central Toda, Eastern Toda, Central Truku, and Eastern Truku dialects. This subgroup shares several changes: (1) sound changes related to the historical schwa *ə, (2) sporadic sound changes, (3) loss of the genitive marker *na, and (4) a few lexical innovations in form or meaning. The Toda-Truku subgroup is further divided into the Toda and Truku subgroups, with the former characterized by the merger of *g and *w, and the latter by the merger of *c and *s followed by the palatalization of *t and *d. The Truku subgroup includes more sporadic sound changes and lexical innovations, while the Toda subgroup has fewer evidential features beyond phonology.