\vspace{-1.25cm}
本論文基於五個賽德克語方言(德固達雅、都達、陶賽、德鹿谷、太魯閣)的語料,以比較方法重建原始賽德克語的音韻、形態句法及詞彙系統,並提出賽德克語的方言分群。

原始賽德克語共可重建出 18 個輔音(*p, *t, *k, *q, *b, *d, *c, *s, *x, *h, *g, *m, *n, *ŋ, *l, *r, *w, *y),4 個單元音(*a, *i, *u, *ə),以及 3 個雙元音(*ay, *aw, *uy)。而原始賽德克語在詞尾位置有較嚴格的限制,這亦造成賽德克語相當豐富的形態音韻變化。

形態句法方面,原始賽德克語區分主格 *ka 及屬格 *na 兩種名詞組標記,但在德固達雅以外的方言皆丟失屬格標記 *na。原始賽德克語共有五套人稱代詞,加上一套主格+屬格的融合附著式人稱代詞。筆者亦重建了原始賽德克語的語態系統(即「焦點系統」),包含四個語態及三套不同詞綴。

原始賽德克語的一些詞彙重建,須仰賴如原始南島語或原始泰雅語的支持。而本文另外一個重要的發現是「特殊雙式詞現象」,即形式相近但不完全相同的近義詞或同義詞。這些「雙式詞」與泰雅語的性別語域系統高度相關。

筆者將賽德克語分為兩個主要分支,其中德固達雅方言獨立一支,其餘方言則屬都達-德鹿谷(太魯閣)群。都達-德鹿谷(太魯閣)群共享變化有:中央元音 *ə 的音變、一些零星音變、屬格標記 *na 的丟失、詞彙詞形或語意創新。都達-德鹿谷(太魯閣)群下則又分為都達群及德鹿谷(太魯閣)群,前者主要音韻證據為 *g、*w 的合流,後者則為 *c、*s 的合流以及隨後的 *t、*d 顎化。

\noindent [關鍵字]:歷史語言學、南島語、賽德克語、原始語重建、方言分群