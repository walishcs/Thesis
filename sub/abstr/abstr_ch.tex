本論文基於五個賽德克語方言(德固達雅、都達、陶賽、德鹿谷、太魯閣)的語料,以比較方法重建原始賽德克語的音韻系統、形態句法系統以及詞彙系統,並提出針對上述五個方言的分群方案。

原始賽德克語共可重建出 18 個輔音(*p, *t, *k, *q, *b, *d, *c, *s, *x, *h, *g, *m, *n, *ŋ, *l, *r, *w, *y),4 個單元音(*a, *i, *u, *ə),以及 3 個雙元音(*ay, *aw, *uy)。就輔音層面而言,德固達雅方言最為保守,保留了所有原始賽德克語的輔音音位。而其餘方言皆有音位數量的減少,主要是來自於音位合流。元音方面最為保守的是陶賽方言,除了保持四個單元音未有變化外,其詞中的雙元音 *ay、*aw 並不如其他方言有高化或進一步的單元音化。德固達雅方言在元音上則有較大的變化,發展出了五元音系統(/a i u e o/)。此外,不論是原始賽德克語或者是當代方言,在末音節元音抑或是詞尾輔音都有較嚴格的限制,這亦造成賽德克語相當豐富的形態音韻變化。

形態句法方面,原始賽德克語區分主格 *ka 及屬格 *na 兩種名詞組標記,但在德固達雅以外的方言,屬格標記 *na 皆丟失。原始賽德克語共有附著式主格、附著式屬格、自由式主格、自由式斜格、自由式所有格五套人稱代詞,加上一套主格+屬格的融合附著式人稱代詞。而在不同方言之間有人稱代詞的形式程度不等的丟失或功能轉換。筆者亦重建了原始賽德克語的語態系統(即「焦點系統」),包含三套不同詞綴。同時我也初步地探討了語態詞綴與不同時貌詞綴、使役詞綴的互動。

我也嘗試探討一些賽德克方言之間不一致的詞彙及其可能的重建。其中有一些須仰賴如原始南島語或原始泰雅語的外部證據,而我也發現部分詞彙形式移借自泰雅語。本文另外一個重要的發現是賽德克語各方言之間存在著形式相近但不完全相同的近義詞或同義詞,即文中所稱的特殊雙式詞現象。這些「雙式詞」與泰雅語的性別語域系統高度相關,但根據現有證據,很難完全確定這個系統在賽德克語中的發展歷程。

筆者將賽德克語分為兩個主要分支,其中德固達雅方言獨立一支,其餘方言則屬都達-德鹿谷(太魯閣)群,包含都達、陶賽、德鹿谷、太魯閣四個主要方言。都達-德鹿谷(太魯閣)群共享以下變化:(1)有關中央元音 *ə 的一些音變、(2)零星音變、(3)丟失屬格標記 *na、(4)少數詞彙上的詞形或語意創新。都達-德鹿谷(太魯閣)群下則又分為都達群及德鹿谷(太魯閣)群,前者主要音韻證據為 *g、*w 的合流,後者則為 *c、*s 的合流以及隨後的 *t、*d 顎化。德鹿谷(太魯閣)群包含較多的零星音變及詞彙創新;而都達群音韻以外的證據較少。
