\chapter{Proto-Seediq reconstruction}\label{ch4}
\lipsum[1-2]

\section{Proto-Seediq phonology}
\lipsum[1]

\subsection{Proto-Seediq phoneme inventory}
\lipsum[1]

\subsection{Reconstructed phonemes and their reflexes}
\lipsum[1]

\subsubsection{Proto-Seediq consonants} \label{sec:psedC}
\lipsum[1]

\subsubsection{Proto-Seediq vowels}
\lipsum[1]

\subsection{Proto-Seediq phonotactics}
\lipsum[1]

\subsection{Proto-Seediq morphophonological alternations} \label{sec:psed_alter}
\lipsum[1]


\section{Proto-Seediq morphosyntax}
\lipsum[1]

\subsection{Proto-Seediq affixes}
\lipsum[1]

\subsubsection{Proto-Seediq voice system affixes}
\lipsum[1]

\subsubsection{Proto-Seediq derivational and functional affixes}
\lipsum[1]

\subsection{A reconstruction of Proto-Seediq Pronominal system}
\lipsum[1]

\subsubsection{Personal pronouns}

The personal pronoun system of \psedf is shown in Table \ref{tab:psedpron}. Just like many of Austronesian languages, Seediq distinguishes the differences of first person plural inclusive and exclusive. 

For the clitic (bound) forms, there are two sets of pronouns, some of the forms are the same between two cases. The oblique free forms are fully preserved in \stof, \sctrf, and \setrf. However in the \stgf dialect, there are only few remained, including \textit{kənan} `\textsc{1sg.obl}', \textit{sunan} `\textsc{2sg.obl}', and \textit{munan} `\textsc{1sg.obl}'. Note that \psedf *munan is for second person plural, but \textit{munan} in \stgf is for first person singular according to \textcite[62]{Sung2018Sedgrammar}. These oblique forms in \stgf are not usually used in daily speeches. The set of original nominative free form replace their function; the original nominative forms thus became neutral in \stgf (i.e. Nominative and oblique cases share the same forms).

\begin{table}[!htbp]
\centering
\caption{\psedf personal pronouns}
\label{tab:psedpron}
\begin{tabular}{lllll}
\hline
         & \multicolumn{2}{c}{Clitic} & \multicolumn{2}{c}{Free} \\ \hline
         & Nominative    & Genitive   & Nominative & Oblique     \\ \hline
\textsc{1sg}      & *=ku          & *=mu       & *yaku      & *kənan      \\
\textsc{2sg}      & *=su          & *=su       & *isu       & *sunan      \\
\textsc{3sg}      & ∅             & *=na; *=niya & *hiya      & *həyaan     \\
\textsc{1pl.incl} & *=ta          & *=ta       & *ita       & *tənan      \\
\textsc{1pl.excl} & *=nami        & *=nami     & *yami      & *mənan      \\
\textsc{2pl}      & *=namu        & *=namu     & *yamu      & *munan      \\
\textsc{3pl}      & ∅             & *=dəha     & *dəhəya    & *dəhəyaan   \\ \hline
\end{tabular}
\end{table}

Furthermore, \psedf also has three portmanteau pronouns, as shown in Table \ref{tab:por}.

\begin{table}[!htbp]
\centering
\caption{Portmanteau personal pronouns in \psedf}
\label{tab:por}
\begin{tabular}{ll}
\hline
Form   & Gloss \\ 
\hline
*=misu & `\textsc{1sg.gen+2sg.nom}' \\
*=saku & `\textsc{2sg.gen+1sg.nom}' \\
*=maku & `\textsc{1sg.gen+2pl.nom}' \\
\hline
\end{tabular}
\end{table}

In general, clitic pronouns are placed after the predicate in the sentence, except for possessive structure. The possessive structure in Seediq can be described as ``possessee=possessor.'' For example, \textit{tama=mu} `father=\textsc{1sg.gen}'. Regarding the detailed usage of more Seediq clitic pronouns in different dialects, please refer to \textcite{ochiai2009sedpron,kuondtrvpronoun,holmer2014clitic}.

Next, I will examine the reflexes of personal pronouns in each dialect set by set.

\subsubsection{Nominative clitic}

The five different personal pronouns in the nominative clitic form are reflected consistently across all dialects, as shown in Table \ref{tab:nomclitic}. 

\begin{table}[!htbp]
\centering
\caption{Reflexes of nominative clitic personal pronouns in Seediq dialects}
\label{tab:nomclitic}
\begin{tabular}{llllll}
\hline
       & =\textsc{1sg.nom} & =\textsc{2sg.nom} & =\textsc{1pl.nom.incl} & =\textsc{1pl.nom.excl}    & =\textsc{2pl.nom} \\ \hline
\psed & *=ku    & *=su    & *=ta         & *=nami          & *=namu  \\
\stg  & =ku     & =su     & =ta          & =nami; (=miyan) & =namu   \\
\sto  & =ku     & =su     & =ta          & =nami           & =namu   \\
\sctr & =ku     & =su     & =ta          & =nami           & =namu   \\
\setr & =ku     & =su     & =ta          & =nami           & =namu   \\ \hline
\end{tabular}
\end{table}

However, there is an additional form in \stgf, which is =\textit{miyan} `\textsc{1pl.nom.incl}'. It is speculated that this form might be borrowed from Atayal, where the corresponding form is =\textit{myan} meaning `\textsc{1pl.gen.incl}'. Furthermore, in Seediq, the first person plural nominative and genitive clitics have the same form, suggesting a possible further expansion in this loanword's function.

\subsubsection{Genitive clitic}

The genitive clitic forms of the five personal pronouns are also reflected consistently across the various dialects. Table \ref{tab:genclitic} shows the reflexes. 

\begin{table}[!htbp]
\centering
\caption{Reflexes of genitive clitic personal pronouns in Seediq dialects}
\label{tab:genclitic}
\begin{tabular}{lllll}
\hline
      & =\textsc{1sg.gen}      & =\textsc{2sg.gen}        & =\textsc{3sg.gen} & =\textsc{3sg.gen}       \\ \hline
\psed & *=mu          & *=su            & *=na     & *=niya         \\
\stg  & =mu           & =su             & =na      & ---            \\
\sto  & =mu           & =su             & =na      & =niya          \\
\sctr & =mu           & =su             & =na      & =niya          \\
\setr & =mu           & =su             & =na      & =niya          \\ \hline
      & =\textsc{1pl.gen.incl} & =\textsc{1pl.gen.excl}   & =\textsc{2pl.gen} & =\textsc{3pl.gen}       \\ \hline
\psed & *=ta          & *=nami          & *=namu   & *=dəha         \\
\stg  & =ta           & =nami; (=miyan) & =namu    & =daha          \\
\sto  & =ta           & =nami           & =namu    & =dəha          \\
\sctr & =ta           & =nami           & =namu    & =dəha          \\
\setr & =ta           & =nami           & =namu    & =dəha; (=nəha) \\ \hline
\end{tabular}
\end{table}

In addition to the previously mentioned borrowing from Atayal in \stgf with the form \textit{miyan}, there are two other notable points. First, `\textsc{3sg.gen}' seems to have two forms in \psedf stage, one short form *=na and one long form *=niya, although the latter is lost in \stgf. Another point is that according to \textcite{Lee2022TrukuPOS}, \setrf has =\textit{nəha} meaning `\textsc{3pl.gen}', but I has not found in other dialects, suggesting it might be borrowed from Atayal as well, where the form is \textit{nəhaʔ} `\textsc{3pl.gen}'.

\subsubsection{Nominative free}

Similarly, the nominative free forms show no significant variation or innovation across dialects. Table \ref{tab:nomfree} presents the forms in each dialect.

\begin{table}[!htbp]
\centering
\caption{Reflexes of nominative free personal pronouns in Seediq dialects}
\label{tab:nomfree}
\begin{tabular}{lllll}
\hline
      & \textsc{1sg.nom}      & \textsc{2sg.nom}      & \textsc{3sg.nom} &         \\ \hline
\psed & *yaku        & *isu         & *hiya   &         \\
\stg  & yaku         & isu          & heya    &         \\
\sto  & yaku         & isu          & hiya    &         \\
\sctr & yaku         & isu          & hiya    &         \\
\setr & yaku         & isu          & hiya    &         \\ \hline
      & \textsc{1pl.nom.incl} & \textsc{1pl.nom.excl} & \textsc{2pl.nom} & \textsc{3pl.nom} \\ \hline
\psed & *ita         & *yami        & *yamu   & *dəhəya \\
\stg  & ita          & yami         & yamu    & deheya  \\
\sto  & ita          & yami         & yamu    & dəhiya  \\
\sctr & ita          & yami         & yamu    & dəhiya  \\
\setr & ita          & yami         & yamu    & dəhiya  \\ \hline
\end{tabular}
\end{table}

However, it may be worth noting the origins of the \textsc{3sg.nom} and \textsc{3pl.nom} forms. Although they may appear similar, when reconstructing them, we need to consider external evidence and their etymology.

The \textsc{3sg.nom} form is derived from \pan *si ia, with its corresponding Atayal form being \textit{hiyaʔ}. As for the \textsc{3pl.nom} form *dəhəya, it may not have a clear \pan source, but it corresponds to Atayal forms such as \textit{ləhaʔ} and \textit{ləhəgaʔ}. The form \textit{ləhaʔ} in Atayal may have originated from *ləhəgaʔ > **ləhaʔaʔ > \textit{ləhaʔ}, as the deletion of /g/ would lead to further vowel coalescence. For discussions on /g/ and /ʔ/ in Atayal, please refer to \textcite{goderich2020phd,song2023Aicgprime}. 

\subsubsection{Oblique free}

Unlike the previous sets, the oblique forms in \stgf are nearly obsolete and are mostly replaced by the nominative forms, as mentioned earlier. The remaining forms, \textit{kenan}, \textit{sunan}, \textit{munan}, are only used in specific situations and vary from person to person (\cite{Sung2018Sedgrammar}).

\begin{table}[!htbp]
\centering
\caption{Reflexes of oblique free personal pronouns in Seediq dialects}
\label{tab:oblfree}
\begin{tabular}{lllll}
\hline
      & \textsc{1sg.obl}      & \textsc{2sg.obl}      & \textsc{3sg.obl} &           \\ \hline
\psed & *kənan       & *sunan       & *həyaan &           \\
\stg  & (kenan)      & (sunan)      & ---     &           \\
\sto  & kənan        & sunan        & həyaan  &           \\
\sctr & kənan        & sunan        & həyaan  &           \\
\setr & kənan        & sunan        & həyaan  &           \\ \hline
      & \textsc{1pl.obl.incl} & \textsc{1pl.obl.excl} & \textsc{2pl.obl} & \textsc{3pl.obl}   \\ \hline
\psed & *tənan       & *mənan       & *munan  & *dəhəyaan \\
\stg  & ---          & ---          & (munan) & ---       \\
\sto  & tənan        & mənan        & munan   & dəhəyaan  \\
\sctr & tənan        & mənan        & munan   & dəhəyaan  \\
\setr & tənan        & mənan        & munan   & dəhəyaan  \\ \hline
\end{tabular}
\end{table}

The oblique forms in \psedf are believed to have inherited from the reconstructed Acc(usative) set of personal pronouns proposed by \textcite{ross2006casepronoun} (Except for the third person forms), with the addition of the suffix *-an. A comparison with Paiwan and Ross's \pan Acc personal pronouns' reconstruction is shown in Table \ref{tab:panacc}

\begin{table}[!htbp]
\centering
\caption{A comparison of \pan \textsc{acc}, Paiwan \textsc{nom} (clitic), and \psedf \textsc{obl} personal pronouns}
\label{tab:panacc}
\begin{tabular}{lllllll}
\hline
       &     & \textsc{1sg}      & \textsc{2sg}        & \textsc{1pl.incl} & \textsc{1pl.incl} & \textsc{2pl}         \\\hline
\pan   & \textsc{acc} & *i-ak-ən & *iSu[qu]-n & *ita-ən  & *i-am-ən & *i-mu[qu]-n \\
Paiwan & \textsc{nom} & =[a]kən  & =[ə]sun    & =[i]cən  & =[a]mən  & =[ə]mun     \\
\psedf & \textsc{obl} & *kən-an  & *sun-an    & *tən-an  & *mən-an  & *mun-an    \\ \hline
\end{tabular}
\end{table}

\subsubsection{Portmanteau forms}

Due to the lack of data from \stof and \sctrf, this presentation will only include data from \stgf (from \cite[62]{Sung2018Sedgrammar}) and \setrf (from \cite[74]{Lee2018Trugrammar}), as shown in Table \ref{tab:porref}.

\begin{table}[!htbp]
\centering
\caption{Reflexes of portmanteau personal pronouns in Seediq dialects}
\label{tab:porref}
\begin{tabular}{llll}
\hline
                     & \textsc{1sg.gen+2sg.nom}                 & \textsc{2sg.gen+1sg.nom}                  & \textsc{1sg.gen+2pl.nom}                \\ 
\multicolumn{1}{c}{} & \multicolumn{1}{c}{(I to thee)} & \multicolumn{1}{c}{(Thou to me)} & \multicolumn{1}{c}{(I to you)} \\ \hline
\psedf               & *=misu                          & *=saku                           & *=maku                         \\
\stg                 & =misu                           & =saku                            & =maku                          \\
\setr                & =misu                           & =saku                            & =maku          \\ \hline                
\end{tabular}
\end{table}

Portmanteau pronouns, like other clitic pronouns, are enclitics placed after the predicate. They consist of two arguments and are found exclusively in (1) nominal predicate; and (2) non-agent voice (NAV) sentences, where they represent the agent or experiencer. Here are two examples from \textcite[74--75]{Lee2018Trugrammar}.

\begin{exe}

    \ex Nominal predicate sentence
    \gll \textit{anay=misu} \textit{ka} \textit{isu.} \\
    sister's.husband=\textsc{1sg.gen+2sg.nom} \textsc{nom} \textsc{2sg.nom}\\
    \trans `You are my sister's husband. (For me, you are my sister's husband.) '
    \ex NAV sentence
    \gll \textit{wada=misu} \textit{qta-an} \textit{da!} \\
    \textsc{perf}=\textsc{1sg.gen+2sg.nom} see-\textsc{lv} \textsc{ptcl}\\
    \trans `I have already seen you!'
\end{exe}

\subsubsection{Demonstrative pronouns}
\lipsum[1]

\section{Issues in Proto-Seediq lexical reconstruction}
\lipsum[1]

\subsection{Special doublet phenomenon in Proto-Seediq lexicon}
\lipsum[1]

\subsubsection{The possible relationship between Seediq's special doublet phenomenon and Atayal's Gender Register System}
\lipsum[1]

\subsection{External evidence for lexical reconstructions}
\lipsum[1]

\section{Interim summary}
\lipsum[1]