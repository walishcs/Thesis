\chapter{Proto-Seediq reconstruction}\label{ch4}

This chapter is divided into three main parts. Section \ref{sec:psed_phono} reconstructs the phonological system of \acl{psed} using the Comparative Method. Section \ref{sec:psed_morpho} provides a preliminary exploration and reconstruction of the morphosyntactic system of \acl{psed}. Finally, Section \ref{sec:psed_lexicon} discusses some lexical reconstruction issues in \acl{psed}.

In Section \ref{sec:psed_phono}, I will first present the phoneme inventory of \acl{psed}. Subsequently, I will list out each reconstructed phoneme along with their reflexes and lexical examples in modern dialects. I will also show and explain irregular and/or inconsistent sound correspondences and discuss what proto-phonemes they should be reconstructed as. Additionally, the phonotactics of \acl{psed} and morphophonological alternations that can be reconstructed to the proto-language level will be addressed.

In Section \ref{sec:psed_morpho}, we will discuss the voice system (the so-called ``focus system'') affixes of \acl{psed}, as well as some functional affixes. Additionally, we will reconstruct the pronominal system of Seediq, including personal pronouns and demonstrative pronouns.

Finally, in Section \ref{sec:psed_lexicon}, I will introduce a unique doublet phenomenon found in \acl{psed} and contemporary dialects, along with its potential connection to the ``Gender Register System'' in the Atayal language (see \cite{li1980gender,li1982gender,li1983gender,goderich2020phd}). Additionally, external evidence for the reconstruction of some \acl{psed} words will also be discussed in this section.

\section{Proto-Seediq phonology} \label{sec:psed_phono}

\subsection{Proto-Seediq phoneme inventory}

Table \ref{tab:psedC} shows the consonant inventory of \acl{psed}. There are 18 consonants in \acl{psed} based on my reconstruction, including:

\begin{enumerate}[label=(\roman*), itemsep=0pt, topsep=0pt]
    \item six stops: *p, *t, *k, *q, *b, *d
    \item one affricate: *c
    \item four fricatives: *s, *x, *h, *g
    \item three nasals: *m, *n, *ŋ
    \item two liquids: *l, *r
    \item two glides: *w, *y
\end{enumerate}

\begin{table}[!htbp]
\centering
\caption{Proto-Seediq consonant inventory}
\label{tab:psedC}
\begin{tabular}{l|ll|ll|ll|ll|ll|ll}
\hline
                    & \multicolumn{2}{c|}{Bilabial} & \multicolumn{2}{c|}{Dental/Alveolar} & \multicolumn{2}{c|}{Palatal} & \multicolumn{2}{c|}{Velar} & \multicolumn{2}{c|}{Uvular} & \multicolumn{2}{c}{Pharyngeal} \\ \hline
Stop                & *p            & *b           & *t  \quad\quad\quad             & *d               &             &               & *k          &           & *q            &            &                 &              \\
Affricate           &               &              & *c               &                  &             &               &             &             &               &            &                 &              \\
Fricative           &               &              & *s               &                  &             &               & *x          & *g         &               &            & *h              &              \\
Nasal               &               & *m           &                  & *n               &             &               &             & *ŋ          &               &            &                 &              \\
Lateral  &               &              &                  & *l               &             &               &             &             &               &            &                 &              \\
Tap/Flap                 &               &              &                  & *r               &             &               &             &             &               &            &                 &              \\
Glide               &               & *w           &                  &                  &             & *y            &             &             &               &            &                 &              \\ \hline
\end{tabular}
\end{table}

The vowel inventory of \acl{psed} is straightforward, as shown in Table \ref{tab:psedV}. There are four monophthongs in \acl{psed}, which are *a, *i, *u, and *ə. Note that mid vowels [e] and [o] cannot be found in \acl{psed}. There are three ``diphthongs'' (VG sequences) in \acl{psed}, including *ay, *aw, and *uy. 

\begin{table}[!htbp]
\centering
\caption{Proto-Seediq vowel inventory}
\label{tab:psedV}
\begin{tabular}{lcccccccc}
\cline{1-4} \cline{7-9}
     & Front & Central & Back &  &  & \multicolumn{3}{c}{``diphthongs''} \\ \cline{1-4} \cline{7-9}
High & *i    &         & *u   &  &  &            & *uy       &           \\
Mid  &       & *ə      &      &  &  &            &           &           \\
Low  &       & *a      &      &  &  & *ay        &           & *aw       \\ \cline{1-4} \cline{7-9}
\end{tabular}
\end{table}

\subsection{Reconstructed phonemes and their reflexes}

This section reconstructs the proto-phonemes in \acl{psed} by comparing segments from words collected from modern dialects. Additionally, I demonstrates the reflexes of these reconstructed phonemes in modern dialects across different positions and environments (if applicable).

\subsubsection{Proto-Seediq consonants} \label{sec:psedC}

Reflexes of \acl{psed} *p can be found in word-initial and medial positions in all dialects, as given in the reflexes of *pahuŋ `gallbladder' and *səpi `dream' in Table \ref{tab:psed_p}. However, no Seediq dialects have a word-final -\textit{p}, because there was a merger of word-final *-k and *-p as *-k from Proto-Atayalic to (Proto-)Seediq (\cite{li1981paic}). Li also points out that this merger is natural because [p] and [k] share the feature [+grave]. 

\begin{table}[!htbp]
\centering
\caption{Reflexes of \acl{psed} *p}
\label{tab:psed_p}
\begin{tabular}{lll}
\hline
           & *p-    & *-p-  \\ \hline
\acs{psed} & *\textbf{p}ahuŋ & *sə\textbf{p}i \\ \hdashline
\acs{tg}   & \textbf{p}ahuŋ  & se\textbf{p}i  \\
\acs{cto}  & \textbf{p}ahuŋ  & sə\textbf{p}i  \\
\acs{eto}  &        &       \\
\acs{ctr}  & \textbf{p}ahuŋ  & sə\textbf{p}i  \\
\acs{etr}  & \textbf{p}ahuŋ  & sə\textbf{p}i  \\ \hline
Gloss      & `gallbladder'   & `dream'        \\ \hline
\end{tabular}
\end{table}

In suffixed forms of \acl{psed} words that contain a final *-k from the historical **-p, we can find the final consonant alternating between *k (in non-suffixed forms, like AV form and bare form) and *p (in suffixed forms such as PV form and LV form). Such as *miyu\textbf{k} `to blow (AV)' and *yu\textbf{p}un `to blow (PV)'. I will provide a more detailed description of the alternations in \acl{psed} in Section \ref{sec:psed_alter}.

Reflexes of \acl{psed} *t are relatively more complicated, as shown in Table \ref{tab:psed_t}. \acl{psed} *t is reflected as \textit{c} before \textit{i} in \acs{eto}, \acs{ctr}, and \acs{etr}. More precisely, *t underwent a palatalization and an affrication processes and changed to an affricate \textit{c} [ʨ] in this environment. The relevant examples can be found in the reflexes of *timu `salt' and *quti `feces' in above-mentioned dialects. 

\begin{table}[!htbp]
\centering
\caption{Reflexes of \acl{psed} *t}
\label{tab:psed_t}
\begin{tabular}{llllll}
\hline
           & *t-    & *t- /\_i & *-t-  & *-t- /\_i & *-t    \\ \hline
\acs{psed} & *\textbf{t}unux & *\textbf{t}imu    & *u\textbf{t}ux & *qu\textbf{t}i     & *səpat \\ \hdashline
\acs{tg}   & \textbf{t}unux  & \textbf{t}imu     & u\textbf{t}ux  & qu\textbf{t}i      & sepac  \\
\acs{cto}  & \textbf{t}unux  & \textbf{t}imu     & u\textbf{t}ux  & qu\textbf{t}i      & səpac  \\
\acs{eto}  &        &          &       &           &        \\
\acs{ctr}  & \textbf{t}unux  & \textbf{c}imu     & u\textbf{t}ux  & qu\textbf{c}i      & səpac  \\
\acs{etr}  & \textbf{t}unux  & \textbf{c}imu     & u\textbf{t}ux  & qu\textbf{c}i      & səpat  \\ \hline
Gloss      & `head' & `salt'   & `spirit' & `feces'  & `four' \\ \hline
\end{tabular}
\end{table}

In word-final position, \acl{psed} *t is reflected as \textit{c} [ʦ(ʰ)] in \acl{tg}, \acl{cto}, \acl{eto}, and \acl{ctr}. However, the final affrication seems to be optional and phonetic because the speakers do not release the final stops all the time, especially if there is another word following it. For example, \textit{sepac papak} `four legs' (\acs{tg}) can be pronounced as either [sepaʦʰ papakʰ] or [sepat̚ \ papakʰ]. 

In addition to the environments mentioned above, \acl{psed} *p is consistently reflected as /t/ in other positions, as seen in the reflexes of *tunux `head' and *utux `spirit' in Table \ref{tab:psed_t}.

\acl{psed} *k is reflected as /k/ in all positions in every dialect, as shown in the reflexes of *kari `language', *yaku `\textsc{1sg.nom}', and *papak `animal foot' in Table \ref{tab:psed_k}.

\begin{table}[!htbp]
\centering
\caption{Reflexes of \acl{psed} *k}
\label{tab:psed_k}
\begin{tabular}{llll}
\hline
           & *k-        & *-k-               & *-k           \\ \hline
\acs{psed} & *\textbf{k}ari      & *ya\textbf{k}u              & *papa\textbf{k}        \\ \hdashline
\acs{tg}   & \textbf{k}ari       & ya\textbf{k}u               & papa\textbf{k}         \\
\acs{cto}  & \textbf{k}ari       & ya\textbf{k}u               & papa\textbf{k}         \\
\acs{eto}  &            &                    &               \\
\acs{ctr}  & \textbf{k}ari       & ya\textbf{k}u               & papa\textbf{k}         \\
\acs{etr}  & \textbf{k}ari       & ya\textbf{k}u               & papa\textbf{k}         \\ \hline
Gloss      & `language' & `\textsc{1sg.nom}' & `animal foot' \\ \hline
\end{tabular}
\end{table}

The voiceless uvular stop *q is also regular in all the Seediq dialects under study, and can be found in all positions, as seen in reflexes of *quyux `rain', *laqi `child', and *dawriq `eye' Table \ref{tab:psed_q}. \textcite{li1981paic} notes that \acl{paic} *q has shifted to an epiglottal (or pharyngealized) stop /ʡ/ in Truwan Seediq. 

\begin{table}[!htbp]
\centering
\caption{Reflexes of \acl{psed} *q}
\label{tab:psed_q}
\begin{tabular}{llll}
\hline
           & *q-    & *-q-    & *-q     \\ \hline
\acs{psed} & *\textbf{q}uyux & *la\textbf{q}i   & *dawri\textbf{q}  \\ \hdashline
\acs{tg}   & \textbf{q}uyux  & la\textbf{q}i    & dori\textbf{q}   \\
\acs{cto}  & \textbf{q}uyux  & la\textbf{q}i    & dowri\textbf{q}   \\
\acs{eto}  &        &         & dawri\textbf{q}  \\
\acs{ctr}  & \textbf{q}uyux  & la\textbf{q}i    & dowri\textbf{q}   \\
\acs{etr}  & \textbf{q}uyux  & la\textbf{q}i    & dowri\textbf{q}   \\ \hline
Gloss      & `rain' & `child' & `eye' \\ \hline
\end{tabular}
\end{table}

Reflexes of *b are generally regular in every dialect in the word-initial and medial positions, as shown in  in Table \ref{tab:psed_b}. Like /p/, /b/ can not be found in word-final position in the modern dialects either. *-b merged with *-k word-finally, but the final consonant alternation can still be noticed between the AV forms and LV/PV forms, as in *məlu\textbf{k} `to close (AV)' and *lə\textbf{b}an `to close (LV)'. 

\begin{table}[!htbp]
\centering
\caption{Reflexes of \acl{psed} *b}
\label{tab:psed_b}
\begin{tabular}{lll}
\hline
           & *b-    & *-b-             \\ \hline
\acs{psed} & *\textbf{b}irat & *u\textbf{b}al            \\ \hdashline
\acs{tg}   & \textbf{b}irac  & u\textbf{b}an             \\
\acs{cto}  & \textbf{b}irac  & u\textbf{b}al             \\
\acs{eto}  &        &                  \\
\acs{ctr}  & \textbf{b}irac  & u\textbf{b}al             \\
\acs{etr}  & \textbf{b}irat  & u\textbf{b}al             \\ \hline
Gloss      & `ear'  & `fur; body hair' \\ \hline
\end{tabular}
\end{table}

\acl{psed} *d is reflected as \textit{ɟ} [ɟ\~{ }ʥ] in \acl{eto}, \acl{ctr}, and \acl{etr} before \textit{i}, which parallels the development of \acl{psed} *t. In other environments, \acl{psed} *d is reflected as /d/ in all other word-initial and medial positions. Note that no word-final /d/ is found in any dialect. For reflexes of \acl{psed} *d, please refer to *dara `blood', *diyan `daytime', *rudan `old (person)', and *budi `bow and arrow' in Table \ref{tab:psed_d}. 

\begin{table}[!htbp]
\centering
\caption{Reflexes of \acl{psed} *d}
\label{tab:psed_d}
\begin{tabular}{lllll}
\hline
           & *d-     & *d- /\_i  & *-d-           & *-d- /\_i       \\ \hline
\acs{psed} & *\textbf{d}ara   & *\textbf{d}iyan    & *ru\textbf{d}an         & *bu\textbf{d}i           \\ \hdashline
\acs{tg}   & \textbf{d}ara    & \textbf{d}iyan     & ru\textbf{d}an          & bu\textbf{d}i            \\
\acs{cto}  & \textbf{d}ara    & \textbf{d}iyan     & ru\textbf{d}an          & bu\textbf{d}i            \\
\acs{eto}  &         &           &                &                 \\
\acs{ctr}  & \textbf{d}ara    & \textbf{ɟ}iyan     & ru\textbf{d}an          & bu\textbf{ɟ}i            \\
\acs{etr}  & \textbf{d}ara    & \textbf{ɟ}iyan     & ru\textbf{d}an          & bu\textbf{ɟ}i            \\ \hline
Gloss      & `blood' & `daytime' & `old (person)' & `bow and arrow' \\ \hline
\end{tabular}
\end{table}

There is only one affricate *c in \acl{psed}; its reflexes is shown in Table \ref{tab:psed_c}, such as those of *cakus `camphor' and *bicur `worm'. \acl{psed} *c is retained as an affricate \textit{c} in \acl{tg} and \acl{cto}, but it is reflected as \textit{s} in \acl{eto}, \acl{ctr}, and \acl{etr}. In addition, no word-final *c can be found in \acl{psed}.

\begin{table}[!htbp]
\centering
\caption{Reflexes of \acl{psed} *c}
\label{tab:psed_c}
\begin{tabular}{lll}
\hline
           & *c-       & *-c-   \\ \hline
\acs{psed} & *\textbf{c}akus    & *bi\textbf{c}ur \\ \hdashline
\acs{tg}   & \textbf{c}akus     & bi\textbf{c}un  \\
\acs{cto}  & \textbf{c}akus     & bi\textbf{c}ur  \\
\acs{eto}  &           &        \\
\acs{ctr}  & \textbf{s}akus     & bi\textbf{s}ur  \\
\acs{etr}  & \textbf{s}akus     & bi\textbf{s}ur  \\ \hline
Gloss      & `camphor' & `worm' \\ \hline
\end{tabular}
\end{table}

%*s

%*x

%*h

%*g

%*m

%*n

%*ŋ

%*l

%*r

%*w

%*y

\subsubsection{Proto-Seediq vowels} \label{sec:psedV}
\lipsum[1-15]

\subsubsection{Irregular and inconsistent sound correspondences}
\lipsum[1-10]

\subsection{Proto-Seediq phonotactics}
\lipsum[1-4]

\subsection{Proto-Seediq morphophonological alternations} \label{sec:psed_alter}
\lipsum[1-15]

% 記得談 *sais *səisun 以及 *CaiC-un > Cay.Cun 的差別

\section{Proto-Seediq morphosyntax} \label{sec:psed_morpho}
\lipsum[1]

\subsection{Proto-Seediq affixes}
\lipsum[1]

\subsubsection{Proto-Seediq voice system affixes}
\lipsum[1-10]

\subsubsection{Proto-Seediq derivational and functional affixes}
\lipsum[1-10]

\subsection{A reconstruction of Proto-Seediq Pronominal system}
\lipsum[1]

\subsubsection{Personal pronouns}

The personal pronoun system of \acl{psed} is shown in Table \ref{tab:psedpron}. Just like many of Austronesian languages, Seediq distinguishes the differences of first person plural inclusive and exclusive. 

For the clitic (bound) forms, there are two sets of pronouns, some of the forms are the same between two cases. The oblique free forms are fully preserved in \acl{cto}, \acl{ctr}, and \acl{etr}. However in the \acl{tg} dialect, there are only few remained, including \textit{kənan} `\textsc{1sg.obl}', \textit{sunan} `\textsc{2sg.obl}', and \textit{munan} `\textsc{1sg.obl}'. Note that \acl{psed} *munan is for second person plural, but \textit{munan} in \acl{tg} is for first person singular according to \textcite[62]{Sung2018Sedgrammar}. These oblique forms in \acl{tg} are not usually used in daily speeches. The set of original nominative free form replace their function; the original nominative forms thus became neutral in \acl{tg} (i.e. Nominative and oblique cases share the same forms).

\begin{table}[!htbp]
\centering
\caption{\acl{psed} personal pronouns}
\label{tab:psedpron}
\begin{tabular}{lllll}
\hline
         & \multicolumn{2}{c}{Clitic} & \multicolumn{2}{c}{Free} \\ \hline
         & Nominative    & Genitive   & Nominative & Oblique     \\ \hline
\textsc{1sg}      & *=ku          & *=mu       & *yaku      & *kənan      \\
\textsc{2sg}      & *=su          & *=su       & *isu       & *sunan      \\
\textsc{3sg}      & ∅             & *=na; *=niya & *hiya      & *həyaan     \\
\textsc{1pl.incl} & *=ta          & *=ta       & *ita       & *tənan      \\
\textsc{1pl.excl} & *=nami        & *=nami     & *yami      & *mənan      \\
\textsc{2pl}      & *=namu        & *=namu     & *yamu      & *munan      \\
\textsc{3pl}      & ∅             & *=dəha     & *dəhəya    & *dəhəyaan   \\ \hline
\end{tabular}
\end{table}

Furthermore, \acl{psed} also has three portmanteau pronouns, as shown in Table \ref{tab:por}.

\begin{table}[!htbp]
\centering
\caption{Portmanteau personal pronouns in \acl{psed}}
\label{tab:por}
\begin{tabular}{ll}
\hline
Form   & Gloss \\ 
\hline
*=misu & `\textsc{1sg.gen+2sg.nom}' \\
*=saku & `\textsc{2sg.gen+1sg.nom}' \\
*=maku & `\textsc{1sg.gen+2pl.nom}' \\
\hline
\end{tabular}
\end{table}

In general, clitic pronouns are placed after the predicate in the sentence, except for possessive structure. The possessive structure in Seediq can be described as ``possessee=possessor.'' For example, \textit{tama=mu} `father=\textsc{1sg.gen}'. Regarding the detailed usage of more Seediq clitic pronouns in different dialects, please refer to \textcite{ochiai2009sedpron,kuondtrvpronoun,holmer2014clitic}.

Next, I will examine the reflexes of personal pronouns in each dialect set by set.

\subsubsubsection{Nominative clitic}

The five different personal pronouns in the nominative clitic form are reflected consistently across all dialects, as shown in Table \ref{tab:nomclitic}. 

\begin{table}[!htbp]
\centering
\caption{Reflexes of nominative clitic personal pronouns in Seediq dialects}
\label{tab:nomclitic}
\begin{tabular}{llllll}
\hline
       & =\textsc{1sg.nom} & =\textsc{2sg.nom} & =\textsc{1pl.nom.incl} & =\textsc{1pl.nom.excl}    & =\textsc{2pl.nom} \\ \hline
\acs{psed} & *=ku    & *=su    & *=ta         & *=nami          & *=namu  \\
\acs{tg}  & =ku     & =su     & =ta          & =nami; (=miyan) & =namu   \\
\acs{cto}  & =ku     & =su     & =ta          & =nami           & =namu   \\
\acs{ctr} & =ku     & =su     & =ta          & =nami           & =namu   \\
\acs{etr} & =ku     & =su     & =ta          & =nami           & =namu   \\ \hline
\end{tabular}
\end{table}

However, there is an additional form in \acl{tg}, which is =\textit{miyan} `\textsc{1pl.nom.incl}'. It is speculated that this form might be borrowed from Atayal, where the corresponding form is =\textit{myan} meaning `\textsc{1pl.gen.incl}'. Furthermore, in Seediq, the first person plural nominative and genitive clitics have the same form, suggesting a possible further expansion in this loanword's function.

\subsubsubsection{Genitive clitic}

The genitive clitic forms of the five personal pronouns are also reflected consistently across the various dialects. Table \ref{tab:genclitic} shows the reflexes. 

\begin{table}[!htbp]
\centering
\caption{Reflexes of genitive clitic personal pronouns in Seediq dialects}
\label{tab:genclitic}
\begin{tabular}{lllll}
\hline
      & =\textsc{1sg.gen}      & =\textsc{2sg.gen}        & =\textsc{3sg.gen} & =\textsc{3sg.gen}       \\ \hline
\acs{psed} & *=mu          & *=su            & *=na     & *=niya         \\
\acs{tg}  & =mu           & =su             & =na      & ---            \\
\acs{cto}  & =mu           & =su             & =na      & =niya          \\
\acs{ctr} & =mu           & =su             & =na      & =niya          \\
\acs{etr} & =mu           & =su             & =na      & =niya          \\ \hline
      & =\textsc{1pl.gen.incl} & =\textsc{1pl.gen.excl}   & =\textsc{2pl.gen} & =\textsc{3pl.gen}       \\ \hline
\acs{psed} & *=ta          & *=nami          & *=namu   & *=dəha         \\
\acs{tg}  & =ta           & =nami; (=miyan) & =namu    & =daha          \\
\acs{cto}  & =ta           & =nami           & =namu    & =dəha          \\
\acs{ctr} & =ta           & =nami           & =namu    & =dəha          \\
\acs{etr} & =ta           & =nami           & =namu    & =dəha; (=nəha) \\ \hline
\end{tabular}
\end{table}

In addition to the previously mentioned borrowing from Atayal in \acl{tg} with the form \textit{miyan}, there are two other notable points. First, `\textsc{3sg.gen}' seems to have two forms in \acl{psed} stage, one short form *=na and one long form *=niya, although the latter is lost in \acl{tg}. Another point is that according to \textcite{Lee2022TrukuPOS}, \acl{etr} has =\textit{nəha} meaning `\textsc{3pl.gen}', but I has not found in other dialects, suggesting it might be borrowed from Atayal as well, where the form is \textit{nəhaʔ} `\textsc{3pl.gen}'.

\subsubsubsection{Nominative free}

Similarly, the nominative free forms show no significant variation or innovation across dialects. Table \ref{tab:nomfree} presents the forms in each dialect.

\begin{table}[!htbp]
\centering
\caption{Reflexes of nominative free personal pronouns in Seediq dialects}
\label{tab:nomfree}
\begin{tabular}{lllll}
\hline
      & \textsc{1sg.nom}      & \textsc{2sg.nom}      & \textsc{3sg.nom} &         \\ \hline
\acs{psed} & *yaku        & *isu         & *hiya   &         \\
\acs{tg}  & yaku         & isu          & heya    &         \\
\acs{cto}  & yaku         & isu          & hiya    &         \\
\acs{ctr} & yaku         & isu          & hiya    &         \\
\acs{etr} & yaku         & isu          & hiya    &         \\ \hline
      & \textsc{1pl.nom.incl} & \textsc{1pl.nom.excl} & \textsc{2pl.nom} & \textsc{3pl.nom} \\ \hline
\acs{psed} & *ita         & *yami        & *yamu   & *dəhəya \\
\acs{tg}  & ita          & yami         & yamu    & deheya  \\
\acs{cto}  & ita          & yami         & yamu    & dəhiya  \\
\acs{ctr} & ita          & yami         & yamu    & dəhiya  \\
\acs{etr} & ita          & yami         & yamu    & dəhiya  \\ \hline
\end{tabular}
\end{table}

However, it may be worth noting the origins of the \textsc{3sg.nom} and \textsc{3pl.nom} forms. Although they may appear similar, when reconstructing them, we need to consider external evidence and their etymology.

The \textsc{3sg.nom} form is derived from \acs{pan} *si ia, with its corresponding Atayal form being \textit{hiyaʔ}. As for the \textsc{3pl.nom} form *dəhəya, it may not have a clear \acs{pan} source, but it corresponds to Atayal forms such as \textit{ləhaʔ} and \textit{ləhəgaʔ}. The form \textit{ləhaʔ} in Atayal may have originated from *ləhəgaʔ > **ləhaʔaʔ > \textit{ləhaʔ}, as the deletion of /g/ would lead to further vowel coalescence. For discussions on /g/ and /ʔ/ in Atayal, please refer to \textcite{goderich2020phd,song2023Aicgprime}. 

\subsubsubsection{Oblique free}

Unlike the previous sets, the oblique forms in \acl{tg} are nearly obsolete and are mostly replaced by the nominative forms, as mentioned earlier. The remaining forms, \textit{kenan}, \textit{sunan}, \textit{munan}, are only used in specific situations and vary from person to person (\cite{Sung2018Sedgrammar}).

\begin{table}[!htbp]
\centering
\caption{Reflexes of oblique free personal pronouns in Seediq dialects}
\label{tab:oblfree}
\begin{tabular}{lllll}
\hline
      & \textsc{1sg.obl}      & \textsc{2sg.obl}      & \textsc{3sg.obl} &           \\ \hline
\acs{psed} & *kənan       & *sunan       & *həyaan &           \\
\acs{tg}  & (kenan)      & (sunan)      & ---     &           \\
\acs{cto}  & kənan        & sunan        & həyaan  &           \\
\acs{ctr} & kənan        & sunan        & həyaan  &           \\
\acs{etr} & kənan        & sunan        & həyaan  &           \\ \hline
      & \textsc{1pl.obl.incl} & \textsc{1pl.obl.excl} & \textsc{2pl.obl} & \textsc{3pl.obl}   \\ \hline
\acs{psed} & *tənan       & *mənan       & *munan  & *dəhəyaan \\
\acs{tg}  & ---          & ---          & (munan) & ---       \\
\acs{cto}  & tənan        & mənan        & munan   & dəhəyaan  \\
\acs{ctr} & tənan        & mənan        & munan   & dəhəyaan  \\
\acs{etr} & tənan        & mənan        & munan   & dəhəyaan  \\ \hline
\end{tabular}
\end{table}

The oblique forms in \acl{psed} are believed to have inherited from the reconstructed Acc(usative) set of personal pronouns proposed by \textcite{ross2006casepronoun} (Except for the third person forms), with the addition of the suffix *-an. A comparison with Paiwan and Ross's \acs{pan} Acc personal pronouns' reconstruction is shown in Table \ref{tab:panacc}

\begin{table}[!htbp]
\centering
\caption{A comparison of \acs{pan} \textsc{acc}, Paiwan \textsc{nom} (clitic), and \acl{psed} \textsc{obl} personal pronouns}
\label{tab:panacc}
\begin{tabular}{lllllll}
\hline
       &     & \textsc{1sg}      & \textsc{2sg}        & \textsc{1pl.incl} & \textsc{1pl.incl} & \textsc{2pl}         \\\hline
\acs{pan}   & \textsc{acc} & *i-ak-ən & *iSu[qu]-n & *ita-ən  & *i-am-ən & *i-mu[qu]-n \\
Paiwan & \textsc{nom} & =[a]kən  & =[ə]sun    & =[i]cən  & =[a]mən  & =[ə]mun     \\
\acl{psed} & \textsc{obl} & *kən-an  & *sun-an    & *tən-an  & *mən-an  & *mun-an    \\ \hline
\end{tabular}
\end{table}

\subsubsubsection{Portmanteau forms}

Due to the lack of data from \acl{cto} and \acl{ctr}, this presentation will only include data from \acl{tg} (from \cite[62]{Sung2018Sedgrammar}) and \acl{etr} (from \cite[74]{Lee2018Trugrammar}), as shown in Table \ref{tab:porref}.

\begin{table}[!htbp]
\centering
\caption{Reflexes of portmanteau personal pronouns in Seediq dialects}
\label{tab:porref}
\begin{tabular}{llll}
\hline
                     & \textsc{1sg.gen+2sg.nom}                 & \textsc{2sg.gen+1sg.nom}                  & \textsc{1sg.gen+2pl.nom}                \\ 
\multicolumn{1}{c}{} & \multicolumn{1}{c}{(I to thee)} & \multicolumn{1}{c}{(Thou to me)} & \multicolumn{1}{c}{(I to you)} \\ \hline
\acl{psed}               & *=misu                          & *=saku                           & *=maku                         \\
\acs{tg}                 & =misu                           & =saku                            & =maku                          \\
\acs{etr}                & =misu                           & =saku                            & =maku          \\ \hline                
\end{tabular}
\end{table}

Portmanteau pronouns, like other clitic pronouns, are enclitics placed after the predicate. They consist of two arguments and are found exclusively in (1) nominal predicate; and (2) non-agent voice (NAV) sentences, where they represent the agent or experiencer. Here are two examples from \textcite[74--75]{Lee2018Trugrammar}.

\begin{exe}

    \ex Nominal predicate sentence
    \gll \textit{anay=misu} \textit{ka} \textit{isu.} \\
    sister's.husband=\textsc{1sg.gen+2sg.nom} \textsc{nom} \textsc{2sg.nom}\\
    \trans `You are my sister's husband. (For me, you are my sister's husband.) '
    \ex NAV sentence
    \gll \textit{wada=misu} \textit{qta-an} \textit{da!} \\
    \textsc{perf}=\textsc{1sg.gen+2sg.nom} see-\textsc{lv} \textsc{ptcl}\\
    \trans `I have already seen you!'
\end{exe}

\subsubsection{Demonstrative pronouns}
\lipsum[1-10]

\section{Issues in Proto-Seediq lexical reconstruction} \label{sec:psed_lexicon}
\lipsum[1]

\subsection{Special doublet phenomenon in Proto-Seediq lexicon} \label{sec:sepical_doublet}
\lipsum[1-15]

\subsubsection{The possible relationship between Seediq's special doublet phenomenon and Atayal's Gender Register System}
\lipsum[1-7]

\subsection{External evidence for lexical reconstructions} \label{sec:external_lexical_evidence}
\lipsum[1-7]

\section{Interim summary}
\lipsum[1-4]
