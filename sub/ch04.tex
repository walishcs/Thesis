\chapter{Proto-Seediq phonology}\label{ch4}

In this chapter, I will first present the phoneme inventory of \acl{psed}. Subsequently, I will list out each reconstructed phoneme along with their reflexes and lexical examples in modern dialects. Please note that data from \acl{eto} is only presented when it is \textbf{crucial for reconstruction}. I will also show and explain irregular and/or inconsistent sound correspondences and discuss what proto-phonemes they should be reconstructed as. Additionally, the phonotactics of \acl{psed} and morphophonological alternations that can be reconstructed to the proto-language level will be addressed.

\section{Proto-Seediq phoneme inventory}

Table \ref{tab:psedC} shows the consonant inventory of \acl{psed}. There are 18 consonants in \acl{psed} based on my reconstruction, including:

\begin{enumerate}[label=(\roman*), itemsep=0pt, topsep=0pt]
    \item six stops: *p, *t, *k, *q, *b, *d
    \item one affricate: *c
    \item four fricatives: *s, *x, *h, *g
    \item three nasals: *m, *n, *ŋ
    \item two liquids: *l, *r
    \item two glides: *w, *y
\end{enumerate}

\begin{table}[!htbp]
\centering
\caption{Proto-Seediq consonant inventory}
\label{tab:psedC}
\begin{tabular}{l|ll|ll|ll|ll|ll|ll}
\hline
                    & \multicolumn{2}{c|}{Bilabial} & \multicolumn{2}{c|}{Dental/Alveolar} & \multicolumn{2}{c|}{Palatal} & \multicolumn{2}{c|}{Velar} & \multicolumn{2}{c|}{Uvular} & \multicolumn{2}{c}{Pharyngeal} \\ \hline
Stop                & *p            & *b           & *t  \quad\quad\quad             & *d               &             &               & *k          &           & *q            &            &                 &              \\
Affricate           &               &              & *c               &                  &             &               &             &             &               &            &                 &              \\
Fricative           &               &              & *s               &                  &             &               & *x          & *g         &               &            & *h              &              \\
Nasal               &               & *m           &                  & *n               &             &               &             & *ŋ          &               &            &                 &              \\
Lateral  &               &              &                  & *l               &             &               &             &             &               &            &                 &              \\
Tap/Flap                 &               &              &                  & *r               &             &               &             &             &               &            &                 &              \\
Glide               &               & *w           &                  &                  &             & *y            &             &             &               &            &                 &              \\ \hline
\end{tabular}
\end{table}

The vowel inventory of \acl{psed} is straightforward, as shown in Table \ref{tab:psedV}. There are four monophthongs in \acl{psed}, which are *a, *i, *u, and *ə. Note that mid vowels [e] and [o] cannot be found in \acl{psed}. There are three ``diphthongs'' (VG sequences) in \acl{psed}, including *ay, *aw, and *uy. 

\begin{table}[!htbp]
\centering
\caption{Proto-Seediq vowel inventory}
\label{tab:psedV}
\begin{tabular}{lcccccccc}
\cline{1-4} \cline{7-9}
     & Front & Central & Back &  &  & \multicolumn{3}{c}{``diphthongs''} \\ \cline{1-4} \cline{7-9}
High & *i    &         & *u   &  &  &            & *uy       &           \\
Mid  &       & *ə      &      &  &  &            &           &           \\
Low  &       & *a      &      &  &  & *ay        &           & *aw       \\ \cline{1-4} \cline{7-9}
\end{tabular}
\end{table}

\section{Reconstructed phonemes and their reflexes} \label{sec:psed_CandV}

This section reconstructs the proto-phonemes in \acl{psed} by comparing segments from words collected from modern dialects. Additionally, I demonstrates the reflexes of these reconstructed phonemes in modern dialects across different positions and environments (if applicable).

\subsection{Proto-Seediq consonants} \label{sec:psedC}

\subsubsection{Proto-Seediq obstruents}

Reflexes of \acl{psed} *p can be found in word-initial and medial positions in all dialects, as given in the reflexes of *pahuŋ `gallbladder' and *səpi `dream' in Table \ref{tab:psed_p}. However, no Seediq dialects have a word-final -\textit{p}, because there was a merger of word-final *-k and *-p as *-k from Proto-Atayalic to (Proto-)Seediq (\cite{li1981paic}). Li also points out that this merger is natural because [p] and [k] share the feature [+grave]. 

\begin{table}[!htbp]
\centering
\caption{Reflexes of \acl{psed} *p}
\label{tab:psed_p}
\begin{tabular}{lll}
\hline
           & *p-    & *-p-  \\ \hline
\acs{psed} & *\textbf{p}ahuŋ & *sə\textbf{p}i \\ \hdashline
\acs{tg}   & \textbf{p}ahuŋ  & se\textbf{p}i  \\
\acs{cto}  & \textbf{p}ahuŋ  & sə\textbf{p}i  \\
% \acs{eto}  &        &       \\
\acs{ctr}  & \textbf{p}ahuŋ  & sə\textbf{p}i  \\
\acs{etr}  & \textbf{p}ahuŋ  & sə\textbf{p}i  \\ \hline
Gloss      & `gallbladder'   & `dream'        \\ \hline
\end{tabular}
\end{table}

In suffixed forms of \acl{psed} words that contain a final *-k from the historical **-p, we can find the final consonant alternating between *k (in non-suffixed forms, like AV form and bare form) and *p (in suffixed forms such as PV form and LV form). Such as *miyu\textbf{k} `to blow (AV)' and *yu\textbf{p}un `to blow (PV)'. I will provide a more detailed description of the alternations in \acl{psed} in Section \ref{sec:psed_alter}.

Reflexes of \acl{psed} *t are relatively more complicated, as shown in Table \ref{tab:psed_t}. \acl{psed} *t is reflected as \textit{c} before \textit{i} in \acs{eto}, \acs{ctr}, and \acs{etr}. More precisely, *t underwent a palatalization and an affrication processes and changed to an affricate \textit{c} [ʨ] in this environment. The relevant examples can be found in the reflexes of *timu `salt' and *quti `feces' in above-mentioned dialects. 

\begin{table}[!htbp]
\centering
\caption{Reflexes of \acl{psed} *t}
\label{tab:psed_t}
\begin{tabular}{llllll}
\hline
           & *t-    & *t- /\_i & *-t-  & *-t- /\_i & *-t    \\ \hline
\acs{psed} & *\textbf{t}unux & *\textbf{t}imu    & *u\textbf{t}ux & *qu\textbf{t}i     & *səpat \\ \hdashline
\acs{tg}   & \textbf{t}unux  & \textbf{t}imu     & u\textbf{t}ux  & qu\textbf{t}i      & sepac  \\
\acs{cto}  & \textbf{t}unux  & \textbf{t}imu     & u\textbf{t}ux  & qu\textbf{t}i      & səpac  \\
\acs{eto}  &        & \textit{c}imu&       &           &        \\
\acs{ctr}  & \textbf{t}unux  & \textbf{c}imu     & u\textbf{t}ux  & qu\textbf{c}i      & səpac  \\
\acs{etr}  & \textbf{t}unux  & \textbf{c}imu     & u\textbf{t}ux  & qu\textbf{c}i      & səpat  \\ \hline
Gloss      & `head' & `salt'   & `spirit' & `feces'  & `four' \\ \hline
\end{tabular}
\end{table}

In word-final position, \acl{psed} *t is reflected as \textit{c} [ʦ(ʰ)] in \acl{tg}, \acl{cto}, \acl{eto}, and \acl{ctr}. However, the final affrication seems to be optional and phonetic because the speakers do not release the final stops all the time, especially if there is another word following it. For example, \textit{sepac papak} `four legs' (\acs{tg}) can be pronounced as either [sepaʦʰ papakʰ] or [sepat̚ \ papakʰ]. 

In addition to the environments mentioned above, \acl{psed} *p is consistently reflected as /t/ in other positions, as seen in the reflexes of *tunux `head' and *utux `spirit' in Table \ref{tab:psed_t}.

\acl{psed} *k is reflected as /k/ in all positions in every dialect, as shown in the reflexes of *kari `language', *yaku `\textsc{1sg.nom}', and *papak `animal foot' in Table \ref{tab:psed_k}.

\begin{table}[!htbp]
\centering
\caption{Reflexes of \acl{psed} *k}
\label{tab:psed_k}
\begin{tabular}{llll}
\hline
           & *k-        & *-k-               & *-k           \\ \hline
\acs{psed} & *\textbf{k}ari      & *ya\textbf{k}u              & *papa\textbf{k}        \\ \hdashline
\acs{tg}   & \textbf{k}ari       & ya\textbf{k}u               & papa\textbf{k}         \\
\acs{cto}  & \textbf{k}ari       & ya\textbf{k}u               & papa\textbf{k}         \\
% \acs{eto}  &            &                    &               \\
\acs{ctr}  & \textbf{k}ari       & ya\textbf{k}u               & papa\textbf{k}         \\
\acs{etr}  & \textbf{k}ari       & ya\textbf{k}u               & papa\textbf{k}         \\ \hline
Gloss      & `language' & `\textsc{1sg.nom}' & `animal foot' \\ \hline
\end{tabular}
\end{table}

The voiceless uvular stop *q is also regular in all the Seediq dialects under study, and can be found in all positions, as seen in reflexes of *quyux `rain', *laqi `child', and *dawriq `eye' Table \ref{tab:psed_q}. \textcite{li1981paic} notes that \acl{paic} *q has shifted to an epiglottal (or pharyngealized glottal) stop /ʡ/ in Truwan Seediq. 

\begin{table}[!htbp]
\centering
\caption{Reflexes of \acl{psed} *q}
\label{tab:psed_q}
\begin{tabular}{llll}
\hline
           & *q-    & *-q-    & *-q     \\ \hline
\acs{psed} & *\textbf{q}uyux & *la\textbf{q}i   & *dawri\textbf{q}  \\ \hdashline
\acs{tg}   & \textbf{q}uyux  & la\textbf{q}i    & dori\textbf{q}   \\
\acs{cto}  & \textbf{q}uyux  & la\textbf{q}i    & dowri\textbf{q}   \\
% \acs{eto}  &        &         & dawri\textbf{q}  \\
\acs{ctr}  & \textbf{q}uyux  & la\textbf{q}i    & dowri\textbf{q}   \\
\acs{etr}  & \textbf{q}uyux  & la\textbf{q}i    & dowri\textbf{q}   \\ \hline
Gloss      & `rain' & `child' & `eye' \\ \hline
\end{tabular}
\end{table}

Reflexes of *b are generally regular in every dialect in the word-initial and medial positions, as shown in *birat `ear' and *ubal `fur; body hair' in Table \ref{tab:psed_b}. Like /p/, /b/ can not be found in word-final position in the modern dialects either. *-b merged with *-k word-finally, but the final consonant alternation can still be noticed between the AV forms and LV/PV forms, as in *məlu\textbf{k} `to close (AV)' and *lə\textbf{b}an `to close (LV)'. 

\begin{table}[!htbp]
\centering
\caption{Reflexes of \acl{psed} *b}
\label{tab:psed_b}
\begin{tabular}{lll}
\hline
           & *b-    & *-b-             \\ \hline
\acs{psed} & *\textbf{b}irat & *u\textbf{b}al            \\ \hdashline
\acs{tg}   & \textbf{b}irac  & u\textbf{b}an             \\
\acs{cto}  & \textbf{b}irac  & u\textbf{b}al             \\
% \acs{eto}  &        &                  \\
\acs{ctr}  & \textbf{b}irac  & u\textbf{b}al             \\
\acs{etr}  & \textbf{b}irat  & u\textbf{b}al             \\ \hline
Gloss      & `ear'  & `fur; body hair' \\ \hline
\end{tabular}
\end{table}

\acl{psed} *d is reflected as \textit{ɟ} [ɟ\~{ }ʥ] in \acl{eto}, \acl{ctr}, and \acl{etr} before \textit{i}, which parallels the development of \acl{psed} *t. In other environments, \acl{psed} *d is reflected as /d/ in all other word-initial and medial positions. Note that no word-final /d/ is found in any dialect. For reflexes of \acl{psed} *d, please refer to *dara `blood', *diyan `daytime', *rudan `old (person)', and *səədiq `human' in Table \ref{tab:psed_d}. 

\begin{table}[!htbp]
\centering
\caption{Reflexes of \acl{psed} *d}
\label{tab:psed_d}
\begin{tabular}{lllll}
\hline
           & *d-     & *d- /\_i  & *-d-           & *-d- /\_i       \\ \hline
\acs{psed} & *\textbf{d}ara   & *\textbf{d}iyan    & *ru\textbf{d}an         & *səə\textbf{d}iq           \\ \hdashline
\acs{tg}   & \textbf{d}ara    & \textbf{d}iyan     & ru\textbf{d}an          & see\textbf{d}iq            \\
\acs{cto}  & \textbf{d}ara    & \textbf{d}iyan     & ru\textbf{d}an          & sə\textbf{d}iq            \\
 \acs{eto}  &         &           &                & sə\textbf{ɟ}iq                \\
\acs{ctr}  & \textbf{d}ara    & \textbf{ɟ}iyan     & ru\textbf{d}an          & səə\textbf{ɟ}iq            \\
\acs{etr}  & \textbf{d}ara    & \textbf{ɟ}iyan     & ru\textbf{d}an          & səə\textbf{ɟ}iq            \\ \hline
Gloss      & `blood' & `daytime' & `old (person)' & `bow and arrow' \\ \hline
\end{tabular}
\end{table}

There is only one affricate *c in \acl{psed}; its reflexes is shown in Table \ref{tab:psed_c}, such as those of *cakus `camphor' and *qəcurux `fish'. \acl{psed} *c is retained as an affricate \textit{c} in \acl{tg} and \acl{cto}, but it is reflected as \textit{s} in \acl{eto}, \acl{ctr}, and \acl{etr}. In addition, no word-final *c can be found in \acl{psed}.

\begin{table}[!htbp]
\centering
\caption{Reflexes of \acl{psed} *c}
\label{tab:psed_c}
\begin{tabular}{lll}
\hline
           & *c-       & *-c-   \\ \hline
\acs{psed} & *\textbf{c}akus    & *qə\textbf{c}urux \\ \hdashline
\acs{tg}   & \textbf{c}akus     & qu\textbf{c}urux  \\
\acs{cto}  & \textbf{c}akus     & qə\textbf{c}urux  \\
\acs{eto}  & \textbf{s}akus     & qə\textbf{s}urux       \\
\acs{ctr}  & \textbf{s}akus     & qə\textbf{s}urux  \\
\acs{etr}  & \textbf{s}akus     & qə\textbf{s}urux  \\ \hline
Gloss      & `camphor' & `fish' \\ \hline
\end{tabular}
\end{table}

For \acl{psed} *s, it is reflected as \textit{s} in all positions in all five dialects under study, as shown in the reflexes of *sapah `house', *misan `winter', *cakus `camphor' in Table \ref{tab:psed_s}.

\begin{table}[!htbp]
\centering
\caption{Reflexes of \acl{psed} *s}
\label{tab:psed_s}
\begin{tabular}{llll}
\hline
           & *s-     & *-s-     & *-s       \\ \hline
\acs{psed} & *\textbf{s}apah  & *mi\textbf{s}an   & *caku\textbf{s}    \\ \hdashline
\acs{tg}   & \textbf{s}apah   & mi\textbf{s}an    & caku\textbf{s}     \\
\acs{cto}  & \textbf{s}apah   & mi\textbf{s}an    & caku\textbf{s}     \\
% \acs{eto}  &         &          &           \\
\acs{ctr}  & \textbf{s}apah   & mi\textbf{s}an    & saku\textbf{s}     \\
\acs{etr}  & \textbf{s}apah   & mi\textbf{s}an    & saku\textbf{s}     \\ \hline
Gloss      & `house' & `winter' & `camphor' \\ \hline
\end{tabular}
\end{table}

\acl{psed} *x is regularly reflected as \textit{x} in all positions and all dialects as well, as shown in the reflexes of *in Table \ref{tab:psed_x}. 

\begin{table}[!htbp]
\centering
\caption{Reflexes of \acl{psed} *x}
\label{tab:psed_x}
\begin{tabular}{llll}
\hline
           & *x-    & *-x-   & *-x    \\ \hline
\acs{psed} & *\textbf{x}iluy & *tə\textbf{x}al & *tunu\textbf{x} \\ \hdashline
\acs{tg}   & \textbf{x}iluy  & te\textbf{x}an  & tunu\textbf{x}  \\
\acs{cto}  & \textbf{x}iluy  & tə\textbf{x}al  & tunu\textbf{x}  \\
% \acs{eto}  &        &        &        \\
\acs{ctr}  & \textbf{x}iluy  & tə\textbf{x}al  & tunu\textbf{x}  \\
\acs{etr}  & \textbf{x}iluy  & tə\textbf{x}al  & tunu\textbf{x}  \\ \hline
Gloss      & `iron' & `once' & `head' \\ \hline
\end{tabular}
\end{table}

However, the occurrence of *x is highly restricted. Word-initial *x appears only in one word --- *xiluy `iron'. This word appear to be a Seediq innovation, since there is no similar form found in other Formosan languages. Words that contain word-medial *x are also rare, they are usually related to the meaning `one', including *təxal `once', *taxa `one person', *maxal `ten', etc. These words all contain the fossilized sequences *-xa- or *-əxa-, which are supposed to be related to \acs{pan} *əsa `one'. In \acl{psed}, word-final *x can be found in quite a number of words. 

\textcite{li1981paic} mentioned that the historical source of this (proto-)phoneme is not yet clear, and \textcite{song2024Aicx} argues that (Proto-)Atayalic /x/ is borrowed from other neighboring languages, such as Pazeh and Saysiyat. 

The regular reflex of \acl{psed} *h is \textit{h} in all positions in all dialects under study, as shown in the reflexes of *hidaw `sun', *quhiŋ `head louse', and *sapah `house' in Table \ref{tab:psed_h}.

\begin{table}[!htbp]
\centering
\caption{Reflexes of \acl{psed} *h}
\label{tab:psed_h}
\begin{tabular}{llll}
\hline
           & *h-    & *-h-         & *-h     \\ \hline
\acs{psed} & *\textbf{h}idaw & *qu\textbf{h}iŋ       & *sapa\textbf{h}  \\ \hdashline
\acs{tg}   & \textbf{h}ido   & qu\textbf{h}iŋ        & sapa\textbf{h}   \\
\acs{cto}  & \textbf{h}idaw  & qu\textbf{h}iŋ        & sapa\textbf{h}   \\
% \acs{eto}  &        &              &         \\
\acs{ctr}  & \textbf{h}idaw  & qu\textbf{h}iŋ        & sapa\textbf{h}   \\
\acs{etr}  & \textbf{h}idaw  & qu\textbf{h}iŋ        & sapa\textbf{h}   \\ \hline
Gloss      & `sun'  & `head louse' & `house' \\ \hline
\end{tabular}
\end{table}

Reflexes of \acl{psed} *g are also complicated in the modern dialects, as shown in *gaya `tradition; law'\footnote{The Seediq term ``\textit{Gaya}/\textit{Waya}'' is difficult to find a direct equivalent in English or other languages like Mandarin and Japanese, so it is currently glossed as such as an expedient measure. ``\textit{Gaya}/\textit{Waya}'' may represent a collective set of norms for life, including what should or should not be done in certain situations. However, as I am not a member of the Seediq Nation, for a genuine interpretation of ``\textit{Gaya}/\textit{Waya}'', please refer to the understanding of Seediq people.}, *baga `hand', *marig `to buy \acs{av}', and *əlug `road; path' in Table \ref{tab:psed_g}.  

\begin{table}[!htbp]
\centering
\caption{Reflexes of \acl{psed} *g}
\label{tab:psed_g}
\begin{tabular}{lllll}
\hline
           & *g-              & *-g-   & *-i\textbf{g}              & *-u\textbf{g}         \\ \hline
\acs{psed} & *\textbf{g}aya            & *ba\textbf{g}a  & *mari\textbf{g}            & *əlu\textbf{g}        \\ \hdashline
\acs{tg}   & \textbf{g}aya             & ba\textbf{g}a   & mar\textbf{u}y             & elu          \\
\acs{cto}  & \textbf{w}aya             & ba\textbf{w}a   & mari              & əlu          \\
\acs{eto}  & \textbf{w}aya             & ba\textbf{w}a   & mari                  & əlu             \\
\acs{ctr}  & \textbf{g}aya             & ba\textbf{g}a   & mari\textbf{g}             & əlu\textbf{g}         \\
\acs{etr}  & \textbf{g}aya             & ba\textbf{g}a   & mari\textbf{g}             & əlu\textbf{g}         \\ \hline
Gloss      & `tradition; law' & `hand' & `to buy \acs{av}' & `road; path' \\ \hline
\end{tabular}
\end{table}

In word-initial and word-medial positions, \acl{psed} *g is reflected as \textit{w} in \acl{cto} and \acl{eto}, but it retains as \textit{g} in all other dialects. 

In the word-final position, the reflexes depend on the last vowel. \acl{psed} *-g is retained as \textit{g} after a high front vowel \textit{i} in \acl{ctr} and \acl{etr}; however, it is deleted in \acl{cto} and \acl{eto} (probably *-ig > **-iy > \textit{i}). In \acl{tg}, *-ig is reflected as -\textit{uy}, which probably underwent the following gliding and metathesis processes: *-ig > **-iw > **-wi > -\textit{uy}. As for *-g after *u, it is deleted (through a glide /w/) in \acl{tg} and \acl{cto}, but preserved as /g/ in \acl{ctr} and \acl{etr}. 

In the relatively early \acl{tg} dialect, the word-final -\textit{uw} was still present, as documented by \textcite{yang1976sedpho} and \textcite{li1981paic}. However, the distinction between -\textit{uw} and -\textit{u} is no longer evident in the speech of my language consultant. The reason might be that the sequence [uw] violates the Obligatory Contour Principle (OCP), as [u] and [w] have nearly identical features, leading to further segment deletion.

Note that no evidence for the presence of word-final *-ag can be found in Seediq. It seems that Pre-\acl{psed} **-ag merged with **-aw as *-aw in the \acl{psed} stage. For example, \acl{psed} *siy\textbf{aw} `next to' and *bab\textbf{aw} `on, above' correspond to \acl{pata} *siy\textbf{ag} `rim; border; edge' and *bab\textbf{aw} `on top of', respectively.

There is a issue regarding the \acl{psed} *g, whether this proto-phoneme should be classified as a stop or a fricative. In this paper, I choose to classify it as a fricative for the following reasons: (1) The reflexes of \acl{psed} *g in modern dialects include [g], [w], [ɣ], indicating that the phonetic value of the proto-phoneme most likely was [ɣ], (2) Seediq exhibits a phenomenon of word-final devoicing (from voiced stops to voiceless stops). However, whether in the \acl{psed} stage or in modern dialects, there is no phenomenon where word-final *g or \textit{g} changes to [k]. Instead, it interacts with different preceding vowels as a unit and becomes different sound sequences, without undergoing devoicing. Therefore, phonologically, *g is not classified in the same category as other voiced stop consonants such as *b and *d.

\subsubsection{Proto-Seediq nasals}

\acl{psed} *m is reflected as \textit{m} in word-initial and word-medial positions in all dialects, as seen in *malu `good' and *tama `father' in Table \ref{tab:psed_m}.

\begin{table}[!htbp]
\centering
\caption{Reflexes of \acl{psed} *m}
\label{tab:psed_m}
\begin{tabular}{lll}
\hline
           & *m-    & *-m-     \\ \hline
\acs{psed} & *\textbf{m}alu  & *ta\textbf{m}a    \\ \hdashline
\acs{tg}   & \textbf{m}alu   & ta\textbf{m}a     \\
\acs{cto}  & \textbf{m}alu   & ta\textbf{m}a     \\
% \acs{eto}  &        &          \\
\acs{ctr}  & \textbf{m}alu   & ta\textbf{m}a     \\
\acs{etr}  & \textbf{m}alu   & ta\textbf{m}a     \\ \hline
Gloss      & `good' & `father' \\ \hline
\end{tabular}
\end{table}

Word-final -\textit{m} cannot be found in any dialects. Like other bilabial phonemes such as *p and *b, \acl{psed} *-m merged with its corresponding velar phoneme *-ŋ. The alternation can still be found in AV forms vs. PV/LV forms, as contrasted in *migi\textbf{ŋ} `to look for (AV)' and *gi\textbf{m}an `to look for (LV)'. 

\acl{psed} *n and *ŋ are regularly reflected as \textit{n} and \textit{ŋ}, respectively in all positions in all dialects, as shown in the reflexes of *naqih `bad', *inu `where', and *rəbagan `summer' in Table \ref{tab:psed_n} and *ŋiraw `mushroom', *məŋari `nine', and *alaŋ `village' in \ref{tab:psed_ŋ}.

\begin{table}[!htbp]
\centering
\caption{Reflexes of \acl{psed} *n}
\label{tab:psed_n}
\begin{tabular}{llll}
\hline
           & *n-    & *-n-    & *-n     \\ \hline
\acs{psed} & *\textbf{n}aqih & *i\textbf{n}u    & rəbaga\textbf{n} \\ \hdashline
\acs{tg}   & \textbf{n}aqah  & i\textbf{n}u     & rubaga\textbf{n} \\
\acs{cto}  & \textbf{n}aqah  & i\textbf{n}u     & rəbawa\textbf{n} \\
% \acs{eto}  &        &         &         \\
\acs{ctr}  & \textbf{n}aqih  & i\textbf{n}u     & rəbaga\textbf{n} \\
\acs{etr}  & \textbf{n}aqih  & i\textbf{n}u     & rəbaga\textbf{n} \\ \hline
Gloss      & `bad'  & `where' & `summer \\ \hline
\end{tabular}
\end{table}

\begin{table}[!htbp]
\centering
\caption{Reflexes of \acl{psed} *ŋ}
\label{tab:psed_ŋ}
\begin{tabular}{llll}
\hline
           & *ŋ-        & *-ŋ-    & *-ŋ       \\ \hline
\acs{psed} & *\textbf{ŋ}iraw     & *mə\textbf{ŋ}ari & *ala\textbf{ŋ}     \\ \hdashline
\acs{tg}   & \textbf{ŋ}iro       & mu\textbf{ŋ}ari  & ala\textbf{ŋ}      \\
\acs{cto}  & \textbf{ŋ}iraw      & məŋ\textbf{ŋ}ri  & ala\textbf{ŋ}      \\
% \acs{eto}  &            &         &           \\
\acs{ctr}  & \textbf{ŋ}iraw      & mə\textbf{ŋ}ari  & ala\textbf{ŋ}      \\
\acs{etr}  & \textbf{ŋ}iraw      & mə\textbf{ŋ}ari  & ala\textbf{ŋ}      \\ \hline
Gloss      & `mushroom' & `nine'  & `village' \\ \hline
\end{tabular}
\end{table}

\subsubsection{Proto-Seediq liquids}

Word-final *l is reflected as \textit{n} in \acl{tg} and is retained as \textit{r} in other dialectl. The reflexes of *l in other position show regular \textit{k}. Table \ref{tab:psed_l} shows the reflexes of \acl{psed} *l, as in the reflexes of *laqi `child', *qawlit `mouse', and *təxal `once'. 

\begin{table}[!htbp]
\centering
\caption{Reflexes of \acl{psed} *l}
\label{tab:psed_l}
\begin{tabular}{llll}
\hline
           & *l-     & *-l-    & *-l    \\ \hline
\acs{psed} & *\textbf{l}aqi   & *qaw\textbf{l}it & *təxa\textbf{l} \\
\acs{tg}   & \textbf{l}aqi    & qo\textbf{l}ic   & texa\textbf{n}  \\
\acs{cto}  & \textbf{l}aqi    & qow\textbf{l}ic  & təxa\textbf{l}  \\
% \acs{eto}  &         &         &        \\
\acs{ctr}  & \textbf{l}aqi    & qow\textbf{l}ic  & təxa\textbf{l}  \\
\acs{etr}  & \textbf{l}aqi    & qow\textbf{l}it  & təxa\textbf{l}  \\ \hline
Gloss      & `child' & `mouse' & `once' \\ \hline
\end{tabular}
\end{table}

The situation of \acl{psed} *r is similar to that of *l. \acl{tg} reflects word-final *r as \textit{n}. Reflexes of *l in other positions in all dialects are \textit{l}, as shown in the reflexes of *rudan `old (person)', *dara `blood', and *bicur `worm' Table \ref{tab:psed_r}. 

\begin{table}[!htbp]
\centering
\caption{Reflexes of \acl{psed} *r}
\label{tab:psed_r}
\begin{tabular}{llll}
\hline
           & *r-            & *-r-    & *-r    \\ \hline
\acs{psed} & *\textbf{r}udan         & *da\textbf{r}a   & *bicu\textbf{r} \\ \hdashline
\acs{tg}   & \textbf{r}udan          & da\textbf{r}a    & bicu\textbf{n}  \\
\acs{cto}  & \textbf{r}udan          & da\textbf{r}a    & bicu\textbf{r}  \\
% \acs{eto}  &                &         &        \\
\acs{ctr}  & \textbf{r}udan          & da\textbf{r}a    & bisu\textbf{r}  \\
\acs{etr}  & \textbf{r}udan          & da\textbf{r}a    & bisu\textbf{r}  \\ \hline
Gloss      & `old (person)' & `blood' & `worm' \\ \hline
\end{tabular}
\end{table}

\subsubsection{Proto-Seediq glides}

\acl{psed} *w is reflected as \textit{w} in word-initial and medial (syllable onset) positions in all dialects, as shown in the reflexes of *waqit `fang', and *rawa `basket' in Table \ref{tab:psed_w}. Syllable-final *w will be discussed together with vowels in Section \ref{sec:psedV}.

\begin{table}[!htbp]
\centering
\caption{Reflexes of \acl{psed} *w}
\label{tab:psed_w}
\begin{tabular}{lll}
\hline
           & *w-    & *-w-     \\ \hline
\acs{psed} & *\textbf{w}aqit & *ra\textbf{w}a    \\ \hdashline
\acs{tg}   & \textbf{w}aqic  & ra\textbf{w}a     \\
\acs{cto}  & \textbf{w}aqic  & ra\textbf{w}a     \\
% \acs{eto}  &        &          \\
\acs{ctr}  & \textbf{w}aqic  & ra\textbf{w}a     \\
\acs{etr}  & \textbf{w}aqit  & ra\textbf{w}a     \\ \hline
Gloss      & `fang' & `basket' \\ \hline
\end{tabular}
\end{table}

\acl{psed} word-initial and word-medial *y is reflected as \textit{y} in all the dialects under study, as shown in the reflexes of *yaku `\textsc{1sg.nom}' and *gaya `tradition; law'. Syllable-final *y will be discussed together with vowels in Section \ref{sec:psedV}.

\begin{table}[!htbp]
\centering
\caption{Reflexes of \acl{psed} *y}
\label{tab:psed_y}
\begin{tabular}{lll}
\hline
           & *y-                & *-y-             \\ \hline
\acs{psed} & *\textbf{y}aku              & *ga\textbf{y}a            \\ \hdashline
\acs{tg}   & \textbf{y}aku               & ga\textbf{y}a             \\
\acs{cto}  & \textbf{y}aku               & wa\textbf{y}a             \\
% \acs{eto}  &                    &                  \\
\acs{ctr}  & \textbf{y}aku               & ga\textbf{y}a             \\
\acs{etr}  & \textbf{y}aku               & ga\textbf{y}a             \\ \hline
Gloss      & `\textsc{1sg.nom}' & `tradition; law' \\ \hline
\end{tabular}
\end{table}

\subsection{Proto-Seediq vowels} \label{sec:psedV}

\subsubsection{Proto-Seediq monophthongs}

\acl{psed} *a, *i and *u are reflected as \textit{a}, \textit{i}, and \textit{u}, respectively in all dialects, as shown in the reflexes of *alaŋ `village', *ini `\acs{neg}', and *rudu `nest' in Table \ref{tab:psed_aiu}. These three vowels can only be found in the penultimate syllable and the final syllable in \acl{psed}.  

\begin{table}[!htbp]
\centering
\caption{Reflexes of \acl{psed} *a, *i, and *u}
\label{tab:psed_aiu}
\begin{tabular}{llll}
\hline
           & *a        & *i          & *u     \\ \hline
\acs{psed} & *\textbf{a}l\textbf{a}ŋ     & *\textbf{i}n\textbf{i}        & *r\textbf{u}d\textbf{u}  \\ \hdashline
\acs{tg}   & \textbf{a}l\textbf{a}ŋ      & \textbf{i}n\textbf{i}         & r\textbf{u}d\textbf{u}   \\
\acs{cto}  & \textbf{a}l\textbf{a}ŋ      & \textbf{i}n\textbf{i}         & r\textbf{u}d\textbf{u}   \\
% \acs{eto}  &           &             &        \\
\acs{ctr}  & \textbf{a}l\textbf{a}ŋ      & \textbf{i}n\textbf{i}         & r\textbf{u}d\textbf{u}   \\
\acs{etr}  & \textbf{a}l\textbf{a}ŋ      & \textbf{i}n\textbf{i}         & r\textbf{u}d\textbf{u}   \\ \hline
Gloss      & `village' & `\acs{neg}' & `nest' \\ \hline
\end{tabular}
\end{table}

As for \acl{psed} *ə, it can only appears at a non-word-final position. However, the penultimate schwa and the prepenultimate ones behave very differently among modern dialects. Therefore, they will be discussed separately. 

The penultimate *ə is regularly reflected as a mid front vowel \textit{e} in \acl{tg}, but it is generally retained as \textit{ə} in \acl{cto}, \acl{eto}, \acl{ctr}, and \acl{etr}. In non-\acl{tg} dialects, if the schwa *ə occurs before \textit{y} or {w}, it will be assimilated as \textit{i} or {u}. Moreover, if there is no consonant between the *ə and the next vowel, the *ə will be assimilated as the same value of the next vowel. \acl{cto} and \acl{eto} tends to reflect *ə as \textit{u} when there is a *g (> \textit{w} in \acl{cto}) before or after it, but this does not occur in every case. In addition, if the historical *g occurs in the onset position, it will be deleted after the assimilation process occurs. Table \ref{tab:psed_ə1} shows the reflexes of \acl{psed} penultimate *ə as the reflexes of *beyax `power', *wəəwa `young woman', *məhəal `to carry on shoulder', *ləiŋ `to hide', *gesak `k.o. bamboo tool', *məgay `to give', and *əlug `road; path' in modern dialects. The environments listed in the table are all situations where schwa \textit{ə} appears in a penultimate syllable.

\begin{table}[!htbp]
\centering
\caption{Reflexes of \acl{psed} penultimate *ə}
\label{tab:psed_ə1}
\begin{tabular}{lllll}
\hline
           & *ə /\_y            & *ə /\_w       & \multicolumn{2}{l}{*ə /\_V\xb{x}}  \\ \hline
\acs{psed} & *b\textbf{ə}yax             & *wə\textbf{ə}wa        & *məh\textbf{ə}al                & *l\textbf{ə}iŋ     \\ \hdashline
\acs{tg}   & b\textbf{e}yax              & we\textbf{e}wa         & meh\textbf{e}an                 & l\textbf{e}iŋ      \\
\acs{cto}  & b\textbf{i}yax              & wə\textbf{u}wa/wowwa   & məhal                  & (tə)liŋ   \\
\acs{eto}  &                    & \textbf{u}wa              &                        &           \\
\acs{ctr}  & b\textbf{i}yax              & wa\textbf{u}wa         & məh\textbf{a}al                 & l\textbf{i}iŋ      \\
\acs{etr}  & b\textbf{i}yax              & wa\textbf{u}wa         & məh\textbf{a}al                 & l\textbf{i}iŋ      \\
Gloss      & `power'            & `young woman' & `to carry on shoulder' & `to hide' \\ \hline \hline
           & *ə /g\_            & *ə /\_g       & *ə                     &           \\ \hline
\acs{psed} & *g\textbf{ə}sak             & *m\textbf{ə}gay        & *\textbf{ə}lug                  &           \\ \hdashline
\acs{tg}   & g\textbf{e}sak              & m\textbf{e}ge          & \textbf{e}lu                    &           \\
\acs{cto}  & \textbf{u}sak              & m\textbf{u}way         & \textbf{ə}lu                    &           \\
 \acs{eto}  &                           &  m\textbf{u}way        &                        &           \\
\acs{ctr}  & g\textbf{i}sak              & m\textbf{ə}gay         & \textbf{ə}lug                   &           \\
\acs{etr}  & g\textbf{ə}sak              & m\textbf{ə}gay         & \textbf{ə}lug                   &           \\
Gloss      & `k.o. bamboo tool' & `to give'     & `road; path'           &           \\ \hline
\end{tabular}
\end{table}

Unlike some Atayal dialects (e.g. Matu'uwal and Plngawan, see \cite{goderich2020phd} for details), none of the modern Seediq dialects still have retained phonemic distinctions of vowel in prepenultimate syllables. Generally speaking, it is better to reconstruct *ə for this prepenultimate neutralized (or reduced) vowel. It is reflected as \textit{ə} in \acl{cto}, \acl{eto}, \acl{ctr}, and \acl{etr}, and \textit{u} in \acl{tg} if there is no other condition, as seen in the reflexes of *dəqəras `face' in Table \ref{tab:psed_ə2}. [u] is more marked compare with [ə], and only one dialect --- \acl{tg} reflects \textit{u} because [ə] is totally absent in its phonology; thus, *ə is assigned because of the above reasons. 

As for the conditioned reflexes, again, in non-\acl{tg} dialects, if the schwa *ə occurs before \textit{y}, it will be assimilated as \textit{i}. Moreover, since \acl{cto} and \acl{eto} reflect \textit{w} of \acl{psed} *g, it affected the vowel quality of the prepenultimate *ə. Specifically, the prepenultimate *ə is reflected as \textit{u} in \acl{cto} if there is a *g before or after it. In addition, if the historical *g occurs in the onset position, it will be deleted after the assimilation process occurs. Relevant cases can be seen in the reflexes of *təyaquŋ `crow', *gəciluŋ `sea' and *bəgihur `wind' in Table \ref{tab:psed_ə2}. 

In the \acl{tg} dialect, the prepenultimate vowel will be assimilated by the following vowel if: (1) there is no consonant between two vowels, (2) the consonant between them is an \textit{h} [ħ]. The reflexes of *məuyas `to sing' and *cəhədil `heavy' in Table \ref{tab:psed_ə2} shows the examples. 

Note that the environments listed in Table \ref{tab:psed_ə2} are all situations where schwa \textit{ə} appears in a \textbf{pre}penultimate syllables.

\begin{table}[!htbp]
\centering
\caption{Reflexes of \acl{psed} prepenultimate *ə}
\label{tab:psed_ə2}
\begin{tabular}{lllllll}
\hline
           & *ə /\_y  & *ə /g\_  & *ə /\_g  & \multicolumn{2}{l}{*ə /\_(h)V\xb{x}} & *ə       \\ \hline
\acs{psed} & *t\textbf{ə}yaquŋ & *g\textbf{ə}ciluŋ & *b\textbf{ə}gihur & *m\textbf{ə}uyas           & *c\textbf{ə}hədil         & *d\textbf{ə}qəras \\ \hdashline
\acs{tg}   & t\textbf{u}yaquŋ  & r\textbf{u}ciluŋ  & b\textbf{u}gihun  & m\textbf{u}uyas            & c\textbf{e}hedin          & d\textbf{u}qeras  \\
\acs{cto}  & t\textbf{i}yaquŋ  & \textbf{u}ciluŋ   & b\textbf{u}wihur  & muyas             & c\textbf{ə}hədil          & d\textbf{ə}qəras  \\
 \acs{eto}  &          & \textbf{u}siluŋ   &          &                   &                  &          \\
\acs{ctr}  & c\textbf{i}yaquŋ  & g\textbf{ə}siluŋ  & b\textbf{ə}gihur  & m\textbf{ə}uyas            & s\textbf{ə}həɟil          & d\textbf{ə}qəras  \\
\acs{etr}  & c\textbf{i}yaquŋ  & g\textbf{ə}siluŋ  & b\textbf{ə}gihur  & m\textbf{ə}uyas            & s\textbf{ə}həɟil          & d\textbf{ə}qəras  \\ \hline
Gloss      & `crow'   & `sea'    & `wind'   & `to sing'         & `heavy'          & `face'   \\ \hline
\end{tabular}
\end{table}

In addition, \acl{cto}, \acl{eto} and some \acl{ctr} speakers tend to ``shorten'' a sequence of two identical schwas (crossing the antepenultimae and penultimate syllable), as shown in the reflexes of *səədiq `human' and *kəəman `night' in Table \ref{tab:psed_VxVx}. Note that this is mandatory in \acl{cto} and \acl{eto}, but it seems not very stable in \acl{ctr}.  

\begin{table}[!htbp]
\centering
\caption{Reflexes of Proto-Seediq *əə sequence}
\label{tab:psed_VxVx}
\begin{tabular}{lll}
\hline
           & \multicolumn{2}{l}{*əə} \\ \hline
\acs{psed} & *s\textbf{əə}diq       & *k\textbf{əə}man            \\ \hdashline
\acs{tg}   & s\textbf{ee}diq        & k\textbf{ee}man             \\
\acs{cto}  & s\textbf{ə}diq         & k\textbf{ə}man              \\
\acs{eto}  & s\textbf{ə}ɟiq         & k\textbf{ə}man                    \\
\acs{ctr}  & s\textbf{əə}ɟiq        & k\textbf{əə}man \~{} k\textbf{ə}man              \\
\acs{etr}  & s\textbf{əə}ɟiq        & k\textbf{əə}man         \\ \hline
Gloss      & `human'       & `night'      \\ \hline
\end{tabular}
\end{table}

Additionally, the phenomenon of shortening identical vowels appears to be found among younger speakers of every dialect, suggesting it is a parallel and ongoing sound change. Take this pair of word in \acl{tg} for example, \textit{pulaale} `do sth. first' contains two `\textit{a}'s originally; however, more people are starting to pronounce it as \textit{pulale}, making it indistinguishable from \textit{pulale} `butterfly'.

``Vowel shortening'' does not occur across the last two syllables, as in the case of \textit{saan} `to go (LV)', which is not pronounced as **\textit{san} in any known dialect.

\subsubsection{Proto-Seediq diphthongs}

\acl{psed} *ay and *aw are reflected as \textit{e} and \textit{o}, respectively in word-medial and final positions in \acl{tg}. By contrast, they are retained -\textit{ay} and -\textit{aw} word-finally, but as -\textit{ey}- and -\textit{ow}- word-medially in \acl{cto}, \acl{ctr}, and \acl{etr}. Among the dialects, \acl{eto} in the most conservative in terms of diphthongs. \acl{eto} reflects *ay and *aw as \textit{ay} and \textit{aw} in both word-medial and word-final positions. 

\acl{psed} *uy can only appear at the word-final position, and it is reflected as -\textit{uy} in all dialects. Reflexes of \acl{psed} diphthongs are shown in the reflexes of *bayluh `k.o. bean', *maytaq `to stab (AV)', *balay `true', *dawriq `eye', *babaw `above; top', and *babuy `pig' in Table \ref{tab:psed_VV}.

\begin{table}[!htbp]
\centering
\caption{Reflexes of Proto-Seediq diphthongs}
\label{tab:psed_VV}
\begin{tabular}{lllllll}\hline
           & \multicolumn{2}{l}{*-ay-}       & *-ay   & *-aw-   & *-aw         & *-uy   \\ \hline
\acs{psed} & *b\textbf{ay}luh  & *m\textbf{ay}taq   & *bal\textbf{ay} & *d\textbf{aw}riq & *bab\textbf{aw}       & *bab\textbf{uy} \\ \hdashline
\acs{tg}   & b\textbf{e}luh    & m\textbf{e}taq     & bal\textbf{e}   & d\textbf{o}riq   & bob\textbf{o}         & bab\textbf{uy}  \\
\acs{cto}  & b\textbf{ey}luh   & m\textbf{ey}taq    & bal\textbf{ay}  & d\textbf{ow}riq  & bab\textbf{aw}        & bab\textbf{uy}  \\
\acs{eto}  &                   & m\textbf{ay}taq    &                 & d\textbf{aw}riq  &              &        \\
\acs{ctr}  & b\textbf{ey}luh   & (m\textbf{i}taq)   & bal\textbf{ay}  & d\textbf{ow}riq  & bab\textbf{aw}        & bab\textbf{uy}  \\
\acs{etr}  & b\textbf{ey}luh   & m\textbf{ey}taq    & bal\textbf{ay}  & d\textbf{ow}riq  & bab\textbf{aw}        & bab\textbf{uy}  \\ \hline
Gloss      & `k.o. bean'       & `to stab (AV)'     & `true' & `eye'   & `above; top' & `pig' 
\\ \hline
\end{tabular}
\end{table}

Except for \acl{eto}, we can observe that some records from the Japanese colonial period indicate that various Seediq dialects retained the word-medial -\textit{ay}- and -\textit{aw}-. For example, the Musha/Paran dialect (related and similar to modern Tgdaya) in \textcite{ogawa2006voc} includes words like \textit{d\textbf{ao}ryak} (22), \textit{d\textbf{ao}deak} (23a) `eye', \textit{b\textbf{ai}roaχ} (23d) `bean', etc.\footnote{The code in the parentheses represents forms collected from different speakers, villages, and/or dialects in the book.} 

Similarly, \citeauthor{tashiro1900easterntw}'s (\citeyear{tashiro1900easterntw}) investigation of the Eastern Truku dialect (Taroko, 太魯閣) contains words such as \textit{t\textbf{au}kan} `net bag', \textit{b\textbf{ai}lo} `k.o. bean' that retained -\textit{ay}- and -\textit{aw}- in their lexicon. 

However, there are also instances where speakers or villages documented in these sources have shifted to -\textit{e(y)}- and -\textit{o(w)}-. The raising of these two diphthongs or their further change to monophthongs appears to be sound changes that occurred independently in different dialects over the past century. Therefore, the diphthongs in the word-medial position should be reconstructed as *-ay- and *-aw- in the \acl{psed} stage and/or at any possible subgroup's ancestral language level (see Chapter \ref{ch7} for subgrouping and the Appendix for the word list).

\subsection{Inconsistent sound correspondences}

\subsubsection{Inconsistent consonantal sound correspondences}

Seediq dialects exhibit some unstable or inconsistent consonant correspondences. These inconsistent consonant correspondences may have various causes. This section will explore some inconsistent consonant correspondences that can be grouped into categories.

%LD (cite Tsuchida)

The first category considered is the sometimes irregular \textit{l} : \textit{d} correspondences between Seediq dialects. Generally, we would expect to see either all \textit{d} or all \textit{l}. However, in the examples below in Table \ref{tab:irr_ld}, there are inconsistent reflexes among the dialects. It is hard to decide which protophoneme I should assign to these correspondences. I will temporarily place [?] as an placeholder for this unknown value.

\begin{table}[!htbp]
\centering
\caption{Examples of \textit{l} : \textit{d}}
\label{tab:irr_ld}
\begin{tabular}{lllll}
\hline
\acs{psed} & *ba[?]aw        & *ə[?]uk    & *[?]ayat             & *kənəda[?]ax    \\ \hdashline
\acs{tg}   & bado            & əluk       & dayac                & kunudalax     \\
\acs{cto}  & balaw           & əluk       & dayac                & kənədadax     \\
\acs{ctr}  & balaw           & əduk       & layac                & kənədadax     \\
\acs{etr}  & balaw           & əduk       & layat                & kənədadax     \\ \hline
Gloss      & `a thorny vine' & `to close' & `Sambucus formosana' & `from; since' \\ \hline
\end{tabular}
\end{table}

Reconstructing this set of correspondences to the proto-level is actually quite challenging. The reflexes as \textit{d} or \textit{l} are distributed across different dialects within each lexical item, as shown in Table \ref{tab:irr_ld} for `to close' and `Sambucus formosana', where the distributions are completely opposite. Therefore, this phenomenon is unlikely to originate from the same protophoneme, suggesting other possibilities.

In fact, a similar ``split'' can be seen in the reflexes of \acl{pan} *N in \acl{psed}. For example, \acs{pan} *paNid > \acs{psed} *palit `wing' and \acs{pan} *siNaR > \acs{psed} *hidaw `sun' show this difference.

\textcite{dyen1987d5} were likely the first scholars to notice this phenomenon. They suggested adding a new \acs{pan} symbol, *D\xb{5}, to distinguish between two sets of reflexes in Seediq: \acs{pan} *N > Seediq \textit{l} and \acs{pan} *D\xb{5} > Seediq \textit{d}\footnote{See \textcite{tsuchida1976tsouic} for the detailed discussion of \acs{pan} *D\xb{1-4}. }. Among all Formosan languages, only Seediq has maintained this distinction, while in others, it has merged with *N.

\textcite{song2024sedq} reevaluated the two reflexes of \acl{pan} *q in \acl{psed}, which are *q and ∅. This study concluded that ∅ is actually the regular reflex in Seediq, while *q /q/ was borrowed from Atayal or its early stages. Supporting evidence suggests that the instances where \acl{pan} *N is reflected as *d in \acl{psed} generally align with the pattern of \acs{pan} *q > \acs{psed} ∅, as seen in \acs{pan} *qaNup > \acs{psed} *aduk `to hunt (w/ dogs)', *qəNeb > *ə[d/l]uk? `to close (door)', *qiNaS > *idas `moon', etc. This viewpoint from \textcite{song2024sedq} indirectly suggests the possibility that Seediq /l/ might also have spread from Atayal. However, the distribution of \textit{l} and \textit{d} does not appear to be as clear as that of \textit{q} and ∅.

I believe that the additional \acs{pan} phoneme proposed by \textcite{dyen1987d5} is unnecessary because the phenomenon might be related to the stratification within Seediq based on discussion in \textcite{song2024sedq}. However, one point mentioned by \textcite{dyen1987d5} is worth considering: when there is an \textit{l} : \textit{d} correspondence between Seediq dialects, the \textit{l} group likely comes from Atayal. Therefore, when encountering such cases, I have a reason to reconstruct these words as \acl{psed} *d, with the reflex as \textit{l} possibly being a loanword from Atayal.

Some examples do not have a clear borrowing source and are often unique to Seediq, as seen in Table \ref{tab:irr_ld} with \textit{kunudalax} : \textit{kənədadax} `from; since', or the \textit{ladiŋ} : \textit{laliŋ} `fishhook' mentioned by \textcite[175]{dyen1987d5}. I speculate that /l/ might not be a native phoneme of Seediq, and that Seediq speakers learning this sound might produce phonetic errors, confusing it with the native /d/. However, this argument needs to be discussed and validated through further research in the future, and it remains a hypothesis for now. Alternatively, this might simply be cases of sporadic assimilation. For these examples, I will temporarily use *[d/l] as a possible reconstruction approach.

The revised reconstructions are shown in Table \ref{tab:irr_ld2}. 

\begin{table}[!htbp]
\centering
\caption{Examples of \textit{l} : \textit{d} with the reconstructions}
\label{tab:irr_ld2}
\begin{tabular}{lllll}
\hline
\acs{psed} & *ba[d/l]aw      & *ə[d/l]uk  & *[d/l]ayat           & *kənəda[d/l]ax    \\ \hdashline
\acs{tg}   & bado            & əluk       & dayac                & kunudalax     \\
\acs{cto}  & balaw           & əluk       & dayac                & kənədadax     \\
\acs{ctr}  & balaw           & əduk       & layac                & kənədadax     \\
\acs{etr}  & balaw           & əduk       & layat                & kənədadax     \\ \hline
Gloss      & `a thorny vine' & `to close' & `Sambucus formosana' & `from; since' \\ \hline
\end{tabular}
\end{table}

The second category is \textit{t} : \textit{l}, which occurs possibly when there is a \textit{b} sound within the word. In the available forms, \acl{tg} and \acl{cto} both show \textit{t}, \acl{etr} consistently shows \textit{l}, and \acl{cto} has instances of both. Based on external evidence from \acl{pan} and Atayal, it is indicated that these \textit{t} sounds likely originated from a historical *l. However, it is difficult to rule out that in Seediq, \textit{t} is the regular reflex in this specific environment, and the dialects reflecting \textit{l} may have borrowed from Atayal. Therefore, I tentatively list both t and l in \acl{psed} as *[t/l].

Cases in \acl{psed} with such special correspondences are: *bə[t/l]uku `flat basket; winnowing basket', *[t/l]ibu `pigsty', *[t/l]ubug `mouth harp; musical instruments', as shown in Table \ref{tab:irr_tl}.

\begin{table}[!htbp]
\centering
\caption{Examples of \textit{t} : \textit{l}}
\label{tab:irr_tl}
\begin{tabular}{llll}
\hline
\acs{pan}  & ---                             & *Nibu    & ---                               \\
\acs{pata} & *baluku                         & *libu    & *lubug                            \\ \hline
\ac{psed}  & *bə[t/l]uku                     & *[t/l]ibu & *[t/l]ubug                        \\ \hdashline
\ac{tg}    & butuku                          & tibu      & tubu                              \\
\ac{cto}   & bətuku                          & tibu      & tubu                              \\
\ac{ctr}   & bəluku                          & libu      & tubug                             \\
\ac{etr}   &                                 & libu      & lubug                             \\ \hline
Gloss      & `flat basket; winnowing basket' & `pigsty'  & `mouth harp; musical instruments' \\ \hline
\end{tabular}
\end{table}

The third group of examples found within Seediq, related to the \textit{c}/\textit{s} : \textit{l} correspondence, includes only two cases: (1) \textit{siŋas} : \textit{liŋas} `food between teeth' and (2) biciq : bilaq `small'\footnote{The sequence \textit{-aq} might come from **-iq sporadically. See Section \ref{sec:weirdV} for details.}. Based on the irregular reflections of \acl{pan} *C and *S in the Atayalic languages, I reconstruct these inconsistent correspondences to \acl{psed} as *c and *s, rather than *l. Relevant examples are shown in Table \ref{tab:irr_cls}.

\begin{table}[!htbp]
\centering
\caption{Examples of \textit{c}/\textit{s} : \textit{l}}
\label{tab:irr_cls}
\begin{tabular}{lll}
\hline
\acs{psed} & *siŋas               & *biciq  \\ \hline
\acs{tg}   & siŋas                & biciq   \\ 
\acs{cto}  & liŋas                &         \\
\acs{ctr}  & siŋas                &         \\
\acs{etr}  & liŋas                & bilaq   \\ \hline
Gloss      & `food between teeth' & `small' \\ \hline
\end{tabular}
\end{table}

I compared the reflexes of \acl{pan} *C and *S in Seediq and Atayal, and found that there are sometimes unexpected reflexes as *l. Such reflexes occur occasionally in Seediq and sometimes in Atayal. For example, the reflexes of \acl{pan} *Cabu `to wrap', *CuSuR `to string', *dakiS `to climb', *RamiS `root', and *SipuR `to count' can be seen in Table \ref{tab:panclsaic}.

\begin{table}[!htbp]
\centering
\caption{Examples of irregular \acs{pan} *C/*S > Atayalic /l/}
\label{tab:panclsaic}
\begin{tabular}{llllll}
\hline
\acs{pan}  & *Cabu     & *CuSuR      & *dakiS     & *RamiS & *SipuR    \\ \hline
\acs{psed} & *\textbf{l}abu     & *\textbf{l}ihug      & *daki\textbf{l}     & *gami\textbf{l} & *səpug    \\
\acs{pata} & *cabu     & *\textbf{l}Vhug      & *rakiyas   & *gami\textbf{l} & *\textbf{l}əpug    \\ \hline
Gloss      & `to wrap' & `to string' & `to climb' & `root' & `to count' \\ \hline
\end{tabular}
\end{table}

This phenomenon can also be observed in Atayal (\cite[173]{goderich2020phd}). Based on external evidence from \acl{pan}, I choose to reconstruct these unique correspondences as \acl{psed} *c or *s. The change to /l/ is likely a later development, occurring multiple times (\acs{pan}/\acs{paic} > \acs{psed}, and \acs{psed} > Seediq dialects). With the current evidence, it is difficult to ascertain why Atayalic languages exhibit such consonant substitutions. However, the direction of *c/*s > \textit{l} is quite clear.

The last group of inconsistent correspondences is \textit{g} (or \ac{cto}, \ac{eto} \textit{w}) : \textit{r}. I name this as the ``GR problem'' of Seediq. Sometimes, the sounds \textit{g} and \textit{r} randomly distributed in specific words and the process seems to be dialect-independent, as given in Table \ref{tab:irr_gr}. However, there is no environment of these correspondences that can be generalized. 

\begin{table}[!htbp]
\centering
\caption{Examples of \textit{g} (or \textit{w}) : \textit{r}}
\label{tab:irr_gr}
\begin{tabular}{lllll}
\hline
\acs{psed}& *gəhak   & *təmaga & *hagat   & *məgipuh \\ \hdashline
\acs{tg}  & rehak    & tumara  & harac    & mugipuh  \\
\acs{cto} & rəhak    & təmawa  & hawac    & məripuh  \\
\acs{ctr} & gəhak    & təmaga  & harac    & məripuh  \\
\acs{etr} & gəhak    & təmaga  & hagat    & məripuh  \\ \hline
Gloss     & `seed'            & `to wait'        & `stone fence'     & `fragile; soft'   \\ \hline \hline
\acs{psed}& *gəciluŋ & *gupun  & *gəbəyuk & *gəmimax \\ \hdashline
\acs{tg}  & ruciluŋ  & rupun   & rubeyuk  & gumimax  \\
\acs{cto} & wuciluŋ  & rupun   & rəbiyuk  & mərimax  \\
\acs{ctr} & rəsiluŋ  & gupun   & ---               & ---               \\
\acs{etr} & gəsiluŋ  & gupun   & gəbiyuk  & gəmimax  \\ \hline
Gloss     & `sea; lake'       & `teeth'          & `canyon'          & `to mix'         \\ \hline
\end{tabular}
\end{table}

\textcite{li1981paic} mentioned that \acl{paic} *g is reflected as \textit{r} in Seediq  when adjacent to the vowel \textit{a}. However, \textcite{song2024Aicg} pointed out that such irregular sound changes had already occurred once in the evolution from \acl{pan} or \acl{paic} to \acl{psed} at the word-initial, word-medial and word-final positions, not limited to the environments found by \textcite{li1981paic}. The aforementioned unconditioned split remains difficult to describe systematically until now. Nevertheless, \textcite{song2024Aicg} used the reflexes of \acl{paic} or Pre-\acl{psed} antepenultimate syllable (and also in the word-initial position) (*)*w- in \acl{psed} and Seediq dialects as evidence, explaining that \textit{g} > \textit{r}, though rare in the sound changes of world languages, is a more likely direction to appear in Seediq. Relevant examples are shown in Table \ref{tab:song2024wgr}.

\begin{table}[!htbp]
\centering
\caption{\acs{pata} antepenultimate *w- : Seediq \textit{g}/\textit{r} (from \cite[13]{song2024Aicg})}
\label{tab:song2024wgr}
\begin{tabular}{lllllll}
\hline
\acs{paic}    & \acs{psed}                & \acs{tg}             & \acs{to}             & \acs{ctr}            & \acs{etr}            & Gloss  \\ \hline
*\textbf{w}aciluŋ & *\textbf{[g/r]}əciluŋ & \textbf{r}uciluŋ & uciluŋ           & \textbf{g}əsiluŋ & \textbf{g}əsiluŋ & `sea'  \\
*\textbf{w}aqanux & *\textbf{r}əqənux    & \textbf{r}uqenux & \textbf{r}əqənux & \textbf{r}əqənux  & \textbf{r}əqənux     & `deer' \\ \hline
\end{tabular}
\end{table}

\textcite[13]{song2024Aicg} argues that the probability of **w directly becoming \textit{r} is relatively low, and there must have been an intermediate stage involving **g. So, we have sufficiently compelling evidence to suggest that \textit{r} may indeed from \textit{g}. Moreover, some instances of \acl{pata} *g > Atayal dialects \textit{r} or \textit{ɹ} may also suggest a similar sporadic sound change pattern in the Atayalic family. 

Therefore, based on the findings of \textcite{song2024Aicg}, I reconstruct the inconsistent \textit{g} : \textit{r} correspondences in Seediq dialects as \acl{psed} *g. This approach is a compromise, but it is the more probable choice given the current evidence.

\subsubsection{Inconsistent vocalic sound correspondences} \label{sec:weirdV}

Among modern dialects of Seediq, there are sometimes irregular sound correspondences. Although these correspondences are unusual, some patterns can be identified. Therefore, these irregular correspondences may arise from analogy or sporadic changes in certain environments.

The first group of irregular correspondences involves the \acl{tg} \textit{i} corresponding to \textit{a} in other dialects. This includes the following examples: (1) \textit{matis} : \textit{matas} `to write', (2) \textit{madis} : \textit{madas} `to bring', (3) \textit{gumabin} : \textit{wumabal}/\textit{gəmabal} `to pull', as shown in Table \ref{tab:irr_i_a}.

\begin{table}[!htbp]
\centering
\caption{Examples of sporadic Tgdaya i : other a}
\label{tab:irr_i_a}
\begin{tabular}{llll}
\hline
Env.      & /[cor]\_[cor]   & /[cor]\_[cor] & /\_[cor]  \\ \hline
\ac{pan}  & *pataS `tattoo' & *adaS         &           \\ \hdashline
\ac{psed} & *matas          & *madas        & *gəmabal  \\ \hdashline
\ac{tg}   & matis           & madis         & gumabin   \\
\ac{cto}  & matas           & madas         & wumabal   \\
\ac{etr}  & matas           & madas         & gəmabal   \\ \hline
Gloss     & `to write'      & `to bring'    & `to pull' \\ \hline
\end{tabular}
\end{table}

The environment for such a sound change is clear. The first two examples indicate that *a may more easily change to \textit{i} in \acl{tg} between two coronal consonants. Similarly, a similar change occurs when only the coda is a coronal consonant, as in *gəmabal > \textit{gumabin}. Interestingly, a sporadic sound change has occurred once before with \ac{pan} *dapaN > \ac{psed} *dapil `sole (foot), footprint' between a bilabial stop and a coronal coda.

This sound change likely results from the influence of coronal features on the vowel, producing sporadic sound changes. Because there is external evidence, we can confidently reconstruct these correspondences to \acl{psed} *a.

In the other set of cases, the correspondence is reversed, with \acl{tg} (and sometimes \acl{cto}) \textit{a} : others \textit{i}. This correspondence occurs when the target vowel is surrounded by two uvular or pharyngeal consonants. Additionally, this may also occur as an \textit{i} \~{} \textit{a} variation within the same dialect before fully changing to the sound \textit{a}. For example: (1) \textit{naqah} : \textit{naqih} `bad', (2) \textit{luqah} : \textit{luqih} `wound', (3) \textit{nuqah} : \textit{nuqih} `hemp fiber', (4) \textit{rumehaq} : \textit{rəməhiq} `to peel bark', etc., as detailed in Table \ref{tab:irr_a_i}.

\begin{table}[!htbp]
\centering
\caption{Examples of sporadic \textit{a} : \textit{i}}
\label{tab:irr_a_i}
\begin{tabular}{lllll}
\hline
Env.      & /q\_h  & /q\_h   & /q\_h            & /h\_q          \\ \hline
\ac{psed} & *naqih & *luqih  & *nuqih           & *rəməhiq       \\ \hdashline
\ac{tg}   & naqah  & luqah   & nuqah            & rumehaq        \\
\ac{cto}  & naqah  & luqah   & nuqih \~{} nuqah & rəməhiq        \\
\ac{etr}  & naqih  & luqih   & nuqih            & rəməhiq        \\ \hline
Gloss     & `bad'  & `wound' & `hemp fiber'     & `to peel bark' \\ \hline
\end{tabular}
\end{table}

These sporadic sound changes are likely due to the vowel lowering effect caused by uvular and pharyngeal consonants. Generally, the vowel lowering forms an allophone of /i/, and is accompanied by a transitional vowel, such as /seediq/ > [seediᵉq] \~ [seediᵃq] `person'. The direct shift to a single [a] is relatively rare and is likely caused when both preceding and following consonants are either uvular or pharyngeal.

\section{Proto-Seediq phonotactics} \label{sec:psed_phonotactics}

Table \ref{tab:psed_syl_type} presents all syllable types in \acl{psed}, including V, CV, VC, VG, CVC, and CVG. Among them, VG and CVG are relatively special in nature, and these two categories are separately listed from VC and CVC.

\begin{table}[!htbp]
\centering
\caption{\acl{psed} syllable types}
\label{tab:psed_syl_type}
\begin{tabular}{lllllll}
\hline
Syl. type     & \multicolumn{2}{l}{Prepenultimate} & \multicolumn{2}{l}{Penultimate} & \multicolumn{2}{l}{Final}        \\ \hline
V             & ---               &                & *\textbf{i}.ma        & `who'            & *sə.kə.lu.\textbf{i} & `startled'         \\
VC            & ---               &                & ---          &                  & *sa.\textbf{un}      & `to go \acs{pv}'   \\
CV            & *\textbf{qə}.si.ya         & `water'        & *qə.\textbf{si}.ya    & `water'          & *qə.si.\textbf{ya}   & `water'            \\
CG            & ---               &                & *\textbf{aw}.raw      & `k.o. bamboo'    & *də.mu.\textbf{uy}   & `to hold \acs{av}' \\
CVC           & ---               &                & ---          &                  & *ba.\textbf{raq}     & `lungs'            \\
CVG           & ---               &                & *\textbf{daw}.riq     & `eye'            & *ru.\textbf{ŋay}     & `monkey'           \\ \hline
\end{tabular}
\end{table}

The only syllable type that can occur in prepenultimate syllable(s) is CV, more precisely, Cə, as in *qə from *qə.si.ya `water' in Table \ref{tab:psed_syl_type}. A standalone V in that position is either deleted or undergoes vowel coalescence (which is to some extent a deletion), hence it does not appear. 

In the penultimate syllable, V, CV, VG, and CVG are all allowed to occur, as seen in the examples from the table: *i.ma `who' where *i appears, *si from *qə.si.ya `water', *aw in *aw.raw `k.o. bamboo', and *daw in *daw.riq `eye'. Of these, the *aw in *aw.raw `k.o. bamboo' is particularly noteworthy as it is the only case of VG found in the penultimate position.

As for the restrictions in the final syllable, there are the least limitations, with examples for all six syllable types. For instance, *i in *sə.kə.lu.i `startled', *un in *sa.un `to go \acs{pv}', *ya in *qə.si.ya `water', *uy in *də.mu.uy `to hold \acs{av}', *raq in *ba.raq `lungs', and *ŋay in *ru.ŋay `monkey'. 

In modern Seediq dialects, there are several segments that are not allowed to occur at the word-final position across all dialects. Since the performance of all dialects is consistent, I reconstruct these restrictions back to the level of \acl{psed}. The sounds not permitted in the word-final position in \acl{psed} include the voiced stops *b, *d, the affricate *c, the labials *p, (*b,) *m, and a specific VC sequence *-ag. The historical schwa *ə also cannot occur in the final syllable.

These phonological constraints further lead to the occurrence of many morphophonological alternations. I will provide detailed explanations about these alternations in Section \ref{sec:psed_alter}. It is worth noting that the absence of *-c did not cause any alternation, unlike in Proto-Atayal (see \cite[60--61,138--139]{goderich2020phd}).

\section{Proto-Seediq morphophonological alternations} \label{sec:psed_alter}

As mentioned in Section \ref{sec:3.2}, many alternations observed in modern Seediq dialects can also be found in \acl{psed}. Additionally, with the reconstructions in Section \ref{sec:psed_CandV} and the discussion of \acl{psed} phonotactics in Section \ref{sec:psed_phonotactics}, it can be inferred that the alternations in \acl{psed} are caused by the constraints imposed by phonotactics. In this subsection, I will introduce the morphophonological alternations in \acl{psed} one by one.

\subsection{Vowel alternations in \acl{psed}}

\subsubsection{Prepenultimate vowel neutralizing (weakening) in \acl{psed}}

Since all modern Seediq dialects exhibit prepenultimate vowel neutralization, I also reconstruct it back to the level of \acl{psed}, with its phonetic value being *ə (please refer to Section \ref{sec:psedV} for the reconstruction of prepenultimate *ə).

Therefore, if originally in the bare form, vowels with phonemic distinctiveness (those located in penultimate and final syllables) may all shift to the prepenultimate position when the base is attached with suffixes of different lengths such as *-un `\textsc{pv}', *-an `\textsc{lv}', or *-ani `\textsc{cv.imp}'. This positional change results in the neutralization of these vowels and subsequently gives rise to alternations between suffixed and non-suffixed forms. Relative examples are shown in *c\textbf{i}kul \~{} *c\textbf{ə}kulun `to push', *b\textbf{ay}taq \~{} *b\textbf{ə}taqan `to stab', and *p\textbf{a}t\textbf{a}s \~{} *p\textbf{ə}t\textbf{ə}sani `to write' in Table \ref{tab:psed_V2ə}. 

\begin{table}[!htbp]
\centering
\caption{*V\~{}*ə alternation in \acl{psed}}
\label{tab:psed_V2ə}
\begin{tabular}{llll}
\hline
Alternation                   & non-suffixed form & suffixed form & Gloss     \\ \hline
\multirow{3}{*}{*V(G)\~{ }*ə} & *c\textbf{i}kul & *c\textbf{ə}kulun     & `to push' \\
                              & *b\textbf{ay}taq & *b\textbf{ə}taqan     & `to stab'   \\ 
                              & *p\textbf{a}t\textbf{a}s & *p\textbf{ə}t\textbf{ə}sani & `to write' \\ \hline
\end{tabular}
\end{table}

\subsubsection{*u\~{}*ə alternation in \acl{psed}}

The restriction against the occurrence of schwa /ə/ in the final syllable is a shared feature among all modern dialects of Seediq. Such a restriction can be reconstructed at the \acl{psed} level. This phenomenon is also observed in Atayal, representing a common innovation in the Atayalic languages \parencite{goderich2020phd, li1981paic}. 

This restriction results in the underlying final-syllabic schwa *ə surface as *u. However, when adding monosyllabic suffixes like *-un `\textsc{pv}' or *-an `\textsc{lv}'\footnote{The voice affixes in \acl{psed} will be elaborated in Section \ref{sec:psed_voice_affixes}.}, the aforementioned condition no longer applies, meaning that the underlying *ə appears in the penultimate syllable and is realized as *ə on the surface. Therefore, the differential realization of *ə in the final and penultimate syllables gives rise to the alternation between *u and *ə, as shown in *həmaŋ\textbf{u}t \~{} *həŋ\textbf{ə}dun `to cook' and *kəmər\textbf{u}t \~{} *kər\textbf{ə}tun `to saw' in Table \ref{tab:psed_ə2u}.

\begin{table}[!htbp]
\centering
\caption{*u\~{}*ə alternation in \acl{psed}}
\label{tab:psed_ə2u}
\begin{tabular}{llll}
\hline
Alternation               & non-suffixed form    & suffixed form & Gloss     \\ \hline
\multirow{2}{*}{*u\~{ }*ə} & *həmaŋ\textbf{u}t & *həŋ\textbf{ə}dun     & `to cook' \\
                          & *kəmər\textbf{u}t & *kər\textbf{ə}tun     & `to saw'   \\ \hline
\end{tabular}
\end{table}

\subsubsection{Alternations related to vowel hiatus *-a.i- and *-a.u- in \acl{psed}}

In Section \ref{sec:sed_ai_au}, we discussed the resolution of hiatus in bare form structures like C\textit{a.i}C or C\textit{a.u}C when attached by monosyllabic suffixes, resulting in various patterns across different dialects such as C\textit{e}.C\textit{un}/C\textit{o}.C\textit{un}, C\textit{ey}.C\textit{un}/Cow.C\textit{un}, and C\textit{ay}.C\textit{un}/C\textit{aw}.C\textit{un}. In fact, following the reconstruction of diphthongs in Section \ref{sec:psedV}, it can be observed that the correspondences of \textit{e/o} : \textit{ey/ow} : \textit{ay/aw} in modern dialects are inherited from \acl{psed} *-ay- and *-aw-. Therefore, these alternations in modern dialects are indeed remnants from \acl{psed}. The examples of alternations between *-a.i-/*-a.u- and *-ay-/*-au- are shown in *n\textbf{ai}s \~{} *n\textbf{ay}sun `rapidly' and *t\textbf{au}s \~{} *t\textbf{aw}sun `to wave hand' in Table \ref{tab:psed_ai_au}.

\begin{table}[!htbp]
\centering
\caption{*u\~{}*ə alternation in \acl{psed}}
\label{tab:psed_ai_au}
\begin{tabular}{llll}
\hline
Alternation        & non-suffixed form    & suffixed form & Gloss     \\ \hline
*a.i \~{ }*ay      & *n\textbf{ai}s & *n\textbf{ay}sun     & `rapidly' \\
*a.u \~{ }*aw      & *t\textbf{au}s & *t\textbf{aw}sun     & `to wave hand'   \\ \hline
\end{tabular}
\end{table}

We also discussed a special case in Section \ref{sec:sed_ai_au}, the word `to sew', which can be reconstructed in \acl{psed} as the bare form *sais and the suffixed form *səisun. However, following the alternation pattern mentioned above, the expected suffixed form should be **saysun.

One possible explanation is that *sais in \acl{psed} is a loanword, and the formation of vowel hiatuses and diphthong alternation patterns predates the borrowing of *sais into Seediq, thus resulting in the suffixed forms of *sais not undergoing the same changes. \acl{psed} *sais could have been borrowed from Pazih, as Pazih has the exact same form, \textit{sais} `to sew' (adapted from \cite{liandtsuchida2001paz}).

\subsubsection{Vowel coalescence in \acl{psed}}

Vowel coalescence is a relatively minor change, occurring in environments as described in Section \ref{sec:V_coal}, where the stem must have the structure CVVCV(C). When such a stem is combined with a suffix, the original VV sequence in the antepenultimate and penultimate positions merges into a neutral vowel. Relevant examples are shown in Table \ref{tab:psed_VV2ə}, such as *pəl\textbf{əa}lay \~{} *pəl\textbf{ə}layun `to do first' and *k\textbf{əi}cug \~{} *k\textbf{ə}cugun `to fear'.

\begin{table}[!htbp]
\centering
\caption{*u\~{}*ə alternation in \acl{psed}}
\label{tab:psed_VV2ə}
\begin{tabular}{llll}
\hline
Alternation        & non-suffixed form    & suffixed form & Gloss     \\ \hline
\multirow{2}{*}{*V.V \~{ }*ə}  & *pəl\textbf{əa}lay & *pəl\textbf{ə}layun     & `to do first' \\ 
                               & *k\textbf{əi}cug   & *k\textbf{ə}cugun       & `to fear' \\ \hline
\end{tabular}
\end{table}

\subsubsection{Apharesis in \acl{psed}} %首音刪除

Apheresis is also evident in \acl{psed}. When a stem with the structure VCV(C) is combined with a suffix, the first V is deleted due to the lack of onset protection. Relevant examples are shown in *\textbf{i}mah \~{} *mahun `to drink' and *\textbf{a}pa \~{} *paun `to carry on back' in Table \ref{tab:psed_V20}. In the table, the deleted vowels are represented by an underscore “\_{}”.

\begin{table}[!htbp]
\centering
\caption{Apharesis in \acl{psed}}
\label{tab:psed_V20}
\begin{tabular}{llll}
\hline
Alternation                & non-suffixed form    & suffixed form & Gloss     \\ \hline
\multirow{2}{*}{*V\~{ }*∅} & *\textbf{i}mah & *\_mahun     & `to drink' \\
                           & *\textbf{a}pa & *\_paun     & `to carry on back'   \\ \hline
\end{tabular}
\end{table}

\subsection{Consonant alternations in \acl{psed}}

\subsubsection{Alternations between *-t and *-d- in \acl{psed}}  \label{sec:psed_d2t}

As mentioned in Section \ref{sec:psed_phonotactics}, \acl{psed} does not allow voiced stops to occur in word-final positions. Therefore, voiced stops like *b and *d undergo devoicing. Additionally, *b is subject to restrictions on word-final labials, which will be discussed later in Section \ref{sed:psed_lab_vel}. The alternation between *-t and *-d- is due to this final devoicing. Relevant examples can be found in *həmaŋu\textbf{t} \~{} *həŋə\textbf{d}un `to cook', *məhaga\textbf{t} \~{} *həga\textbf{d}un `to pile stones' in Table \ref{tab:psed_d2t}.

\begin{table}[!htbp]
\centering
\caption{Alternations between *-t and *-d- in \acl{psed}}
\label{tab:psed_d2t}
\begin{tabular}{llll}
\hline
Alternation                   & non-suffixed form    & suffixed form & Gloss     \\ \hline
\multirow{2}{*}{*-t\~{ }*-d-} & *həmaŋu\textbf{t} & *həŋə\textbf{d}un     & `to cook' \\
                              & *məhaga\textbf{t} & *həga\textbf{d}un     & `to pile stones'   \\ \hline
\end{tabular}
\end{table}

\subsubsection{Alternations between velar and bilabial consonants in \acl{psed}} \label{sed:psed_lab_vel}

Bilabial consonants including *p, *b, and *m do not occur in the word-final position in \acl{psed}. However, if we consider the suffixed forms in \acl{psed}, the alternation between bilabial velar and bilabial sounds can be seen, as shown *madu\textbf{k} \~{} *du\textbf{p}un `to hunt (by chasing)', *məlu\textbf{k} \~{} *lə\textbf{b}an `to close', and *migi\textbf{ŋ} \~{} *gi\textbf{m}an  `to look for' in Table \ref{tab:psed_lab_vel}.

\begin{table}[!htbp]
\centering
\caption{Velar and bilabial consonantal alternations in \acl{psed}}
\label{tab:psed_lab_vel}
\begin{tabular}{llll}
\hline
Alternation & non-suffixed form & suffixed form & Gloss                  \\ \hline
*-k\~{ }*-p-     & *madu\textbf{k}  & *du\textbf{p}un       & `to hunt (by chasing)' \\
*-k\~{ }*-b-     & *mədu\textbf{k}  & *də\textbf{b}an       & `to close'             \\
*-ŋ\~{ }*-m-     & *migi\textbf{ŋ}  & *gi\textbf{m}an       & `to look for'          \\ \hline
\end{tabular}
\end{table}

\subsubsection{Alternation between *-aw and *-ag- in \acl{psed}}

The *-ag sequence cannot occur in the word-final position in \acl{psed} due to the merger of **-ag and **-aw from pre-\acl{psed} to \acl{psed}  (see Section \ref{sec:psedC}, \ref{sec:psedV} for the reconstruction of *g and *aw). However, when the underlying *-ag- sequence is not in word-final position, i.e., after suffixation, it is not subject to the restriction of word-final position and is realized as *-ag- in the surface. Relevant examples are shown in *rəməŋ\textbf{aw} \~{} *rəŋ\textbf{ag}un `to speak' and  *dəmay\textbf{aw} \~{} *dəy\textbf{ag}un `to help' in Table \ref{tab:psed_agaw}.

\begin{table}[!htbp]
\centering
\caption{*-aw and *-ag- alternations in paradigm forms of \acl{psed}}
\label{tab:psed_agaw}
\begin{tabular}{llll}
\hline
Alternation                      & non-suffixed form  & suffixed form & Gloss                   \\ \hline
\multirow{2}{*}{*-aw\~{ }*-ag-}  & *rəməŋ\textbf{aw} & *rəŋ\textbf{ag}un     & `to speak'               \\
                                & *dəmay\textbf{aw} & *dəy\textbf{ag}un     & `to help' \\ \hline
\end{tabular}
\end{table}

\subsubsection{Alternations between *p/*b- and *m- in \acl{psed}}

Alternations between *p/*b- and *m- in \acl{psed} occur in environments identical to all modern dialects, as mentioned in Section \ref{sec:pbm_alter}. When the actor voice infix *<əm> is inserted to stems with structures such as *pVCV(C) or *bVCV(C), the dissimilation of two consecutive labial consonants results in the deletion of the initial syllable *pə- or *bə-. Relevant examples are shown in *\textbf{p}atas \~{} *\textbf{m}atas `to write' and *\textbf{b}ahu \~{} *\textbf{m}ahu `to wash clothes' in Table \ref{tab:psed_bp2m}.

\begin{table}[!htbp]
\centering
\caption{*p/*b- and *m- alternations in \acl{psed}}
\label{tab:psed_bp2m}
\begin{tabular}{llll}
\hline
Alternation  & non-infixed form  & infixed form & Gloss                   \\ \hline
*p-\~{ }*m-  & *\textbf{p}atas & *\textbf{m}atas   & `to write'               \\
*b-\~{ }*m-  & *\textbf{b}ahu  & *\textbf{m}ahu     & `to wash clothes' \\ \hline
\end{tabular}
\end{table}

\section{Summary}

In this chapter, I discussed the phonology of \acl{psed}. The reconstructed \acl{psed} phonemes are based on the Comparative Method. For some irregular correspondences, I have made temporary arrangements according to the specific circumstances of each group.

In addition to reconstructing the \acl{psed} phonemes, this chapter also discusses its phonotactics and the rich morphophonological alternations, further deepening the understanding of \acl{psed}'s characteristics based on the Seediq dialects.

Although the reflexes of \acl{psed} in the Seediq dialects under studied do not show significant differences, distinct reflexes can still be observed. These sound changes can serve as phonological evidence for subgrouping. The reflexes of \acl{psed} *g, *c, *ə in the dialects of Seediq are potentially crucial in the subgrouping process. I will combine the morphosyntactic evidence from Chapter \ref{ch5} and the lexical evidence from Chapter \ref{ch6} to perform internal subgrouping of Seediq in Chapter \ref{ch7} .