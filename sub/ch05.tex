\chapter{Proto-Seediq morphosyntax} \label{ch5}

This chapter will discuss the morphosyntactic system in \acl{psed}. Section \ref{sec:psed_case} talks about the case-marking system in \acl{psed}. Additionally, I will reconstruct the pronominal system of Seediq, including personal pronouns and demonstrative pronouns in Section \ref{sec:psed_pron}. The voice system (the so-called ``focus system'') affixes of \acl{psed}, as well as some derivational affixes are discussed in Section \ref{sec:psed_aff}. Lastly, there were some irregular verb conjugation in \acl{psed}, which is discussed in \ref{sec:psed_irr_v}.

\section{Proto-Seediq case-marking system} \label{sec:psed_case}
\lipsum[1-10]

\section{Proto-Seediq pronominal system} \label{sec:psed_pron}
\lipsum[1]

\subsection{Personal pronouns}

The personal pronoun system of \acl{psed} is shown in Table \ref{tab:psedpron}. Just like many of Austronesian languages, Seediq distinguishes the differences of first person plural inclusive and exclusive. 

For the clitic (bound) forms, there are two sets of pronouns, some of the forms are the same between two cases. The oblique free forms are fully preserved in \acl{cto}, \acl{ctr}, and \acl{etr}. However in the \acl{tg} dialect, there are only few remained, including \textit{kənan} `\textsc{1sg.obl}', \textit{sunan} `\textsc{2sg.obl}', and \textit{munan} `\textsc{1sg.obl}'. Note that \acl{psed} *munan is for second person plural, but \textit{munan} in \acl{tg} is for first person singular according to \textcite[62]{Sung2018Sedgrammar}. These oblique forms in \acl{tg} are not usually used in daily speeches. The set of original nominative free form replace their function; the original nominative forms thus became neutral in \acl{tg} (i.e. Nominative and oblique cases share the same forms).

\begin{table}[!htbp]
\centering
\caption{\acl{psed} personal pronouns}
\label{tab:psedpron}
\begin{tabular}{lllll}
\hline
         & \multicolumn{2}{c}{Clitic} & \multicolumn{2}{c}{Free} \\ \hline
         & Nominative    & Genitive   & Nominative & Oblique     \\ \hline
\textsc{1sg}      & *=ku          & *=mu       & *yaku      & *kənan      \\
\textsc{2sg}      & *=su          & *=su       & *isu       & *sunan      \\
\textsc{3sg}      & ∅             & *=na; *=niya & *hiya      & *həyaan     \\
\textsc{1pl.incl} & *=ta          & *=ta       & *ita       & *tənan      \\
\textsc{1pl.excl} & *=nami        & *=nami     & *yami      & *mənan      \\
\textsc{2pl}      & *=namu        & *=namu     & *yamu      & *munan      \\
\textsc{3pl}      & ∅             & *=dəha     & *dəhəya    & *dəhəyaan   \\ \hline
\end{tabular}
\end{table}

Furthermore, \acl{psed} also has three portmanteau pronouns, as shown in Table \ref{tab:por}.

\begin{table}[!htbp]
\centering
\caption{Portmanteau personal pronouns in \acl{psed}}
\label{tab:por}
\begin{tabular}{ll}
\hline
Form   & Gloss \\ 
\hline
*=misu & `\textsc{1sg.gen+2sg.nom}' \\
*=saku & `\textsc{2sg.gen+1sg.nom}' \\
*=maku & `\textsc{1sg.gen+2pl.nom}' \\
\hline
\end{tabular}
\end{table}

In general, clitic pronouns are placed after the predicate in the sentence, except for possessive structure. The possessive structure in Seediq can be described as ``possessee=possessor.'' For example, \textit{tama=mu} `father=\textsc{1sg.gen}'. Regarding the detailed usage of more Seediq clitic pronouns in different dialects, please refer to \textcite{ochiai2009sedpron,kuondtrvpronoun,holmer2014clitic}.

Next, I will examine the reflexes of personal pronouns in each dialect set by set.

\subsubsection{Nominative clitic}

The five different personal pronouns in the nominative clitic form are reflected consistently across all dialects, as shown in Table \ref{tab:nomclitic}. 

\begin{table}[!htbp]
\centering
\caption{Reflexes of nominative clitic personal pronouns in Seediq dialects}
\label{tab:nomclitic}
\begin{tabular}{llllll}
\hline
       & =\textsc{1sg.nom} & =\textsc{2sg.nom} & =\textsc{1pl.nom.incl} & =\textsc{1pl.nom.excl}    & =\textsc{2pl.nom} \\ \hline
\acs{psed} & *=ku    & *=su    & *=ta         & *=nami          & *=namu  \\
\acs{tg}  & =ku     & =su     & =ta          & =nami; (=miyan) & =namu   \\
\acs{cto}  & =ku     & =su     & =ta          & =nami           & =namu   \\
\acs{eto}  & =ku     & =su     & =ta          & =nami           & =namu   \\
\acs{ctr} & =ku     & =su     & =ta          & =nami           & =namu   \\
\acs{etr} & =ku     & =su     & =ta          & =nami           & =namu   \\ \hline
\end{tabular}
\end{table}

However, there is an additional form in \acl{tg}, which is =\textit{miyan} `\textsc{1pl.nom.incl}'. It is speculated that this form might be borrowed from Atayal, where the corresponding form is =\textit{myan} meaning `\textsc{1pl.gen.incl}'. Furthermore, in Seediq, the first person plural nominative and genitive clitics have the same form, suggesting a possible further expansion in this loanword's function.

\subsubsection{Genitive clitic}

The genitive clitic forms of the five personal pronouns are also reflected consistently across the various dialects. Table \ref{tab:genclitic} shows the reflexes. 

\begin{table}[!htbp]
\centering
\caption{Reflexes of genitive clitic personal pronouns in Seediq dialects}
\label{tab:genclitic}
\begin{tabular}{lllll}
\hline
      & =\textsc{1sg.gen}      & =\textsc{2sg.gen}        & =\textsc{3sg.gen} & =\textsc{3sg.gen}       \\ \hline
\acs{psed} & *=mu          & *=su            & *=na     & *=niya         \\
\acs{tg}  & =mu           & =su             & =na      & ---            \\
\acs{cto}  & =mu           & =su             & =na      & =niya          \\
\acs{eto}  & =mu           & =su             & =na      & =niya          \\
\acs{ctr} & =mu           & =su             & =na      & =niya          \\
\acs{etr} & =mu           & =su             & =na      & =niya          \\ \hline
      & =\textsc{1pl.gen.incl} & =\textsc{1pl.gen.excl}   & =\textsc{2pl.gen} & =\textsc{3pl.gen}       \\ \hline
\acs{psed} & *=ta          & *=nami          & *=namu   & *=dəha         \\
\acs{tg}  & =ta           & =nami; (=miyan) & =namu    & =daha          \\
\acs{cto}  & =ta           & =nami           & =namu    & =dəha          \\
\acs{eto}  & =ta           & =nami           & =namu    & =nəha          \\
\acs{ctr} & =ta           & =nami           & =namu    & =dəha          \\
\acs{etr} & =ta           & =nami           & =namu    & =dəha     \\ \hline
\end{tabular}
\end{table}

In addition to the previously mentioned borrowing from Atayal in \acl{tg} with the form \textit{miyan}, there are two other notable points. First, `\textsc{3sg.gen}' seems to have two forms in \acl{psed} stage, one short form *=na and one long form *=niya, although the latter is lost in \acl{tg}. Another point is that according to \textcite{lee2015tawsa}, \acl{etr} has =\textit{nəha} meaning `\textsc{3pl.gen}', but I has not found in other dialects, suggesting it might be borrowed from Atayal as well, where the form is \textit{nəhaʔ} `\textsc{3pl.gen}'.

\subsubsection{Nominative free}

Similarly, the nominative free forms show no significant variation or innovation across dialects. Table \ref{tab:nomfree} presents the forms in each dialect.

\begin{table}[!htbp]
\centering
\caption{Reflexes of nominative free personal pronouns in Seediq dialects}
\label{tab:nomfree}
\begin{tabular}{lllll}
\hline
      & \textsc{1sg.nom}      & \textsc{2sg.nom}      & \textsc{3sg.nom} &         \\ \hline
\acs{psed} & *yaku        & *isu         & *hiya   &         \\
\acs{tg}  & yaku         & isu          & heya    &         \\
\acs{cto}  & yaku         & isu          & hiya    &         \\
\acs{eto}  & yaku         & isu          & hiya    &         \\
\acs{ctr} & yaku         & isu          & hiya    &         \\
\acs{etr} & yaku         & isu          & hiya    &         \\ \hline
      & \textsc{1pl.nom.incl} & \textsc{1pl.nom.excl} & \textsc{2pl.nom} & \textsc{3pl.nom} \\ \hline
\acs{psed} & *ita         & *yami        & *yamu   & *dəhəya \\
\acs{tg}  & ita          & yami         & yamu    & deheya  \\
\acs{cto}  & ita          & yami         & yamu    & dəhiya  \\
\acs{eto}  & ita          & yami         & yamu    & dəhiya  \\
\acs{ctr} & ita          & yami         & yamu    & dəhiya  \\
\acs{etr} & ita          & yami         & yamu    & dəhiya  \\ \hline
\end{tabular}
\end{table}

However, it may be worth noting the origins of the \textsc{3sg.nom} and \textsc{3pl.nom} forms. Although they may appear similar, when reconstructing them, we need to consider external evidence and their etymology.

The \textsc{3sg.nom} form is derived from \acs{pan} *si ia, with its corresponding Atayal form being \textit{hiyaʔ}. As for the \textsc{3pl.nom} form *dəhəya, it may not have a clear \acs{pan} source, but it corresponds to Atayal forms such as \textit{ləhaʔ} and \textit{ləhəgaʔ}. The form \textit{ləhaʔ} in Atayal may have originated from \acl{paic} *lVhəjaʔ > **ləhəgaʔ > **ləhaʔaʔ > \textit{ləhaʔ}, as the deletion of /g/ would lead to further vowel coalescence. For discussions on /g/ and /ʔ/ in Atayal, please refer to \textcite{goderich2020phd,song2023Aicgprime}. 

\subsubsection{Oblique free}

Unlike the previous sets, the oblique forms in \acl{tg} are nearly obsolete and are mostly replaced by the nominative forms, as mentioned earlier. The remaining forms, \textit{kenan}, \textit{sunan}, \textit{munan}, are only used in specific situations and vary from person to person (\cite{Sung2018Sedgrammar}).

\begin{table}[!htbp]
\centering
\caption{Reflexes of oblique free personal pronouns in Seediq dialects}
\label{tab:oblfree}
\begin{tabular}{lllll}
\hline
      & \textsc{1sg.obl}      & \textsc{2sg.obl}      & \textsc{3sg.obl} &           \\ \hline
\acs{psed} & *kənan       & *sunan       & *həyaan &           \\
\acs{tg}  & (kenan)      & (sunan)      & ---     &           \\
\acs{cto}  & kənan        & sunan        & həyaan  &           \\
\acs{ctr} & kənan        & sunan        & həyaan  &           \\
\acs{etr} & kənan        & sunan        & həyaan  &           \\ \hline
      & \textsc{1pl.obl.incl} & \textsc{1pl.obl.excl} & \textsc{2pl.obl} & \textsc{3pl.obl}   \\ \hline
\acs{psed} & *tənan       & *mənan       & *munan  & *dəhəyaan \\
\acs{tg}  & ---          & ---          & (munan) & ---       \\
\acs{cto}  & tənan        & mənan        & munan   & dəhəyaan  \\
\acs{ctr} & tənan        & mənan        & munan   & dəhəyaan  \\
\acs{etr} & tənan        & mənan        & munan   & dəhəyaan  \\ \hline
\end{tabular}
\end{table}

The oblique forms in \acl{psed} are believed to have inherited from the reconstructed Acc(usative) set of personal pronouns proposed by \textcite{ross2006casepronoun} (Except for the third person forms), with the addition of the suffix *-an. A comparison with Paiwan and Ross's \acs{pan} Acc personal pronouns' reconstruction is shown in Table \ref{tab:panacc}

\begin{table}[!htbp]
\centering
\caption{A comparison of \acs{pan} \textsc{acc}, Paiwan \textsc{nom} (clitic), and \acl{psed} \textsc{obl} personal pronouns}
\label{tab:panacc}
\begin{tabular}{lllllll}
\hline
       &     & \textsc{1sg}      & \textsc{2sg}        & \textsc{1pl.incl} & \textsc{1pl.incl} & \textsc{2pl}         \\\hline
\acs{pan}   & \textsc{acc} & *i-ak-ən & *iSu[qu]-n & *ita-ən  & *i-am-ən & *i-mu[qu]-n \\
Paiwan & \textsc{nom} & =[a]kən  & =[ə]sun    & =[i]cən  & =[a]mən  & =[ə]mun     \\
\acl{psed} & \textsc{obl} & *kən-an  & *sun-an    & *tən-an  & *mən-an  & *mun-an    \\ \hline
\end{tabular}
\end{table}

\subsubsection{Portmanteau forms}

Data from \acl{tg} (from \cite[62]{Sung2018Sedgrammar}) and \acl{etr} (from \cite[74]{Lee2018Trugrammar}) show three different portmanteau personal pronouns in Seediq, as listed in Table \ref{tab:porref}.

\begin{table}[!htbp]
\centering
\caption{Reflexes of portmanteau personal pronouns in Seediq dialects}
\label{tab:porref}
\begin{tabular}{llll}
\hline
                     & \textsc{1sg.gen+2sg.nom}                 & \textsc{2sg.gen+1sg.nom}                  & \textsc{1sg.gen+2pl.nom}                \\ 
\multicolumn{1}{c}{} & \multicolumn{1}{c}{(I to thee)} & \multicolumn{1}{c}{(Thou to me)} & \multicolumn{1}{c}{(I to you)} \\ \hline
\acl{psed}               & *=misu                          & *=saku                           & *=maku                         \\
\acs{tg}                 & =misu                           & =saku                            & =maku                          \\
\acs{etr}                & =misu                           & =saku                            & =maku          \\ \hline                
\end{tabular}
\end{table}

Portmanteau pronouns, like other clitic pronouns, are enclitics placed after the predicate. They consist of two arguments and are found exclusively in (a) nominal predicate; and (b) non-agent voice (NAV) sentences, where they represent the agent or experiencer. Here are two examples (\ref{ex:Pron_nominalP}) and (\ref{ex:Pron_NAV}) from \textcite[74--75]{Lee2018Trugrammar}.

\begin{exe}

    \ex Nominal predicate sentence \label{ex:Pron_nominalP}
    \gll anay=misu ka isu. \\
    sister's.husband=\textsc{1sg.gen+2sg.nom} \textsc{top} \textsc{2sg.nom}\\
    \glt `You are my sister's husband. (For me, you are my sister's husband.)'
    \ex NAV sentence \label{ex:Pron_NAV}
    \gll wada=misu qta-an da! \\
    \textsc{perf}=\textsc{1sg.gen+2sg.nom} see-\textsc{lv} \textsc{ptcl}\\
    \glt `I have already seen you!'
\end{exe}



\subsection{Demonstratives and deixis}

\acl{dei}, \acl{dem}

\begin{table}[!htbp]
\centering
\caption{b}
\label{tab:psed_dem_dei}
\begin{tabular}{lll}
\hline
                & Demonstrative & Spacial deixis \\ \hline
Near            & *nii          & *hini          \\
Far (visible)   & *gaga/*ga     & *hiya          \\
Far (invisible) & *gayi         & *gayi?         \\ \hline
\end{tabular}
\end{table}

\begin{table}[!htbp]
\centering
\caption{a}
\label{tab:sed_dia_dem_dei}
\begin{tabular}{llll}
\hline
           & \acs{dem}: near & \acs{dem}: far (visible) & \acs{dem}: far (invisible) \\ \hline
\acs{psed} & *nii            & *gaga/*ga                & *gayi                      \\
\acs{tg}   & nii             & gaga/ga                  &                            \\
\acs{cto}  & nii             & wawa/wa                  & wayi                       \\
\acs{eto}  & nii             & wawa/wa                  & wayi                       \\
\acs{ctr}  & nii             & gaga/ga                  &                            \\
\acs{etr}  & nii             & gaga/ga                  & gayi                       \\ \hline\hline
           & \acs{dei}: near & \acs{dei}: far (visible) & \acs{dei}: far (invisible) \\ \hline
\acs{psed} & *hini           & *hiya                    & *gayi?                     \\
\acs{tg}   & hini            & hiya/(hii)               &                            \\
\acs{cto}  & hini            & hiya/(ha)                & wayi                       \\
\acs{eto}  & hini            & hiya                     & (wowi)                     \\
\acs{ctr}  & hini            & hiya                     &                            \\
\acs{etr}  & hini            & hiya                     & gayi                       \\ \hline
\end{tabular}
\end{table}

\lipsum[1-10]

\section{Proto-Seediq affixes} \label{sec:psed_aff}

Seediq is generally described as an agglutinative language (\cite{tsukida2012}), but \textcite{holmer1996parametric} does not consider Seediq to be a purely agglutinative language. Despite differing opinions, it is clear that Seediq expresses different verbal voices by affixation; and processes such as causativization, nominalization, and verbalization are also predominantly achieved through adding affixes. This section will explore \acl{psed} inflectional affixes (Section \ref{sec:psed_voice_affixes}) and provide an initial examination of derivational affixes (Section \ref{sec:psed_deri_affixes}). The data in this section primarily comes from the \acl{tg} and \acl{etr} dialects.

\subsection{Proto-Seediq voice system (inflectional) affixes} \label{sec:psed_voice_affixes}

The voice system in Seediq is traditionally referred to as the ``focus system'', relating to its morphosyntactic alignment. Although the term ``focus system'' might be generally considered outdated, it is still frequently used in the literature on Formosan languages especially in Taiwan, particularly as ``焦點系統'' in Chinese. The ``focus'' in Austronesian languages does not correspond to the syntactic focus commonly discussed in general linguistic sense, but refers to the main noun phrase in a sentence that agrees with the verb voice. However, to avoid confusion, this thesis employs the term ``voice''.

This thesis principally follows the classification by \textcite{Sung2018Sedgrammar,Lee2018Trugrammar}, categorizing Seediq voices into Agent/Actor Voice (AV), Patient Voice (PV), Locative Voice (LV), and Circumstantial Voice (CV)\footnote{I combined the Instrumental(工具)/Benefactive(受惠)/Referential(參考) Focuses/Voices as a single Circumstantial Voice since they all share the same verbal marking affixes.}. PV, LV, and CV can further be combined into Undergoer Voice (UV) or non-Actor Voice (NAV), with three distinct subcategories: UV-P, UV-L, and UV-C. Here, I adopt the simplest four-way notation (AV, PV, LV, and CV) to distinguish between these voices.

\acl{psed} can generally be divided into three groups of voice affixes, as shown in Table \ref{tab:psed_voice}. I will first discuss the forms and the basic usages of the voice affixes, followed by the interaction between voice and aspect/mood.

\begin{table}[!htbp]
\centering
\caption{Proto-Seediq voice affixes}
\label{tab:psed_voice}
\begin{tabular}{llll}
\hline
   & I           & II                   & III                        \\ \hline
AV & *mə-; *<əm> & ∅                    & *-a                        \\
PV & *-un        & \multirow{2}{*}{*-i} & \multirow{2}{*}{*-aw/*-ay} \\
LV & *-an        &                      &                            \\
CV & *sə-        & *-ani                & *-anay                     \\ \hline
\end{tabular}
\end{table}



\begin{table}[!htbp]
\centering
\caption{Interactions of Proto-Seediq voices and aspects/moods}
\label{tab:psed_voice_tam}
\begin{tabular}{lllll}
\hline
                  & AV                    & PV            & LV             & CV      \\ \hline
Realis            & *mə-√; *<əm>√         & *√-un         & *√-an          & *sə-√   \\
Realis perfective & *m<ən>ə-√; *<əm><ən>√ & *<ən>√        & *(<ən>)√-an    & ?       \\
Irrealis          & *mə-pə-√; *mə-√       & *(Cə\~{})√-un & *Cə\~{}√-an    & ?       \\
Subjuctive        & *√                    & \multicolumn{2}{c}{*√-i}       & *√-ani  \\
Hortative         & *√-a                  & \multicolumn{2}{c}{*√-aw/*-ay} & *√-anay \\ \hline
\end{tabular}
\end{table}



\subsection{Some Proto-Seediq derivational affixes} \label{sec:psed_deri_affixes}
\lipsum[1-10]

\section{Irregular verb conjugation in Proto-Seediq} \label{sec:psed_irr_v}
% 有空再寫



\section{Summary}
\lipsum[1-4]