\chapter{Proto-Seediq morphosyntax} \label{ch5}

This chapter discuss the morphosyntactic system in \acl{psed}. Section \ref{sec:psed_case} talks about the case-marking system in \acl{psed}. Additionally, I reconstructs the personal pronominal system of Seediq in Section \ref{sec:psed_pron}. The voice system (the so-called ``focus system'') affixes of \acl{psed}, as well as some derivational affixes are discussed in Section \ref{sec:psed_aff}. Lastly, there were some irregular verb conjugation in \acl{psed}, which is discussed in Section \ref{sec:psed_irr_v}.

\section{Proto-Seediq case-marking system} \label{sec:psed_case}

This section discusses the reconstruction of \acl{psed} case markers. Table \ref{tab:psed_case} lists the reconstructed case markers for \acl{psed} and their reflexes in \acl{tg}, \acl{cto}, and \acl{etr}. The case marking system of \acl{psed} is quite simple, consisting only of *ka `\acs{nom}' and *na `\acs{gen}'.

\begin{table}[!htbp]
\centering
\caption{Proto-Seediq case markers and their reflexes in \acl{tg}, \acl{cto}, and \acl{etr}}
\label{tab:psed_case}
\begin{tabular}{lll}
\hline
           & \ac{nom} & \ac{gen} \\ \hline
\acs{psed} & *ka      & *na      \\
\ac{tg}    & ka       & na       \\
\ac{cto}   & ka       & ∅        \\
\ac{etr}   & ka       & ∅        \\ \hline
\end{tabular}
\end{table}

According to \textcite{Sung2018Sedgrammar}, the case markers in \acl{tg} follow a two-way distinction system, with the nominative marker \textit{ka} and the genitive marker \textit{na}. However, in other dialects, such as \acl{etr}, the genitive marker no longer has a phonological form (\cite{Lee2018Trugrammar}). Dialects other than \acl{tg} retain only the nominative marker \textit{ka}. The loss of the genitive marker *na can be considered a shared innovation among these dialects. 

Compare the two sentences with the same meaning in \acl{tg} and \acl{etr} in (\ref{ex:uka_na_etr}). In (1a), Tgdaya marks the genitive noun phrase \textit{huliŋ=su} `your dog' by the genitive case marker \textit{na} (although it can be omitted in colloquial speeches), while in (1b), Eastern Truku does not require a genitive marker at all. 

\begin{exe}
\ex \label{ex:uka_na_etr} \textcite[69]{Lee2018Trugrammar}
    \begin{xlist}
    \ex \acl{tg} (with \textit{na})
    \gll wada quyut-un na huliŋ=su ka ŋiyo=mu \\
    \textsc{pfv} bite-\textsc{pv} \textsc{gen} dog=\textsc{2sg.gen} \textsc{nom} cat=\textsc{1sg.gen}\\
    \glt `Your dog bit my cat.'
    \ex \acl{etr} (without \textit{na})
    \gll wada qəyut-un huliŋ=su ka ŋiyaw=mu \\
    \textsc{pfv} bite-\textsc{pv} dog.\textsc{gen}=\textsc{2sg.gen} \textsc{nom} cat=\textsc{1sg.gen}\\
    \glt `Your dog bit my cat.'
    \end{xlist}
\end{exe}

Seediq case markers do not distinguish between noun categories; that is, the same marking system is used regardless of singular/plural, human/non-human, definite/indefinite, and so on. This phenomenon differs from its sister language Atayal (\cite{huang1995}) and may represent significant simplification during the development from \acl{paic} to \acl{psed}.

\section{Proto-Seediq personal pronominal system} \label{sec:psed_pron}

The personal pronoun system of \acl{psed} is shown in Table \ref{tab:psedpron}. Just like many of Austronesian languages, Seediq distinguishes the differences of first person plural inclusive and exclusive. 

For the clitic (bound) forms, there are two sets of pronouns, some of the forms are the same between two cases. The oblique free forms are fully preserved in \acl{cto}, \acl{ctr}, and \acl{etr}. However in the \acl{tg} dialect, there are only few remained, including \textit{kenan} `\textsc{1sg.obl}', \textit{sunan} `\textsc{2sg.obl}', and \textit{munan} `\textsc{2pl.obl}'. These oblique forms in \acl{tg} are not usually used in daily speeches. The set of original nominative free form replace their function; the original nominative forms thus became neutral in \acl{tg} (i.e. Nominative and oblique cases share the same forms).

\begin{table}[!htbp]
\centering
\caption{\acl{psed} personal pronouns}
\label{tab:psedpron}
\begin{tabular}{lllll}
\hline
         & \multicolumn{2}{c}{Clitic} & \multicolumn{2}{c}{Free} \\ \hline
         & Nominative    & Genitive   & Nominative & Oblique     \\ \hline
\textsc{1sg}      & *=ku          & *=mu       & *yaku      & *kənan      \\
\textsc{2sg}      & *=su          & *=su       & *isu       & *sunan      \\
\textsc{3sg}      & ∅             & *=na; *=niya & *hiya      & *həyaan     \\
\textsc{1pl.incl} & *=ta          & *=ta       & *ita       & *tənan      \\
\textsc{1pl.excl} & *=nami        & *=nami     & *yami      & *mənan      \\
\textsc{2pl}      & *=namu        & *=namu     & *yamu      & *munan      \\
\textsc{3pl}      & ∅             & *=dəha     & *dəhəya    & *dəhəyaan   \\ \hline
\end{tabular}
\end{table}

Furthermore, \acl{psed} also has three portmanteau pronouns, as shown in Table \ref{tab:por}.

\begin{table}[!htbp]
\centering
\caption{Portmanteau personal pronouns in \acl{psed}}
\label{tab:por}
\begin{tabular}{ll}
\hline
Form   & Gloss \\ 
\hline
*=misu & `\textsc{1sg.gen+2sg.nom}' \\
*=saku & `\textsc{2sg.gen+1sg.nom}' \\
*=maku & `\textsc{1sg.gen+2pl.nom}' \\
\hline
\end{tabular}
\end{table}

In general, clitic pronouns are placed after the predicate in the sentence, except for possessive structure. The possessive structure in Seediq can be described as ``possessee=possessor.'' For example, \textit{tama=mu} `father=\textsc{1sg.gen}'. Regarding the detailed usage of more Seediq clitic pronouns in different dialects, please refer to \textcite{ochiai2009sedpron,kuondtrvpronoun,holmer2014clitic}.

Next, I will examine the reflexes of personal pronouns in each dialect set by set.

\subsubsection{Nominative clitic}

The five different personal pronouns in the nominative clitic form are reflected consistently across all dialects, as shown in Table \ref{tab:nomclitic}. 

\begin{table}[!htbp]
\centering
\caption{Reflexes of nominative clitic personal pronouns in Seediq dialects}
\label{tab:nomclitic}
\begin{tabular}{llllll}
\hline
       & =\textsc{1sg.nom} & =\textsc{2sg.nom} & =\textsc{1pl.nom.incl} & =\textsc{1pl.nom.excl}    & =\textsc{2pl.nom} \\ \hline
\acs{psed} & *=ku    & *=su    & *=ta         & *=nami          & *=namu  \\
\acs{tg}  & =ku     & =su     & =ta          & =nami; (=miyan) & =namu   \\
\acs{cto}  & =ku     & =su     & =ta          & =nami           & =namu   \\
\acs{eto}  & =ku     & =su     & =ta          & =nami           & =namu   \\
\acs{ctr} & =ku     & =su     & =ta          & =nami           & =namu   \\
\acs{etr} & =ku     & =su     & =ta          & =nami           & =namu   \\ \hline
\end{tabular}
\end{table}

However, there is an additional form in \acl{tg}, which is =\textit{miyan} `\textsc{1pl.nom.incl}'. It is speculated that this form might be borrowed from Atayal, where the corresponding form is =\textit{myan} meaning `\textsc{1pl.gen.incl}'. Furthermore, in Seediq, the first person plural nominative and genitive clitics have the same form, suggesting a possible further expansion in this loanword's function.

\subsubsection{Genitive clitic}

The genitive clitic forms of the five personal pronouns are also reflected consistently across the various dialects. Table \ref{tab:genclitic} shows the reflexes. 

\begin{table}[!htbp]
\centering
\caption{Reflexes of genitive clitic personal pronouns in Seediq dialects}
\label{tab:genclitic}
\begin{tabular}{lllll}
\hline
      & =\textsc{1sg.gen}      & =\textsc{2sg.gen}        & =\textsc{3sg.gen} & =\textsc{3sg.gen}       \\ \hline
\acs{psed} & *=mu          & *=su            & *=na     & *=niya         \\
\acs{tg}  & =mu           & =su             & =na      & ---            \\
\acs{cto}  & =mu           & =su             & =na      & =niya          \\
\acs{eto}  & =mu           & =su             & =na      & =niya          \\
\acs{ctr} & =mu           & =su             & =na      & =niya          \\
\acs{etr} & =mu           & =su             & =na      & =niya          \\ \hline
      & =\textsc{1pl.gen.incl} & =\textsc{1pl.gen.excl}   & =\textsc{2pl.gen} & =\textsc{3pl.gen}       \\ \hline
\acs{psed} & *=ta          & *=nami          & *=namu   & *=dəha         \\
\acs{tg}  & =ta           & =nami; (=miyan) & =namu    & =daha          \\
\acs{cto}  & =ta           & =nami           & =namu    & =dəha          \\
\acs{eto}  & =ta           & =nami           & =namu    & =nəha          \\
\acs{ctr} & =ta           & =nami           & =namu    & =dəha          \\
\acs{etr} & =ta           & =nami           & =namu    & =dəha     \\ \hline
\end{tabular}
\end{table}

In addition to the previously mentioned borrowing from Atayal in \acl{tg} with the form \textit{miyan}, there are two other notable points. First, `\textsc{3sg.gen}' seems to have two forms in \acl{psed} stage, one short form *=na and one long form *=niya, although the latter is lost in \acl{tg}. Another point is that according to \textcite{lee2015tawsa}, \acl{eto} has =\textit{nəha} meaning `\textsc{3pl.gen}', but I has not found in other dialects, suggesting it might be borrowed from Atayal as well, where the form is \textit{nəhaʔ} `\textsc{3pl.gen}'.

\newpage
\subsubsection{Nominative free}

Similarly, the nominative free forms show no significant variation or innovation across dialects. Table \ref{tab:nomfree} presents the forms in each dialect.

\begin{table}[!htbp]
\centering
\caption{Reflexes of nominative free personal pronouns in Seediq dialects}
\label{tab:nomfree}
\begin{tabular}{lllll}
\hline
      & \textsc{1sg.nom}      & \textsc{2sg.nom}      & \textsc{3sg.nom} &         \\ \hline
\acs{psed} & *yaku        & *isu         & *hiya   &         \\
\acs{tg}  & yaku         & isu          & heya    &         \\
\acs{cto}  & yaku         & isu          & hiya    &         \\
\acs{eto}  & yaku         & isu          & hiya    &         \\
\acs{ctr} & yaku         & isu          & hiya    &         \\
\acs{etr} & yaku         & isu          & hiya    &         \\ \hline
      & \textsc{1pl.nom.incl} & \textsc{1pl.nom.excl} & \textsc{2pl.nom} & \textsc{3pl.nom} \\ \hline
\acs{psed} & *ita         & *yami        & *yamu   & *dəhəya \\
\acs{tg}  & ita          & yami         & yamu    & deheya  \\
\acs{cto}  & ita          & yami         & yamu    & dəhiya  \\
\acs{eto}  & ita          & yami         & yamu    & dəhiya  \\
\acs{ctr} & ita          & yami         & yamu    & dəhiya  \\
\acs{etr} & ita          & yami         & yamu    & dəhiya  \\ \hline
\end{tabular}
\end{table}

However, it may be worth noting the origins of the \textsc{3sg.nom} and \textsc{3pl.nom} forms. Although they may appear similar, when reconstructing them, we need to consider external evidence and their etymology.

The \textsc{3sg.nom} form is derived from \acs{pan} *si ia, with its corresponding Atayal form being \textit{hiyaʔ}. As for the \textsc{3pl.nom} form *dəhəya, it may not have a clear \acs{pan} source, but it corresponds to Atayal forms such as \textit{ləhaʔ} and \textit{ləhəgaʔ}. The form \textit{ləhaʔ} in Atayal may have originated from \acl{paic} *lVhəjaʔ > **ləhəgaʔ > **ləhaʔaʔ > \textit{ləhaʔ}, as the deletion of /g/ would lead to further vowel coalescence. For discussions on /g/ and /ʔ/ in Atayal, please refer to \textcite{goderich2020phd,song2023Aicgprime}. 

\subsubsection{Oblique free}

Unlike the previous sets, the oblique forms in \acl{tg} are nearly obsolete and are mostly replaced by the nominative forms, as mentioned earlier. The remaining forms, \textit{kenan}, \textit{sunan}, \textit{munan}, are only used in specific situations and vary from person to person (\cite{Sung2018Sedgrammar}).

\begin{table}[!htbp]
\centering
\caption{Reflexes of oblique free personal pronouns in Seediq dialects}
\label{tab:oblfree}
\begin{tabular}{lllll}
\hline
      & \textsc{1sg.obl}      & \textsc{2sg.obl}      & \textsc{3sg.obl} &           \\ \hline
\acs{psed} & *kənan       & *sunan       & *həyaan &           \\
\acs{tg}  & (kenan)      & (sunan)      & ---     &           \\
\acs{cto}  & kənan        & sunan        & həyaan  &           \\
\acs{ctr} & kənan        & sunan        & həyaan  &           \\
\acs{etr} & kənan        & sunan        & həyaan  &           \\ \hline
      & \textsc{1pl.obl.incl} & \textsc{1pl.obl.excl} & \textsc{2pl.obl} & \textsc{3pl.obl}   \\ \hline
\acs{psed} & *tənan       & *mənan       & *munan  & *dəhəyaan \\
\acs{tg}  & ---          & ---          & (munan) & ---       \\
\acs{cto}  & tənan        & mənan        & munan   & dəhəyaan  \\
\acs{ctr} & tənan        & mənan        & munan   & dəhəyaan  \\
\acs{etr} & tənan        & mənan        & munan   & dəhəyaan  \\ \hline
\end{tabular}
\end{table}

The oblique forms in \acl{psed} are believed to have inherited from the reconstructed Acc(usative) set of personal pronouns proposed by \textcite{ross2006casepronoun} (Except for the third person forms), with the addition of the suffix *-an. A comparison with Paiwan and Ross's \acs{pan} Acc personal pronouns' reconstruction is shown in Table \ref{tab:panacc}

\begin{table}[!htbp]
\centering
\caption{A comparison of \acs{pan} \textsc{acc}, Paiwan \textsc{nom} (clitic), and \acl{psed} \textsc{obl} personal pronouns}
\label{tab:panacc}
\begin{tabular}{lllllll}
\hline
       &     & \textsc{1sg}      & \textsc{2sg}        & \textsc{1pl.incl} & \textsc{1pl.incl} & \textsc{2pl}         \\\hline
\acs{pan}   & \textsc{acc} & *i-ak-ən & *iSu[qu]-n & *ita-ən  & *i-am-ən & *i-mu[qu]-n \\
Paiwan & \textsc{nom} & =[a]kən  & =[ə]sun    & =[i]cən  & =[a]mən  & =[ə]mun     \\
\acl{psed} & \textsc{obl} & *kən-an  & *sun-an    & *tən-an  & *mən-an  & *mun-an    \\ \hline
\end{tabular}
\end{table}

\newpage
\subsubsection{Portmanteau forms}

Data from \acl{tg} (from \cite[62]{Sung2018Sedgrammar}) and \acl{etr} (from \cite[74]{Lee2018Trugrammar}) show three different portmanteau personal pronouns in Seediq, as listed in Table \ref{tab:porref}.

\begin{table}[!htbp]
\centering
\caption{Reflexes of portmanteau personal pronouns in Seediq dialects}
\label{tab:porref}
\begin{tabular}{llll}
\hline
                     & \textsc{1sg.gen+2sg.nom}                 & \textsc{2sg.gen+1sg.nom}                  & \textsc{1sg.gen+2pl.nom}                \\ 
\multicolumn{1}{c}{} & \multicolumn{1}{c}{(I to thee)} & \multicolumn{1}{c}{(Thou to me)} & \multicolumn{1}{c}{(I to you)} \\ \hline
\acl{psed}               & *=misu                          & *=saku                           & *=maku                         \\
\acs{tg}                 & =misu                           & =saku                            & =maku                          \\
\acs{etr}                & =misu                           & =saku                            & =maku          \\ \hline                
\end{tabular}
\end{table}

Portmanteau pronouns, like other clitic pronouns, are enclitics placed after the predicate. They consist of two arguments and are found exclusively in (a) nominal predicate; and (b) non-agent voice (NAV) sentences, where they represent the agent or experiencer. Here are two examples (\ref{ex:Pron_nominalP}) and (\ref{ex:Pron_NAV}) from \textcite[74--75]{Lee2018Trugrammar}.

\begin{exe}

    \ex Nominal predicate sentence \label{ex:Pron_nominalP}
    \gll anay=misu ka isu. \\
    sister's.husband=\textsc{1sg.gen+2sg.nom} \textsc{top} \textsc{2sg.nom}\\
    \glt `You are my sister's husband. (For me, you are my sister's husband.)'
    \ex NAV sentence \label{ex:Pron_NAV}
    \gll wada=misu qta-an da! \\
    \textsc{perf}=\textsc{1sg.gen+2sg.nom} see-\textsc{lv} \textsc{ptcl}\\
    \glt `I have already seen you!'
\end{exe}

% \subsection{Demonstratives and deixis} \label{sec:psed_dei}

% \acl{dei}, \acl{dem}

% \begin{table}[!htbp]
% \centering
% \caption{b}
% \label{tab:psed_dem_dei}
% \begin{tabular}{lll}
% \hline
%                 & Demonstrative & Spacial deixis \\ \hline
% Near            & *nii          & *hini          \\
% Far (visible)   & *gaga/*ga     & *hiya          \\
% Far (invisible) & *gayi         & *gayi?         \\ \hline
% \end{tabular}
% \end{table}

% \begin{table}[!htbp]
% \centering
% \caption{a}
% \label{tab:sed_dia_dem_dei}
% \begin{tabular}{llll}
% \hline
%            & \acs{dem}: near & \acs{dem}: far (visible) & \acs{dem}: far (invisible) \\ \hline
% \acs{psed} & *nii            & *gaga/*ga                & *gayi                      \\
% \acs{tg}   & nii             & gaga/ga                  &                            \\
% \acs{cto}  & nii             & wawa/wa                  & wayi                       \\
% \acs{eto}  & nii             & wawa/wa                  & wayi                       \\
% \acs{ctr}  & nii             & gaga/ga                  &                            \\
% \acs{etr}  & nii             & gaga/ga                  & gayi                       \\ \hline\hline
%            & \acs{dei}: near & \acs{dei}: far (visible) & \acs{dei}: far (invisible) \\ \hline
% \acs{psed} & *hini           & *hiya                    & *gayi?                     \\
% \acs{tg}   & hini            & hiya/(hii)               &                            \\
% \acs{cto}  & hini            & hiya/(ha)                & wayi                       \\
% \acs{eto}  & hini            & hiya                     & (wowi)                     \\
% \acs{ctr}  & hini            & hiya                     &                            \\
% \acs{etr}  & hini            & hiya                     & gayi                       \\ \hline
% \end{tabular}
% \end{table}

\section{Proto-Seediq Verbal morphology} \label{sec:psed_aff}

Seediq is generally described as an agglutinative language (\cite{tsukida2012}), but \textcite{holmer1996parametric} does not consider Seediq to be a purely agglutinative language. Despite differing opinions, it is clear that Seediq expresses different verbal voices by affixation; and processes such as causativization, nominalization, and verbalization are also predominantly achieved through adding affixes. This section will explore \acl{psed} inflectional affixes (Section \ref{sec:psed_voice_affixes}) and provide an initial examination of derivational affixes (Section \ref{sec:psed_deri_affixes}). The data in this section primarily comes from the \acl{tg} and \acl{etr} dialects.

\subsection{Proto-Seediq voice system (inflectional) affixes} \label{sec:psed_voice_affixes}

The voice system in Seediq is traditionally referred to as the ``focus system'', relating to its morphosyntactic alignment. Although the term ``focus system'' might be generally considered outdated, it is still frequently used in the literature on Formosan languages especially in Taiwan, particularly as ``焦點系統'' in Chinese. The ``focus'' in Austronesian languages does not correspond to the syntactic focus commonly discussed in general linguistic sense, but refers to the main noun phrase in a sentence that agrees with the verb voice. However, to avoid confusion, this thesis employs the term ``voice''.

This thesis principally follows the classification by \textcite{Sung2018Sedgrammar,Lee2018Trugrammar}, categorizing Seediq voices into Agent/Actor Voice (AV), Patient Voice (PV), Locative Voice (LV), and Circumstantial Voice (CV)\footnote{I combined the Instrumental (工具)/Benefactive (受惠)/Referential (參考) Focuses/Voices as a single Circumstantial Voice since they all share the same verbal marking affixes.}. PV, LV, and CV can further be combined into Undergoer Voice (UV) or non-Actor Voice (NAV), with three distinct subcategories: UV-P, UV-L, and UV-C. Here, I adopt the simplest four-way notation (AV, PV, LV, and CV) to distinguish between these voices.

\acl{psed} can generally be divided into three groups of voice affixes, as shown in Table \ref{tab:psed_voice}. I will first discuss the forms and the basic usages of the voice affixes, followed by the interaction between voice and aspect/mood.

\begin{table}[!htbp]
\centering
\caption{Proto-Seediq voice affixes}
\label{tab:psed_voice}
\begin{tabular}{llll}
\hline
   & I           & II                   & III                        \\ \hline
AV & *mə-; *<əm> & ∅                    & *-a                        \\
PV & *-un        & \multirow{2}{*}{*-i} & \multirow{2}{*}{*-aw/*-ay} \\
LV & *-an        &                      &                            \\
CV & *sə-        & *-ani                & *-anay                     \\ \hline
\end{tabular}
\end{table}

Set I consists of the five affixes *mə-/*<əm> (AV), *-un (PV), *-an (LV), *sə- (CV). This set of affixes is primarily used in affirmative declarative sentences. Table \ref{tab:psed_voi_I} shows their \acl{pan} origins and the reflexes in two modern representative dialects (\acl{tg}, and \acl{etr}).

The AV\xb{1} prefix *mə- is generally used with stative verbs and unaccusative verbs, such as \acl{psed} *mə-hada `cooked; ripe' or *mə-tutiŋ `to fall'. The AV\xb{2} infix *<əm> can occur in various types of AV verbs, such as avalent verbs (e.g., some weather verbs) *q<əm>uyux `to rain', as well as monovalent and divalent verbs. Typically, verbs that use *<əm> are dynamic verbs. 

\begin{table}[!htbp]
\centering
\caption{Proto-Seediq Set I voice affixes and their reflexes}
\label{tab:psed_voi_I}
\begin{tabular}{llllll}
\hline
Set I     & AV\xb{1} & AV\xb{2} & PV   & LV   & CV        \\ \hline
\ac{pan}  & *ma-     & *<um>    & *-ən & *-an & *Sa-/*Si- \\
\ac{psed} & *mə-     & *<əm>    & *-un & *-an & *sə-      \\
\ac{tg}   & mu-      & <um>     & -un  & -an  & su-       \\
\ac{etr}  & mə-      & <əm>     & -un  & -an  & sə-       \\ \hline
\end{tabular}
\end{table}

Some verbs can combine with both *mə- and *<əm>, with the latter adding an additional argument to the valency. For instance, the difference between *mə-tutuy `to wake up' in (\ref{ex:ma_vs_um}a) and *t<əm>utuy `to wake sb. up' in (\ref{ex:ma_vs_um}b) illustrates this distinction. This is an example of the so-called ``causative-inchoative alternation'' in (Proto-)Seediq.

\begin{exe}
\ex \label{ex:ma_vs_um}
    \begin{xlist}
    \ex \textbf{Inchoative}
    \gll *mə-tutuy=ku\\
    \textsc{av}-wake.up=\textsc{1sg.nom}\\
    \glt `I wake up.'
    \ex \textbf{Causative}
    \gll *t<əm>utuy=ku həyaan\\
    <\textsc{av}>wake.up=\textsc{1sg.nom} \textsc{3sg.obl}\\
    \glt `I wake him/her up.'
    \end{xlist}
\end{exe}

The Set I PV suffix *-un emphasizes the patient/theme that is completely affected by the action in affirmative declarative sentences, while the LV suffix *-an highlights a broad sense of location, such as (physical) location, goal, recipient, or source. In Seediq, the distinction between these two types of voices is somewhat blurred, with signs of a gradual loss of differentiation in their usage. \textcite{tsukida2009} categorizes both under the Goal Voice (GV1 and GV2), and \textcite{tang2015} also follows Tsukida's opinion and explains the difference between \textit{-un} and \textit{-an} as a matter of transitivity (\textit{-un} being more transitive than \textit{-an}). Sometimes, the choice of using \textit{-un} or \textit{-an} also depends on the verb. Further research is needed to explore this issue.

% \aikai{補張2000、李2018 提到單獨 -un / -an 的時制差別}

The Set I CV prefix *sə- emphasizes the beneficiary, instrument, or conveyee (the theme being transferred) in affirmative declarative sentences, serving more functions.

Set II consists of only two affixes: *-i (PV/LV) and *-ani (CV), while this class of AV verbs lacks any marking. This set of affixes is primarily used in negative declarative sentences (negating unrealized events) and imperative sentences. Table \ref{tab:psed_voi_II} shows their \acl{pan} origins and the reflexes in two modern representative dialects.

\begin{table}[!htbp]
\centering
\caption{Proto-Seediq Set II voice affixes and their reflexes}
\label{tab:psed_voi_II}
\begin{tabular}{lll}
\hline
Set II    & PV/LV & LV     \\ \hline
\ac{pan}  & *-i   & *-an-i \\
\ac{psed} & *-i   & *-ani  \\
\ac{tg}   & -i    & -ani   \\
\ac{etr}  & -i    & -ani   \\ \hline
\end{tabular}
\end{table}

We can categorize Set II as the subjunctive (or subjunctive-imperative) mood. In negative declarative sentences about past events, the action represents an unrealized fact; whereas in imperatives, it can be interpreted as the speaker's intention. Hungarian also exhibits a phenomenon where the subjunctive and imperative are indistinguishable in morphology (\cite{toth2007}).

Examples (\ref{ex:SetII_1}--\ref{ex:SetII_3}) illustrates the use of Set II affixes in sentences with different moods. (\ref{ex:SetII_1}a--b) are negative declarative sentences (past events), (\ref{ex:SetII_2}a--b) are affirmative imperative sentences, and (\ref{ex:SetII_3}a--b) are negative imperative sentences.

\begin{exe}
\ex Negative declarative \label{ex:SetII_1}
    \begin{xlist}
    \ex \textbf{PV} *-i
    \gll *ini=mu qəta-i ka tama=mu\\
    \textsc{neg}=\textsc{1sg.gen} see-\textsc{pv.subj} \textsc{nom} father=\textsc{1sg.gen}\\
    \glt `I did not see my father.'
    \ex \textbf{CV} *-ani
    \gll *ini=mu həpəy-ani idaw ka tama=mu\\
    \textsc{neg}=\textsc{1sg.gen} cook-\textsc{cv.subj} rice.\textsc{obl} \textsc{nom} father=\textsc{1sg.gen}\\
    \glt `I did not cook (a meal) for my father.'
    \end{xlist}
\ex Affirmative imperative \label{ex:SetII_2}
    \begin{xlist}
    \ex \textbf{PV} *-i
    \gll *qəta-i ka tama=su\\
    see-\textsc{pv.subj} \textsc{nom} father=\textsc{2sg.gen}\\
    \glt `Look at your father!'
    \ex \textbf{CV} *-ani
    \gll *həpəy-ani idaw ka tama=su\\
    cook-\textsc{cv.subj} rice.\textsc{obl} \textsc{nom} father=\textsc{2sg.gen}\\
    \glt `Cook for your father!'
    \end{xlist}
\ex Negative imperative \label{ex:SetII_3}
    \begin{xlist}
    \ex \textbf{PV} *-i
    \gll *iya qəta-i ka tama=su\\
    \textsc{neg} see-\textsc{pv.subj} \textsc{nom} father=\textsc{2sg.gen}\\
    \glt `Don't look at your father!'
    \ex \textbf{CV} *-ani
    \gll *iya həpəy-ani idaw ka tama=su\\
    \textsc{neg} cook-\textsc{cv.subj} rice.\textsc{obl} \textsc{nom} father=\textsc{2sg.gen}\\
    \glt `Don't cook for your father!'
    \end{xlist}
\end{exe}

Set III is used exclusively in sentences with hortative mood and appears to lack evidence of usage in negative sentences. Similar to Set II, Seediq does not distinguish between PV and LV in this set. Historically, \acl{psed} *-aw derives from \acl{pan} *-aw (PV), while *-ay originates from \acl{pan} *-ay (LV). \textcite[103]{Lee2018Trugrammar} notes that *-ay is typically used for suggestions, proposals, or advice, whereas *-aw is used for requests. However, upon reviewing the example sentences in the dictionary of \textcite{ILRDFEdict}, the usage of the two forms may not show significant differences. Until more evidence becomes available, this thesis does not differentiate between the two. \textcite[97]{Sung2018Sedgrammar} mentions that in Tgdaya, there is a form -\textit{ano} (< **-anaw) in addition to -\textit{ane}, but I have not found any instances of -\textit{ano} in dictionaries or other corpora, nor does \citeauthor{ochiai2016phd}'s (\citeyear{ochiai2016phd}) grammatical sketch mention such a form. Furthermore, external evidence, such as other Austronesian languages or \acl{pan}, does not attest to a form like **-anaw.

Table \ref{tab:psed_voi_III} shows the \acl{pan} origins of the Set III affixes and their reflexes in two modern representative dialects. Note that *-a (AV) has not been retained in \acl{tg}, and its status in \acl{cto} and \acl{eto} is uncertain. It seems that only \acl{ctr} and \acl{etr} have preserved the form -\textit{a}.

\begin{table}[!htbp]
\centering
\caption{Proto-Seediq voice affixes (Set III) and their reflexes}
\label{tab:psed_voi_III}
\begin{tabular}{llll}
\hline
Set III   & AV  & PV/LV                & CV      \\ \hline
\ac{pan}  & *-a & *-aw (PV); *-ay (LV) & *-an-ay \\
\ac{psed} & *-a & *-aw/*-ay            & *-anay  \\
\ac{tg}   & --- & -o/-e                & -ane    \\
\ac{etr}  & -a  & -aw/-ay              & -anay   \\ \hline
\end{tabular}
\end{table}

Set III (hortative) produces semantic or pragmatic distinctions when combined with different personal pronouns. With first-person pronouns, it is used for purposes such as requests or suggestions, similar to English expressions like ``Let me...'' or ``Let's...,'' as in (\ref{ex:SetIII_1}a--b). When used with second-person pronouns, it conveys a sense of prohibition or discouragement, even without the presence of a negator, possibly reflecting an ironic tone, as shown in (\ref{ex:SetIII_2}). Examples with third-person pronouns have so far only been mentioned in Tgdaya (\cite[97--98]{ochiai2016phd}), also expressing prohibition or discouragement. Until further evidence becomes available, this usage with third-person pronouns is not reconstructed back to \acl{psed}. Relevant examples can be seen in (\ref{ex:SetIII_3}).


\begin{exe}
\ex Hortative + first-person pronouns \label{ex:SetIII_1}
    \begin{xlist}
    \ex \textbf{AV} *-a
    \gll *uq-a=ta han\\
    eat-\textsc{av.hort}=\textsc{1pl.incl.gen} first\\
    \glt `Let's eat (something)!'
    \ex \textbf{PV/LV} *-aw/*-ay
    \gll *uq-ay=ta ka idaw han\\
    eat-\textsc{pv/lv.hort}=\textsc{1pl.incl.gen} \textsc{nom} rice first\\
    \glt `Let's eat the meal. (the food is visible)'
    \end{xlist}
\ex Hortative + second-person pronouns \label{ex:SetIII_2}
    \begin{xlist}
    \ex \textbf{PV/LV} *-aw/*-ay
    \gll *cəkul-ay=saku\\
    push-\textsc{pv/lv.hort}=\textsc{1sg.nom+2sg.gen}\\
    \glt `Don't push me!'
    \end{xlist}
\end{exe}


\begin{exe}
\ex Hortative + third-person pronouns (Tgdaya) \label{ex:SetIII_3}
    \begin{xlist}
    \ex \textbf{PV} -\textit{e} (\cite[98]{ochiai2016phd})
    \gll tubug-e=na ∅ babuy\\
    feed-\textsc{pv/lv.hort}=\textsc{3sg.gen} \textsc{nom} pig\\
    \glt `Don't let him feed the pig(s)!' (It seems like he is going to feed the pigs, so we have to stop him from feeding the pigs.)
    \end{xlist}
\end{exe}

I propose a preliminary analysis of the interaction between \acl{psed} voices and aspects/moods, as shown in Table \ref{tab:psed_voice_tam}. Some parts of the table are incomplete and remain uncertain, temporarily marked with question marks ``?''. These gaps may be due to a lack of data or the possible absence of corresponding usages.

\begin{table}[!htbp]
\centering
\caption{Interactions of Proto-Seediq voices and aspects/moods}
\label{tab:psed_voice_tam}
\begin{tabular}{lllll}
\hline
                  & AV                    & PV            & LV             & CV      \\ \hline
Realis            & *mə-√; *<əm>√         & *√-un         & *√-an          & *sə-√   \\
Realis perfective & *m<ən>ə-√; *<əm><ən>√ & *<ən>√        & *<ən>√-an      & ?       \\
Irrealis          & *mə-pə-√; *mə-√       & *Cə\~{}√-un   & *Cə\~{}√-an    & ?       \\
Subjuctive        & *√                    & \multicolumn{2}{c}{*√-i}       & *√-ani  \\
Hortative         & *√-a                  & \multicolumn{2}{c}{*√-aw/*-ay} & *√-anay \\ \hline
\end{tabular}
\end{table}

The newly introduced affixes in the table include *<ən> and *pə-. *<ən> expresses the perfective aspect, while *pə- functions as an irrealis prefix, generally indicating future events. When *<ən> is used independently, it can simultaneously express patient voice and perfective aspect. This phenomenon is observed in many Austronesian languages, such as \textit{<in>} `\textsc{pv.pfv}' in Atayal (\cite{huang1995}) and Tagalog (\cite{schachter_otanes1972tagalog}). On the other hand, *pə- is often used in combination with *mə-. The irrealis form of PV/LV is expressed through *Cə\~{} reduplication combined with *-un/*-an. This analysis may still have shortcomings and awaits further research in the future.

An alternative analysis is to interpret *<ən>/*pə- as markers of past tense and future tense, respectively. \textcite{chen2018, chen2024} argue that in Squliq Atayal, \textit{<in>} and \textit{p-} serve as an existential past marker and a future marker, and have conducted extensive testing to support this claim. However, modern Seediq dialects have yet to undergo similarly comprehensive research. Given this situation, this thesis does not address the fine usages of these markers at the \acl{psed} stage. The terms currently used in Table \ref{tab:psed_voice_tam} can be treated as placeholders, pending further discussion and examination in the future.

\subsection{Some Proto-Seediq verbal derivational affixes} \label{sec:psed_deri_affixes}

This section discusses some \acl{psed} derivational affixes, such as causative, reciprocal, and other verbalizers.

\subsubsection{Causative}

The causative prefix in \acl{psed} is *pə-, derived from \acl{pan} *pa- (\cite{ACD,ross_zeitoun2023}). Similar to many Formosan languages, when used with stative verbs, the causative prefix requires the addition of the stative verb prefix *kə-, which is a reflex of \ac{pan} *ka-. Reflexes in representative dialects and examples are shown in Table \ref{tab:psed_pa}.

\begin{table}[!htbp]
\centering
\caption{Proto-Seediq causative prefix *pə-}
\label{tab:psed_pa}
\begin{tabular}{llllll}
\hline
 & \acs{cau} & \multicolumn{2}{l}{\textsc{cau} + dynamic verb} & \multicolumn{2}{l}{\textsc{cau} + stative verb} \\ \hline
\ac{pan}  & *pa- & *pa-...  & `\textsc{cau}-...' & *pa-ka-...  & `\textsc{cau-stat}-...'  \\
\ac{psed} & *pə- & *pə-qita & `\textsc{cau}-see.\textsc{av}' & *pə-kə-dəŋu & `\textsc{cau-stat}-dry.\textsc{av}' \\
\ac{tg}   & pu-  & pu-qita  & `\textsc{cau}-see.\textsc{av}' & pu-ku-deŋu  & `\textsc{cau-stat}-dry.\textsc{av}' \\
\ac{etr}  & pə-  & pə-qita  & `\textsc{cau}-see.\textsc{av}' & pə-kə-dəŋu  & `\textsc{cau-stat}-dry.\textsc{av}' \\ \hline
\end{tabular}
\end{table}

The causative prefix *pə- can also interact with different voices, as shown in Table \ref{tab:psed_pa_voice}. It is important to note that when using the causative prefix in affirmative declarative sentences in the Actor voice, neither infixes nor prefixes for AV (*<um>/*mə-) appear.

\begin{table}[!htbp]
\centering
\caption{Proto-Seediq causative prefix *pə-with voice affixes}
\label{tab:psed_pa_voice}
\begin{tabular}{llll}
\hline
   & I           & II                   & III                        \\ \hline
AV & *pə-√       & *pə-√                & *pə-√-a                    \\
PV & *pə-√-un    & \multirow{2}{*}{*pə-√-i} & \multirow{2}{*}{*pə-√-aw/-ay} \\
LV & *pə-√-an    &                      &                            \\
CV & *sə-pə-√    & *pə-√-ani            & *pə-√-anay                 \\ \hline
\end{tabular}
\end{table}

\subsubsection{Reciprocal}

\textcite{ochiai2016reciprocals} extensively discussed the reciprocals in two dialects of Seediq, \acl{tg} and \acl{etr}. Based on Ochiai's analysis, I reconstructed two sets of \acl{psed} reciprocal patterns: (i) *mə-Cə\~{}/ *pə-Cə\~{} (where C stands for any consonant), and (ii) *mə-sə-. Table \ref{tab:psed_recp} provides the reflexes in both dialects.

\begin{table}[!htbp]
\centering
\caption{Proto-Seediq reciprocal prefixes}
\label{tab:psed_recp}
\begin{tabular}{lll}
\hline
          & \acs{recp}\xb{1}      & \textsc{recp}\xb{2} \\ \hline
\ac{pan}  & *ma-Ca\~{}/*pa-Ca\~{} & ---                 \\
\ac{psed} & *mə-Cə\~{}/*pə-Cə\~{} & *mə-sə-/*pə-sə-             \\
\ac{tg}   & mu-Cu\~{}/pu-Cu\~{}   & mu-su-/pu-su-              \\
\ac{etr}  & mə-Cə\~{}/pə-Cə\~{}   & mə-sə-/pə-sə-              \\ \hline
\end{tabular}
\end{table}

\textcite{ochiai2016reciprocals} compared two sets of reciprocal prefixes and made the following comparison, as shown in Table \ref{tab:psed_recp_ochi}. Ochiai argues that the *mə-Cə\~{} (her \textit{mVCV-}) set is more productive than *mə-sə- (her \textit{mVsV-}) and is applied to verbs with higher transitivity. 

\begin{table}[!htbp]
\centering
\caption{The two reciprocal prefixes (adapted form \textcite[182]{ochiai2016reciprocals})}
\label{tab:psed_recp_ochi}
\begin{tabular}{lll}
\hline
\ac{psed}           & *mə-Cə\~{} (mCVC-) & *mə-sə- (mVsV-) \\ \hline
m/p pairing         & Yes                & Yes             \\
Productivity        & high               & low             \\
Transitive pair     & Yes                & No              \\
Kin-term derivation & No                 & Yes             \\
Typological type    & grammatical        & lexical         \\ \hline
\end{tabular}
\end{table}

Based on the data provided by Ochiai, I propose the following principles (exceptions may exist): (1) *mə-Cə\~{} is paired with dynamic verbs of high transitivity, (2) *mə-sə- is used with dynamic verbs of low transitivity, static verbs, and nouns. In Table \ref{tab:psed_recp_ex}, I have listed examples of the usage of the two sets of reciprocal prefixes under different circumstances.

\begin{table}[!htbp]
\centering
\caption{Examples with two reciprocal prefixes in Proto-Seediq}
\label{tab:psed_recp_ex}
\begin{tabular}{llllll}
\hline
           &            & Root  &                  & Reciprocal    &                            \\ \hline
*mə-Cə\~{} & + dnmc. v. & *qita  & `to see'         & *mə-qə\~{}qita & `to see e.o.'              \\
*mə-sə-    & + dnmc. v. & *barux & `to lend/borrow' & *mə-sə-barux   & `to work for e.o.' \\
*mə-sə-    & + stat. v. & *daliŋ & `te be close'    & *mə-sə-daliŋ   & `to be close to e.o.'      \\
*mə-sə-    & + n.       & *daŋi  & `friend'         & *mə-sə-daŋi    & `to be e.o.'s friend'  \\ \hline
\end{tabular}
\end{table}

\subsubsection{Other verbalizers}

In \acl{psed}, there are several verbalizers (morphemes that turn nouns into verbs) that can be reconstructed. This section will enumerate some of these affixes, accompanied by examples from modern dialects, as shown in Table \ref{tab:psed_vblz}. 

\begingroup
\setstretch{1}
\renewcommand\arraystretch{1.2}
\begin{table}[!htbp]
\centering
\caption{Some Proto-Seediq verbalizers}
\label{tab:psed_vblz}
\begin{tabular}{lllll}
\hline
*gə-       & `to get; to let be'       & \ac{psed} & *g<əm>ə-həyi    & `to pick meat'                 \\
     &                    & \ac{tg}   & g<um>u-heyi      & \quad < `meat'                           \\
     &                    & \ac{etr}  & g<əm>ə-hiyi      &                            \\ \cline{3-5} 
     &                    & \ac{psed} & *g<əm>ə-bəliŋ    & `to make hole'             \\
     &                    & \ac{tg}   & g<um>u-beliŋ     & \quad < `hold'                           \\
     &                    & \ac{etr}  & g<əm>ə-bəliŋ     &                            \\ \hline
*sə- & `to (make) appear' & \ac{psed} & *sə-bəgihur      & `windy'                    \\
     &                    & \ac{tg}   & su-bugihun       & \quad < `wind'                           \\
     &                    & \ac{etr}  & sə-bəgihur       &                            \\ \cline{3-5} 
     &                    & \ac{psed} & *sə-quwaq        & `noisy'                    \\
     &                    & \ac{tg}   & su-quwaq         & \quad < `mouth'                          \\
     &                    & \ac{etr}  & sə-quwaq         &                            \\ \hline
*tə- & `to produce'       & \ac{psed} & *tə-laqi         & `to give birth'            \\
     &                    & \ac{tg}   & tu-laqi          & \quad < `child'                           \\
     &                    & \ac{etr}  & tə-laqi          &                            \\ \cline{3-5} 
     &                    & \ac{psed} & *tə-ŋayan        & `to name'                  \\
     &                    & \ac{tg}   & tu-ŋayan         & \quad < `name (n.)'                           \\
     &                    & \ac{etr}  & tə-haŋan         &                            \\ \hline
*tə- & `to catch'         & \ac{psed} & *tə-camat        & `to catch animal'          \\
     &                    & \ac{tg}   & tu-camac         & \quad < `wild animal'                           \\
     &                    & \ac{etr}  & tə-samat         &                            \\ \hline
*sənə-     & `to fight because of...'  & \ac{psed} & *sənə-laqi      & `to fight because of child'    \\
     &                    & \ac{tg}   & sunu-laqi        & \quad < `child'                           \\
     &                    & \ac{etr}  & sənə-laqi        &                            \\ \hline
*məgə-     & `to be like; to become'   & \ac{psed} & *məgə-laqi      & `to be like a child'           \\
     &                    & \ac{tg}   & mugu-laqi        & \quad < `child'                           \\
     &                    & \ac{etr}  & məgə-laqi        &                            \\ \cline{3-5} 
     &                    & \ac{psed} & *məgə-risaw      & `to become a youth'        \\
     &                    & \ac{tg}   & mugu-riso        & \quad < `youth (male)'                           \\
     &                    & \ac{etr}  & (ˣmaa=)risaw     &                            \\ \hline
*sə-Cə\~{} & `to have a smell of...'   & \ac{psed} & *sə-cə\~{}cakus & `to have a smell of camphor' \\
     &                    & \ac{tg}   & su-cu\~{}cakus   & \quad < `camphor tree'                           \\
     &                    & \ac{etr}  & sə\~{}sakus      &                            \\ \cline{3-5} 
     &                    & \ac{psed} & *sə-hə\~{}huliŋ  & `to have a smell of dog' \\
     &                    & \ac{tg}   & su-hu\~{}huliŋ   & \quad < `dog'                           \\
     &                    & \ac{etr}  & sə-hə\~{}huliŋ   &                            \\ \hline
*kə-Cə\~{} & `to do sth. along...'     & \ac{psed} & *kə-lə\~{}alaŋ  & `to do sth. along villages'      \\
     &                    & \ac{tg}   & ku-la\~{}alaŋ    & \quad < `village'                           \\
     &                    & \ac{etr}  & mə-kə-lə\~{}alaŋ &                            \\ \hline
\end{tabular}
\end{table}
\endgroup

In \acl{psed}, the core meaning of *gə- is `to get; to let be', as seen in *g<əm>ə-həyi `to pick meat (to get meat)' and *g<əm>ə-bəliŋ `to make a hole (to let there be a hole)'. The prefix *sə- conveys the meaning of `to (make) appear', as in *sə-bəgihur `windy (wind appears)' and *sə-quwaq `noisy (mouth appears > noises)'. The prefix *tə- has two distinct meanings which may have different origins. The first meaning is `to produce', such as *tə-laqi `to give birth (to produce child)' and *tə-ŋayan `to name (to produce a name)'; the second meaning is `to catch', as in *tə-camat `to catch (wild) animal'.

There are also disyllabic verbalizers like *sənə- `to fight because of...', such as *sənə-laqi `to fight because of a child'. Currently, there is no direct evidence to prove that this is composed of two morphemes, but the most likely combination is *sə- plus an infix *<ən>.

Another verbalizer is *məgə- `to be like; to become', such as *məgə-laqi `to be like a child (childish?)', *məgə-risaw `to become a youth'. In the Tgdaya language, there is no morphological distinction between these two meanings, and they share a semantic similarity in expressing `having the characteristics of N'. However, in \acl{etr}, `to become...' is replaced with \textit{maa=} (\cite{tsukida2009}). I suspect \textit{ma(ʔ)a=} is borrowed from an Atayal form similar to \textit{məʔə-} \~{} \textit{mɐʔɐ-} `to become...' (Atayal's prepenultimate neutralized vowel could be \textit{ə} \~{} \textit{ɐ} \~{} \textit{a}). First, Seediq should not have a non-neutralized prepenultimate vowel, which is why Tsukida analyzes it as a proclitic. Second, Atayal conveniently has corresponding examples, where Atayal \textit{məʔə-} \~{} \textit{mɐʔɐ-} should be cognate with Proto-Seediq *məgə-, and with the common change of **g to *ʔ in Atayal (\cite{goderich2020phd,song2024Aicg}). Therefore, I hypothesize that \acl{etr} borrowed \textit{mɐʔɐ-} from a certain Atayal dialect (possibly a subdialect of Squliq, given \acl{etr}'s migration routes to the east), adopting it as \textit{maa=}, replacing some functions originally expected of **məgə- and creating a semantic division.

% 剩餘的兩個 verbalizers 都必須和 *Cə\~{} 重疊搭配使用。第一個是 *sə-Cə\~{} & `to have a smell of...',如 *sə-cə\~{}cakus `to have a smell of camphor'。請注意在 \acl{etr} 的 reflex \textit{səsakus} 中,比預期的 **səsəsakus 少了一個音節,這是因為重複音節出現太多所造成的異化現象,詳見\textcite[4]{lee2009odor}。而 Tgdaya 沒有這種限制,如 su-su\~{}sama `to have a smell of vegetable'。 

The remaining two verbalizers also require use with the *Cə\~{} reduplication. The first is *sə-Cə\~{} `to have a smell of...', such as *sə-cə\~{}cakus `to have a smell of camphor'. Note that in the \acl{etr} reflex \textit{səsakus}, there is one less syllable than the expected **səsəsakus. This reduction is due to a dissimilation phenomenon caused by too many repeated syllables, as detailed in \textcite[4]{lee2009odor}. In Tgdaya, this restriction is absent, as shown in \textit{su-su\~{}sama} `to have a smell of vegetable'.

% 另一個是 *kə-Cə\~{} & `to do sth. along...',如 *kə-lə\~{}alaŋ  & `to do sth. along villages'。這個詞綴加上名詞後,常作為副詞性動詞使用。有趣的是,因為root *alaŋ `village' 是元音起首,故重疊時必須使用第二個音節的 onset *l。正常來說,是重疊 root 的第一個音節,如 \acl{tg} \textit{ku-ru\~{}ruru} `to do sth. along the creek', \acl{etr} \textit{mə-kə-yə\~{}yayuŋ} `to do sth. along the river'.

The other verbalizer is *kə-Cə\~{} `to do something along...', such as *kə-lə\~{}alaŋ `to do something along villages'. This affix, when added to a noun, is often used as an adverbial verb. Interestingly, because the root *alaŋ `village' starts with a vowel, the reduplication must use the onset *l of the second syllable. Normally, the reduplication would involve the onset of the first syllable of the root, as in \acl{tg} \textit{ku-ru\~{}ruru} `to do something along the creek' or \acl{etr} \textit{mə-kə-yə\~{}yayuŋ} `to do something along the river'.

% 本論文只是列舉一些明顯能重建回\acl{psed}的例子。\acl{psed} 及各個方言必定還有更多的 derivatoinal 詞彙,包含此處並沒有提到的一些 nominalizer 或單純改變意思用的功能性詞綴,如已故之人名字會加上前綴 *səkə- `late' 等等。這個議題還有待未來更多語料的搜集與重建。

This thesis only lists some examples that can clearly be reconstructed back to \acl{psed}. \acl{psed} and its various dialects undoubtedly possess many more derivational words, including some nominalizers or semantic functional affixes not mentioned here, such as the prefix *səkə- `late' added to the names of deceased individuals. This topic awaits further collection and reconstruction of data in the future.

\section{Irregular verb conjugation in Proto-Seediq} \label{sec:psed_irr_v}

In modern Seediq dialects, it is sometimes observed that the same verb tends to use different stems when combined with different affixes. Some examples may result from processes like haplology, while others seem to involve suppletion. However, in some cases, the reasons are difficult to identify. Many of these examples are consistent across Seediq dialects and can be reconstructed back to \acl{psed}. This section will provide a discussion of the examples I have been able to identify.

The first two examples are both related to haplology, such as *tinun \~{} *tun- `to weave' and *hadut \~{} *had- `to escort'. When no suffix is added (i.e., only with prefixes or infixes, or in bare form), both use the first stem. However, when a suffix like *-un is attached, the second stem is used to avoid potential repetitive sound sequences, as shown in Table \ref{tab:psed_hap}.

\begin{table}[!htbp]
\centering
\caption{Haplology in verbal conjunctions in Seediq}
\label{tab:psed_hap}
\begin{tabular}{llll}
\hline
Non-suffixed form & Suffixed form & **Suffixed form & Gloss       \\ \hline
*t<ə.m>i.nun        & *tu.n-un       & **tə.\textbf{nu}.\textbf{n-u}n      & `to weave'  \\
*h<ə.m>a.dut        & *ha.d-un       & **hə.\textbf{du}.\textbf{d-u}n      & `to escort' \\ \hline
\end{tabular}
\end{table}

The third and fourth examples can be categorized as suppletion, such as *bəgay \~{} *biq- `to give' and *əkan \~{} *uq- `to eat'. The conditions for usage are the same as the previous examples: the second stem is used when a suffix is attached. Relevant examples are shown in Table \ref{tab:psed_supp}. The morpheme *biq- `to give' may have been borrowed from Atayal, while *uq- `to eat' appears to have no identifiable source.

\begin{table}[!htbp]
\centering
\caption{Suppletion in verbal conjunctions in Seediq}
\label{tab:psed_supp}
\begin{tabular}{llll}
\hline
Non-suffixed form & Suffixed form & **Suffixed form & Gloss     \\ \hline
*məgay            & *biq-un       & **bəgay-un      & `to give' \\
*m-əkan           & *uq-un        & **kan-un        & `to eat'  \\ \hline
\end{tabular}
\end{table}

The final example is *adas \~{} *dəs- `to bring' (compare also \acl{pata} *maras \~{} *rasun `to bring'), which seems to be a case of irregular vowel alternation, as shown in Table \ref{tab:psed_v_irr}. In contrast, a similar environment, such as *patas \~{} *pətasun `to write', does not exhibit such alternation.

\begin{table}[!htbp]
\centering
\caption{Pure irregularity in verbal conjunctions in Seediq}
\label{tab:psed_v_irr}
\begin{tabular}{llll}
\hline
Non-suffixed form & Suffixed form & **Suffixed form & Gloss     \\ \hline
*m-adas           & *dəs-un       & **das-un        & `to eat'  \\ \hline
\end{tabular}
\end{table}

\section{Summary}

% 這一章初步地討論了 \acl{psed} 的 morphosyntactic system,包含 case-marking system、personal pronouns、voices、其他 derivational morphemes 以及一些在 \acl{psed} 就已經可以發現的不規則動詞變化。

% 在 case markers 的討論當中,我發現了除了 Tgdaya 以外的方言皆丟失了 genitive marker \textit{na},這或許可視為一個可能的共同創新。而關於 voice affixes 的部分,則發現 Tgdaya (可能包含 \acl{cto} 及 \acl{eto}) 皆丟失了 AV 的 hortative suffix \textit{-a},可能也是潛在的共同創新。只是 \acl{cto} 及 \acl{eto} 是否真的缺乏該後綴,抑或是尚未被記載,則有待後續調查。

% Seediq 各個現代方言的基本句法可能沒有太大的差異,只是因為時間限制而無法搜集到每個方言的語料,無法進一步探討其異同。這章的方言差異將作為 Chapter \ref{ch7} 分群的輔助性證據使用。

This chapter preliminarily discusses the morphosyntactic system of \acl{psed}, including its case-marking system, personal pronouns, voice affixes, other derivational morphemes, and some irregular verb changes already evident in \acl{psed}.

In the discussion of case markers, I discovered that dialects other than Tgdaya have all lost the genitive marker \textit{na}, which could be seen as a shared innovation. Regarding voice affixes, it has been found that Tgdaya (possibly including \acl{cto} and \acl{eto}) have lost the AV hortative suffix \textit{-a}, which may also be a potential shared innovation. Whether \acl{cto} and \acl{eto} truly lack this suffix or it has simply not yet been documented remains to be investigated further.

The basic morphosyntax of the modern Seediq dialects might not differ significantly, but due to time constraints, it was not possible to collect data from each dialect to explore their differences further. The dialectal differences discussed in this chapter will be used as corroborative evidence for subgrouping in Chapter \ref{ch7}.