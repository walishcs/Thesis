\chapter{Seediq subgrouping}\label{ch7}

In this section, I will propose a subgrouping hypothesis of Seediq dialects mainly based on shared phonological innovations. The historical development of \acl{psed} *ə plays an important role in this subgrouping. Lexical evidence will be used as well, especially those contains sporadic sound change and those with replacement of the whole lexical item. 

\subsection{Subgrouping hypothesis}

The subgrouping of Seediq dialects based on the phonological and lexical evidence is presented in Figure \ref{fig:qhuni}. 

\begin{figure}[!htbp] 
\centering
\begin{forest}
for tree={l sep=5em, s sep=2em, inner sep=0, anchor=north, parent anchor=south, child anchor=north}
    [Seediq
        [\acl{tg}, tier=word]
        [\acl{totr}
            [\acl{to}
                [\acl{cto}, tier=word]
                [\acl{eto}, tier=word]   
            ]
            [\acl{tr}
                [\acl{ctr}, tier=word]
                [\acl{etr}, tier=word]
            ]
        ]
    ]
\end{forest}
\caption{Seediq subgrouping tree}\label{fig:qhuni}
\end{figure}

There is a subgroup of Toda-Truku, which is consisted of \acl{cto}, \acl{eto}, \acl{ctr}, and \acl{etr}. Moreover, \acl{cto} and \acl{eto} form a Toda subgroup; while \acl{ctr} and \acl{etr} form a Truku subgroup under Toda-Truku. 

\acl{tg} is thus divided as the other primary branch since there is no further evidence to link \acl{tg} with the other dialects for now. On the other hand, \acl{tg} itself has some phonological change which are very unique among Seediq dialects, especially the historical penultimate schwa *ə shift to /e/ in penultimate syllable and the pre-penultimate *ə shift to /u/ as the ``neutralized/reduced'' vowel. That is, Tgdaya is considered a primary offspring of \acl{psed}. 

\subsection{Evidence for a Toda-Truku subgroup}
\lipsum[1-7]
\subsubsection{Evidence for a Toda subgroup}
\lipsum[1-7]
\subsubsection{Evidence for a Truku subgroup}
\lipsum[1-7]

\section{Interim summary}
\lipsum[1-2]