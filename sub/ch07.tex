\chapter{Seediq subgrouping}\label{ch7}

% 在這一章中,我會propose一個 subgrouping hypothesis of Seediq dialects 同時基於 phonological, morphosyntactic, and lexical evidence. Phonological 的證據中,historical schwa *ə 的split-merger以及在某些詞項中的 sporadic sound changes扮演了最重要的角色;morphosyntactic 的證據主要是與格位標記有關;而 lexical evidence 則涉及到純粹的 lexical innovation 以及在 Chapter \ref{ch6} 中所提到的可能共同創新。

In this chapter, I will propose a subgrouping hypothesis of Seediq dialects based on phonological, morphosyntactic, and lexical evidence. Phonologically, the most crucial evidence includes the development of the historical schwa *ə and sporadic sound changes found in certain lexical items. The morphosyntactic evidence primarily relates to the case marking system, while the lexical evidence involves pure lexical innovation and potential shared innovations discussed in Chapter \ref{ch6}.

\section{Subgrouping hypothesis}

The subgrouping of Seediq dialects based on the phonological and lexical evidence is presented in Figure \ref{fig:qhuni}. 

\begin{figure}[!htbp] 
\centering
\begin{forest}
for tree={l sep=5em, s sep=2em, inner sep=0, anchor=north, parent anchor=south, child anchor=north}
    [Seediq
        [\acl{tg}, tier=word]
        [\acl{totr}
            [\acl{to}
                [\acl{cto}, tier=word]
                [\acl{eto}, tier=word]   
            ]
            [\acl{tr}
                [\acl{ctr}, tier=word]
                [\acl{etr}, tier=word]
            ]
        ]
    ]
\end{forest}
\caption{Seediq subgrouping tree}\label{fig:qhuni}
\end{figure}

There is a subgroup of Toda-Truku, which is composed of \acl{cto}, \acl{eto}, \acl{ctr}, and \acl{etr}. Moreover, \acl{cto} and \acl{eto} form a Toda subgroup; while \acl{ctr} and \acl{etr} form a Truku subgroup under Toda-Truku. 

\begin{figure}[!htbp] 
\centering
\begin{forest}
for tree={l sep=5em, s sep=2em, inner sep=0, anchor=north, parent anchor=south, child anchor=north, edge={black, thick}}
    [Seediq
        [\acl{tg}, tier=word]
        [\acl{totr}, for tree={edge={gray, opacity=0.5}, text opacity=0.5} % This applies gray color to all edges under \acl{totr}
            [\acl{to}
                [\acl{cto}, tier=word]
                [\acl{eto}, tier=word]   
            ]
            [\acl{tr}
                [\acl{ctr}, tier=word]
                [\acl{etr}, tier=word]
            ]
        ]
    ]
\end{forest}
\caption{The position of \acl{tg} in Seediq}\label{fig:qhuni_tg}
\end{figure}

\acl{tg} forms a primary branch by itself because there is no evidence to link \acl{tg} with the other dialects. The position of the \acl{tg} dialect in Seediq is shown in Figure \ref{fig:qhuni_tg}. 

Moreover, \acl{tg} itself has some phonological changes which are very unique among Seediq dialects, as listed below:

\begin{enumerate}
    \item \textbf{*ə (penultimate) and *ay merged as \textit{e}.} The historical schwa *ə in the penultimate syllable became \textit{e} in \acl{tg}, and the diphthong *ay monophthongized into an \textit{e} as well: \acl{psed} *məgay > \acl{tg} \textit{mege} `to give (AV)', \acl{psed} *maytaq > \acl{tg} \textit{metaq} `to stab (AV)'.
    \item \textbf{*ə (prepenultimate) and *u merged as \textit{u}.}: The historical schwa *ə in the prepenultimate syllables (the weakened/neutralized vowel) became \textit{u} in \acl{tg}: \acl{psed} *dəqəras > \acl{tg} \textit{duqeras} `face'. 
    \item \textbf{*-uy and *-ig merged as \textit{uy}.} The sequence *-ig changed to -\textit{uy} and merged with the original *-uy in \acl{tg}. \acl{psed} *marig > \acl{tg} \textit{maruy} `to buy (AV)', \acl{psed} *babuy > \acl{tg} \textit{babuy} `pig'.
\end{enumerate}


\section{Evidence for a Toda-Truku subgroup}

The Toda-Truku subgroup is supported by (1) shared sound changes, (2) sporadic sound changes, (3) lexical innovations, and (4) a morphosyntactic change. Any one of these occurring individually could be due to dialect contact or coincidence. However, when these four types of shared changes occur together, the likelihood of establishing this subgroup is significantly increased. This section will discuss examples of each type of shared change. 

Due to the lack of data, examples from \acl{eto} are not presented in this section dedicated to grouping the Toda-Truku. Nonetheless, there are compelling reasons to assert that \acl{eto} is closely related to \acl{cto}, as detailed in \textcite{lee2015tawsa}, as well in Section \ref{sec:anth_lit} concerning migration history and Section \ref{sec:toda_group} discussing other linguistic evidence in this thesis. Therefore, given the close relationship between \acl{eto} and \acl{cto}, it is reasonable to infer that \acl{eto} can also be included in the Toda-Truku subgroup.

\begin{figure}[!htbp] 
\centering
\begin{forest}
for tree={l sep=5em, s sep=2em, inner sep=0, anchor=north, parent anchor=south, child anchor=north}
    [Seediq
        [\acl{tg}, tier=word, edge={gray, opacity=0.5}, text={gray, opacity=0.5}]
        [\acl{totr}
            [\acl{to}
                [\acl{cto}, tier=word]
                [\acl{eto}, tier=word]   
            ]
            [\acl{tr}
                [\acl{ctr}, tier=word]
                [\acl{etr}, tier=word]
            ]
        ]
    ]
\end{forest}
\caption{The position of Toda-Truku in Seediq}\label{fig:qhuni_to_tr}
\end{figure}

The phonological evidence supporting the Toda-Truku subgroup is related to changes in the historical schwa *ə. Under different conditions, members of the Toda-Truku subgroup have undergone two sets of the shared sound changes of \acl{psed} *ə.

One of their shared changes is the full assimilation of the penultimate schwa to the following vowel if there is no intervening consonant between them. The rule can be generalized as: \acl{psed} *ə > V\xb{x} /\_V\xb{x}(C)\#. The examples in Table \ref{tab:totrsc1} show the full assimilation of the historical schwa *ə to any following cardinal vowel, such as \acl{psed} *məhəal > \acl{ptotr} *məhaal `to carry on shoulder (AV)', *ləməiŋ > *ləmiiŋ `to hide (AV), and *qəhədəun > *qəhəduun `to end (PV)'. The \acl{etr} is the representative of modern dialects in Table \ref{tab:totrsc1}. 

\begin{table}[!htbp]
\centering
\caption{Full assimilation of *ə in \acl{ptotr}}
\label{tab:totrsc1}
\begin{tabular}{llll}
\hline
\acs{psed}     & \acs{ptotr}    & \acs{etr}   & Gloss                       \\ \hline
*məh\textbf{ə}al   & *məh\textbf{a}al   & məh\textbf{a}al   & `to carry on shoulder (AV)' \\
*məgəc\textbf{ə}as & *məgəc\textbf{a}as & məgəs\textbf{a}as & `brown-ish color'           \\
*cəm\textbf{ə}ax   & *cəm\textbf{a}ax   & səm\textbf{a}ax   & `to chop (AV)'              \\
*məs\textbf{ə}aŋ   & *məs\textbf{a}aŋ   & məs\textbf{a}aŋ   & `to be angry (AV)'          \\
*səməl\textbf{ə}an & *səməl\textbf{a}an & səməl\textbf{a}an & `to make (LV)'              \\
*ləm\textbf{ə}iŋ   & *ləm\textbf{i}iŋ   & ləm\textbf{i}iŋ   & `to hide (AV)'              \\
*qəhəd\textbf{ə}un & *qəhəd\textbf{u}un & qəhəd\textbf{u}un & `to end (PV)'               \\ \hline
\end{tabular}
\end{table}

The other shared change also involves the assimilation of schwa, \acl{psed} *ə will be assimilated as high front vowels *i/*u by the homorganic glides *y/*w. Table \ref{tab:totrsc2} shows the examples of this sound change, such as \acl{psed} *bəyax > \acl{ptotr} *biyax `power', and *wəəwa > *wəuwa `young woman'. 

\begin{table}[!htbp]
\centering
\caption{*ə > *i/*u in \acl{ptotr}}
\label{tab:totrsc2}
\begin{tabular}{llll}
\hline
\acs{psed}     & \acs{ptotr}    & \acs{etr}   & Gloss                       \\ \hline
*b\textbf{ə}yax   & *b\textbf{i}yax   & b\textbf{i}yax   & `power'               \\
*b\textbf{ə}yuq   & *b\textbf{i}yuq   & b\textbf{i}yuq   & `juice'               \\
*məh\textbf{ə}yug & *məh\textbf{i}yug & məh\textbf{i}yug & `to be standing (AV)' \\
*m\textbf{ə}yah   & *m\textbf{i}yah   & m\textbf{i}yah   & `to come (AV)'        \\
*kət\textbf{ə}yun & *kət\textbf{i}yun & kəc\textbf{i}yun & `to pluck (AV)'       \\ 
*wə\textbf{ə}wa & *wə\textbf{u}wa & wa\textbf{u}wa & `young woman'       \\ \hline
\end{tabular}
\end{table}

In the context of Seediq, the distinction between sporadic sound change and lexical innovation can sometimes be blurry. In this thesis, sporadic sound changes do not include the segmental replacements discussed in Chapter \ref{ch6}, but are limited to more rare sound changes that do not fit replacement patterns. On the other hand, lexical innovation encompasses the complete replacement of words, shared semantic innovations or divergences, and partial changes in word form, such as the aforementioned affixation or segmental replacements in Chapter \ref{ch6}. This delineation helps clarify the distinction between the two category.

In the \acl{ptotr}, several sporadic sound changes can be observed, as presented in Table \ref{tab:totr_sp}. For instance, the initial consonant *s of \acl{psed} *sumay `bear' changes to *k in \acl{ptotr}, as seen in *kumay. Additionally, the initial consonant *g of *gəlabaŋ is completely assimilated by the onset of following syllable *l in \acl{ptotr}, resulting in *ləlabaŋ. Such assimilation from *gVl... to *lVl... is unique and not found in other instances. The medial consonant *y of \acl{psed} *ŋayan `name', a regular reflex of \acl{pan} *j, sporadically changed into *h in \acl{ptotr}. In Toda-Truku, both \textit{ŋahan} and the metathesized \textit{haŋan} can be found in different (sub)dialects. Furthermore, the final consonant *k of \acl{psed} *dəhuk `to arrive (AV)' is assimilated to the onset *h [ħ] of the same syllable, turning into *q.

\begin{table}[!htbp]
\centering
\caption{Shared sporadic sound changes in Toda-Truku}
\label{tab:totr_sp}
\begin{tabular}{llllll}
\hline
\ac{psed} & \ac{ptotr} & \ac{cto} & \ac{ctr} & \ac{etr} & Gloss               \\ \hline
*\textbf{s}umay    & *\textbf{k}umay     & \textbf{k}umay    & \textbf{k}umay    & \textbf{k}umay    & `bear'              \\
*\textbf{g}əlabaŋ  & *\textbf{l}əlabaŋ   & \textbf{l}əlabaŋ  & \textbf{l}əlabaŋ  & \textbf{l}əlabaŋ  & `broad; wide'       \\
*ŋa\textbf{y}an & *ŋa\textbf{h}an; *\textbf{h}aŋan   & \textbf{h}aŋan  & \textbf{h}aŋan  & ŋa\textbf{h}an; \textbf{h}aŋan  & `name'       \\
*dəhu\textbf{k}    & *dəhu\textbf{q}     & dəhu\textbf{q}    & dəhu\textbf{q}    & dəhu\textbf{q}    & `to arrive (AV)'    \\
*\textbf{d}əmahug  & *\textbf{c}əmahug   & \textbf{c}əmahu   & \textbf{s}əmahug  & \textbf{s}əmahug  & `to scoop up (AV)'  \\
*\textbf{s}əka\textbf{d}u   & *\textbf{c}əka\textbf{c}u    & \textbf{c}əka\textbf{c}u   & \textbf{s}əka\textbf{s}u   & \textbf{s}əka\textbf{s}u   & `Formosan Michelia' \\
*\textbf{rə}sənaw  & *sənaw     & sənaw    & sənaw    & sənaw    & `man; husband'     \\ \hline    
\end{tabular}
\end{table}

In terms of \acl{psed} *dəmahug `to scoop up (AV)' and *səkadu `Formosan Michelia', there are sporadic instances of coronal obstruents turning into *c, resulting in \acl{ptotr} *cəmahug and *cəkacu respectively. This pattern of *s...d > *c...c is also observed in \acl{psed} *səmudal `old (thing)' > Tgdaya \textit{cumucac} /cumucat/ (also with the final *l > \textit{c} /t/). However, such a change in *səkadu > *cəkacu are uncommon and shared among the reflexes in \acl{cto}, \acl{ctr}, and \acl{etr}, which makes it a good piece of evidence of sporadic change.

Finally, the \acl{psed} *rəsənaw `man; husband' unexpectedly drops its initial syllable *rə- in \acl{ptotr}, resulting in *sənaw. In Toda-Truku, the meaning remains unchanged, still denoting `man; husband'. Contrastingly, Tgdaya differentiates this term into \textit{ruseno} `man (general)' and \textit{seno} `husband', in parallel to the differentiation of *məqədil `woman; wife' into \textit{muqedin} `woman' and \textit{qedin} `wife'. This demonstrates that Tgdaya does not share the innovations found in the Toda-Truku subgroup.

The lexical innovations in the Toda-Truku subgroup are quite complex, involving morphological adjustments such as \acl{psed} *daliŋ `close' > \acl{ptotr} *dalih, *(mə)qədil `woman; wife' > *q<ər>idil (with a sporadic *ə > *i), and *maaw `Litsea cubeba' > *məqərig. Notably, \acl{psed} *papak `foot (general)' in \acl{ptotr} shifted semantically to denote `foot (animal/non-human)'. Simultaneously, there was a borrowing from Atayal *kakay `foot' to signify `foot (human)', undergoing a sporadic \textit{k} > \textit{q} change to become \acl{ptotr} *qaqay.

Additionally, several instances suggest borrowing from Atayal during the \acl{ptotr} period, replacing the original \acl{psed} words. For instance, \ac{ptotr} *diyax `day; time' (cf. \ac{pata} *riʔax `day; time', replacing \ac{psed} *ali) and *təhiyaq `far' (cf. \ac{pata} *tuhiyaq `far', replacing \ac{psed} *təhiya). It is also possible that these borrowings occurred among these dialects through later diffusion.

Lastly, several cases in \acl{psed} reconstructed with the suffix *-\cvc show exclusive shared forms in the Toda-Truku dialects, differing from Tgdaya. However, in the absence of external evidence, it is challenging to ascertain whether the shared forms among the Toda-Truku dialects are shared retentions. These shared forms are only used as supplementary evidence. For example, \acl{psed} *dəgə\cvc\,`narrow' > \acl{ptotr} *dəgəril, *təhə\cvc\, `to migrate' > *təhədil, *bəru\cvc\,`vine basket (woman)' > *bəruŋuy.\footnote{In the Tgdaya dialect, the three forms are \textit{dugehiŋ}, \textit{teheruy}, and \textit{bururu}, respectively.}

The aforementioned lexical innovations and potential candidates are listed in Table \ref{tab:totr_lx} in descending order of evidential strength. The modern dialects are represented by \acl{cto} and \acl{etr}.

\begin{table}[!htbp]
\centering
\caption{Shared lexical innovations in Toda-Truku}
\label{tab:totr_lx}
\begin{tabular}{llllll}
\hline
\ac{psed}     & \ac{ptotr} & \ac{cto} & \ac{etr} & Gloss                & Evidential strength \\ \hline
*daliŋ        & *dalih     & dalih    & dalih    & `close'              & higher              \\
*(mə)qedil    & *qəridil   & qəridil  & qəriɟil  & `woman; wife'        & higher              \\
*papak `foot' & *papak     & papak    & papak    & `animal foot'        & higher              \\
*papak `foot' & *qaqay     & qaqay    & qaqay    & `human foot'         & higher              \\ \hdashline
*ali          & *diyax     & diyax    & ɟiyax    & `day; time'          & lower               \\
*təhiya       & *təhiyaq   & təhiyaq  & təhiyaq  & `far'                & lower               \\
*dəgə\cvc     & *dəgəril   & dəgəril  & dəgəril  & `narrow'             & lower               \\
*təhə\cvc     & *təhədil   & təhədil  & təhəɟil  & `to migrate'         & lower               \\
*bəru\cvc     & *bəruŋuy   & bəruŋuy  & bəruŋuy  & `vine basket (woman)' & lower               \\ \hline
\end{tabular}
\end{table}

Lastly, the members of the Toda-Truku subgroup share a morphosyntactic change involving the loss of the genitive marker *na. This change affects the language in two main ways: (1) possessive structures *A na B `B's A; A of B' have shifted to *A B, and (2) in non-Actor voice sentences, the agent, which was marked by the genitive marker *na in \acl{psed}, is now ∅-marked/unmarked. The loss of \acl{psed} genitive marker *na can also serve as evidence for subgrouping within the Toda-Truku group.

\subsection{Evidence for a Toda subgroup} \label{sec:toda_group}

The Toda subgroup includes two members, \acl{cto} and \acl{eto}. The position of Toda in Seediq is shown in Figure \ref{fig:qhuni_to}. This subgroup is primarily supported by phonological evidence, namely the merger of \acl{psed} non-final *g and *w as \acl{pto} *w. Although this is the only phonological evidence for subgrouping, it is quite substantial. Among all known dialects of Seediq, and even among all the Formosan languages, only \acl{cto} and \acl{eto} exhibit the merger of *g and *w. Coupled with their clear oral history, the connection between the two is strong. 

\begin{figure}[!htbp] 
\centering
\begin{forest}
for tree={l sep=5em, s sep=2em, inner sep=0, anchor=north, parent anchor=south, child anchor=north}
    [Seediq
        [\acl{tg}, tier=word, edge={gray, opacity=0.5}, text={gray, opacity=0.5}]
        [\acl{totr}
            [\acl{to}
                [\acl{cto}, tier=word]
                [\acl{eto}, tier=word]   
            ]
            [\acl{tr}, for tree={edge={gray, opacity=0.5}, text={gray, opacity=0.5}}
                [\acl{ctr}, tier=word]
                [\acl{etr}, tier=word]
            ]
        ]
    ]
\end{forest}
\caption{The position of Toda in Seediq}\label{fig:qhuni_to}
\end{figure}

Table \ref{tab:gw_mgr} displays the *g/*w merger in Toda (represented by \acl{cto}). For instance, \acl{psed} words such as *gaga `that' and *wawa `meat' have merged in \acl{pto} as *wawa `that; meat'. In the change of *gəciluŋ `lake; sea' to \textit{uciluŋ}, the initial syllable underwent a process from *gə- > **wə- > **wu- > \textit{u}- (see Section \ref{sec:psedV_mono}).

\begin{table}[!htbp]
\centering
\caption{The merger of *g and *w in Toda}
\label{tab:gw_mgr}
\begin{tabular}{llll}
\hline
\ac{psed} & \ac{ptotr} & \ac{cto} & Gloss           \\ \hline
*\textbf{g}a\textbf{g}a     & *\textbf{g}a\textbf{g}a      & \textbf{w}a\textbf{w}a     & `that'          \\
*\textbf{gə}ciluŋ  & *\textbf{gə}ciluŋ   & \textbf{u}ciluŋ   & `lake; sea'     \\
*ba\textbf{g}a     & *ba\textbf{g}a      & ba\textbf{w}a     & `hand'          \\
*wawa     & *wawa      & wawa     & `meat'          \\
*wada     & *wada      & wada     & `\textsc{perf}' \\
*rawa     & *rawa      & rawa     & `basket'        \\ \hline
\end{tabular}
\end{table}

As for the phenomenon where *ə tends to become \textit{u} before and after /w/ (< *g) , it seems only stable in cases like the aforementioned initial *gə- > \textit{u}-. It is currently difficult to confirm whether the *ə > \textit{u} in other environments represent shared innovations or are later parallel sound changes. For instance, *təgə- `towards, cardinal number' > \acl{cto} \textit{tu}-, *təgəsa `to teach' > \textit{tusa} \~{} \textit{təsa}, but *dəgə\cvc\,`narrow' > \textit{dəwəril} \~{} \textit{duril}. Therefore, I currently maintain a reserved stance and do not consider this vowel change after the *g, *w merger as a shared innovation between \acl{cto} and \acl{eto}.

Due to the lack of lexical data from \acl{eto}, only one set of shared lexical innovation for the Toda subgroup has been identified so far. Specifically, \acl{psed} *sərəqəmu `corn' was replaced in \acl{pto} by *ləhəŋay. Investigations into the vocabulary of \acl{eto} will rely on further field work in the future.

\subsection{Evidence for a Truku subgroup} \label{sec:truku_sb}

The Truku subgroup consists of two members, \acl{ctr} and \acl{etr}. This subgroup is supported by phonological evidence, sporadic sound changes, and lexical innovations. The position of Truku in Seediq is shown in Figure \ref{fig:qhuni_tr}.

\begin{figure}[!htbp] 
\centering
\begin{forest}
for tree={l sep=5em, s sep=2em, inner sep=0, anchor=north, parent anchor=south, child anchor=north}
    [Seediq
        [\acl{tg}, tier=word, edge={gray, opacity=0.5}, text={gray, opacity=0.5}]
        [\acl{totr}
            [\acl{to}, for tree={edge={gray, opacity=0.5}, text={gray, opacity=0.5}}
                [\acl{cto}, tier=word]
                [\acl{eto}, tier=word]   
            ]
            [\acl{tr}
                [\acl{ctr}, tier=word]
                [\acl{etr}, tier=word]
            ]
        ]
    ]
\end{forest}
\caption{The position of Truku in Seediq}\label{fig:qhuni_tr}
\end{figure}

The first shared sound change in the Truku subgroup involves the merger of \acl{psed} *c and *s as \textit{s} in \acl{cto} and \acl{eto}. Examples include \acl{psed} *camat `animal' becoming \acl{ptr} *samat, and *cakus `camphor' becoming *sakus, and so on. Particularly, words like *macu `millet' > *masu, and *masug > *masug `to divide (AV)' demonstrate this merger in the same environment. Note that the -\textit{c} in \acl{ctr} \textit{samac} `animal', resulting from phonetic affrication (underlying /t/), is unrelated to the original *c and does not participate in the merger. Examples related to the *c/*s merger can be seen in Table \ref{tab:trcs}.

\begin{table}[!htbp]
\centering
\caption{The merger of *c and *s in Truku}
\label{tab:trcs}
\begin{tabular}{llllll}
\hline
\ac{psed}      & \ac{ptotr}   & \ac{ptr}     & \acs{ctr}   & \acs{etr}   & Gloss               \\ \hline
(*səkadu)  & *\textbf{c}əka\textbf{c}u  & *\textbf{s}əka\textbf{s}u  & \textbf{s}əka\textbf{s}u  & \textbf{s}əka\textbf{s}u  & `Formosan Michelia' \\
(*dəmahug) & *\textbf{c}əmahug & *\textbf{s}əmahug & \textbf{s}əmahug & \textbf{s}əmahug & `to scoop up (AV)'       \\
*\textbf{c}amat     & *\textbf{c}amat   & *\textbf{s}amat   & \textbf{s}amac   & \textbf{s}amat   & `animal'            \\
*\textbf{c}aku\textbf{s}     & *\textbf{c}aku\textbf{s}   & *\textbf{s}aku\textbf{s}   & \textbf{s}aku\textbf{s}   & \textbf{s}aku\textbf{s}   & `camphor tree'      \\
*ma\textbf{c}u      & *ma\textbf{c}u    & *ma\textbf{s}u    & ma\textbf{s}u    & ma\textbf{s}u    & `millet'    \\
*ma\textbf{s}ug     & *ma\textbf{s}ug   & *ma\textbf{s}ug   & ma\textbf{s}ug   & ma\textbf{s}ug   & `to divide (AV)'    \\ \hline
\end{tabular}
\end{table}

The second shared sound change within the Truku subgroup is the palatalization of \acl{psed} *t and *d before the high front vowel *i, resulting in \acl{ptr} *c [ʨ] and *ɟ [ʥ\~{}ɟ]. Examples include *təyaquŋ `crow' > *ciyaquŋ, *timu `salt' > *cimu, *quti `faeces' > *quci, *dəyagun `to help (PV)' > *ɟiyagun, *diyan `daytime' > *ɟiyan, and *budi `bow and arrow' > *buɟi, as demonstrated in Table \ref{tab:trtdpal}.

\begin{table}[!htbp]
\centering
\caption{The palatalization of *t and *d in Truku}
\label{tab:trtdpal}
\begin{tabular}{llllll}
\hline
\ac{psed}      & \ac{ptotr}   & \ac{ptr}     & \acs{ctr}   & \acs{etr}   & Gloss               \\ \hline
*\textbf{t}əyaquŋ  & *\textbf{t}əyaquŋ  & *\textbf{c}iyaquŋ  & \textbf{c}iyaquŋ   & \textbf{c}iyaquŋ   & `crow'          \\
*\textbf{t}imu  & *\textbf{t}imu  & *\textbf{c}imu  & \textbf{c}imu   & \textbf{c}imu   & `salt'          \\
*qu\textbf{t}i  & *qu\textbf{t}i  & *qu\textbf{c}i  & qu\textbf{c}i   & qu\textbf{c}i   & `faeces'        \\
*\textbf{d}əyagun & *\textbf{d}əyagun & *\textbf{ɟ}iyagun & \textbf{ɟ}iyagun  & \textbf{ɟ}iyagun  & `to help (PV)'       \\
*\textbf{d}iyan & *\textbf{d}iyan & *\textbf{ɟ}iyan & \textbf{ɟ}iyan  & \textbf{ɟ}iyan  & `daytime'       \\
*bu\textbf{d}i  & *bu\textbf{d}i  & *bu\textbf{ɟ}i  & bu\textbf{ɟ}i   & bu\textbf{ɟ}i   & `bow and arrow' \\ \hline
\end{tabular}
\end{table}

In the examples mentioned, such as \acl{ptr} *ciyaquŋ `crow' and *ɟiyagun `to help (PV)', in the colloquial speech of modern dialects, they may further simplify to bisyllabic words like \textit{cyaquŋ} [ˈʨja.quŋ], \textit{ɟyagun} [ˈʥja.ɣun] with a CGV.CVC structure.

The palatalization of *t, *d must occurred after the merger of *c and *s, as we do not see a change like \acl{psed} *timu `salt' > **cimu > **simu. Moreover, although \acl{eto} also exhibits these two sound changes, according to \textcite{lee2015tawsa}, these sound changes were influenced by the contact with \acl{etr}. I have no reason to include \acl{eto} in the Truku subgroup based on this evidence. 

The members of Truku also share some sporadic sound changes, such as the initial consonant deletion in \acl{psed} *nunuh `breast' > \acl{ptr} *unuh, the nasal insertion at the end of *saya `now; today' > *sayaŋ, the sporadic *ŋi > *ni in *məŋihur `spicy' > *mənihur, the unexpected complete assimilation in *puru `axe' > *pupu, and the consonant changes in *məluk \~{} *ləban /əlub/ `to close' > *məduk \~{} *dəpan /ədup/. Additionally, in some words, the diphthong *aw following a penultimate coronal sound in \acl{psed} became *u in \acl{ptr}, as seen in *sawni `just' > *suni, *rawdux `chicken' > *rudux, and *cəlawkah `light (thing)' > *səlukah. However, some words like *dawriq `eye' remain unchanged as *dawriq. Relevant examples are displayed in Table \ref{tab:tr_sp}.

\begin{table}[!hbtp]
\centering
\caption{Shared sporadic sound changes in Truku}
\label{tab:tr_sp}
\begin{tabular}{llllll}
\hline
\ac{psed}      & \ac{ptotr}   & \ac{ptr}     & \acs{ctr}   & \acs{etr}   & Gloss               \\ \hline
*\textbf{n}unuh  & *\textbf{n}unuh  & *unuh    & unuh    & unuh    & `breast'        \\
*saya   & *saya   & *saya\textbf{ŋ}   & saya\textbf{ŋ}   & saya\textbf{ŋ}   & `today'         \\
*mə\textbf{ŋ}ihur& *mə\textbf{ŋ}ihur& *mə\textbf{n}ihur & mə\textbf{n}ihur & mə\textbf{n}ihur & `spicy'        \\
*pu\textbf{r}u   & *pu\textbf{r}u   & *pu\textbf{p}u    & pu\textbf{p}u    & pu\textbf{p}u    & `axe'           \\
*mə\textbf{l}uk  & *mə\textbf{l}uk  & *mə\textbf{d}uk   & mə\textbf{d}uk   & mə\textbf{d}uk   & `to close (AV)' \\
*\textbf{l}ə\textbf{b}an  & *\textbf{l}ə\textbf{b}an  & *\textbf{d}ə\textbf{p}an   & \textbf{d}ə\textbf{p}an   & \textbf{d}ə\textbf{p}an   & `to close (LV)' \\ 
*s\textbf{aw}ni  & *s\textbf{aw}ni  & *s\textbf{u}ni    & s\textbf{u}ni    & s\textbf{u}ni    & `just; now'     \\
*r\textbf{aw}dux & *r\textbf{aw}dux & *r\textbf{u}dux   & r\textbf{u}dux   & r\textbf{u}dux   & `chicken'       \\ 
*cəl\textbf{aw}kah & *cəl\textbf{aw}kah & *səl\textbf{u}kah & səl\textbf{u}kah & (ləhəkah) & `light (thing)'\\ \hline
\end{tabular}
\end{table}

Lastly, I found three shared lexical innovations among Truku subgroup members. For example, \acl{psed} *ariŋ `peach' was replaced with \acl{ptr} *qəlupas, *kəduruk `forehead' was replaced with *pəŋəlux, and *ruma `inside' was replaced by *ruwan (the final example likely involves syllable substitution). Relevant examples are shown in Table \ref{tab:tr_lx}.

\begin{table}[!hbtp]
\centering
\caption{Shared lexical innovations in Truku}
\label{tab:tr_lx}
\begin{tabular}{llllll}
\hline
\ac{psed}      & \ac{ptotr}   & \ac{ptr}     & \acs{ctr}   & \acs{etr}   & Gloss               \\ \hline
*ariŋ   & *ariŋ   & *qəlupas & qəlupas & qəlupas & `peach'         \\
*kəduruk& *kəduruk& *pəŋəlux & pəŋəlux & pəŋəlux & `forehead'      \\
*ruma  & *ruma   & *ruwan   & ruwan   & ruwan   & `inside'        \\ \hline
\end{tabular}
\end{table}

The two Truku dialects have nearly the same phonological system, except some minor phonetic differences. The crucial evidence that distinguishes \acl{etr} from \acl{ctr} is lexical innovations, which involve mainly basic words, such as `woman', `fire', `many', etc, as shown in Table \ref{tab:etrlex}.

\begin{table}[!hbtp]
\centering
\caption{Lexical innovations in \acl{etr}}
\label{tab:etrlex}
\begin{tabular}{llllll}
\hline
\ac{psed}      & \ac{ptotr}   & \ac{ptr}     & \acs{ctr}   & \acs{etr}   & Gloss               \\ \hline
*(mə)qədil & *qəridil & *qəriɟil & qəriɟil & kuyuh   & `woman'         \\
*puniq   & *puniq   & *puniq   & puniq   & tahut   & `fire'          \\
*əgu     & *əgu     & *əgu     & əgu     & lala    & `many'          \\
*tuya    & *tuya    & *tuya    & tuya    & tuwil   & `eel'           \\
*cilu    & *cilu    & *silu    & silu    & təqətaq & `gecko'         \\ \hline
\end{tabular}
\end{table}

\section{Summary}

I have proposed a Seediq subgrouping hypothesis based on the Comparative Method and shared innovations among each subgroup. Simply put, Tgdaya is the most divergent dialect, not sharing any crucial phonological changes with other dialects and having its unique sound changes. Other dialects can be grouped together based on historical changes of *ə, sporadic sound changes, lexical innovations, and a syntactic change. At first glance, my subgrouping hypothesis appears similar to that mentioned in Section \ref{sec:cheng_sb} of Cheng, Nanu Ruwiq, et al. (2017) and Cheng, Dakis Pawan, et al. (2019). However, my approach is based on shared innovations for subgrouping, rather than on mutual intelligibility or shared retentions.