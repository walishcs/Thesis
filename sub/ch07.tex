\chapter{Seediq subgrouping}\label{ch7}

% 在這一章中,我會propose一個 subgrouping hypothesis of Seediq dialects 同時基於 phonological, morphosyntactic, and lexical evidence. Phonological 的證據中,historical schwa *ə 的split-merger以及在某些詞項中的 sporadic sound changes扮演了最重要的角色;morphosyntactic 的證據主要是與格位標記有關;而 lexical evidence 則涉及到純粹的 lexical innovation 以及在 Chapter \ref{ch6} 中所提到的可能共同創新。

In this chapter, I will propose a subgrouping hypothesis of Seediq dialects based on phonological, morphosyntactic, and lexical evidence. Phonologically, the most crucial evidence includes the development of the historical schwa *ə and sporadic sound changes found in certain lexical items. The morphosyntactic evidence primarily relates to the case marking system, while the lexical evidence involves pure lexical innovation and potential shared innovations discussed in Chapter \ref{ch6}.

\section{Subgrouping hypothesis}

The subgrouping of Seediq dialects based on the phonological and lexical evidence is presented in Figure \ref{fig:qhuni}. 

\begin{figure}[!htbp] 
\centering
\begin{forest}
for tree={l sep=5em, s sep=2em, inner sep=0, anchor=north, parent anchor=south, child anchor=north}
    [Seediq
        [\acl{tg}, tier=word]
        [\acl{totr}
            [\acl{to}
                [\acl{cto}, tier=word]
                [\acl{eto}, tier=word]   
            ]
            [\acl{tr}
                [\acl{ctr}, tier=word]
                [\acl{etr}, tier=word]
            ]
        ]
    ]
\end{forest}
\caption{Seediq subgrouping tree}\label{fig:qhuni}
\end{figure}

There is a subgroup of Toda-Truku, which is composed of \acl{cto}, \acl{eto}, \acl{ctr}, and \acl{etr}. Moreover, \acl{cto} and \acl{eto} form a Toda subgroup; while \acl{ctr} and \acl{etr} form a Truku subgroup under Toda-Truku. 

\acl{tg} forms a primary branch by itself because there is no evidence to link \acl{tg} with the other dialects. The position of the \acl{tg} dialect in Seediq is shown in Figure \ref{fig:qhuni_tg}. 

\begin{figure}[!htbp] 
\centering
\begin{forest}
for tree={l sep=5em, s sep=2em, inner sep=0, anchor=north, parent anchor=south, child anchor=north, edge={black, thick}}
    [Seediq
        [\acl{tg}, tier=word]
        [\acl{totr}, for tree={edge={gray, opacity=0.5}, text opacity=0.5} % This applies gray color to all edges under \acl{totr}
            [\acl{to}
                [\acl{cto}, tier=word]
                [\acl{eto}, tier=word]   
            ]
            [\acl{tr}
                [\acl{ctr}, tier=word]
                [\acl{etr}, tier=word]
            ]
        ]
    ]
\end{forest}
\caption{The position of \acl{tg} in Seediq}\label{fig:qhuni_tg}
\end{figure}

Moreover, \acl{tg} itself has some phonological changes which are very unique among Seediq dialects, as listed below:

\begin{enumerate}
    \item \textbf{*ə (penultimate) and *ay merged as \textit{e}.} The historical schwa *ə in the penultimate syllable became \textit{e} in \acl{tg}, and the diphthong *ay monophthongized into an \textit{e} as well: \acl{psed} *məgay > \acl{tg} \textit{mege} `to give (AV)', \acl{psed} *maytaq > \acl{tg} \textit{metaq} `to stab (AV)'.
    \item \textbf{*ə (prepenultimate) and *u merged as \textit{u}.}: The historical schwa *ə in the prepenultimate syllables (the weakened/neutralized vowel) became \textit{u} in \acl{tg}: \acl{psed} *dəqəras > \acl{tg} \textit{duqeras} `face'. 
    \item \textbf{*-uy and *-ig merged as \textit{uy}.} The sequence *-ig changed to -\textit{uy} and merged with the original *-uy in \acl{tg}. \acl{psed} *marig > \acl{tg} \textit{maruy} `to buy (AV)', \acl{psed} *babuy > \acl{tg} \textit{babuy} `pig'.
\end{enumerate}


\section{Evidence for a Toda-Truku subgroup}

The Toda-Truku subgroup is supported by (1) shared sound changes, (2) sporadic changes, (3) lexical innovations, and (4) morphosyntactic changes. Any one of these occurring individually could be due to dialect contact or coincidence. However, when these four types of shared changes occur together, the likelihood of establishing this subgroup is significantly increased. This section will discuss examples of each type of shared change. 

\begin{figure}[!htbp] 
\centering
\begin{forest}
for tree={l sep=5em, s sep=2em, inner sep=0, anchor=north, parent anchor=south, child anchor=north}
    [Seediq
        [\acl{tg}, tier=word, edge={gray, opacity=0.5}, text={gray, opacity=0.5}]
        [\acl{totr}
            [\acl{to}
                [\acl{cto}, tier=word]
                [\acl{eto}, tier=word]   
            ]
            [\acl{tr}
                [\acl{ctr}, tier=word]
                [\acl{etr}, tier=word]
            ]
        ]
    ]
\end{forest}
\caption{The position of Toda-Truku in Seediq}\label{fig:qhuni_to_tr}
\end{figure}

The phonological evidence supporting the Toda-Truku subgroup is related to changes in the historical schwa *ə. Under different conditions, members of the Toda-Truku subgroup have undergone two sets of the shared sound changes of \acl{psed} *ə.

One of their shared changes is the full assimilation of the penultimate schwa to the following vowel if there is no intervening consonant between them. The rule can be generalized as: \psed *ə > V\xb{x} /\_V\xb{x}(C)\#. The examples in Table \ref{tab:totrsc1} show the full assimilation of the historical schwa *ə to any following cardinal vowel, such as \acl{psed} *məhəal > \acl{ptotr} *məhaal `to carry on shoulder (AV)', *ləməiŋ > *ləmiiŋ `to hide (AV), and *qəhədəun > *qəhəduun `to end (PV)'. The \acl{etr} is the representative of modern dialects in Table \ref{tab:totrsc1}. 

\begin{table}[!htbp]
\centering
\caption{Full assimilation of *ə in \acl{ptotr}}
\label{tab:totrsc1}
\begin{tabular}{llll}
\hline
\acs{psed}     & \acs{ptotr}    & \acs{etr}   & Gloss                       \\ \hline
*məh\textbf{ə}al   & *məh\textbf{a}al   & məh\textbf{a}al   & `to carry on shoulder (AV)' \\
*məgəc\textbf{ə}as & *məgəc\textbf{a}as & məgəs\textbf{a}as & `brown-ish color'           \\
*cəm\textbf{ə}ax   & *cəm\textbf{a}ax   & səm\textbf{a}ax   & `to chop (AV)'              \\
*məs\textbf{ə}aŋ   & *məs\textbf{a}aŋ   & məs\textbf{a}aŋ   & `to be angry (AV)'          \\
*səməl\textbf{ə}an & *səməl\textbf{a}an & səməl\textbf{a}an & `to make (LV)'              \\
*ləm\textbf{ə}iŋ   & *ləm\textbf{i}iŋ   & ləm\textbf{i}iŋ   & `to hide (AV)'              \\
*qəhəd\textbf{ə}un & *qəhəd\textbf{u}un & qəhəd\textbf{u}un & `to end (PV)'               \\ \hline
\end{tabular}
\end{table}

The other shared change also involves the assimilation of schwa, \acl{psed} *ə will be assimilated as high front vowels *i/*u by the homorganic glides *y/*w. Table \ref{tab:totrsc2} shows the examples of this sound change, such as \acl{psed} *bəyax > \acl{ptotr} *biyax `power', and *wəəwa > *wəuwa `young woman'. 

\begin{table}[!htbp]
\centering
\caption{*ə > *i/*u in \acl{ptotr}}
\label{tab:totrsc2}
\begin{tabular}{llll}
\hline
\acs{psed}     & \acs{ptotr}    & \acs{etr}   & Gloss                       \\ \hline
*b\textbf{ə}yax   & *b\textbf{i}yax   & b\textbf{i}yax   & `power'               \\
*b\textbf{ə}yuq   & *b\textbf{i}yuq   & b\textbf{i}yuq   & `juice'               \\
*məh\textbf{ə}yug & *məh\textbf{i}yug & məh\textbf{i}yug & `to be standing (AV)' \\
*m\textbf{ə}yah   & *m\textbf{i}yah   & m\textbf{i}yah   & `to come (AV)'        \\
*kət\textbf{ə}yun & *kət\textbf{i}yun & kəc\textbf{i}yun & `to pluck (AV)'       \\ 
*wə\textbf{ə}wa & *wə\textbf{u}wa & wa\textbf{u}wa & `young woman'       \\ \hline
\end{tabular}
\end{table}

\subsection{Evidence for a Toda subgroup}

\begin{figure}[!htbp] 
\centering
\begin{forest}
for tree={l sep=5em, s sep=2em, inner sep=0, anchor=north, parent anchor=south, child anchor=north}
    [Seediq
        [\acl{tg}, tier=word, edge={gray, opacity=0.5}, text={gray, opacity=0.5}]
        [\acl{totr}
            [\acl{to}
                [\acl{cto}, tier=word]
                [\acl{eto}, tier=word]   
            ]
            [\acl{tr}, for tree={edge={gray, opacity=0.5}, text={gray, opacity=0.5}}
                [\acl{ctr}, tier=word]
                [\acl{etr}, tier=word]
            ]
        ]
    ]
\end{forest}
\caption{The position of Toda in Seediq}\label{fig:qhuni_to}
\end{figure}

\lipsum[1-7]
\subsection{Evidence for a Truku subgroup}

\begin{figure}[!htbp] 
\centering
\begin{forest}
for tree={l sep=5em, s sep=2em, inner sep=0, anchor=north, parent anchor=south, child anchor=north}
    [Seediq
        [\acl{tg}, tier=word, edge={gray, opacity=0.5}, text={gray, opacity=0.5}]
        [\acl{totr}
            [\acl{to}, for tree={edge={gray, opacity=0.5}, text={gray, opacity=0.5}}
                [\acl{cto}, tier=word]
                [\acl{eto}, tier=word]   
            ]
            [\acl{tr}
                [\acl{ctr}, tier=word]
                [\acl{etr}, tier=word]
            ]
        ]
    ]
\end{forest}
\caption{The position of Truku in Seediq}\label{fig:qhuni_tr}
\end{figure}

\lipsum[1-7]

\section{Summary}
\lipsum[1-2]