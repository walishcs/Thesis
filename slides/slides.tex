\PassOptionsToPackage{table, xcdraw, dvipsnames}{xcolor}

\documentclass[11pt, presentation, aspectratio=169]{beamer}
% \documentclass[11pt, handout, aspectratio=169]{beamer}
\usetheme{yeast}

\usepackage{lipsum}
\usepackage{tabularx}
\renewcommand{\arraystretch}{2}
\renewcommand\tabularxcolumn[1]{m{#1}}

\usepackage{appendixnumberbeamer}
\setbeamertemplate{frametitle continuation}{}

\newcommand \albf[1]{\textbf{\alert{#1}}}

% defs
\def\cmd#1{\texttt{\color{red}\footnotesize $\backslash$#1}}
\def\env#1{\texttt{\color{blue}\footnotesize #1}}
\definecolor{db}{rgb}{0,0,0.5}
\definecolor{dr}{rgb}{0.6,0,0}
\definecolor{dg}{rgb}{0,0.5,0}
\definecolor{halfgray}{gray}{0.55}
\definecolor{hl}{HTML}{fafe7b}

\newcommand<>\hlbox[2]{%
  \alt#3{\makebox[\dimexpr\width-1\fboxsep]{\colorbox{#1}{#2}}}{#2}%
}

% Allow for overlay of cellcolor function
\renewcommand<>\cellcolor[1]{\only#2{\beameroriginal\cellcolor{#1}}}
\renewcommand<>\rowcolor[1]{\only#2{\beameroriginal\rowcolor{#1}}}

\usepackage{xspace}
\newcommand{\pan}{\textsc{PAn}\xspace}
\newcommand{\paic}{\textsc{PAic}\xspace}
\newcommand{\pata}{\textsc{PAta}\xspace}
\newcommand{\ata}{\textsc{Ata}\xspace}
\newcommand{\psed}{\textsc{PSed}\xspace}
\newcommand{\sed}{\textsc{Sed}\xspace}
\newcommand{\psedf}{Proto-Seediq\xspace}
\newcommand{\stg}{\textsc{Tg}\xspace}
\newcommand{\stgf}{Tgdaya\xspace}
\newcommand{\sto}{\textsc{To}\xspace}
\newcommand{\stof}{Toda\xspace}
\newcommand{\sctr}{\textsc{CTr}\xspace}
\newcommand{\sctrf}{Central Truku\xspace}
\newcommand{\setr}{\textsc{ETr}\xspace}
\newcommand{\setrf}{Eastern Truku\xspace}

\addbibresource{ref.bib}

\setbeamertemplate{headline}[progress bar]


\title[Seediq: Phonology, Reconstruction, and Subgrouping]{\textbf{Seediq: Phonology, Reconstruction, and Subgrouping}\\\large{~}\\\large{\textit{Kari Seediq: Snlhayan hengak, Bnrahan smmalu, \\ma Pnssugan qpuruh}}}
\author[W. Song]{Walis Hian-chi Song\\宋硯之}
\institute[NTHU]{國立清華大學語言學研究所}
\date{October 23, 2023}

\begin{document}

\maketitle

\begin{frame}{Outline}
  \tableofcontents
\end{frame}

\section{Introduction}

\begin{frame}{Abbreviations}
    \begin{itemize}
        \item Proto-languages
        \begin{itemize}
        \item \paic = Proto-Atayalic
        \item \pan = Proto-Austronesian
        \item \pata = Proto-Atayal
        \item \psed = \psedf
        \end{itemize}
    \end{itemize}
        \begin{itemize}
        \item Seediq dialects
        \begin{itemize}
            \item \sctr = \sctrf
            \item \setr = \setrf
            \item \stg = \stgf
            \item \sto = \stof
        \end{itemize}
    \end{itemize}
\end{frame}

\begin{frame}{Basic introduction - 1}
    \begin{itemize}
        \item<1-> Seediq is an Austronesian language spoken in Central and Eastern mountainous area of Taiwan. 
        \item<2-> Together with Atayal, they form an Atayalic subgroup (``self-evident''), which is considered to be a primary branch of Austronesian family (\cite{blust1999subgrouping}). 
        \item<3-> Four dialects will be included: Tgdaya (德固達雅), Toda (都達), Central Truku (德鹿谷) and Eastern Truku (太魯閣).
    \end{itemize}
\end{frame}

\begin{frame}{Basic introduction - 2}
    \begin{figure}
        \centering
        \includegraphics[height=0.9\textheight]{images/map.pdf}
    \end{figure}
\end{frame}

\begin{frame}{Methodology - 1}{The Comparative Method}
\begin{itemize}
    \item<1-> This paper employs the Comparative Method in historical linguistics to reconstruct Proto-Seediq and classify Seediq dialects, following essential steps (\cite{fox1995}).
    \item<2-> \albf{Step 1:} Collecting words and identify cognates among Seediq dialects, while ruling out thoses with chance, universals, and borrowing as explanations.
    \begin{itemize}
        \item<3-> \albf{Cognates:} words in basic vocabulary that are similar or same in both form and meaning, showing regular sound correspondences.
    \end{itemize}
    \item<4-> \albf{Step 2:} Setting up sound correspondences and group the correspondence sets together based on phonetic similarity.
\end{itemize}
\end{frame}

\begin{frame}{Methodology - 2}{The Comparative Method}
\begin{itemize}
    \item<1-> \albf{Step 3:} Reconstructing the protophonemes and assign their phonetic values.
    \begin{itemize}
        \item<2-> sound change should be plausible and natural in language developments, and regular
        \item<3-> a reconstructed phoneme should minimize changes in both place and manner of articulation
    \end{itemize}
    \item<4-> \albf{Step 4:} Reconstructing the lexicon based on the sound correspondences and protophonemes.
\end{itemize}
\end{frame}

\begin{frame}{Methodology - 3}{The Comparative Method}
\begin{itemize}
    \item<1-> \albf{Step 5:} Determining a subgrouping of the Seediq language.
    \begin{itemize}
        \item<2-> based on shared innovations, not retentions
        \item<3-> sporadic/rare SCs are better than common ones
        \item<4-> ``drift'' --- common sound changes occurring individually in closely related languages/dialects --- must be ruled out
        \item<5-> \albf{Phonological evidence:} Mergers, conditioned shifts/splits, etc.
        \item<6-> \albf{Lexical evidence:} Shared lexical innovations (including exclusive ``word-modifications'' such as affixing, partial replacements on words. See the discussion of ``Special doublet phenomenon'')
    \end{itemize}
\end{itemize}
\end{frame}

\begin{frame}{Methodology - 4}{Top-down reconstruction method}
    \begin{itemize}
        \item<1-> Is also called ``inverted reconstruction'' (\cites[512--16]{hockett1958course}[346]{anttila1972introduction})
        \item<2-> If there is a relevant reconstruction in language X's ancestral language(s) based on other languages, we can use those forms to reconstruct the form's in X \albf{from above} (\cite[88]{fox1995}).
        \item<3-> In this paper, evidence from \pan can be considered. 
        \item<4-> However, Seediq (and Atayal) has numerous lexical innovations from \pan, so I have to highly depend on evidence from Atayal. 
        \item<5-> For example, \stg \albf{\textit{ali}}, \sto \albf{\textit{diyax}}, \sctr and \setr \albf{\textit{ɟiyax}} `day; time'. 
            \begin{itemize}
                \item<6-> \psed can be *ali or *diyax
                \item<7-> \pata *riʔax `day; time'
                \item<8-> \psed → \albf{*diyax}
            \end{itemize}
    \end{itemize}
\end{frame}

\begin{frame}{Data sources}
\begin{table}[!htbp]
    \centering
    \resizebox{0.99\textwidth}{!}{%
    \begin{tabular}{l|llll}
       & Wordlist     & Database/Dictionary & Dictionary & Dictionary   \\ \hline
    \stg  & \textcite{ILRDF1000words} & \textcite{Seddatabase}   & \textcite{ILRDFEdict} & \textcite{watandiro2009seddict} \\
    \sto  & \textcite{ILRDF1000words} & \textcite{Seddatabase}   &              & \textcite{watandiro2009seddict} \\
    \sctr & \textcite{ILRDF1000words} & \textcite{Seddatabase}   &              & \textcite{watandiro2009seddict} \\
    \setr & \textcite{ILRDF1000words} &                     & \textcite{ILRDFEdict} &
    \end{tabular}%
    }
    \end{table}
- \pan (ACD, \cite{ACD}); \pata (\cite{goderich2020phd})\\
- Seediq Assoc. = Seediq Language and Culture Sustainable Development Association
\end{frame}

\begin{frame}{Orthographic conventions}
    \begin{table}[]
    \centering
    \resizebox{0.9\textwidth}{!}{%
    \begin{tabular}{l|ccccccc}
    \textbf{This paper} & c     & ɟ & h   & ŋ   & r   & l          & y   \\ \hline
    \textbf{CIP/MOE}    & c     & j & h   & ng  & r   & l          & y   \\
    \textbf{IPA}        & [ts\~{}tɕ] & [ɟ\~{}dʑ]  & [ħ] & [ŋ] & [ɾ] & [l\~{}ɮ] & [j]
    \end{tabular}%
    }
    \end{table}
    \begin{itemize}
        \item {<}r> has some other variants, such as [ɾ], [ɹ], [ɻ], [ɺ], [r], etc. The pronunciation depends on dialect or individual.
        \item Here I only listed those symbol that different from IPA. 
    \end{itemize}
\end{frame}

\section{Proto-Seediq Phonology}

\subsection{Phoneme inventory}

\begin{frame}{Consonants}
    \begin{table}[]
\centering
\resizebox{\textwidth}{!}{%
\begin{tabular}{l|cc|cc|cc|cc|cc|cc}
                    & \multicolumn{2}{c|}{Bilabial} & \multicolumn{2}{c|}{Dental/Alveolar} & \multicolumn{2}{c|}{Palatal} & \multicolumn{2}{c|}{Velar} & \multicolumn{2}{c|}{Uvular} & \multicolumn{2}{c}{Pharyngeal} \\ \hline
Stop                & *p            & *b           & *t               & *d               &             &               & *k          & *g          & *q            &            &                 &              \\
Affricate           &               &              & *c               &                  &             &               &             &             &               &            &                 &              \\
Fricative           &               &              & *s               &                  &             &               & *x          &             &               &            & *h              &              \\
Nasal               &               & *m           &                  & *n               &             &               &             & *ŋ          &               &            &                 &              \\
Lateral  &               &              &                  & *l               &             &               &             &             &               &            &                 &              \\
Tap                 &               &              &                  & *r               &             &               &             &             &               &            &                 &              \\
Glide               &               & *w           &                  &                  &             & *y            &             &             &               &            &                 &             
\end{tabular}%
}
\end{table}
\end{frame}

\begin{frame}{Vowels}
\begin{table}[]
\centering
\resizebox{0.6\textwidth}{!}{%
\begin{tabular}{c|ccc}
     & Front & Central & Back \\ \hline
High & *i    &         & *u   \\
Mid  &       & *ə      &      \\
Low  &       & *a      &     
\end{tabular}%
}
\end{table}

\begin{center}
``diphthongs'' (VG sequences): *aw, *ay and *uy.
\end{center}

\end{frame}

\subsection{Regular sound correspondences}

\begin{frame}{Stops}

\begin{minipage}[t]{0.48\linewidth}
\begin{table}[]
\centering
\resizebox{0.65\textheight}{!}{%
\begin{tabular}{c|cccc|l}
\hline
\textbf{\psed}      & \textbf{\stg} & \textbf{\sto} & \textbf{\sctr} & \textbf{\setr} & \multicolumn{1}{c}{Env.} \\ \hline
*p                  & p             & p             & p              & p              &                          \\ \hline
\multirow{3}{*}{*t} & t             & t             & \cellcolor<2>{hl} c              & \cellcolor<2>{hl} c              & /\_i, /\_əy              \\
                    & \cellcolor<3>{hl} c             & \cellcolor<3>{hl} c             & \cellcolor<3>{hl} c              & t              & /\_\#                    \\
                    & t             & t             & t              & t              & /elsewhere               \\ \hline
*k                  & k             & k             & k              & k              &                          \\ \hline
*q                  & q             & q             & q              & q              &                         \\ \hline
\end{tabular}%
}
\end{table}
\end{minipage}\hfill
\begin{minipage}[t]{0.48\linewidth}
\begin{table}[]
\centering
\resizebox{0.68\textheight}{!}{%
\begin{tabular}{c|cccc|l}
\hline
\textbf{\psed} & \textbf{\stg} & \textbf{\sto} & \textbf{\sctr} & \textbf{\setr} & \multicolumn{1}{c}{Env.}                         \\ \hline
*b             & b             & b             & b              & b              &                          \\ \hline
\multirow{2}{*}{*d}& d             & d             & \cellcolor<2>{hl} ɟ              & \cellcolor<2>{hl} ɟ              & /\_i, /\_əy              \\
               & d             & d             & d              & d              & /elsewhere               \\ \hline
*g-/-g-        & g             & \cellcolor<4>{hl} w             & g              & g              &                          \\
*-ig           & \cellcolor<5>{hl} uy            & \cellcolor<6>{hl} i             & ig             & ig             &                          \\
*-ug           & \cellcolor<7>{hl} u             & \cellcolor<7>{hl} u             & ug             & ug             &                         \\ \hline
\end{tabular}%
}
\end{table}
\end{minipage}
\end{frame}

\begin{frame}{Affricate \& Fricatives}
    \begin{table}[]
\centering
\resizebox{0.75\textheight}{!}{%
\begin{tabular}{c|cccc|c}
\hline
\textbf{\psed} & \textbf{\stg} & \textbf{\sto} & \textbf{\sctr} & \textbf{\setr} & Env. \\ \hline
*c             & c             & c             & \cellcolor<2>{hl} s              & \cellcolor<2>{hl} s              &      \\ \hline
*s             & s             & s             & \cellcolor<2>{hl} s              & \cellcolor<2>{hl} s              &      \\ \hline
*x             & x             & x             & x              & x              &      \\ \hline
*h             & h             & h             & h              & h              &      \\ \hline
\end{tabular}%
}
\end{table}
\end{frame}

\begin{frame}{Nasals, Liquids, Glides}
\begin{table}[]
\centering
\resizebox{0.7\textheight}{!}{%
\begin{tabular}{c|cccc|l}
\hline
\textbf{\psed} & \textbf{\stg} & \textbf{\sto} & \textbf{\sctr} & \textbf{\setr} & \multicolumn{1}{c}{Env.} \\ \hline
*m             & m             & m             & m              & m              &      \\ \hline
*n             & n             & n             & n              & n              &      \\ \hline
*ŋ             & ŋ             & ŋ             & ŋ              & ŋ              &      \\ \hline
\multirow{2}{*}{*r}             & \cellcolor<2>{hl} n             & r             & r              & r              &  /\_\#    \\ 
               & r             & r             & r              & r              & /elsewhere     \\  \hline
\multirow{2}{*}{*l}             & \cellcolor<2>{hl} n             & l             & l              & l              &  /\_\#    \\ 
               & l             & l             & l              & l              & /elsewhere     \\  \hline
*w-/-w-  &  w & w & w & w \\ \hline
*y-/-y-  & y & y & y & y \\ \hline
\end{tabular}%
}
\end{table}
\end{frame}


\begin{frame}{Vowels - 1}
    \begin{table}[]
\centering
\resizebox{0.95\textwidth}{!}{%
\begin{tabular}{c|c|cccc|l}
\hline 
\textbf{\psed}         & \textbf{\psed word} & \textbf{\stg}                        & \textbf{\sto}      & \textbf{\sctr}                                                & \textbf{\setr}                       & \multicolumn{1}{c}{Env./Gloss} \\ \hline
*a                     &                     & a                                    & a                  & a                                                             & a                                    &                                \\ \hline
*i                     &                     & i                                    & i                  & i                                                             & i                                    &                                \\ \hline
*u                     &                     & u                                    & u                  & u                                                             & u                                    &                                \\ \hline
\multirow{3}{*}{*VxVx} &                     & V\textsubscript{x}V\textsubscript{x} & \cellcolor<2>{hl} V\textsubscript{x} & \cellcolor<3>{hl} V\textsubscript{x}V\textsubscript{x} \~{ } V\textsubscript{x} & V\textsubscript{x}V\textsubscript{x} &                                \\ \cline{2-7} 
                       & *s\textbf{əə}diq    & s\textbf{ee}diq                      & s\textbf{ə}diq     & s\textbf{əə}ɟiq                                               & s\textbf{əə}ɟiq                      & `human'                        \\
                       & *təl\textbf{əə}tu   & tul\textbf{ee}tu                     & təl\textbf{ə}tu    & təl\textbf{ə}tu                                               & mətəl\textbf{əə}tu                   & `cold (thing)'                 \\ \hline 

\end{tabular}%
}
\end{table}
\end{frame}

\begin{frame}{Vowels - 2 (prepenultimate schwa)}
    \begin{table}[]
\centering
\resizebox{0.99\textwidth}{!}{%
\begin{tabular}{c|c|cccc|l}
\hline
\textbf{\psed} & \textbf{\psed word}      & \textbf{\stg} & \textbf{\sto} & \textbf{\sctr} & \textbf{\setr} & \multicolumn{1}{c}{\textbf{Env./Gloss}} \\ \hline
\multirow{8}{*}{*ə} & & \cellcolor<2>{hl} u                  & ə                  & ə                  & ə                  & /\_...σσ\#                      \\ \cline{2-7}
& *d\textbf{ə}qəras & d\textbf{u}qeras & d\textbf{ə}qəras & d\textbf{ə}qəras & d\textbf{ə}qəras & `face' \\ \cline{2-7}
                    & & \cellcolor<2>{hl} u                  & \cellcolor<4>{hl} u                  & ə                  & ə                  & /g\_CVCV(C)\#; /\_gVCV(C)\#       \\ \cline{2-7}
                    & *g\textbf{ə}bian & b\textbf{u}bian & w\textbf{u}bian & g\textbf{ə}bian & g\textbf{ə}bian & `evening' \\ 
                    & *b\textbf{ə}gihur & b\textbf{u}gihun & b\textbf{u}wihur & b\textbf{ə}gihur & b\textbf{ə}gihur & `wind' \\ \cline{2-7}
                    & & \cellcolor<3>{hl} V\textsubscript{x} & ə                  & ə                  & ə                  & /\_(h)V\textsubscript{x}CV(C)\#   \\ \cline{2-7}
                    & *m\textbf{ə}uyas & m\textbf{u}uyas & muyas & m\textbf{ə}uyas & m\textbf{ə}uyas & `to sing.AV' \\ 
                    & *c\textbf{ə}hədil & c\textbf{e}hedin & c\textbf{ə}hədil & s\textbf{ə}həɟil & s\textbf{ə}həɟil & `heavy' \\ \hline
\end{tabular}%
}
\end{table}
\end{frame}

\begin{frame}{Vowels - 3 (penultimate schwa)}
    \begin{table}[]
\centering
\resizebox{0.99\textwidth}{!}{%
\begin{tabular}{c|c|cccc|l}
\hline
\textbf{\psed} & \textbf{\psed word}      & \textbf{\stg} & \textbf{\sto} & \textbf{\sctr} & \textbf{\setr} & \multicolumn{1}{c}{\textbf{Env./Gloss}} \\ \hline
\multirow{10}{*}{*ə} & & \cellcolor<2>{hl} e                  & \cellcolor<3>{hl}  i                  & \cellcolor<3>{hl} i                  & \cellcolor<3>{hl} i                  & /\_yV(C)\#            \\  \cline{2-7}
                    & *b\textbf{ə}yax & b\textbf{e}yax & b\textbf{i}yax & b\textbf{i}yax & b\textbf{i}yax & `power' \\ \cline{2-7}
                    & & \cellcolor<2>{hl} e                  & \cellcolor<3>{hl} V\textsubscript{x} & \cellcolor<3>{hl} V\textsubscript{x} & \cellcolor<3>{hl} V\textsubscript{x} & /\_V\textsubscript{x}(C)\#        \\ \cline{2-7}
                    & *məh\textbf{ə}al & meh\textbf{e}an & məhal & məh\textbf{a}al & məh\textbf{a}al & `to carry on shoulder' \\
                    & *l\textbf{ə}iŋ & l\textbf{e}iŋ & tliŋ & l\textbf{i}iŋ & l\textbf{i}iŋ & `to hide' \\ \cline{2-7}
                    & & \cellcolor<2>{hl} e                  & \cellcolor<4>{hl} u                  & ə                  & ə                  & /g\_CV(C)\# (not all cases); /\_gV(C)\#           \\ \cline{2-7}
                    & *g\textbf{ə}sak & g\textbf{e}sak & w\textbf{u}sak & g\textbf{i}sak & g\textbf{ə}sak & `bamboo tool (removing ramie bark)' \\
                    & *m\textbf{ə}gay & m\textbf{e}ge & m\textbf{u}way & m\textbf{ə}gay & m\textbf{ə}gay & `to give.AV' \\ \cline{2-7}
                    & & \cellcolor<2>{hl} e                  & ə                  & ə                  & ə                  & /elsewhere                        \\ \cline{2-7}
                    & *\textbf{ə}lug & \textbf{e}lu & \textbf{ə}lu & \textbf{ə}lug & \textbf{ə}lug & `road; path' \\ \hline
\end{tabular}%
}
\end{table}
\end{frame}

\begin{frame}{Vowels - 4 (``diphthongs'')}
\begin{table}[]
\centering
\resizebox{0.79\textwidth}{!}{%
\begin{tabular}{c|c|cccc|l}
\hline
\textbf{\psed} & \textbf{\psed word}      & \textbf{\stg} & \textbf{\sto} & \textbf{\sctr} & \textbf{\setr} & \multicolumn{1}{c}{\textbf{Env./Gloss}} \\ \hline
\multirow{4}{*}{*ay} & & \cellcolor<2>{hl} e  & \cellcolor<2>{hl} ey & \cellcolor<2>{hl} ey & \cellcolor<2>{hl} ey & /\_σ        \\ \cline{2-7}
                     & *b\textbf{ay}luh & b\textbf{e}luh & b\textbf{ey}luh & b\textbf{ey}luh & b\textbf{ey}luh & `beans' \\ \cline{2-7}
                     & & e  & ay & ay & ay & /elsewhere \\ \cline{2-7}
                     & *bal\textbf{ay} & bal\textbf{e} & bal\textbf{ay} & bal\textbf{ay} & bal\textbf{ay} & `true' \\ \hline
\multirow{4}{*}{*aw} & & \cellcolor<2>{hl} o  & \cellcolor<2>{hl} ow & \cellcolor<2>{hl} ow & \cellcolor<2>{hl} ow & /\_σ        \\ \cline{2-7}
                     & *d\textbf{aw}riq & d\textbf{o}riq & d\textbf{ow}riq & d\textbf{ow}riq & d\textbf{ow}riq & `eye' \\ \cline{2-7}
                     & & o  & aw & aw & aw & /elsewhere \\ \cline{2-7}
                     & *bab\textbf{aw} & bob\textbf{o} & bab\textbf{aw} & bab\textbf{aw} & bab\textbf{aw} & `above; top' \\ \hline 
\multirow{2}{*}{*uy} & & uy & uy & uy & uy &            \\ \cline{2-7}
                     & *bab\textbf{uy} & bab\textbf{uy} & bab\textbf{uy} & bab\textbf{uy} & bab\textbf{uy} & `pig' \\ \hline 
\end{tabular}%
}
\end{table}
\end{frame}

\begin{frame}{Evidence for *-ay- and *-aw- (\cite{lee2012tawsa})}
    \begin{table}[!htbp]
\centering
\caption{\textit{e} : \textit{ay} and \textit{o} : \textit{aw} correspondences in Central Toda and Eastern Toda (adapted from \cite[95-96]{lee2012tawsa})}
\label{tab:ctoeto}
\resizebox{0.79\textwidth}{!}{%
\begin{tabular}{llll}
\hline
                              & Central Toda & Eastern Toda & Gloss                                 \\ \hline
e : ay                  & metaq        & m\textbf{ay}taq       & `\textsc{av}.stab'   \\ \hline
\multirow{5}{*}{o : aw} & soki         & s\textbf{aw}ki        & `sickle'                              \\
                              & doriq        & d\textbf{aw}riq       & `eye'                                 \\
                              & tokan        & t\textbf{aw}kan       & `man's basket'                        \\
                              & rodux        & r\textbf{aw}dux       & `chicken'                             \\
                              & mosa         & m\textbf{aw}sa/mosa   & `\textsc{av.irr}-go' \\ \hline
\end{tabular}%
}
\end{table}
\end{frame}

\section{Seediq subgrouping}

\begin{frame}{Subgrouping hypothesis}
    \begin{figure}
        \centering
        \includegraphics[height=0.75\textheight]{images/0-1}
    \end{figure}
\end{frame}

\begin{frame}{Subgrouping hypothesis}
    \begin{figure}
        \centering
        \includegraphics[height=0.75\textheight]{images/0-2}
    \end{figure}
\end{frame}

\begin{frame}{Subgrouping hypothesis}
    \begin{figure}
        \centering
        \includegraphics[height=0.75\textheight]{images/0-3}
    \end{figure}
\end{frame}

\begin{frame}{Subgrouping hypothesis}
    \begin{figure}
        \centering
        \includegraphics[height=0.75\textheight]{images/0-4}
    \end{figure}
\end{frame}

\section{The special doublet phenomenon}

\begin{frame}{Overview}
    \begin{itemize}
        \item<1-> Across the modern dialects of the Seediq language, there are words with similar, but not identical forms, where their meanings are the same or related. 
        \item<2-> Often the difference between the two forms is just one or few segment(s). 
        \item<3-> Sometimes the two similar forms appear together in the same dialect, and sometimes they appear separately in different dialects without conditions or rules.
    \end{itemize}
\end{frame}

\begin{frame}{The special doublet phenomenon - 1}
Infixation:
\begin{table}[]
\centering
\resizebox{0.99\textwidth}{!}{%
\begin{tabular}{lllllll}
\hline
\psed    &               & \stg                         & \sto    & \sctr & \setr         & Gloss                    \\ \hline
*babaw   &               & bobo                         & babaw   & babaw & babaw         & `above; on top of'       \\
*ba\albf{ra}w   & < **bəba<ra>w & baro                         & baraw   & baraw & baraw         & `above; on top of'       \\
*rahut   &               & rahuc                        & tərahuc &       &               & `downslope' \\
*hu\albf{na}t   & < **rəhu<na>t & hunac                        &         &       & hunat `south' & `downslope' \\
*ŋaŋut   &               & ŋaŋuc                        & ŋaŋuc   & ŋaŋuc & ŋaŋut         & `outside'                  \\
(*ŋə\albf{ra}t)   & < **ŋəŋə<ra>t & ŋerac                        &         &       &               & `outside'                  \\
*ahiŋ    &               & ahiŋ                         &         &       &               & `shoulder'                 \\
*hi\albf{ra}ŋ   & < **ahi<ra>ŋ  &                              & hiraŋ   & hiraŋ & hiraŋ         & `shoulder'                 \\ \hline
\end{tabular}%
}
\end{table}
\end{frame}

\begin{frame}{The special doublet phenomenon - 2}
Substitution (segment):
\begin{table}[]
\centering
\resizebox{0.85\textwidth}{!}{%
\begin{tabular}{llllll}
\hline
\psed    & \stg    & \sto    & \sctr   & \setr   & Gloss               \\ \hline
*ləla\albf{b}aŋ &         & ləlabaŋ & ləlabaŋ & ləlabaŋ & `broad; wide'         \\
*gəla\albf{h}aŋ & gulahaŋ &         &         &         & `broad; wide'         \\
         & dali\albf{ŋ}   &         &         &         & `close; near; nearby' \\
         &         & dali\albf{h}   & dali\albf{h}   & dali\albf{h}   & `close; near; nearby' \\
*məridi\albf{l} & muridin & məridil & məriɟil & məriɟil & `crooked; diagonal'  \\
         & muridi\albf{h} &         &         &         & `crooked; diagonal'   \\
*\albf{c}əka    & ceka    & cəka    & səka    & səka    & `half; to halve'     \\
*\albf{d}əka    & deka    & dəka    & dəka    & dəka    & `half; to halve'     \\ \hline
\end{tabular}%
}
\end{table}
\end{frame}

\begin{frame}{The special doublet phenomenon - 3}
\small{Substitution (syllable):}
\begin{table}[]
\centering
\resizebox{\textwidth}{!}{%
\begin{tabular}{l|l|llllll}
\hline
\pan & \pata & \psed   & \stg    & \sto    & \sctr  & \setr                             & Gloss                 \\ \hline
*qudaS&*quras & *qu\albf{das}  &         & qudas   &        &                                   & `gray hair'             \\
&*quriʔ & *qu\albf{di}   & qudi    &         &        & quɟi                              & `gray hair'             \\
*laŋaw&*ɹaŋaw & *raŋ\albf{aw}  & ruŋedi raŋo  & raŋaw   &        &                                   & `blowfly' \\
&*ɹaŋəriʔ&*rəŋ\albf{ədi} & ruŋedi  & rəŋədi  & rəŋəɟi & rəŋəɟi                            & `housefly'             \\
*iRi&&*i\albf{gi}    &         & iwi     & igi    &                                   & `left'                  \\
&*ʔiɹil&*i\albf{ril}   & irin    & iril    &        & iril                              & `left'                  \\
*uka&*ʔukas&*u\albf{ka}    & uka     & uka     & uka    &                                   & `not exist'             \\
&*ʔuŋat&*u\albf{ŋat}   &         &         & uŋac   & uŋat                              & `not exist'             \\
&&(*dəgə-CVC) & duge\albf{hiŋ} & dəwə\albf{liŋ} &        & dəgə\albf{ril}                           & `narrow'                \\
&&(*təhə-CVC) & tehe\albf{ruy} & təhə\albf{dil} & təhə\albf{ɟil}& təhə\albf{ɟil}                           & `to migrate; move'      \\ \hline
\end{tabular}%
}
\end{table}
\end{frame}

\begin{frame}{The Gender Register System in (Proto-)Atayal}{(\cite{li1980gender,li1982gender,li1983gender}; \cite{goderich2020phd})}
\begin{itemize}
    \item<1-> Atayal has a gender register system (or ``men's and women's speech'') in its lexicon, but is only fully preserved in some elderly speakers in Matu'uwal.
    \item<2-> The gender register system has collapsed in other dialects, with only occasional remnants remaining. 
    \item<3-> The derivation from the female form to male form is irregular. 
    \item<4-> There are several strategies to derive words --- affixation, deletion, substitution, etc. 
\end{itemize}
\end{frame}

\begin{frame}{A comparison with \pata ~{}- typology}
\begin{table}[]
\centering
\resizebox{0.78\columnwidth}{!}{%
\begin{tabular}{ccc}
\hline
Type                            & Atayal & Seediq \\ \hline
affixation (<ra>/<na>/suffixes) & O      & O      \\
segment substitution            & O      & O      \\
syllable substitution           & O      & O      \\ \hline
\end{tabular}%
}
\end{table}
\end{frame}

\begin{frame}{How about Seediq}
\begin{itemize}
    \item<1-> None of the modern dialect distinguishes M/F vocabularies.
    \item<2-> But the way of modifying words are still productive until very recent time. (< 400 years)
    \item<3-> So, it is related to the one in Atayal, but it modifies words for unknown reasons. 
\end{itemize}
\end{frame}


\section{\psed reflexes of \pan segments -- The five ``problems''}

\begin{frame}{The 5 ``problems''}
\begin{table}
\centering
\resizebox{0.95\textwidth}{!}{%
\begin{tabular}{cccc}
\hline
Type                                 & \pan    & \psed (regular) & \psed (irregular)  \\ \hline
the ``DC'' problem                   & *d/(z?) & *d               & *c                  \\ \hline
the ``SH'' problem                   & *S      & *s               & *h                  \\ \hline
\multirow{2}{*}{the ``CLS'' problem} & *C      & *c               & \multirow{2}{*}{*l} \\ 
                                     & *S      & *s               &                     \\ \hline
the ``LD'' problem                   & *N      & *l               & *d                  \\ \hline
the ``GR'' problem                   & *R      & *g               & *r                  \\ \hline
\end{tabular}%
}
\end{table}
\end{frame}

\begin{frame}{Where do the problems happen?}
\begin{table}[]
\centering
\resizebox{\columnwidth}{!}{%
\begin{tabular}{ccccc}
\hline
Type                & Among Formosan lgs. & Between \textsc{Aic} & Among \textsc{Sed} dialects & Borrowing? \\ \hline
``DC''  & ---                 & O           & O   & Less likely               \\ \hline
``SH''  & \cellcolor<2>{hl} O & O           & O   & Unknown                   \\ \hline
``CLS'' & ---                 & O           & O   & Less likely               \\ \hline
``LD''  & ---                 & O           & O   & Less likely               \\ \hline
``GR''  & ---                 & O           & O   & Some cases                \\ \hline
\end{tabular}%
}
\end{table}
\end{frame}

\begin{frame}{Example -- The ``LD'' problem}{Regular *N > *l}
\begin{table}[]
\centering
\resizebox{0.5125\textwidth}{!}{%
\begin{tabular}{lll}
\hline
\pan    & \psed               & Gloss                      \\ \hline 
*\albf{N}aŋuy  & *\albf{l}aŋuy              & `to swim'                  \\
*\albf{N}ataD  & *\albf{l}atat (`to leave') & `outside'                  \\
*\albf{N}uka   & (*\albf{l}u-qih)           & `wound'                    \\
*\albf{N}utud  & *\albf{l}utut              & `to join things'           \\
*bə\albf{N}bə\albf{N} & *bə\albf{l}əbu\albf{l}            & `banana'                   \\
*qə\albf{N}uR & *ə\albf{l}ug  `road'       & `animal trail'             \\
*na\albf{N}aq  & *na\albf{l}aq              & `pus'                      \\
*pa\albf{N}id  & *pa\albf{l}it              & `wing'                     \\
*Səba\albf{N}  & *səba\albf{l}              & `cloth (carrying a child)' \\
*dapa\albf{N}  & *dapi\albf{l}              & `footprint'                \\
*quba\albf{N}  & *uba\albf{l} `hair'        & `gray hair'                \\
*SiSi\albf{N}  & *sisi\albf{l}              & `Alcippe spp.'             \\ \hline
\end{tabular}
}
\end{table}
\end{frame}

\begin{frame}{Example -- The ``LD'' problem}{Irregular ones}
\begin{table}[]
\centering
\resizebox{0.9\textwidth}{!}{%
\begin{tabular}{ccccccc}
\hline
\pan   & \psedf    & \stg  & \sto & \sctr & \setr & Gloss                     \\ \hline
*ba\albf{N}aR & *balaw & ba\albf{d}o   & balaw & balaw  & balaw  & `a thorny vine'           \\
*\albf{N}a    & *\albf{d}a       & \albf{d}a    & \albf{d}a    & \albf{d}a     & \albf{d}a     & `already'                 \\
*\albf{N}ayad & *layat & \albf{d}ayac  & \albf{d}ayac & layac  & layat  & `Sambucus formosana'      \\
*\albf{N}usuŋ & *\albf{d}uhuŋ    & \albf{d}uhuŋ  & \albf{d}uhuŋ & \albf{d}uhuŋ  & \albf{d}uhuŋ  & `mortar'                  \\
*qa\albf{N}up & *a\albf{d}uk     & a\albf{d}uk   & a\albf{d}uk  & a\albf{d}uk   & a\albf{d}uk   & `to hunt'                 \\
*qə\albf{N}əb & *əluk  & əluk   & əluk  & ə\albf{d}uk   & ə\albf{d}uk   & `to close'                \\
*qi\albf{N}aS & *i\albf{d}as     & i\albf{d}as   & i\albf{d}as  & i\albf{d}as   & i\albf{d}as   & `moon'                    \\
*si\albf{N}aR & *hi\albf{d}aw    & hi\albf{d}o   & hi\albf{d}aw & hi\albf{d}aw  & hi\albf{d}aw  & `sun'                     \\
*sulə\albf{N} & *huru\albf{t}   & huru\albf{c}  & huru\albf{c} & həri\albf{c}  & huri\albf{t}  & `to plug' > `to keep sth.' \\
*wana\albf{N} & *na<ra>\albf{t}? & nara\albf{c}  & nara\albf{c} & nara\albf{c}  & nara\albf{t}  & `right side'              \\ \hline
\end{tabular}
}
\end{table}
\end{frame}

\section{Conclusion}

\begin{frame}{Summary}
    \begin{itemize}
    \item<1-> The paper reconstructs Proto-Seediq phonology and lexicon from four dialects, providing around 700 reconstructed words.
    \item<2-> It suggests a new Seediq subgrouping hypothesis with linguistic evidence. 
    \item<3-> The paper discusses the special doublet phonomenon in Seediq, comparing it to Atayal's Gender Register System, but uncertainty remains about gender-based vocabulary distinctions.
    \item<4-> It compiles reflexes from Proto-Austronesian to Proto-Seediq, highlighting the five big problems with possible explanations. 
    \end{itemize}
\end{frame}

\begin{frame}{Remaining issues}
    \begin{itemize}
        \item<1-> Better explanation for the big five problems. 
        \item<2-> Fieldwork: more sufficient data from different speech varieties. (Tawsa [-aw-\&-ay-], Truwan/Sadu [*q>ʡ]), etc.
    \end{itemize}
\end{frame}

\begin{frame}[noframenumbering, allowframebreaks]{Selected references}
\tiny\printbibliography
\end{frame}

\begin{frame}
\vspace*{\fill}
    \begin{center}
        {\Huge \albf{*məhuway namu balay!}\\ \\Thank you for listening!}
    \end{center}
\vspace*{\fill}
\end{frame}

\section*{}
\appendix

\begin{frame}{Vl. Stops - 1}
    \begin{table}[]
\centering
\resizebox{0.78\textwidth}{!}{%
\begin{tabular}{c|c|cccc|l}
\textbf{\psed} & \textbf{\psed word}      & \textbf{\stg} & \textbf{\sto} & \textbf{\sctr} & \textbf{\setr} & \multicolumn{1}{c}{\textbf{Env./Gloss}} \\ \hline
\multirow{3}{*}{*p} & & p    & p    & p     & p     &                          \\             \cline{2-7}         
 & *\textbf{p}ada   & \textbf{p}ada & \textbf{p}ada & \textbf{p}ada & \textbf{p}ada & `muntjac' \\ 
 & *ra\textbf{p}it  & ra\textbf{p}ic & ra\textbf{p}ic & ra\textbf{p}ic & ra\textbf{p}it & `flying squirrel' \\ \hline
\multirow{8}{*}{*t} &  & t    & t    & c     & c     & /\_i, /\_əy                \\  \cline{2-7}
                    & *qu\textbf{t}i & qu\textbf{t}i & qu\textbf{t}i & qu\textbf{c}i & qu\textbf{c}i & `faeces'                          \\
                    & *\textbf{t}əyaquŋ & \textbf{t}uyaquŋ & \textbf{t}əyaquŋ & \textbf{c}əyaquŋ & \textbf{c}əyaquŋ & `crow'          \\ \cline{2-7}
                   &  & c    & c    & c     & t     & /\_\#                         \\ \cline{2-7}
                   & *səpa\textbf{t} & sepa\textbf{c} & səpa\textbf{c} & səpa\textbf{c} & səpa\textbf{t} & `four'                  \\ \cline{2-7}
                   & & t    & t    & t     & t     & /elsewhere               \\ \cline{2-7}
                   & *\textbf{t}unux & \textbf{t}unux & \textbf{t}unux & \textbf{t}unux & \textbf{t}unux  & `head'                   \\ 
                   & *u\textbf{t}ux & u\textbf{t}ux & u\textbf{t}ux & u\textbf{t}ux & u\textbf{t}ux  & `spirit'                        \\ \hline
\end{tabular}%
}
\end{table}
\end{frame}

\begin{frame}{Vl. Stops - 2}
    \begin{table}[]
\centering
\resizebox{0.85\textwidth}{!}{%
\begin{tabular}{c|c|cccc|l}
\textbf{\psed} & \textbf{\psed word}      & \textbf{\stg} & \textbf{\sto} & \textbf{\sctr} & \textbf{\setr} & \multicolumn{1}{c}{\textbf{Env./Gloss}} \\ \hline
\multirow{4}{*}{*k}  & & k    & k    & k     & k     &                          \\ \cline{2-7}
                    & *\textbf{k}ari & \textbf{k}ari & \textbf{k}ari & \textbf{k}ari & \textbf{k}ari & `language' \\ 
                    & *ba\textbf{k}a & ba\textbf{k}a & ba\textbf{k}a & ba\textbf{k}a & ba\textbf{k}a & `enough' \\
                    & *papa\textbf{k} & papa\textbf{k} & papa\textbf{k} & papa\textbf{k} & papa\textbf{k} & `animal foot' \\ \hline
\multirow{4}{*}{*q}  & & q    & q    & q     & q     &                          \\ \cline{2-7}
                    & *\textbf{q}uyux & \textbf{q}uyux & \textbf{q}uyux & \textbf{q}uyux & \textbf{q}uyux & `rain' \\
                    & *la\textbf{q}i & la\textbf{q}i & la\textbf{q}i & la\textbf{q}i & la\textbf{q}i & `child' \\
                    & *quwa\textbf{q} & quwa\textbf{q} & quwa\textbf{q} & quwa\textbf{q} & quwa\textbf{q} & `mouth' \\ \hline
\end{tabular}%
}
\end{table}
\end{frame}

\begin{frame}{Vd. Stops - 1}
    \begin{table}[]
\centering
\resizebox{0.85\textwidth}{!}{%
\begin{tabular}{c|c|cccc|l}
\textbf{\psed} & \textbf{\psed word}      & \textbf{\stg} & \textbf{\sto} & \textbf{\sctr} & \textbf{\setr} & \multicolumn{1}{c}{\textbf{Env./Gloss}} \\ \hline
\multirow{3}{*}{*b} & & b    & b    & b     & b     &                          \\ \cline{2-7}
                    & *\textbf{b}aki & \textbf{b}aki & \textbf{b}aki & \textbf{b}aki & \textbf{b}aki & `grandfather'         \\
                    & *ba\textbf{b}aw & bo\textbf{b}o & ba\textbf{b}aw & ba\textbf{b}aw & ba\textbf{b}aw & `above, top'      \\ \hline
\multirow{5}{*}{*d} &  & d    & d    & ɟ     & ɟ     & /\_i, /\_əy                \\ \cline{2-7}
                    & *səə\textbf{d}iq & see\textbf{d}iq & sə\textbf{d}iq & səə\textbf{ɟ}iq & səə\textbf{ɟ}iq & `human'      \\ \cline{2-7}
                    & & d    & d    & d     & d     & /elsewhere               \\ \cline{2-7}
                    & *\textbf{d}ara & \textbf{d}ara & \textbf{d}ara & \textbf{d}ara & \textbf{d}ara & `blood'               \\ 
                    & *ru\textbf{d}an & ru\textbf{d}an & ru\textbf{d}an & ru\textbf{d}an & ru\textbf{d}an & `old (person)'   \\ \hline
\end{tabular}%
}
\end{table}
\end{frame}

\begin{frame}{Vd. Stops - 2}
    \begin{table}[]
\centering
\resizebox{0.9\textwidth}{!}{%
\begin{tabular}{c|c|cccc|l}
\textbf{\psed} & \textbf{\psed word}      & \textbf{\stg} & \textbf{\sto} & \textbf{\sctr} & \textbf{\setr} & \multicolumn{1}{c}{\textbf{Env./Gloss}} \\ \hline
\multirow{2}{*}{*g-/*-g-}   & & g  & w  & g  & g  &  \\ \cline{2-7}
                            & *\textbf{g}a\textbf{g}a & \textbf{g}a\textbf{g}a & \textbf{w}a\textbf{w}a & \textbf{g}a\textbf{g}a & \textbf{g}a\textbf{g}a & `that; at; \textsc{Prog.}' \\ \cline{2-7}
\multirow{2}{*}{*-a\textbf{g}} & & \textbf{o} & a\textbf{w} & a\textbf{w} & a\textbf{w} &  \\ \cline{2-7}
                                & *hida\textbf{g} & hido & hida\textbf{w} & hida\textbf{w} & hida\textbf{w} & `sun' \\ \cline{2-7}
\multirow{2}{*}{*-i\textbf{g}}  & & \textbf{u}y & i  & i\textbf{g} & i\textbf{g} &  \\ \cline{2-7}
                                & *mari\textbf{g} & mar\textbf{u}y & mari & mari\textbf{g} & mari\textbf{g} & `to buy.AV' \\ \cline{2-7}
\multirow{2}{*}{*-u\textbf{g}}  & & u  & u  & u\textbf{g} & u\textbf{g} &  \\ \cline{2-7}
                                & *əlu\textbf{g} & elu & əlu & əlu\textbf{g} & əlu\textbf{g} & `road; path' \\ \hline
\end{tabular}%
}
\end{table}
\end{frame}

\begin{frame}{Affricate}
    \begin{table}[]
\centering
\resizebox{0.9\textwidth}{!}{%
\begin{tabular}{c|c|cccc|l}
\textbf{\psed} & \textbf{\psed word}      & \textbf{\stg} & \textbf{\sto} & \textbf{\sctr} & \textbf{\setr} & \multicolumn{1}{c}{\textbf{Env./Gloss}} \\ \hline
\multirow{3}{*}{*c} & & c   & c   & s   & s   &  \\ \cline{2-7}
    & *\textbf{c}akus & \textbf{c}akus & \textbf{c}akus & \textbf{s}akus & \textbf{s}akus & `camphor' \\
    & *bi\textbf{c}ur & bi\textbf{c}un & bi\textbf{c}ur & bi\textbf{s}ur & bi\textbf{s}ur & `worm' \\ \hline
\end{tabular}%
}
\end{table}
\end{frame}

\begin{frame}{Fricatives - 1}
    \begin{table}[]
\centering
\resizebox{0.85\textwidth}{!}{%
\begin{tabular}{c|c|cccc|l}
\textbf{\psed} & \textbf{\psed word}      & \textbf{\stg} & \textbf{\sto} & \textbf{\sctr} & \textbf{\setr} & \multicolumn{1}{c}{\textbf{Env./Gloss}} \\ \hline
\multirow{4}{*}{*s} & & s & s & s & s &  \\ \cline{2-7}
                    & *\textbf{s}apah & \textbf{s}apah & \textbf{s}apah & \textbf{s}apah & \textbf{s}apah & `house' \\
                    & *mi\textbf{s}an & mi\textbf{s}an & mi\textbf{s}an & mi\textbf{s}an & mi\textbf{s}an & `winter' \\
                    & *caku\textbf{s} & caku\textbf{s} & caku\textbf{s} & saku\textbf{s} & saku\textbf{s} & `camphor' \\ \hline
\multirow{4}{*}{*x} & & x & x & x & x &  \\ \cline{2-7}
                    & *\textbf{x}iluy & \textbf{x}iluy & \textbf{x}iluy & \textbf{x}iluy & \textbf{x}iluy & `iron' \\
                    & *tə\textbf{x}al & te\textbf{x}an & tə\textbf{x}al & tə\textbf{x}al & tə\textbf{x}al & `once' \\
                    & *tunu\textbf{x} & tunu\textbf{x} & tunu\textbf{x} & tunu\textbf{x} & tunu\textbf{x}  & `head' \\  \hline
\end{tabular}%
}
\end{table}
\end{frame}

\begin{frame}{Fricatives - 2}
    \begin{table}[]
\centering
\resizebox{0.99\textwidth}{!}{%
\begin{tabular}{c|c|cccc|l}
\textbf{\psed} & \textbf{\psed word}      & \textbf{\stg} & \textbf{\sto} & \textbf{\sctr} & \textbf{\setr} & \multicolumn{1}{c}{\textbf{Env./Gloss}} \\ \hline
\multirow{4}{*}{*h} & & h & h & h & h &  \\ \cline{2-7}
                    & *\textbf{h}idag & \textbf{h}ido & \textbf{h}idaw & \textbf{h}idaw & \textbf{h}idaw & `sun' \\
                    & *qu\textbf{h}iŋ & qu\textbf{h}iŋ & qu\textbf{h}iŋ & qu\textbf{h}iŋ & qu\textbf{h}iŋ & `head louse' \\
                    & *sapa\textbf{h} & sapa\textbf{h} & sapa\textbf{h} & sapa\textbf{h} & sapa\textbf{h} & `house' \\ \hline
\end{tabular}%
}
\end{table}
\end{frame}

\begin{frame}{Nasals}
    \begin{table}[]
\centering
\resizebox{0.8\textwidth}{!}{%
\begin{tabular}{c|c|cccc|l}
\textbf{\psed} & \textbf{\psed word}      & \textbf{\stg} & \textbf{\sto} & \textbf{\sctr} & \textbf{\setr} & \multicolumn{1}{c}{\textbf{Env./Gloss}} \\ \hline
\multirow{3}{*}{*m} & & m   & m   & m   & m   \\ \cline{2-7}
                    & *\textbf{m}alu & \textbf{m}alu & \textbf{m}alu & \textbf{m}alu & \textbf{m}alu & `good' \\
                    & *ta\textbf{m}a & ta\textbf{m}a & ta\textbf{m}a & ta\textbf{m}a & ta\textbf{m}a  & `father' \\ \hline
\multirow{4}{*}{*n} & & n   & n   & n   & n \\ \cline{2-7}
                    & *\textbf{n}aqih & \textbf{n}aqah & \textbf{n}aqah & \textbf{n}aqih & \textbf{n}aqih & `bad' \\ 
                    & *i\textbf{n}u & i\textbf{n}u & i\textbf{n}u & i\textbf{n}u & i\textbf{n}u & `where' \\
                    & *rəbaga\textbf{n} & rubaga\textbf{n} & rəbawa\textbf{n} & rəbaga\textbf{n} & rəbaga\textbf{n} & `summer' \\ \hline
\multirow{3}{*}{*ŋ} & & ŋ   & ŋ   & ŋ   & ŋ     \\ \cline{2-7}
                    & *\textbf{ŋ}a\textbf{ŋ}ut & \textbf{ŋ}a\textbf{ŋ}uc & \textbf{ŋ}a\textbf{ŋ}uc & \textbf{ŋ}a\textbf{ŋ}uc & \textbf{ŋ}a\textbf{ŋ}ut & `outside' \\
                    & *aru\textbf{ŋ}  & aru\textbf{ŋ} & aru\textbf{ŋ} & aru\textbf{ŋ} & aru\textbf{ŋ} & `pangolin' \\ \hline
\end{tabular}%
}
\end{table}
\end{frame}

\begin{frame}{Liquids}
    \begin{table}[]
\centering
\resizebox{0.78\textwidth}{!}{%
\begin{tabular}{c|c|cccc|l}
\textbf{\psed} & \textbf{\psed word}      & \textbf{\stg} & \textbf{\sto} & \textbf{\sctr} & \textbf{\setr} & \multicolumn{1}{c}{\textbf{Env./Gloss}} \\ \hline
\multirow{5}{*}{*r} &  & n & r & r & r & /\_\# \\ \cline{2-7}
                    & *bicu\textbf{r} & bicu\textbf{n} & bicu\textbf{r} & bisu\textbf{r} & bisu\textbf{r} & `worm' \\ \cline{2-7}
                    & & r & r & r & r & /elsewhere \\ \cline{2-7}
                    & *\textbf{r}udan & \textbf{r}udan & \textbf{r}udan & \textbf{r}udan & \textbf{r}udan & `old (person)' \\
                    & *da\textbf{r}a & da\textbf{r}a & da\textbf{r}a & da\textbf{r}a & da\textbf{r}a & `blood'               \\ \hline
\multirow{5}{*}{*l} & & n & l & l & l & /\_\# \\ \cline{2-7}
                    & *təxa\textbf{l} & texa\textbf{n} & təxa\textbf{l} & təxa\textbf{l} & təxa\textbf{l} & `once' \\  \cline{2-7}
                   &  & l & l & l & l & /elsewhere \\  \cline{2-7}
                   & *\textbf{l}aqi & \textbf{l}aqi & \textbf{l}aqi & \textbf{l}aqi & \textbf{l}aqi & `child' \\ 
                   & *qaw\textbf{l}it & qo\textbf{l}ic & qow\textbf{l}ic & qow\textbf{l}ic & qow\textbf{l}it & `mouse' \\ \hline
\end{tabular}%
}
\end{table}
\end{frame}

\begin{frame}{Glides}
    \begin{table}[]
\centering
\resizebox{0.99\textwidth}{!}{%
\begin{tabular}{c|c|cccc|l}
\textbf{\psed} & \textbf{\psed word}      & \textbf{\stg} & \textbf{\sto} & \textbf{\sctr} & \textbf{\setr} & \multicolumn{1}{c}{\textbf{Env./Gloss}} \\ \hline
\multirow{3}{*}{*w-/-w-} & &  w & w & w & w \\ \cline{2-7}
                         & *\textbf{w}ili & \textbf{w}ili & \textbf{w}ili & \textbf{w}ili & \textbf{w}ili & `leech' \\
                         & *ra\textbf{w}a & ra\textbf{w}a & ra\textbf{w}a & ra\textbf{w}a & ra\textbf{w}a & `basket' \\ \hline
\multirow{2}{*}{*y-/-y-} & & y & y & y & y \\ \cline{2-7}
                         & *\textbf{y}a\textbf{y}u & \textbf{y}a\textbf{y}u & \textbf{y}a\textbf{y}u & \textbf{y}a\textbf{y}u & \textbf{y}a\textbf{y}u & `small knife' \\ \hline
\end{tabular}%
}
\end{table}
\end{frame}

\begin{frame}{The ``DC'' problem (\pan *d > \psed *d/*c)}
\begin{table}[!htbp]
\centering
\resizebox{1\textheight}{!}{%
\begin{tabular}{cccc}
\hline
\pan     & \psedf & Proto-Atayal & Gloss                      \\ \hline
*\textbf{d}akəS   & *\textbf{c}akus & *\textbf{r}akus       & `camphor tree'             \\
*\textbf{d}amaR   & *\textbf{c}amaw & ---          & `fire or light of candles' \\
*qaRi\textbf{d}aŋ & *gi\textbf{c}aŋ & *qagi\textbf{r}aŋ     & `k.o. bean'                \\ \hline
\end{tabular}%
}
\end{table}

\begin{table}[!htbp]
\centering
\resizebox{0.82\textheight}{!}{%
\begin{tabular}{ccc}
\hline
\psedf   & Proto-Atayal & Gloss                   \\ \hline
*\textbf{d}əŋusun & *\textbf{c}Vŋusan     & `target; goal; meaning' \\
*\textbf{c}əruhiŋ & *\textbf{r}aɹuhiŋ     & `nest fern'             \\
*u\textbf{c}ik    & *qu\textbf{r}ip       & `ginger'                \\ \hline
\end{tabular}
}
\end{table}

\end{frame}

\begin{frame}{The ``SH'' problem (\pan *S > \psed *s/*h)}
\begin{small}This is a higher-level issue, please refer to \textcite{tsuchida1976tsouic, ross2015coronals} for more details. I will only list the examples in Seediq here.\end{small}
\begin{table}[!htbp]
\centering
\resizebox{0.75\textheight}{!}{%
\begin{tabular}{ccc}
\hline
\pan    & \psedf & Gloss           \\ \hline
*\textbf{S}auni  & *\textbf{s}awni & `just now'      \\
*\textbf{S}əpat  & *\textbf{s}əpat & `four'          \\
*\textbf{S}əpi   & *\textbf{s}əpi  & `dream'         \\
*\textbf{S}apuy  & *\textbf{h}apuy & `fire' > `cook' \\
*\textbf{S}əma   & *\textbf{h}əma  & `tongue'        \\
*\textbf{S}uRəNa & *\textbf{h}uda  & `snow'          \\ \hline
\end{tabular}
}
\end{table}
\end{frame}

\begin{frame}{The ``CLS'' problem (\pan *C or *S > \psed *c or *s / *l)}
    \begin{table}[!htbp]
\centering
\resizebox{0.99\textwidth}{!}{%
\begin{tabular}{ccccccc}
\hline
\pan           & \psedf    & \stg & \sto & \sctr & \setr & Gloss                \\ \hline
*\textbf{C}iŋaS         & *[s\textbf{l}]iŋas & siŋas & \textbf{l}iŋas & siŋas  & \textbf{l}iŋas  & `food between teeth' \\
*\textbf{C}uSuR         & *\textbf{l}ihug    & \textbf{l}ihu  & \textbf{l}ihu  & \textbf{l}ihug  & \textbf{l}ihug  & `to string'          \\
*\textbf{C}abu          & *\textbf{l}abu     & \textbf{l}abu  & \textbf{l}abu  & \textbf{l}abu   & \textbf{l}abu   & `to wrap'            \\
*\textbf{C}aŋis         & *\textbf{l}iŋis    & \textbf{l}iŋis & \textbf{l}iŋis & \textbf{l}iŋis  & \textbf{l}iŋis  & `to cry'             \\
*daki\textbf{S}         & *daki\textbf{l}    & daki\textbf{n} & daki\textbf{l} & daki\textbf{l}  & daki\textbf{l}  & `to climb'           \\
*Ramə\textbf{C}; *Rami\textbf{S} & *gami\textbf{l}    & gami\textbf{n} & gami\textbf{l} & gami\textbf{l}  & gami\textbf{l}  & `root'               \\ \hline
\end{tabular}
}
\end{table}
\end{frame}

\begin{frame}{The ``LD'' problem (\pan *N > \psed *l/*d)}
\begin{table}[!htbp]
\centering
\resizebox{0.96\textwidth}{!}{%
\begin{tabular}{ccccccc}
\hline
\pan   & \psedf    & \stg  & \sto & \sctr & \setr & Gloss                     \\ \hline
*ba\textbf{N}aR & *balaw & ba\textbf{d}o   & balaw & balaw  & balaw  & `a thorny vine'           \\
*\textbf{N}a    & *\textbf{d}a       & \textbf{d}a    & \textbf{d}a    & \textbf{d}a     & \textbf{d}a     & `already'                 \\
*\textbf{N}ayad & *layat & \textbf{d}ayac  & \textbf{d}ayac & layac  & layat  & `Sambucus formosana'      \\
*\textbf{N}usuŋ & *\textbf{d}uhuŋ    & \textbf{d}uhuŋ  & \textbf{d}uhuŋ & \textbf{d}uhuŋ  & \textbf{d}uhuŋ  & `mortar'                  \\
*qa\textbf{N}up & *a\textbf{d}uk     & a\textbf{d}uk   & a\textbf{d}uk  & a\textbf{d}uk   & a\textbf{d}uk   & `to hunt'                 \\
*qə\textbf{N}əb & *əluk  & əluk   & əluk  & ə\textbf{d}uk   & ə\textbf{d}uk   & `to close'                \\
*qi\textbf{N}aS & *i\textbf{d}as     & i\textbf{d}as   & i\textbf{d}as  & i\textbf{d}as   & i\textbf{d}as   & `moon'                    \\
*si\textbf{N}aR & *hi\textbf{d}aw    & hi\textbf{d}o   & hi\textbf{d}aw & hi\textbf{d}aw  & hi\textbf{d}aw  & `sun'                     \\
*sulə\textbf{N} & *huru\textbf{t}?   & huru\textbf{c}  & huru\textbf{c} & həri\textbf{c}  & huri\textbf{t}  & `to plug' > `to keep sth.' \\
*wana\textbf{N} & *na<ra>\textbf{t}? & nara\textbf{c}  & nara\textbf{c} & nara\textbf{c}  & nara\textbf{t}  & `right side'              \\ \hline
\end{tabular}
}
\end{table}
\end{frame}

\begin{frame}{The ``GR'' problem (\pan *R > \psed *g/*r)}
\begin{table}[!htbp]
\centering
\resizebox{0.99\textwidth}{!}{%
\begin{tabular}{cccccc}
\hline
\pan      & \psedf   & Gloss  & \pan      & \psedf   & Gloss                \\ \hline
*\textbf{R}abi(qi) & *\textbf{r}abi    & `night (all night)'   & *da\textbf{R}iŋ    & *da\textbf{r}iŋ   & `moan'                 \\
*\textbf{R}iNuk    & *\textbf{r}əhənuk & `k.o. berry'          & *ga\textbf{R}aŋ    & *ka\textbf{r}aŋ   & `crab'                 \\
*\textbf{R}ubu     & *\textbf{r}udu    & `nest'                & *hu\textbf{R}aC    & *u\textbf{r}at    & `tendon; blood vessel' \\
*\textbf{R}udaŋ    & *\textbf{r}udan   & `old man'             & *ka\textbf{R}i     & *ka\textbf{r}i    & `language'             \\
*ba\textbf{R}aq    & *ba\textbf{r}aq   & `lungs'               & *kə\textbf{R}ət    & *kə\textbf{r}ut   & `to cut' > `to saw'    \\
*da\textbf{R}a     & *da\textbf{r}a    & `maple tree'          & *qa\textbf{R}əm    & *a\textbf{r}uŋ    & `pangolin'             \\
*da\textbf{R}aq    & *da\textbf{r}a    & `blood'               & *tagə\textbf{R}aŋ  & *təkə\textbf{r}aŋ & `ribcage'              \\
*u\textbf{R}əŋ     & *u\textbf{r}uŋ    & `horn'                 \\ \hline
\end{tabular}
}
\end{table}
\end{frame}

\begin{frame}{The ``GR'' problem (Seediq-internal)}
    \begin{table}[]
\centering
\resizebox{\textwidth}{!}{%
\begin{tabular}{llll|l|l|l}
\stg    & \sto    & \sctr   & \setr   & \psed    & \pata  & Gloss           \\ \hline
\textbf{r}ehak   & \textbf{r}əhak   & \textbf{g}əhak   & \textbf{g}əhak   & *\textbf{g}əhak   & *\textbf{g}əhap   & `seed'          \\
tuma\textbf{r}a  & təma\textbf{w}a  & təma\textbf{g}a  & təma\textbf{g}a  & *təma\textbf{g}a  & *-na\textbf{g}aʔ  & `to wait'       \\
ha\textbf{r}ac   & ha\textbf{w}ac   & ha\textbf{r}ac   & ha\textbf{g}at   & *ha\textbf{g}at   & *ha\textbf{g}aʔ   & `stone fence'   \\
mu\textbf{g}ipuh & mə\textbf{r}ipuh & mə\textbf{r}ipuh & mə\textbf{r}ipuh & *mə\textbf{g}ipuh & *ma\textbf{ʔ}ipuh & `fragile; soft' \\
\textbf{r}uciluŋ & \textbf{w}uciluŋ & \textbf{r}əsiluŋ & \textbf{g}əsiluŋ & *\textbf{g}əciluŋ & *\textbf{w}aciluŋ & `sea; lake'     \\
\textbf{r}upun   & \textbf{r}upun   & \textbf{g}upun   & \textbf{g}upun   & *\textbf{g}upun   & *\textbf{g}ipun   & `teeth'   \\
\textbf{r}ubeyuk & \textbf{r}əbiyuk & ---              & \textbf{g}əbiyuk & *\textbf{g}əbəyuk  & ---               & `canyon' \\
\textbf{g}umimax & mə\textbf{r}imax & ---              & \textbf{g}əmimax & *\textbf{g}əmimax  & *\textbf{ʔ}umimag & `to mix'
\end{tabular}%
}
\end{table}
- \pata *ʔ can be from **g.
\end{frame}


\end{document}
