\documentclass[master]{NTHUthesis}
\usepackage[utf8]{inputenc}
\usepackage{libertine}
\usepackage{libertinust1math}
\usepackage[utf8]{inputenc}
\usepackage{xeCJK}
\setCJKmainfont{AR PL UKai TW}
\setCJKmonofont{AR PL UKai TW}
\usepackage{pdfpages}
\usepackage{pdflscape}


%%% Customization  %%%
\usepackage{custom}
\usepackage{zhlipsum, lipsum}   % dummy text
\usepackage{graphicx}           % figures
\usepackage{float}
\usepackage{enumitem}
\usepackage{booktabs}           % tables
\usepackage[ruled]{algorithm2e} % algorithms, list of algorithms
\usepackage{gb4e} %Glossing 用
\noautomath %防止gb4e爆掉
\usepackage[normalem]{ulem}
\useunder{\uline}{\ul}{}
\usepackage{multicol}
\usepackage{caption} 
\captionsetup[table]{skip=7pt} %表格caption間距
\usepackage{chngpage} % allows for temporary adjustment of side margins
%\usepackage{cite}               % bibliography

% Bibliography
\usepackage[backend=biber, style=authoryear-comp, dashed=false, maxcitenames=2, maxbibnames=99]{biblatex}
\addbibresource{ref.bib} %Import the bibliography file
\renewcommand*{\nameyeardelim}{\space}
\renewcommand*{\postnotedelim}{\addcolon\space}
\renewcommand*{\bibpagespunct}{\addcolon\space}
\DefineBibliographyStrings{english}{%
  page             = {},
  pages            = {},
} 
%\renewcommand*{\bibfont}{\footnotesize}

\newcommand{\sameyear}[1]{}

\usepackage{xpatch}
\AtEveryBibitem{\clearfield{month}}
\AtEveryCitekey{\clearfield{month}}
\xpatchbibmacro{date+extradate}{%
  \printtext[parens]%
}{%
  \setunit{\addperiod\space}%
  \printtext%
}{}{}

\NewBibliographyString{toappear}
\DefineBibliographyStrings{english}{%
  toappear = {to appear},
}

%Add comma between two authors in bibliography only
\AtBeginBibliography{\renewcommand*{\finalnamedelim}{
  \finalandcomma
  \addspace\bibstring{and}\space}  
 }

\DeclareFieldFormat[article,inproceedings,misc,inbook,incollection]{citetitle}{#1}
\DeclareFieldFormat[article,inproceedings,misc,inbook,incollection]{title}{#1}
\DeclareFieldFormat[thesis]{title}{\mkbibemph{#1}}
\DeclareFieldFormat[article,book,inproceedings,misc,inbook,incollection,thesis]{addendum}{#1\nopunct}
\DeclareFieldFormat[article,inproceedings,misc,inbook,incollection,thesis]{volume}{#1}
\renewbibmacro{in:}{}

\DefineBibliographyStrings{american}{phdthesis = {PhD dissertation}}
\urlstyle{same}

%%END Bibliography Setup%%

%%% Necessary %%%
\titleZH{Kari Seediq 這是一個中文標題 }
\titleEN{This an English title}
\instituteZH{語言學研究所}
\studentID{111044501}
\studentZH{宋硯之}
\studentEN{Walis Hian-chi Song}
\advisorZH{廖秀娟~~博士}
\advisorEN{Dr. Hsiu-chuan Liao}
\yearZH{2024}
\monthZH{7}

% Comment the following to have chapters numbered without interruption (numbering through parts)
\makeatletter\@addtoreset{chapter}{part}\makeatother%

\begin{document}

\makecover

\pagenumbering{Roman}

%%% Necessary %%%
\begin{abstractEN}
\lipsum[1-2]
\end{abstractEN}

%%% Optional %%%
\begin{abstractSED}
\lipsum[1-2]
\end{abstractSED}

%%% Necessary %%%
\begin{abstractZH}
\zhlipsum[1][name=trad]
\end{abstractZH}

%%% Optional %%%
\begin{acknowledgementsEN}
\lipsum[1-2]
\end{acknowledgementsEN}

%%% Necessary %%%
\maketoc

%%% Optional %%%
\phantomsection
\listofalgorithms
\addcontentsline{toc}{chapter}{List of Algorithms}
\clearpage

\pagenumbering{arabic}

%\part{English Version}

\chapter{Introduction}
\lipsum[1-2] \textcite{blust1999subgrouping} \pan

\section{The Seediq language and its people}
\lipsum[1-3]

\subsection{Seediq dialects}
\lipsum[1]

\subsubsection{Tgdaya}
\lipsum[1-2]

\subsubsection{Central Toda}
\lipsum[1]

\subsubsection{Eastern Toda}
\lipsum[1]

\subsubsection{Central Truku}
\lipsum[1]

\subsubsection{Eastern Truku}
\lipsum[1]

\subsubsubsection{The Truku Nation and their ethnolect \textit{Kari Truku}}
\lipsum[1]

\section{Research questions and goals}
\lipsum[1-4]

\section{Methodology}
\lipsum[1-7]

\section{Data sources}
\lipsum[1-3]

\section{Orthographic conventions}
\lipsum[1-5]

\section{Outline of the thesis}
\lipsum[1-2]

\chapter{Literature review}
\lipsum[1]

\section{The position of Seediq in the Austronesian family}
\lipsum[1-5]

\section{Internal relationships of Seediq dialects}
\lipsum[1-5]

\section{Relevant anthropology studies}
\lipsum[1-5]

\section{Interim summary}
\lipsum[1]

\chapter{Phonologies of Seediq dialects}
\lipsum[1]

\section{Phoneme inventories and phonotactics}

\subsection{Tgdaya phonology}
\lipsum[1]

\subsubsection{Tgdaya consonant inventory}
\subsubsection{Tgdaya vowel inventory}
\subsubsection{Tgdaya phonotactics}

\subsection{Central Toda phonology}
\lipsum[1]

\subsubsection{Central Toda consonant inventory}
\subsubsection{Central Toda vowel inventory}
\subsubsection{Central Toda phonotactics}

\subsection{Eastern Toda phonology}
\lipsum[1]

\subsubsection{Eastern Toda consonant inventory}
\subsubsection{Eastern Toda vowel inventory}
\subsubsection{Eastern Toda phonotactics}

\subsection{Central and Eastern Truku phonologies}
\lipsum[1]

\subsubsection{Truku consonant inventory}
\subsubsection{Truku vowel inventory}
\subsubsection{Truku phonotactics}

\section{Synchronic alternations}
\lipsum[1]
\subsection{Consonant alternations}
\lipsum[1]
\subsection{Vowel alternations}
\lipsum[1]

\section{Interim summary}
\lipsum[1]

\chapter{Proto-Seediq reconstruction}
\lipsum[1-2]

\section{Proto-Seediq phonology}

\subsection{Proto-Seediq phoneme inventory}

\subsection{Reconstructed phonemes and their reflexes}

\subsubsection{Proto-Seediq consonants}

\subsubsection{Proto-Seediq vowels}

\subsection{Proto-Seediq phonotactics}

\section{Interim summary}
\lipsum[1]

\section{Proto-Seediq morphosyntax}

\subsection{Proto-Seediq affixes}

\subsubsection{Proto-Seediq voice system affixes}

\subsubsection{Proto-Seediq derivational and functional affixes}

\subsection{Morphophonological alternations in Proto-Seediq}

\subsection{A reconstruction of Proto-Seediq Pronominal system}

\subsubsection{Personal pronouns}

\subsubsection{Demonstrative pronouns}

\section{Issues in Proto-Seediq lexical reconstruction}

\subsection{Special doublet phenomenon in Proto-Seediq lexicon}

\subsubsection{The possible relationship between Seediq's special doublet phenomenon and Atayal's Gender Register System}

\subsection{External evidence for lexical reconstructions}

\section{Interim summary}
\lipsum[1]

\chapter{Seediq subgrouping}

\section{Subgrouping hypothesis}
\lipsum[1]
\subsection{Evidence for a Toda-Truku subgroup}
\lipsum[1]
\subsubsection{Evidence for a Toda subgroup}
\lipsum[1]
\subsubsection{Evidence for a Truku subgroup}
\lipsum[1]

\section{Interim summary}
\lipsum[1]

\chapter{External relationships}

\section{Proto-Seediq reflexes of Proto-Austronesian segments}
\lipsum[1]
\subsection{Regular reflexes}
\lipsum[1]
\subsection{Irregular reflexes}
\lipsum[1]
\subsubsection{The ``DC'' problem}
\lipsum[1]
\subsubsection{The ``CLS'' problem}
\lipsum[1]
\subsubsection{The ``LD'' problem}
\lipsum[1]
\subsubsection{The ``GR'' problem}
\lipsum[1]
\subsubsection{The ``Q∅'' problem}
\lipsum[1]
\subsubsection{The ``SH'' problem}
\lipsum[1]

\section{Language contact}
\lipsum[1]

% Wait for Liao laoshi's comments on the term paper % 

\section{Implications for Proto-Atayalic reconstruction}
\lipsum[1]

\subsection{Proto-Atayalic *-d}
\lipsum[1]
\subsection{Proto-Atayalic *j}
\lipsum[1]
\subsection{Proto-Atayalic *g}
\lipsum[1]
\subsection{Proto-Atayalic lexicon}
\lipsum[1]
%\subsection{Proto-Atayalic *q}

\section{Interim summary}
\lipsum[1]

\chapter{Conclusion}
\lipsum[1]
\section{Summary}
\lipsum[1]
\section{Remaining issues and future reseach directions}
\lipsum[1]

%%%%%%%%%%% 中文版 %%%%%%%%%%%

% \part{中文版}
% ffdf
% \chapter{前言}
% \section{研究方法}
% 
% \section{測試標題}

%%%%%%%%%%% 中文版 %%%%%%%%%%%

\newpage
\addcontentsline{toc}{part}{References}
\printbibliography[title={References}]

\end{document}
