\documentclass[12pt]{article}

\usepackage[full]{textcomp}
\usepackage{CJKutf8}
\usepackage[T1]{fontenc}
\usepackage{enumitem}
\usepackage[a4paper,left=1in, right=1in, top=1in, bottom=1in]{geometry}
\usepackage{url}
\usepackage[dvipsnames]{xcolor}
\usepackage{amsmath}
\usepackage{multirow}
\usepackage{graphicx}
\graphicspath{ {./images/} }
\usepackage{booktabs}
\usepackage[normalem]{ulem}
\useunder{\uline}{\ul}{}
\usepackage{longtable}
\usepackage{makecell}
\usepackage{float}
\usepackage{enumitem}
\usepackage[linguistics]{forest}
\usepackage{pdfpages}
\usepackage{setspace}
\setstretch{2}
\renewcommand{\arraystretch}{2} %表格行高
\usepackage{lipsum} %假字
\usepackage{pdflscape} %橫向
\usepackage{tablefootnote}
\usepackage{comment}
\usepackage{multicol}

\usepackage{gb4e} %Glossing 用
\noautomath %防止gb4e爆掉

\usepackage{chngcntr} %將圖表號碼改成Section.X
\counterwithin{table}{section}
\counterwithin{figure}{section}

\usepackage{caption} 
\captionsetup[table]{skip=7pt} %表格caption間距

\usepackage{tocloft} %TOC裡的Table Figure

% for figures
\renewcommand{\cftfigpresnum}{Figure } % put before the number
%\renewcommand{\cftfigaftersnum}{: }  % put after the number
% adjust spacing
\newlength{\mylen}
\settowidth{\mylen}{\cftfigpresnum}
\addtolength{\cftfignumwidth}{\mylen}
% for tables
\renewcommand{\cfttabpresnum}{Table } % put before the number
%\renewcommand{\cfttabaftersnum}{: }  % put after the number
% adjust spacing
\newlength{\mylentwo}
\settowidth{\mylentwo}{\cfttabpresnum}
\addtolength{\cfttabnumwidth}{\mylentwo}

%%% 這段是for\paragraph的
\usepackage{titlesec}

\setcounter{secnumdepth}{4}

\titleformat{\paragraph}
{\normalfont\normalsize\bfseries}{\theparagraph}{1em}{}
\titlespacing*{\paragraph}
{0pt}{3.25ex plus 1ex minus .2ex}{1.5ex plus .2ex}
%%%

% allows for temporary adjustment of side margins
\usepackage{chngpage}

\usepackage{xspace}
\newcommand{\pan}{\textsc{PAn}\xspace}
\newcommand{\panf}{Proto-Austronesian\xspace}
\newcommand{\paic}{\textsc{PAic}\xspace}
\newcommand{\paicf}{Proto-Atayalic\xspace}
\newcommand{\pata}{\textsc{PAta}\xspace}
\newcommand{\pataf}{Proto-Atayal\xspace}
\newcommand{\ata}{\textsc{Ata}\xspace}
\newcommand{\psed}{\textsc{PSed}\xspace}
\newcommand{\sed}{\textsc{Sed}\xspace}
\newcommand{\psedf}{Proto-Seediq\xspace}
\newcommand{\stg}{\textsc{Tg}\xspace}
\newcommand{\stgf}{Tgdaya\xspace}
\newcommand{\ptotr}{\textsc{PTo-Tr}\xspace}
\newcommand{\ptotrf}{Proto-Toda-Truku\xspace}
\newcommand{\totrf}{Toda-Truku\xspace}
\newcommand{\sto}{\textsc{To}\xspace}
\newcommand{\stof}{Toda\xspace}
\newcommand{\ptr}{\textsc{PTr}\xspace}
\newcommand{\ptrf}{Proto-Truku\xspace}
\newcommand{\sctr}{\textsc{CTr}\xspace}
\newcommand{\sctrf}{Central Truku\xspace}
\newcommand{\setr}{\textsc{ETr}\xspace}
\newcommand{\setrf}{Eastern Truku\xspace}

\newcommand{\xb}[1]{\textsubscript{#1}}

% Fonts
\usepackage{libertine}
\usepackage{libertinust1math}
\usepackage[utf8]{inputenc}
\usepackage{xeCJK}
\setCJKmainfont{Noto Serif CJK TC}


% package settings
\usepackage[
    hidelinks,
    pdfnewwindow=true,
    pdfauthor={Walis Hian-chi Song},
    pdftitle={HistLing Fall 2022},
]{hyperref}

\setlength\parindent{1.5em}

\pagenumbering{roman}
% begin
\begin{document}



\newpage
\addcontentsline{toc}{section}{List of Abbreviations}
\section*{List of Abbreviations}

    \begin{multicols}{2} 
    \begin{itemize}[leftmargin=*]

    \setlength\itemsep{-.5em}
    \item[ ] (AV) = Actor Voice verb form
    \item[ ] (LV) = Locative Voice verb form
    \item[ ] (PV) = Patient Voice verb form
    \item[ ] 1 = first person
    \item[ ] 2 = second person
    \item[ ] 3 = third person
    \item[ ] \sctr = \sctrf
    \item[ ] \setr = \setrf
    \item[ ] \textsc{excl} = exclusive
    \item[ ] \textsc{gen} = genitive
    \item[ ] \textsc{hort} = hortative
    \item[ ] \textsc{imp} = imperative
    \item[ ] \textsc{incl} = inclusive
    \item[ ] \textsc{irr} = irrealis
    \item[ ] \textsc{neg} = negator 
    \item[ ] \textsc{nom} = nominative
    \item[ ] \textsc{obl} = oblique
    \item[ ] P- = Proto-
    \item[ ] \paic = Proto-Atayalic
    \item[ ] \pan = Proto-Austronesian
    \item[ ] \textsc{pl} = plural
    \item[ ] \textsc{prog} = progressive
    \item[ ] \psed = \psedf
    \item[ ] \ptotr = \ptotrf
    \item[ ] \ptr = \ptrf
    \item[ ] \textsc{sg} = singular
    \item[ ] \stg = \stgf
    \item[ ] \sto = \stof

    \end{itemize}
    \end{multicols}

\newpage
\pagenumbering{arabic}

\section{Introduction}

\subsection{The Seediq language}

Seediq is an Austronesian language spoken in central and eastern mountainous area of Taiwan. 

Along with Atayal, it belongs to the Atayalic subgroup of Austronesian. \textcite{blust1999subgrouping} considers Atayalic to be a primary branch of Austronesian, while \textcite{ross2009morphology} view it as one of a member of the Nuclear Austronesian subgroup. The different proposals of the position of Seediq in the Austronesian family are shown in Figure \ref{fig:sedinAnblust} and \ref{fig:sedinAnross} respectively.  

    \begin{figure}[H]
    \centering
           \begin{forest}
           for tree={l sep=5em, s sep=6em, inner sep=0, anchor=north, parent anchor=south, child anchor=north}
            [Austronesian
                [Atayalic
                    [Atayal, tier=word]
                    [\textbf{Seediq}, draw , tier=word]
                ]
                [Other 9 subgroups, roof, tier=word]
            ]
            \end{forest}
        \caption{Seediq's position in the Austronesian family (\cite{blust1999subgrouping})}
        \label{fig:sedinAnblust}
    \end{figure}

\begin{figure}[H]
    \centering
           \begin{forest}
           for tree={l sep=3em, s sep=3em, inner sep=0, anchor=north, parent anchor=south, child anchor=north}
            [Austronesian
                [Puyuma, tier=word]
                [Rukai, tier=word]
                [Tsou, tier=word]
                [Nuclear Austronesian
                    [Atayalic
                        [Atayal, tier=word]
                        [\textbf{Seediq}, draw, tier=word]
                    ]
                    [Other 10 subgroups, roof, tier=word]
                ]
            ]
            \end{forest}
        \caption{Seediq's position in the Austronesian family (\cite{ross2009morphology})}
        \label{fig:sedinAnross}
\end{figure}

Seediq is the ethnic language of two indigenous nations recognized by the government of Taiwan --- the Seediq Nation (賽德克族) and the Truku Nation (太魯閣族). The people of the Seediq Nation call their language \textit{Kari Seediq}/\textit{Sediq}/\textit{Seejiq} (賽德克語); while the people of the Truku Nation call it \textit{Kari Truku} (太魯閣語). 

Generally, \textit{Kari Truku} is very close to the \textit{Truku} dialect of Seediq (賽德克語德鹿谷方言). Their relationship will be described in Section \ref{subgrouping}. Based on the linguistic evidence and the mutual intelligibility of these two groups of language, this paper treats \textit{Kari Truku} as a dialect of Seediq. Please note that there is nothing to do with Truku people's ethnic identity. 

Since there will be two ``Truku'' groups, the ``\textit{Kari Truku}'' and the Truku dialect of Seediq will be referred as ``Eastern Truku'' and ``Central Truku'' respectively according to their geographical area.

As of May 2023, the total population of the Seediq Nation and the Truku Nation is 44,734 (Seediq 10,978 + Truku 33,756).\footnote{Data from Council of Indigenous Peoples, Taiwan, at \href{https://www.cip.gov.tw/zh-tw/news/data-list/940F9579765AC6A0/83C63F954CB5EB1BA4B571F18AE92066-info.html}{https://www.cip.gov.tw/zh-tw/news/data-list/940F9579765AC6A0/83C63F954CB5EB1BA4B571F18AE92066-info.html} (last visited July 4, 2023).} However, not all of these individuals are native speakers of Seediq, as the number of native speakers has declined rapidly due to long-term language assimilation policies. Meanwhile, it is also difficult to estimate the exact number of speakers nowadays.

\subsubsection{Seediq dialects}

First of all, the term ``dialect'' lacks a consensus in both linguistic and general usage. Sometimes, two ``dialects'' can differ significantly and without mutual intelligibility, but are classified under the same language. Other times, two ``languages'' can be mutually intelligible but are still classified as separate languages rather than dialects. This involves many social or political factors.

In this paper, the term ``dialect'' refers to the ``language group (語群)'' names used by the Seediq people as their primary distinction, apart from the specific case of the Truku dialect mentioned earlier. 

Therefore, this paper will discuss the following dialects as a basis for reconstructing \psedf: (1) Tgdaya dialect (Chinese: 德固達雅; also referred to as Paran, Tkdaya, Wushe in other literature); (2) Toda dialect (Chinese: 都達; also known as Teuda, Towda); (3) Central Truku dialect (Chinese: 德鹿谷; used by Truku people residing in central Taiwan, Nantou); (4) Eastern Truku dialect (Chinese: 太魯閣; used by Truku people who migrated to Hualien in eastern Taiwan).

Figure \ref{fig:sedmap} displays the current geographical locations of Seediq dialects on a map of Taiwan. The smallest administrative unit shown on this map is ``village'', which indicates the phenomenon of some villages having different dialects. 

It is worth mentioning that some Tgdaya and Toda people migrated to eastern Taiwan in the past. However, due to changes in power dynamics and continuous migration over the past century, only a few of them currently reside in the eastern region. 

The Tgdaya people who migrated eastward are referred to as Pribaw or the Bokkui group (木瓜 bo̍k-kue `papaya' in Taiwanese; from Seediq \textit{bukuy} `back'). They currently reside in  Upper Mingli Tribe; Mingli Village, Wanrong Township, Hualien County, but their language usage and characteristics are not well-documented. Only few words were recorded in \textcite{tashiro1900easterntw}.

The Toda people who migrated eastward are known by different names, such as Tawsa, Tawsay, Tuda, etc. Some of them assimilated into the Klesan Atayal group and switched to speak different variants of Yilan Creole. Another part gradually moved southward to their current location in Tawsa Tribe, Lishan Village, Zhuoxi Township, Hualien County. There are some linguistic records of their language. \textcite{lee2012tawsa} introduced its phonological system and vocabulary, which has Toda features but is also influenced by Truku, making it a contact-induced dialect. 

For more information about the history of further eastward migration of Seediq, please refer to \textcite{liao1977Sedtheruy,liao1978Sedtheruy}. Note that the Tgdaya and Toda dialects referred to in this paper are based on the central variation (Nantou County).

\begin{figure}[!t]
    \centering
    \includegraphics[scale=0.8]{images/map.pdf}
    \caption{Seediq(-Truku) speaking area and dialectal distribution}
    \label{fig:sedmap}
\end{figure}
    
\subsection{Methodology}

\subsubsection{The Comparative Method --- for bottom-up reconstruction and subgrouping}

This paper mainly employs the Comparative Method, a fundamental tool in historical linguistics, for the purpose of Proto-Seediq reconstruction and subgrouping of Seediq dialects. The Comparative Method is a systematic approach that involves several essential steps in order to uncover linguistic relationships and reconstruct ancestral languages (\cite{fox1995linguistic}).

The first step in this process is the collection of words from different Seediq dialects. By comparing these words, I aim to identify cognates, which are words that are similar or same in both form and meaning. The similar words among dialects also have to show regular sound correspondences and in basic vocabulary. In addition, it is necessary to exclude the possibilities of ``chance'', ``universals'', and ``borrowing''. Cognates provide valuable evidence for language comparison and reconstruction.

Once cognates have been identified, the next step involves setting up sound correspondences. Sound correspondences are regular patterns observed across cognate words. I will group these correspondence sets together based on their phonetic similarity. 

The third step employs a bottom-up reconstruction of protophonemes, which are the phonemes believed to have existed in the proto-language. In the process of reconstructing phonemes of a proto-language, careful attention must be given to the directionality of sound changes. Two fundamental reconstruction principles must be taken, as pointed out by \textcite[85]{crowley2010introduction}. Firstly, the sound change should be plausible and natural in language developments. Secondly, a reconstructed phoneme should minimize changes in both place and manner of articulation. By adhering to these principles, it becomes possible to reconstruct protophonemes for each corresponding group and identify consistent sound changes from the proto-language to its descendant languages.

Also, sound changes occurring from the proto-language to the descendant languages or dialects adhere to the principles of the Neogrammarian Hypothesis, alternatively recognized as the Regularity Hypothesis. According to this notion, sound changes follow a strict rule without any exceptions (\cite{brugmann1878morphologische}). In essence, a sound change is anticipated to be universally applied to all forms under defined conditions in a language or dialect. Irregular cases are typically observed solely in loanwords, sporadic sound changes, and accidental possibilities. For the phoneme inventory in the proto-language, we also expect it to be a balanced system. 

Using the protophonemes as a guide, I will also reconstruct the lexicon of \psedf. By applying the sound correspondences and protophonemes to the cognate words, they can infer the likely forms of words in \psedf. This step enables the reconstruction of a substantial portion of the lexicon.

Finally, the Comparative Method allows for the determination of subgrouping within the Seediq language. The subgrouping hypothesis in this paper is carried out based on the following criteria: (1) The primary method of subgrouping involves the use of exclusively shared innovations to form a subgroup, while shared retentions must not be employed as subgrouping criteria. (2) Sporadic or rare sound changes are preferred over common ones as subgrouping criteria, as common sound changes might occur individually in closely related languages, representing a phenomenon known as ``drift'' or ``convergence among genetically related languages'', such as the ``Umlaut'' phenomenon in West-Germanic languages (\cite[47--48]{greenberg1957subgroupings}). (3) Phonological evidence includes conditioned shared sound changes, unique mergers, and more. (4) The lexical evidence referred to in this paper includes categories like lexical innovations, sporadic sound changes or segment replacements within words, and others. 

The paper will simultaneously consider the various aforementioned types of evidence as the basis for subgrouping, as situations might arise where a combination of two or more types of evidence is needed for judgment. These subgroups represent distinct historical stages or divisions within the Seediq language.

Overall, the application of the Comparative Method in this study of Seediq dialects involves collecting data, identifying cognates, establishing sound correspondences, reconstructing protophonemes and the lexicon, and determining subgroups. By following these steps, we are able to gain valuable insights into the historical development and relationships within the Seediq language, contributing to our understanding of its linguistic evolution.

\subsubsection{Top-down reconstruction method}

The top-down reconstruction method is also known as ``inverted reconstruction'' (\cites[512--16]{hockett1958course}[346]{anttila1972introduction}). 

For example, consider a language family shown in Figure \ref{fig:topdown}. We reconstruct proto-language *Y by comparing languages A and B, and proto-language *Z by comparing languages C and D. Proto-language *X is then reconstructed from *Y and *Z. However, evidence for the forms of *Y and *Z is not only found in the languages they originate from but also in *X. Once we've reconstructed the relevant forms of *X based on *Z and other languages, we can use these forms to reconstruct the forms of *Y from above (\cite[88]{fox1995linguistic}).

\begin{figure}[H]
    \centering
           \begin{forest}
           for tree={l sep=5em, s sep=6em, inner sep=0, anchor=north, parent anchor=south, child anchor=north}
            [*X
                [*Y
                    [A]
                    [B]
                ]
                [*Z
                    [C]
                    [D]
                ]
            ]
            \end{forest}
        \caption{An assumed language family as an example for top-down reconstruction}
        \label{fig:topdown}
    \end{figure}

In this paper, some of the reconstruction of \psedf's lexical forms is primarily guided by such a method. At times, modern Seediq dialects may exhibit two or more competing forms for a given word among them. In such cases, external evidence from \pan can assist in determining the \psedf form. 

However, considering that Atayalic languages often display numerous lexical innovations, it is frequently challenging to find evidence directly from \pan. Therefore, this paper employs a methodology which is similar to that of \textcite[187--88]{goderich2020phd} for the reconstruction of certain forms. If \psedf has two candidates *a and *b for a specific lexical item, and if corresponding forms *b' or b'' exist in \pataf or Atayal dialects, I will choose *b for the reconstruction in \psedf. Relevent cases will be discussed in Section 6.2. 

Certainly, in cases where a particular form appears in only one dialect, lacks agreement among the majority of dialects, or its meaning aligns with the characteristics of loanwords (such as expressing a highly specific meaning), the possibility of borrowing must be considered. Given that the Atayalic subbranch only contains two languages, the risk of circular reasoning is difficult to entirely avoid at this stage. I can only endeavor to eliminate instances fitting the aforementioned criteria as much as possible. To address this issue more effectively, a broader comparison of various Seediq and Atayal dialects, as well as an exploration of historical contact relationships between the two distinct language branches, may be necessary.

\subsection{Data sources}

I collected the data mainly from the learning wordlists by \textcite{ILRDF1000words}, the online dictionaries by \textcite{ILRDFEdict}, and \citeauthor{watandiro2009seddict}'s (\citeyear{watandiro2009seddict}) dictionary. There is also a database from Seediq Language and Culture Sustainable Development Association (n.d., shortened in Table \ref{tab:data} as ``Seediq Assoc.''), which was once used on an online dictionary.\footnote{The database file was downloaded from the website of Seediq Language and Culture Sustainable Development Association, but the source is currently unavailable.} The detail is shown in Table \ref{tab:data}.

\begin{table}[!htbp]
    \centering
    \caption{Data soureces}
    \label{tab:data}
    \begin{tabular}{l|llll}
       & Wordlist     & Database/Dictionary & Dictionary & Dictionary   \\ \hline
    \stg  & \textcite{ILRDF1000words} & \textcite{Seddatabase}   & \textcite{ILRDFEdict} & \textcite{watandiro2009seddict} \\
    \sto  & \textcite{ILRDF1000words} & \textcite{Seddatabase}   &              & \textcite{watandiro2009seddict} \\
    \sctr & \textcite{ILRDF1000words} & \textcite{Seddatabase}   &              & \textcite{watandiro2009seddict} \\
    \setr & \textcite{ILRDF1000words} &                     & \textcite{ILRDFEdict} &
    \end{tabular}
    \end{table}

All \pan data for comparing purpose are from \textcite{ACD}. Note that the symbol use of *e [ə] will be changed to *ə in this paper. \pataf forms are mainly from \textcite{goderich2020phd}, exceptions will be mentioned later in the paper.  

\subsection{Orthographic conventions}

I apply the orthographic conventions in the paper in order to make an easier comparison of Seediq dialects. In principle, the symbol use will be the same as the International Phonetic Alphabet (IPA), but there are still some exceptions. The conventions are shown in Table \ref{tab:orth}, and the current orthography systems (MOE/CIP\footnote{Abbreviations for \textbf{M}inistry \textbf{o}f \textbf{E}ducation and \textbf{C}ouncil of \textbf{I}ndigenous \textbf{P}eoples of Taiwan.}) that is used by the peoples are also listed for reference. 

\begin{table}[!htbp]
\centering
\caption{Orthographic conventions}
\label{tab:orth}
\begin{tabular}{ccc|ccc}
\hline
This paper & MOE/CIP & IPA    & This paper & MOE/CIP                   & IPA   \\ \hline
p          & p       & [p]    & r          & r                         & [ɾ]   \\
t          & t       & [t]    & l          & l                         & [l\~{ }ɮ] \\
k          & k       & [k]    & w          & w                         & [w]   \\
q          & q       & [q]    & y          & y                         & [j]   \\
b          & b       & [b]    & a          & a                         & [a]   \\
d          & d       & [d]    & i          & i                         & [i]   \\
g          & g       & [g\~{ }ɣ]  & u          & u                         & [u]   \\
c          & c       & [ʦ]   & e          & e (\stg)                  & [e]   \\
ɟ          & j       & [ɟ\~{ }dʑ] & ey         & ey (\sto, \sctr, \setr)   & [ej]  \\
s          & s       & [s]    & o          & o (\stg)                  & [o]   \\
x          & x       & [x]    & ow         & o/ow (\sto, \sctr, \setr) & [ow]  \\
h          & h       & [ħ]    & ə          & e (\sto, \sctr, \setr)    & [ə]   \\
m          & m       & [m]    &            &                           &       \\
n          & n       & [n]    &            &                           &       \\
ŋ          & ng      & [ŋ]    &            &                           &       \\ \hline
\end{tabular}
\end{table}

There is an issue about the ``neutralized'' (or reduced) vowel in prepenultimate syllables. Any kinds of vowel reduced into [u] (\stg)/[ə] (non-\stg) in this specific environment in general. Since the vowel quality is predictable, it is omitted in the MOE/CIP system. For example, \textit{dtnqlian} (\stg) `a group of slaves' is pronounced as [dutunuquli(j)an]. In this paper, the ``omitted'' vowel in the MOE/CIP system will be retranscribed based on the phonological system of each dialect. 

\subsection{Outline of the paper}

This paper will be organized as follows. Section 2 shows a review of previous studies on the Seediq language, especially those about reconstructions and internal classifications. Section 3 discuss the phonological systems of Seediq dialects in a synchronic aspect. Section 4 focus on the reconstruction of \psedf phonology. Section 5 presents a Seediq subgrouping hypothesis based on linguistic evidence. Section 6 discusses some issues on lexical reconstruction. Section 7 discusses the \psedf reflexes of \panf segments and related issues. Lastly, section 8 is for the conclusion, including a summary and remaining issues for future studies. 

\section{Literature review}

In this section, several works about Seediq or Atayalic reconstruction, subgrouping or dialectal differences will be discussed. 

\subsection{Reconstruction}

Generally speaking, only very few papers focused on the aspect of Seediq historical development or reconstruction.

\subsubsection{Li (1981)}

\textcite{li1981paic} directly reconstructed Proto-Atayalic with several Atayal and Seediq dialects, instead of reconstructing Proto-Atayal and Proto-Seediq first. 

Although that phonologically speaking, Seediq is relatively more conservative in some sense, we still need an intermediate \psedf reconstruction to clarify the Seediq-internal relationships and the historical development from Proto-Atayalic. 

\subsubsection{Ochiai (2016)}

\textcite{ochiai2016buhwan} based on the ``Bu-hwan'' vocabulary recorded in \textcite{bullock1874formosan} and two modern dialects --- (Modern-)Paran (a.k.a Tgdaya) and Taroko (Eastern Truku) to reconstruct 179 lexical items. 

The main issue of Ochiai's reconstruction is that she chose *ð as the protophonemic value of the correspondence of Bu-hwan \textit{dz}\~{ }\textit{z} [ð?] : Paran \textit{y} : Taroko \textit{y}. This reconstruction is questionable since (1) [j] > [ð] should be a more common sound change; (2) the corresponding protophoneme in Proto-Atayal and \panf is *y. Thus, we can just assume that there is still a glide *y in \psedf, then underwent a fortition process in the Paran/Bu-hwan dialect based on the principle of naturalness of sound change and external evidence from both the sister language and the ancestral language of \psedf. 

\subsection{Subgrouping}

In the previous studies, only few of them discussed the subgrouping inside the Seediq language. Most of the works take Atayalic as a whole, without a deeper look into Seediq. 

In this subsection, those studies that contain subgrouping and dialectal differences will be included. These works can be roughly divided into two types: two-branch subgrouping and three-or-more-branch subgrouping. 

\subsubsection{Ogawa and Asai (1935)}

\textcite{ogawaandasai1935} pointed out that Seediq is the language of the Seediq ethnic group that has been living in \textit{Taichūshū} (Taichung) and \textit{Karenkōchō} (Hualien) for a long time.

They argued that Seediq can be roughly divided into two dialects: 霧社 (Musya or Paran) and タロコ (Taroko). Table \ref{tab:ona1935} the phonological differences between two dialects (including their subdialects):

\begin{table}[H]
\centering
\caption{Phonological relationship between Paran and Taroko dialects in \textcite{ogawaandasai1935}}
\label{tab:ona1935}
\begin{tabular}{cc|c}
\multicolumn{2}{c|}{Musya}                                                               & Taroko       \\
Paran                                                       & Gungu & Ibuh \\ \hline
ð                                                                     & j              & j           \\
\makecell[c]{ts\\(except for final)} & ts             & s           \\
g?                                                                   & q (ǥ [ɣ])/w     & ǥ [ɣ]       \\
g                                                                     & g              & g           \\ \hline
\end{tabular}
\end{table}

Based on their description, the subgrouping tree can be drawn as is Figure \ref{fig:onatree}:

\begin{figure}[H]
    \centering
    \begin{forest}
           for tree={l sep=5em, s sep=6em, inner sep=0, anchor=north, parent anchor=south, child anchor=north}
            [Seediq
                [Musya
                    [Paran, tier=word]
                    [Gungu, tier=word]
                ]
                [Taroko, tier=word]
            ]
            \end{forest}
    \caption{Seediq subgrouping tree based on Ogawa and Asai's description}
    \label{fig:onatree}
\end{figure}

\subsubsection{Cheng, Dakis Pawan, and Pidu Nokan (2019)}

\textcite{Chengetal2019Tgdaya} refer that Seediq has three ethnic subgroups: Tgdaya, Toda, and Truku. However, they indicate that Seediq can be roughly divided into two linguistic dialects: Tgdaya and Truku-Toda. Truku-Toda can be further divided into 都達土語(Toda), 太魯閣土語(Eastern Truku), 德路固土語(Central Truku), and 見晴土語(Kbayan). See Figure \ref{fig:chengtree} for their subgrouping tree. 

\begin{figure}[H]
    \centering
    \begin{forest}
           for tree={l sep=5em, s sep=3em, inner sep=0, anchor=north, parent anchor=south, child anchor=north}
            [Seediq
                [Tgdaya, tier=word] 
                [Truku-Toda
                    [Eastern Truku, tier=word]
                    [Central Truku, tier=word]
                    [Toda, tier=word]
                    [Kbayan, tier=word]
                ]
            ]
     \end{forest}
    \caption{Seediq subgrouping tree based on Cheng et al.'s description}
    \label{fig:chengtree}
\end{figure}

Eastern Truku in Hualien and Central Truku in Nantou are slightly different due to the geographic isolation, but they can still communicate with each other without problems. Towda is different from Truku, but not a lot. They can also communicate. There are also differences between Kbayan and \setrf. 

In Cheng et al.'s introduction of dialects, there is no description of their differences at all. They only mentioned that the dialects are different, but without any evidence to combine them or separate them. In my opinion, the reason that they combines Toda and Truku is their mutual intelligibility, since that Toda and Truku have some shared retentions on vowel. This is not the accurate method to subgroup the dialects, so the model provided by them needs to be revised.

\subsubsection{Sung (2018)}

The distribution and the description of Seediq dialects are also in \citeauthor{Sung2018Sedgrammar}'s (\citeyear{Sung2018Sedgrammar}) reference grammar book. 

Sung only points out the dialectal differences in the book, here are the correspondences:

\begin{itemize}[itemsep=0pt, topsep=0pt]
    \item Tg /e/ : To, Tr /ə/
    \item Tg /-e/ : To, Tr /-aj/
    \item Tg /-o/ : To, Tr /-aw/
    \item Tg, To /t, d/ : Tr /ts , ɟ/ (/\_i)
    \item Tg, Tr /g-, -g-/ : To /w-; -w-/
    \item Tg /-w/ : To, Tr /-g/\footnote{Toda should not have [g], this could be a typo.}
    \item Tg, To /ʦ/ : Tr /s/
\end{itemize}

\subsubsection{Li (1981)}

In \citeauthor{li1981paic}'s (\citeyear{li1981paic}) reconstruction of Proto-Atayalic, he mentioned that Atayal has two major dialects: Squliq and Cʔuliʔ; however, he only refer Seediq has four dialects: Toŋan, Toda, Truwan and Inago. 

Li considered that the Seediq dialects are not completely uniform with significant dialectal differences, but also not very divergent. Li also did not clarify the relationship between these Seediq ``dialects''. In my opinion, these ``dialects'' in this paper are more like the ``dialect spots''.

\subsection{Issues}

There is a research gap of a reconstruction of \psedf. \textcite{li1981paic} reconstructed Proto-Atayalic directly, and \textcite{goderich2020phd} reconstructed Proto-Atayalic. By the discussion of Li's reconstruction of Proto-Atayalic in Goderich's dissertation, I realized that there are some Atayal or Seediq internal issues should be solved first, before directly go into the whole subgroup. Thence, a reconstruction of \psedf is needed for further discussion of the Atayalic subgroup. 

On the other hand, since there is no reliable reconstruction of \psedf phonology and lexicon. Thus, previous 
Seediq subgrouping proposals are not based on the basic principle of the Comparative Method.  

\section{Phonologies of Seediq dialects}

Generally, Seediq dialects all have similar phoneme inventories. The primary differences are \textit{g} and \textit{c} /ʦ/, this will be introduced in the following part. 

For the vowel system, \stgf is relatively a more different one, which has a five-monophthong system, but only one diphthong. The other dialects has four monophthongs and three phonemic diphthongs. 

\subsection{Tgdaya phonology}

\subsubsection{Tgdaya consonant inventory} \label{sectgC}

The \stgf dialect has 18 consonants, including seven stops \textit{p, t, k, q, b, d, g}, one affricate \textit{c} [ʦ], three fricatives \textit{s, x, h} [ħ], three nasals \textit{m, n, ŋ}, one lateral \textit{l}, one tap \textit{r} [ɾ], and two glides \textit{w, y} [j] (\cite{Sung2018Sedgrammar}). Table \ref{tab:tgC} shows the consonant inventory of \stgf.

\begin{table}[!htbp]
\centering
\caption{Tgdaya consonant inventory\\}
\label{tab:tgC}
\begin{tabular}{l|cc|cc|cc|cc|cc|cc}
\hline
                    & \multicolumn{2}{c|}{Bilabial} & \multicolumn{2}{c|}{Dental/Alveolar} & \multicolumn{2}{c|}{Palatal} & \multicolumn{2}{c|}{Velar} & \multicolumn{2}{c|}{Uvular} & \multicolumn{2}{c}{Pharyngeal} \\ \hline
Stop                & p            & b           & t               & d               &             &               & k          & g          & q            &            &                 &              \\
Affricate           &               &              & c [ʦ]              &                  &             &               &             &             &               &            &                 &              \\
Fricative           &               &              & s               &                  &             &               & x          &             &               &            & h [ħ]             &              \\
Nasal               &               & m           &                  & n               &             &               &             & ŋ          &               &            &                 &              \\
Lateral &               &              &                  & l               &             &               &             &             &               &            &                 &              \\
Tap                 &               &              &                  & r [ɾ]              &             &               &             &             &               &            &                 &              \\
Glide               &               & w           &                  &                  &             & y [j]           &             &             &               &            &                 &              \\ \hline
\end{tabular}
\end{table}

<h> is pronounced as [ħ] by older speakers; on the contrary, younger speakers pronounce it as [h]. The difference between the place of articulation seems to cause the occurrence of vowel lowering: e.g. \stgf (younger) \textit{hido} [hido] \~{} Truku \textit{hidaw} [ħeʲdaw] `sun'.

<r> is usually pronounced as a tap [ɾ] or a retracted [ɾ̠], but some speakers give an approximant [ɹ] sometimes, probably as a free variation. 

Glottal stop makes no makes no contrast in \stgf Seediq, and is usually not phonemic. There are some morphological evidence that shows the glottal stop is not a phoneme in \stgf. For the word formation of `along N', the structure can be described as \textit{ku}-C\textsubscript{1}\textit{u}\~{ }\textsc{Base}. If the base starts with a consonant, the derived form will be like: \textit{ku-yu-yayuŋ} `along the river'. However, if the base starts with a vowel, the ``first'' consonant will be the onset of the second syllable, such as \textit{ku-la-alaŋ} `along the village(s)' instead of **ku-ʔa-alaŋ. 

<y> is pronounced as a palatal approximant [j]. There is a near-extinct subdialect of \stgf --- Paran, which has a corresponding dental fricative [ð]. \textcite{asai1953sedik} mentioned that  [ð] is general in Paran, and ``changed to'' [j] in the Gungu pronunciation.

\subsubsection{Tgdaya vowel inventory}

\stgf has five monophthongs: \textit{a, i, e, u, o}, as shown in Table \ref{tab:tgV}.

\begin{table}[!htbp]
\centering
\caption{Tgdaya vowel inventory}
\label{tab:tgV}
\begin{tabular}{lccc}
\hline
     & Front & Central & Back \\ \hline
High &  i    &         &  u   \\
Mid  &  e    &         &  o   \\
Low  &       &  a      &      \\ \hline
\end{tabular}
\end{table}

\stgf also has a diphthong -\textit{uy} [uj]. 

\subsection{Toda phonology}

\subsubsection{Toda consonant inventory}

The \stof dialect has 17 consonant, including six stops \textit{p, t, k, q, b, d}, one affricate \textit{c} [ʦ], three fricatives \textit{s, x, h} [ħ], three nasals \textit{m, n, ŋ}, one lateral \textit{l}, one tap \textit{r} [ɾ], and two glides \textit{w, y} [j], the inventory is as given in Table \ref{tab:toC}.

\begin{table}[!htbp]
\centering
\caption{Toda consonant inventory}
\label{tab:toC}
\begin{tabular}{l|cc|cc|cc|cc|cc|cc}
\hline
                    & \multicolumn{2}{c|}{Bilabial} & \multicolumn{2}{c|}{Dental/Alveolar} & \multicolumn{2}{c|}{Palatal} & \multicolumn{2}{c|}{Velar} & \multicolumn{2}{c|}{Uvular} & \multicolumn{2}{c}{Pharyngeal} \\ \hline
Stop                & p            & b           & t               & d               &             &               & k          &            & q            &            &                 &              \\
Affricate           &               &              & c [ʦ]              &                  &             &               &             &             &               &            &                 &              \\
Fricative           &               &              & s               &                  &             &               & x          &             &               &            & h [ħ]             &              \\
Nasal               &               & m           &                  & n               &             &               &             & ŋ          &               &            &                 &              \\
Lateral &               &              &                  & l               &             &               &             &             &               &            &                 &              \\
Tap                 &               &              &                  & r [ɾ]              &             &               &             &             &               &            &                 &              \\
Glide               &               & w           &                  &                  &             & y [j]           &             &             &               &            &                 &              \\ \hline
\end{tabular}
\end{table}

\stof lacks the voiced counterpart of /k/. Apart from this, the consonant inventory is the same to the one in \stgf. The phoneme <r> is also pronounced as a tap [ɾ] or a retracted one [ɾ̠]. 

\subsubsection{Toda vowel inventory}

\stof has four monophthongs: \textit{a, i, u, ə}, as shown in Table \ref{tab:toV}.

\begin{table}[!htbp]
\centering
\caption{Toda vowel inventory}
\label{tab:toV}
\begin{tabular}{lccc}
\hline
     & Front & Central & Back \\ \hline
High &  i    &         &  u   \\
Mid  &       &  ə      &      \\
Low  &       &  a      &      \\ \hline
\end{tabular}
\end{table}

\stof has three phonemic diphthongs: -\textit{ow}-/-\textit{aw} /aw/, -\textit{ey}-/-\textit{ay} /aj/, and -\textit{uy} /uj/. 

\subsection{Truku (Central and Eastern) phonology}

\subsubsection{Truku consonant inventory}

The two Truku dialects, which are the Central one and the Eastern one respectively, have totally the same phoneme inventory. Thus, they will be discussed together here. 

The Truku dialects have 17 consonants, including six stops \textit{p, t, k, q, b, d}, four fricatives \textit{s, x, g} [ɣ], \textit{h} [ħ], three nasals \textit{m, n, ŋ}, one lateral fricative \textit{l} [ɮ], one tap \textit{r} [ɾ], and two glides \textit{w, y} [j]. Table \ref{tab:trC} shows the inventory of consonant.

\begin{table}[!htbp]
\centering
\caption{Truku consonant inventory}
\label{tab:trC}
\begin{tabular}{l|cc|cc|cc|cc|cc|cc}
\hline
                    & \multicolumn{2}{c|}{Bilabial} & \multicolumn{2}{c|}{Dental/Alveolar} & \multicolumn{2}{c|}{Palatal} & \multicolumn{2}{c|}{Velar} & \multicolumn{2}{c|}{Uvular} & \multicolumn{2}{c}{Pharyngeal} \\ \hline
Stop                & p            & b           & t               & d               &             &               & k          &            & q            &            &                 &              \\
Affricate           &               &              &                &                  &             &               &             &             &               &            &                 &              \\
Fricative           &               &              & s               &                  &             &               & x          &   g [ɣ]          &               &            & h [ħ]             &              \\
Nasal               &               & m           &                  & n               &             &               &             & ŋ          &               &            &                 &              \\
Lateral fricative &               &              &                  & l [ɮ]              &             &               &             &             &               &            &                 &              \\
Tap                 &               &              &                  & r [ɾ]              &             &               &             &             &               &            &                 &              \\
Glide               &               & w           &                  &                  &             & y [j]           &             &             &               &            &                 &              \\ \hline
\end{tabular}
\end{table}

Truku lacks the phonemic affricate \textit{c}, but the phonetic \textit{c} can occur as an allophone of /t/ in word-final position. <g> is pronounced as [ɣ] in Truku, such as \textit{gaya} [ˈɣaja] `law, tradition', \textit{gaga} [ˈɣaɣa] `that'.

The phoneme <l> sounds more like a lateral fricative [ɮ], it clear when it is at the initial or intervocalic positions, like \textit{laqi} [ˈɮaqi] `child' and \textit{elug} [ˈəɮuɣ] `road'. 

The phoneme <r> has more variants. The \sctrf speakers tend to pronounce it as a tap [ɾ], [ɾ̠], or an approximant [ɹ̠]; while the \setrf pronounce it more like a lateral [l]. However, we are not sure if the phonetic value of \setrf [l] is the same as \stgf and \stof's [l] since it is a bit different from auditory in my experience. Thence, from a historical perspective, I do not take it as a shift for now.

The other consonants are the same to the ones in \stgf. 

\subsubsection{Truku vowel inventory}

Truku dialects also have four monophthongs: \textit{a, i, u, ə}, as shown in Table \ref{tab:trV}.

\begin{table}[!htbp]
\centering
\caption{Truku vowel inventory}
\label{tab:trV}
\begin{tabular}{lccc}
\hline
     & Front & Central & Back \\ \hline
High &  i    &         &  u   \\
Mid  &       &  ə      &      \\
Low  &       &  a      &      \\ \hline
\end{tabular}
\end{table}

Truku has three diphthongs as well: -\textit{ow}-/-\textit{aw} /aw/, -\textit{ey}-/-\textit{ay} /aj/, and -\textit{uy} /uj/.

\subsection{Syllable structure}

Seediq dialects consist of six syllable types: V, CV, VC, VG, CVG, and CVC. Generally, there are fewer restrictions in the final position, and all six types have some examples exist in each dialect.

However, non-final positions exhibit different patterns in various dialects. \stgf does not permit any closed syllables (including CVG) in this position, while \stof allows VG and CVG to appear there. \sctrf and \setrf have non-final VC syllables, which are relatively rare. The examples are shown in Table \ref{tab:dialectsyltype}.

\begin{table}[!htbp]
\centering
\caption{Syllable types in Seediq dialects}
\label{tab:dialectsyltype}
\begin{tabular}{lllllll}
\hline
Types        & Position    & \stg    & \sto    & \sctr   & \setr   & Gloss                  \\ \hline
\multirow{2}{*}{V}   & Non-final & \textbf{i}.ma     & \textbf{i}.ma     & \textbf{i}.ma     & \textbf{i}.ma     & `who'                  \\
                     & Final       & lu.bu.\textbf{i}  & lə.bu.\textbf{i}  & lə.bu.\textbf{i}  & lə.bu.\textbf{i}  & `to wrap-\textsc{imp}' \\ \hline
\multirow{2}{*}{CV}  & Non-final & \textbf{ru}.dan   & \textbf{ru}.dan   & \textbf{ru}.dan   & \textbf{ru}.dan   & `old (person)'         \\
                     & Final       & pa.\textbf{ru}    & pa.\textbf{ru}    & pa.\textbf{ru}    & pa.\textbf{ru}    & `big'                  \\ \hline
\multirow{2}{*}{VC}  & Non-final & ---      & ---      & \textbf{əm}.pi.tu   & \textbf{əm}.pi.tu   & `seven'                \\
                     & Final       & ku.la.\textbf{un} & kə.la.\textbf{un} & kə.la.\textbf{un} & kə.la.\textbf{un} & `to know (PV)'         \\ \hline
\multirow{2}{*}{VG}  & Non-final & ---      & \textbf{ow}.raw   & \textbf{ow}.raw   & ?        & `k.o. bamboo'          \\
                     & Final       & du.mu.\textbf{uy} & də.mu.\textbf{uy} & də.mu.\textbf{uy} & də.mu.\textbf{uy} & `to hold; to use (AV)' \\ \hline
\multirow{2}{*}{CVG} & Non-final & ---      & \textbf{dow}.riq  & \textbf{dow}.riq  & \textbf{dow}.riq  & `eye'                  \\
                     & Final       & lu.\textbf{buy}   & lu.\textbf{buy}   & lu.\textbf{buy}   & lu.\textbf{buy}   & `bag'                  \\ \hline
\multirow{2}{*}{CVC} & Non-final & ---      & ---      & ---      & ---      & ---                    \\
                     & Final       & ba.\textbf{raq}   & ba.\textbf{raq}   & ba.\textbf{raq}   & ba.\textbf{raq}   & `lungs'                \\ \hline
\end{tabular}
\end{table}

In casual spoken language, which tends to be faster-paced, non-final syllable structures like CGV, CVN, or even CVC might occur, but this varies depending on the dialect or region.

For instance, in \setrf, /bəjutux/ might be pronounced as [bju.tux] `pigeon', \stgf 
 /luŋuluŋan/ as [luŋ.lu.ŋan] `to think (LV)', and \setrf /mətəɣəsa/ as [mə.təɣ.sa] `to teach (AV).' However, a more detailed analysis awaits further investigation of (sub)dialects in the future.

\section{Proto-Seediq phonology}

In this section, a reconstruction of \psedf phonology will be provided based on the modern dialects. Furthermore, this section will cover the following subtopics: (1) \psedf phoneme inventory, (2) phonological reconstruction and sound changes, (3) \psedf phonotactics. 

\subsection{Proto-Seediq phoneme inventory} \label{psedphoin}

Table \ref{tab:psedC} shows the consonant inventory of \psedf. There are 18 consonants in \psedf based on my reconstruction, including:

\begin{enumerate}[label=(\roman*), itemsep=0pt, topsep=0pt]
    \item seven stops: *p, *t, *k, *q, *b, *d, *g
    \item one affricate: *c
    \item three fricatives: *s, *x, *h
    \item three nasals: *m, *n, *ŋ
    \item two liquids: *l, *r
    \item two glides: *w, *y
\end{enumerate}

\begin{table}[!htbp]
\centering
\caption{Proto-Seediq consonant inventory}
\label{tab:psedC}
\begin{tabular}{l|cc|cc|cc|cc|cc|cc}
\hline
                    & \multicolumn{2}{c|}{Bilabial} & \multicolumn{2}{c|}{Dental/Alveolar} & \multicolumn{2}{c|}{Palatal} & \multicolumn{2}{c|}{Velar} & \multicolumn{2}{c|}{Uvular} & \multicolumn{2}{c}{Pharyngeal} \\ \hline
Stop                & *p            & *b           & *t               & *d               &             &               & *k          & *g          & *q            &            &                 &              \\
Affricate           &               &              & *c               &                  &             &               &             &             &               &            &                 &              \\
Fricative           &               &              & *s               &                  &             &               & *x          &             &               &            & *h              &              \\
Nasal               &               & *m           &                  & *n               &             &               &             & *ŋ          &               &            &                 &              \\
Lateral  &               &              &                  & *l               &             &               &             &             &               &            &                 &              \\
Tap                 &               &              &                  & *r               &             &               &             &             &               &            &                 &              \\
Glide               &               & *w           &                  &                  &             & *y            &             &             &               &            &                 &              \\ \hline
\end{tabular}
\end{table}

The vowel inventory of \psedf is straightforward, as shown in Table \ref{tab:psedV}. There are four monophthongs in \psedf, which are *a, *i, *u, and *ə. Note that mid vowels [e] and [o] cannot be found in \psedf. 

\begin{table}[!htbp]
\centering
\caption{Proto-Seediq vowel inventory}
\label{tab:psedV}
\begin{tabular}{lcccccccc}
\cline{1-4} \cline{7-9}
     & Front & Central & Back &  &  & \multicolumn{3}{c}{``diphthongs''} \\ \cline{1-4} \cline{7-9}
High & *i    &         & *u   &  &  &            & *uy       &           \\
Mid  &       & *ə      &      &  &  &            &           &           \\
Low  &       & *a      &      &  &  & *ay        &           & *aw       \\ \cline{1-4} \cline{7-9}
\end{tabular}
\end{table}

There are also four ``diphthongs'' (VG sequences) in \psedf, including: *ay, *aw, and *uy. 

\subsection{Reconstructed phonemes and their reflexes}

\subsubsection{Reconstructed consonants and their reflexes}

The correspondence of \psedf *p can be found in word-initial and medial positions in all dialects, as given in Table \ref{tab:p}. However, all Seediq dialects are lacking in word-final -\textit{p}, since there is a final *-k and *-p merger as *-k from Proto-Atayalic to (Proto-)Seediq (\cite{li1981paic}). Li also points out that this merger is natural because [p] and [k] share the feature [+grave]. 

\begin{table}[!htbp]
\centering
\caption{Proto-Seediq *p}
\label{tab:p}
\begin{tabular}{c|c|cccc|l}
\textbf{\psed} & \textbf{\psed word}      & \textbf{\stg} & \textbf{\sto} & \textbf{\sctr} & \textbf{\setr} & \multicolumn{1}{c}{\textbf{Env./Gloss}} \\ \hline
\multirow{3}{*}{*p} & & p    & p    & p     & p     &                          \\             \cline{2-7}         
 & *\textbf{p}ada   & \textbf{p}ada & \textbf{p}ada & \textbf{p}ada & \textbf{p}ada & `muntjac' \\ 
 & *ra\textbf{p}it  & ra\textbf{p}ic & ra\textbf{p}ic & ra\textbf{p}ic & ra\textbf{p}it & `flying squirrel' \\ \hline
\end{tabular}
\end{table}

In suffixed form of those \psedf words contains a final *-k from the historical **-p, we can find the final consonant alternating. Such as *miyuk `to blow (AV)' and *yupun `to blow (PV)'. 

The reflexes of \psedf *t are relatively more complicated, as seen in Table \ref{tab:t}. \psedf *t is reflected as \textit{c} before \textit{i} or \textit{əy} in \sctrf and \setrf. Or, more precisely, *t changed to an affricate \textit{c} in this environment and underwent a palatalization process then became [tɕ] phonetically. 

\begin{table}[!htbp]
\centering
\caption{Proto-Seediq *t}
\label{tab:t}
\begin{tabular}{c|c|cccc|l}
\textbf{\psed} & \textbf{\psed word}      & \textbf{\stg} & \textbf{\sto} & \textbf{\sctr} & \textbf{\setr} & \multicolumn{1}{c}{\textbf{Env./Gloss}} \\ \hline
\multirow{8}{*}{*t} &  & t    & t    & c     & c     & /\_i, /\_əy                \\  \cline{2-7}
                    & *qu\textbf{t}i & qu\textbf{t}i & qu\textbf{t}i & qu\textbf{c}i & qu\textbf{c}i & `faeces'                          \\
                    & *\textbf{t}əyaquŋ & \textbf{t}uyaquŋ & \textbf{t}əyaquŋ & \textbf{c}əyaquŋ & \textbf{c}əyaquŋ & `crow'          \\ \cline{2-7}
                   &  & c    & c    & c     & t     & /\_\#                         \\ \cline{2-7}
                   & *səpa\textbf{t} & sepa\textbf{c} & səpa\textbf{c} & səpa\textbf{c} & səpa\textbf{t} & `four'                  \\ \cline{2-7}
                   & & t    & t    & t     & t     & /elsewhere               \\ \cline{2-7}
                   & *\textbf{t}unux & \textbf{t}unux & \textbf{t}unux & \textbf{t}unux & \textbf{t}unux  & `head'                   \\ 
                   & *u\textbf{t}ux & u\textbf{t}ux & u\textbf{t}ux & u\textbf{t}ux & u\textbf{t}ux  & `spirit'                        \\ \hline
\end{tabular}
\end{table}

In word-final position, \psedf *t is reflected as \textit{c} in \stgf, \stof, and \sctrf. However, this final affrication seems to be optional, since the speakers do not release the final stops all the time, especially if there is another word after it. For example, \textit{sepac papak} (\stg) can be pronounced as either [sepatsʰ papakʰ] or [sepat̚  papak̚ ]. 

\psedf *k is reflected as /k/ in all positions in every dialect, as shown in Table \ref{tab:k}. 

\begin{table}[!htbp]
\centering
\caption{Proto-Seediq *k}
\label{tab:k}
\begin{tabular}{c|c|cccc|l}
\textbf{\psed} & \textbf{\psed word}      & \textbf{\stg} & \textbf{\sto} & \textbf{\sctr} & \textbf{\setr} & \multicolumn{1}{c}{\textbf{Env./Gloss}} \\ \hline
\multirow{4}{*}{*k}  & & k    & k    & k     & k     &                          \\ \cline{2-7}
                    & *\textbf{k}ari & \textbf{k}ari & \textbf{k}ari & \textbf{k}ari & \textbf{k}ari & `language' \\ 
                    & *ba\textbf{k}a & ba\textbf{k}a & ba\textbf{k}a & ba\textbf{k}a & ba\textbf{k}a & `enough' \\
                    & *papa\textbf{k} & papa\textbf{k} & papa\textbf{k} & papa\textbf{k} & papa\textbf{k} & `animal foot' \\ \hline
\end{tabular}%
\end{table}

The voiceless uvular stop *q is also regular in the following dialects, and can be found in all positions, as seen in Table \ref{tab:q}. Besides, \textcite{li1981paic} notes that \paic *q has become an epiglottal stop /ʡ/ in Truwan Seediq. 

\begin{table}[!htbp]
\centering
\caption{Proto-Seediq *q}
\label{tab:q}
\begin{tabular}{c|c|cccc|l}
\textbf{\psed} & \textbf{\psed word}      & \textbf{\stg} & \textbf{\sto} & \textbf{\sctr} & \textbf{\setr} & \multicolumn{1}{c}{\textbf{Env./Gloss}} \\ \hline
\multirow{4}{*}{*q}  & & q    & q    & q     & q     &                          \\ \cline{2-7}
                    & *\textbf{q}uyux & \textbf{q}uyux & \textbf{q}uyux & \textbf{q}uyux & \textbf{q}uyux & `rain' \\
                    & *la\textbf{q}i & la\textbf{q}i & la\textbf{q}i & la\textbf{q}i & la\textbf{q}i & `child' \\
                    & *quwa\textbf{q} & quwa\textbf{q} & quwa\textbf{q} & quwa\textbf{q} & quwa\textbf{q} & `mouth' \\ \hline
\end{tabular}%
\end{table}

The correspondence of *b is generally regular in every dialect in the word-initial and medial positions, as given in Table \ref{tab:b}. Like /p/, /b/ can not be found in word-final position as well. *-b merged with *-k word-finally, but the final consonant alternation can still be noticed in the paradigm process, such as *məluk `to close (AV)' and *ləban `to close (LV)'. 

\begin{table}[!htbp]
\centering
\caption{Proto-Seediq *b}
\label{tab:b}
\begin{tabular}{c|c|cccc|l}
\textbf{\psed} & \textbf{\psed word}      & \textbf{\stg} & \textbf{\sto} & \textbf{\sctr} & \textbf{\setr} & \multicolumn{1}{c}{\textbf{Env./Gloss}} \\ \hline
\multirow{3}{*}{*b} & & b    & b    & b     & b     &                          \\ \cline{2-7}
                    & *\textbf{b}aki & \textbf{b}aki & \textbf{b}aki & \textbf{b}aki & \textbf{b}aki & `grandfather'         \\
                    & *ba\textbf{b}aw & bo\textbf{b}o & ba\textbf{b}aw & ba\textbf{b}aw & ba\textbf{b}aw & `above, top'      \\ \hline
\end{tabular}
\end{table}

\psedf *d is reflected as [ɟ] in \sctrf and \setrf before \textit{i} or \textit{əy}, which is also parallel to the reflex of \psedf *t. Apart from this case, \psedf *d is reflected as /d/ in all other word-initial and medial positions. Note that there is no word-final /d/ found in any dialect. For the correspondences of \psedf *d, please see Table \ref{tab:d}. 

\begin{table}[!htbp]
\centering
\caption{Proto-Seediq *d}
\label{tab:d}
\begin{tabular}{c|c|cccc|l}
\textbf{\psed} & \textbf{\psed word}      & \textbf{\stg} & \textbf{\sto} & \textbf{\sctr} & \textbf{\setr} & \multicolumn{1}{c}{\textbf{Env./Gloss}} \\ \hline
\multirow{5}{*}{*d} &  & d    & d    & ɟ     & ɟ     & /\_i, /\_əy                \\ \cline{2-7}
                    & *səə\textbf{d}iq & see\textbf{d}iq & sə\textbf{d}iq & səə\textbf{ɟ}iq & səə\textbf{ɟ}iq & `human'      \\ \cline{2-7}
                    & & d    & d    & d     & d     & /elsewhere               \\ \cline{2-7}
                    & *\textbf{d}ara & \textbf{d}ara & \textbf{d}ara & \textbf{d}ara & \textbf{d}ara & `blood'               \\ 
                    & *ru\textbf{d}an & ru\textbf{d}an & ru\textbf{d}an & ru\textbf{d}an & ru\textbf{d}an & `old (person)'   \\ \hline
\end{tabular}
\end{table}

Regular \psedf *g reflexes is also complicated among dialects, as given in Table \ref{tab:g}.  In word-initial and medial positions, \stof reflects \psedf *g as \textit{w}. All the other dialects retains \textit{g}. 

In word-final position, the reflexes depend on the last vowel. \psedf *-g is retained as \textit{g} after a high front vowel \textit{i} in \sctrf and \setrf; however, it is deleted in \stof (probably *-ig > **-iy > \textit{i}). In \stgf, *-ig is reflected as -\textit{uy}, which probably underwent the following gliding and metathesis process: *-ig > **-iw > -\textit{uy}. As for *-g after *u, it is deleted (through a glide /w/) in \stgf and \stof, but preserved as /g/ in \sctrf and \setrf. 

Note that there is no evidence for final *-ag can be found in Seediq. It seems that Pre-\psedf **-ag merged with **-aw as *-aw in \psedf stage. For example, \psedf *siyaw `next to', *babaw `on, above', while \pataf *siyag `rim; border; edge', *babaw `on top of.' 

\begin{table}[!htbp]
\centering
\caption{Proto-Seediq *g}
\label{tab:g}
\begin{tabular}{c|c|cccc|l}
\textbf{\psed} & \textbf{\psed word}      & \textbf{\stg} & \textbf{\sto} & \textbf{\sctr} & \textbf{\setr} & \multicolumn{1}{c}{\textbf{Env./Gloss}} \\ \hline
\multirow{2}{*}{*g-/*-g-}   & & g  & w  & g  & g  &  \\ \cline{2-7}
                            & *\textbf{g}a\textbf{g}a & \textbf{g}a\textbf{g}a & \textbf{w}a\textbf{w}a & \textbf{g}a\textbf{g}a & \textbf{g}a\textbf{g}a & `that; at; \textsc{prog}' \\ \hline
\multirow{2}{*}{*-i\textbf{g}}  & & \textbf{u}y & i  & i\textbf{g} & i\textbf{g} &  \\ \cline{2-7}
                                & *mari\textbf{g} & mar\textbf{u}y & mari & mari\textbf{g} & mari\textbf{g} & `to buy.AV' \\ \hline
\multirow{2}{*}{*-u\textbf{g}}  & & u  & u  & u\textbf{g} & u\textbf{g} &  \\ \cline{2-7}
                                & *əlu\textbf{g} & elu & əlu & əlu\textbf{g} & əlu\textbf{g} & `road; path' \\ \hline
\end{tabular}
\end{table}

Moreover, \psedf *g is not always reflected regularly among dialects. I name this as the ``GR problem'' of Seediq. Sometimes, the reflex is \textit{r} in specific words and the process seems to be dialect-independent, as given in Table \ref{tab:g3}. However, there is no environment of these correspondences that can be generalized. 

\begin{table}[]
\centering
\caption{Irregular reflexes of \psedf *g}
\label{tab:g3}
\begin{tabular}{l|l|llll|l}
\makecell[l]{\pata\\(\cite{goderich2020phd})} & \psed             & \sctr            & \setr            & \stg             & \sto             & Gloss           \\ \hline
*\textbf{g}əhap                               & *\textbf{g}əhak   & \textbf{g}əhak   & \textbf{g}əhak   & \textbf{r}ehak   & \textbf{r}əhak   & `seed'          \\
*-na\textbf{g}aʔ                              & *təma\textbf{g}a  & təma\textbf{g}a  & təma\textbf{g}a  & tuma\textbf{r}a  & təma\textbf{w}a  & `to wait'       \\
*ha\textbf{g}aʔ                               & *ha\textbf{g}at   & ha\textbf{r}ac   & ha\textbf{g}at   & ha\textbf{r}ac   & ha\textbf{w}ac   & `stone fence'   \\
*ma\textbf{ʔ}ipuh                             & *mə\textbf{g}ipuh & mə\textbf{r}ipuh & mə\textbf{r}ipuh & mu\textbf{g}ipuh & mə\textbf{r}ipuh & `fragile; soft' \\
*\textbf{w}aciluŋ                             & *\textbf{g}əciluŋ & \textbf{r}əsiluŋ & \textbf{g}əsiluŋ & \textbf{r}uciluŋ & \textbf{w}uciluŋ & `sea; lake'     \\
*\textbf{g}ipun                               & *\textbf{g}upun   & \textbf{g}upun   & \textbf{g}upun   & \textbf{r}upun   & \textbf{r}upun   & `teeth'         \\
---                                           & *\textbf{g}əbəyuk & ---              & \textbf{g}əbiyuk & \textbf{r}ubeyuk & \textbf{r}əbiyuk & `canyon'        \\
*\textbf{ʔ}umimag                             & *\textbf{g}əmimax & ---              & \textbf{g}əmimax & \textbf{g}umimax & mə\textbf{r}imax & `to mix'       
\end{tabular}
\end{table}

Please note that the regular reflex of \psedf *g in \stof is \textit{w}, but it shows \textit{r} in some words. Thus, this sound change occurred before \psedf *g become \textit{w} in \stof, or some of the words are borrowed from other dialects. 

As for the whole phenomenon, it cannot be simply attributed to borrowings, since some words are only found in Seediq or Atayal. At the same time, Atayal do not reflect Proto-Atayalic *g as *r or *ɹ. 

There is a possible reason for [g] becoming an \textit{r}-like sound, which is probably due to a perception issue. However, this assumption needs a phonetic perception experiment to proof it; thence, further studies should be carried out to clarify the cause of this irregular sound change. 

This can be also seen in the reflexes of \pan *R in \psedf as *g and *r. I will discuss this irregularity later in Section 7.2.5.

There is only one affricate *c in \psedf, its reflexes is shown in Table \ref{tab:c}. \psedf *c is preserved as affricate \textit{c} in \stgf and \stof, but is reflected as \textit{s} in \sctrf and \setrf. In addition, there is generally no word-final *c can be found in \psedf.

\begin{table}[!htbp]
\centering
\caption{Proto-Seediq *c}
\label{tab:c}
\begin{tabular}{c|c|cccc|l}
\textbf{\psed} & \textbf{\psed word}      & \textbf{\stg} & \textbf{\sto} & \textbf{\sctr} & \textbf{\setr} & \multicolumn{1}{c}{\textbf{Env./Gloss}} \\ \hline
\multirow{3}{*}{*c} & & c   & c   & s   & s   &  \\ \cline{2-7}
    & *\textbf{c}akus & \textbf{c}akus & \textbf{c}akus & \textbf{s}akus & \textbf{s}akus & `camphor' \\
    & *bi\textbf{c}ur & bi\textbf{c}un & bi\textbf{c}ur & bi\textbf{s}ur & bi\textbf{s}ur & `worm' \\ \hline
\end{tabular}
\end{table}

For \psedf *s, it is reflected as \textit{s} in all positions in all four dialects, as given in Table \ref{tab:s}. 

\begin{table}[!htbp]
\centering
\caption{Proto-Seediq *s}
\label{tab:s}
\begin{tabular}{c|c|cccc|l}
\textbf{\psed} & \textbf{\psed word}      & \textbf{\stg} & \textbf{\sto} & \textbf{\sctr} & \textbf{\setr} & \multicolumn{1}{c}{\textbf{Env./Gloss}} \\ \hline
\multirow{4}{*}{*s} & & s & s & s & s &  \\ \cline{2-7}
                    & *\textbf{s}apah & \textbf{s}apah & \textbf{s}apah & \textbf{s}apah & \textbf{s}apah & `house' \\
                    & *mi\textbf{s}an & mi\textbf{s}an & mi\textbf{s}an & mi\textbf{s}an & mi\textbf{s}an & `winter' \\
                    & *caku\textbf{s} & caku\textbf{s} & caku\textbf{s} & saku\textbf{s} & saku\textbf{s} & `camphor' \\ \hline
\end{tabular}
\end{table}

\psedf *x is regularly reflected as \textit{x} in all positions and all dialects as well, as seen in Table \ref{tab:x}. 

However, the occurrence of *x is highly restricted. Word-initial *x only appears in one word --- *xiluy `iron'. This word appear to be a Seediq innovation, since there is no similar form found in other languages. 

\begin{table}[!htbp]
\centering
\caption{Proto-Seediq *x}
\label{tab:x}
\begin{tabular}{c|c|cccc|l}
\textbf{\psed} & \textbf{\psed word}      & \textbf{\stg} & \textbf{\sto} & \textbf{\sctr} & \textbf{\setr} & \multicolumn{1}{c}{\textbf{Env./Gloss}} \\ \hline
\multirow{4}{*}{*x} & & x & x & x & x &  \\ \cline{2-7}
                    & *\textbf{x}iluy & \textbf{x}iluy & \textbf{x}iluy & \textbf{x}iluy & \textbf{x}iluy & `iron' \\
                    & *tə\textbf{x}al & te\textbf{x}an & tə\textbf{x}al & tə\textbf{x}al & tə\textbf{x}al & `once' \\
                    & *tunu\textbf{x} & tunu\textbf{x} & tunu\textbf{x} & tunu\textbf{x} & tunu\textbf{x}  & `head' \\  \hline
\end{tabular}
\end{table}


The words that contain word-medial *x are also rare. Also, those words are usually related to the meaning of `one', including *təxal `once', *taxa `one person', *maxal `ten', etc. These words are all with fossilized *-xa- or *-əxa- sequences, which are supposed to be related to \pan *əsa `one'. 

In \psedf, word-final *x can be found in large quantities. However, the historical source of this (proto-)phoneme is not yet clear. 

The regular reflex of \psedf *h is \textit{h} in all positions among all dialects, as shown in Table \ref{tab:h}.

\begin{table}[!htbp]
\centering
\caption{Proto-Seediq *h}
\label{tab:h}
\begin{tabular}{c|c|cccc|l}
\textbf{\psed} & \textbf{\psed word}      & \textbf{\stg} & \textbf{\sto} & \textbf{\sctr} & \textbf{\setr} & \multicolumn{1}{c}{\textbf{Env./Gloss}} \\ \hline
\multirow{4}{*}{*h} & & h & h & h & h &  \\ \cline{2-7}
                    & *\textbf{h}idaw & \textbf{h}ido & \textbf{h}idaw & \textbf{h}idaw & \textbf{h}idaw & `sun' \\
                    & *qu\textbf{h}iŋ & qu\textbf{h}iŋ & qu\textbf{h}iŋ & qu\textbf{h}iŋ & qu\textbf{h}iŋ & `head louse' \\
                    & *sapa\textbf{h} & sapa\textbf{h} & sapa\textbf{h} & sapa\textbf{h} & sapa\textbf{h} & `house' \\ \hline
\end{tabular}
\end{table}

\psedf *m is reflected as \textit{m} in word-initial and medial positions in all dialects, as seen in Table \ref{tab:m}.

\begin{table}[!htbp]
\centering
\caption{Proto-Seediq *m}
\label{tab:m}
\begin{tabular}{c|c|cccc|l}
\textbf{\psed} & \textbf{\psed word}      & \textbf{\stg} & \textbf{\sto} & \textbf{\sctr} & \textbf{\setr} & \multicolumn{1}{c}{\textbf{Env./Gloss}} \\ \hline
\multirow{3}{*}{*m} & & m   & m   & m   & m   \\ \cline{2-7}
                    & *\textbf{m}alu & \textbf{m}alu & \textbf{m}alu & \textbf{m}alu & \textbf{m}alu & `good' \\
                    & *ta\textbf{m}a & ta\textbf{m}a & ta\textbf{m}a & ta\textbf{m}a & ta\textbf{m}a  & `father' \\ \hline
\end{tabular}
\end{table}

Word-final \textit{m} cannot be found in any dialects. Like other bilabial phonemes such as *p and *b, \psedf *-m merged with its velar corresponding phoneme *-ŋ. The alternation can still be found in paradigm forms. For example, *migiŋ `to look for (AV)' and *giman `to look for (PV)'. 

\psedf *n and *ŋ are reflected as \textit{n} and \textit{ŋ} regularly in all positions among dialects, as seen in Table \ref{tab:n} and \ref{tab:ŋ} respectively. 

\begin{table}[!htbp]
\centering
\caption{Proto-Seediq *n}
\label{tab:n}
\begin{tabular}{c|c|cccc|l}
\textbf{\psed} & \textbf{\psed word}      & \textbf{\stg} & \textbf{\sto} & \textbf{\sctr} & \textbf{\setr} & \multicolumn{1}{c}{\textbf{Env./Gloss}} \\ \hline
\multirow{4}{*}{*n} & & n   & n   & n   & n \\ \cline{2-7}
                    & *\textbf{n}aqih & \textbf{n}aqah & \textbf{n}aqah & \textbf{n}aqih & \textbf{n}aqih & `bad' \\ 
                    & *i\textbf{n}u & i\textbf{n}u & i\textbf{n}u & i\textbf{n}u & i\textbf{n}u & `where' \\
                    & *rəbaga\textbf{n} & rubaga\textbf{n} & rəbawa\textbf{n} & rəbaga\textbf{n} & rəbaga\textbf{n} & `summer' \\ \hline
\end{tabular}
\end{table}

\begin{table}[!htbp]
\centering
\caption{Proto-Seediq *ŋ}
\label{tab:ŋ}
\begin{tabular}{c|c|cccc|l}
\textbf{\psed} & \textbf{\psed word}      & \textbf{\stg} & \textbf{\sto} & \textbf{\sctr} & \textbf{\setr} & \multicolumn{1}{c}{\textbf{Env./Gloss}} \\ \hline
\multirow{3}{*}{*ŋ} & & ŋ   & ŋ   & ŋ   & ŋ     \\ \cline{2-7}
                    & *\textbf{ŋ}a\textbf{ŋ}ut & \textbf{ŋ}a\textbf{ŋ}uc & \textbf{ŋ}a\textbf{ŋ}uc & \textbf{ŋ}a\textbf{ŋ}uc & \textbf{ŋ}a\textbf{ŋ}ut & `outside' \\
                    & *aru\textbf{ŋ}  & aru\textbf{ŋ} & aru\textbf{ŋ} & aru\textbf{ŋ} & aru\textbf{ŋ} & `pangolin' \\ \hline
\end{tabular}
\end{table}

\begin{table}[!htbp]
\centering
\caption{Proto-Seediq *r}
\label{tab:r}
\begin{tabular}{c|c|cccc|l}
\textbf{\psed} & \textbf{\psed word}      & \textbf{\stg} & \textbf{\sto} & \textbf{\sctr} & \textbf{\setr} & \multicolumn{1}{c}{\textbf{Env./Gloss}} \\ \hline
\multirow{5}{*}{*r} &  & n & r & r & r & /\_\# \\ \cline{2-7}
                    & *bicu\textbf{r} & bicu\textbf{n} & bicu\textbf{r} & bisu\textbf{r} & bisu\textbf{r} & `worm' \\ \cline{2-7}
                    & & r & r & r & r & /elsewhere \\ \cline{2-7}
                    & *\textbf{r}udan & \textbf{r}udan & \textbf{r}udan & \textbf{r}udan & \textbf{r}udan & `old (person)' \\
                    & *da\textbf{r}a & da\textbf{r}a & da\textbf{r}a & da\textbf{r}a & da\textbf{r}a & `blood'               \\ \hline
\end{tabular}
\end{table}

Word-final *r is reflected as \textit{n} in \stgf and is preserved as \textit{r} in other dialects. The reflexes of *r in other position show regular \textit{r}. Table \ref{tab:r} shows the reflexes of \psedf *r.  

The situation of \psedf *l is similar to the one of *r. \stgf reflects word-final *l as \textit{n}. All other reflexes in each dialect are \textit{l} in general, as given in Table \ref{tab:l}. 

\begin{table}[!htbp]
\centering
\caption{Proto-Seediq *l}
\label{tab:l}
\begin{tabular}{c|c|cccc|l}
\textbf{\psed} & \textbf{\psed word}      & \textbf{\stg} & \textbf{\sto} & \textbf{\sctr} & \textbf{\setr} & \multicolumn{1}{c}{\textbf{Env./Gloss}} \\ \hline
\multirow{5}{*}{*l} & & n & l & l & l & /\_\# \\ \cline{2-7}
                    & *təxa\textbf{l} & texa\textbf{n} & təxa\textbf{l} & təxa\textbf{l} & təxa\textbf{l} & `once' \\  \cline{2-7}
                   &  & l & l & l & l & /elsewhere \\  \cline{2-7}
                   & *\textbf{l}aqi & \textbf{l}aqi & \textbf{l}aqi & \textbf{l}aqi & \textbf{l}aqi & `child' \\ 
                   & *qaw\textbf{l}it & qo\textbf{l}ic & qow\textbf{l}ic & qow\textbf{l}ic & qow\textbf{l}it & `mouse' \\ \hline
\end{tabular}
\end{table}

\psedf *w is reflected as \textit{w} in word-initial and medial (syllable onset) positions and in all dialects, as seen in Table \ref{tab:w}. Syllable-final *w will be discussed together with vowels.

\begin{table}[!htbp]
\centering
\caption{Proto-Seediq *w}
\label{tab:w}
\begin{tabular}{c|c|cccc|l}
\textbf{\psed} & \textbf{\psed word}      & \textbf{\stg} & \textbf{\sto} & \textbf{\sctr} & \textbf{\setr} & \multicolumn{1}{c}{\textbf{Env./Gloss}} \\ \hline
\multirow{3}{*}{*w-/-w-} & &  w & w & w & w \\ \cline{2-7}
                         & *\textbf{w}ili & \textbf{w}ili & \textbf{w}ili & \textbf{w}ili & \textbf{w}ili & `leech' \\
                         & *ra\textbf{w}a & ra\textbf{w}a & ra\textbf{w}a & ra\textbf{w}a & ra\textbf{w}a & `basket' \\ \hline
\end{tabular}
\end{table}

\psedf word-initial and medial *y is reflected as \textit{y} in every dialect mentioned in this paper. As mentioned in Section \ref{sectgC}, the Paran dialect fortioned it as [ð]. However, the Paran dialect is (near-)extinct due to the collective emigration policy implemented by the Japanese colonial government. Thus, the [ð] sound is only preserved in few words of some elderly speakers from Paran. 

\begin{table}[!htbp]
\centering
\caption{Proto-Seediq *y}
\label{tab:y}
\begin{tabular}{c|c|cccc|l}
\textbf{\psed} & \textbf{\psed word}      & \textbf{\stg} & \textbf{\sto} & \textbf{\sctr} & \textbf{\setr} & \multicolumn{1}{c}{\textbf{Env./Gloss}} \\ \hline
\multirow{2}{*}{*y-/-y-} & & y & y & y & y \\ \cline{2-7}
                         & *\textbf{y}a\textbf{y}u & \textbf{y}a\textbf{y}u & \textbf{y}a\textbf{y}u & \textbf{y}a\textbf{y}u & \textbf{y}a\textbf{y}u & `small knife' \\ \hline
\end{tabular}
\end{table}

Syllable-final *y will be discussed together with vowels as well.

\subsubsection{Reconstructed vowels and their reflexes}

\paragraph{Monophthongs}

\psedf *a, *i and *u are reflected in all dialects as \textit{a}, \textit{i}, and \textit{u} respectively, as given in Table \ref{tab:aiu}. These three vowels can be only found in penultimate and the final syllable in \psedf.  

\begin{table}[!htbp]
\centering
\caption{Proto-Seediq *a, *i and *u}
\label{tab:aiu}
\begin{tabular}{c|c|cccc|l}
\textbf{\psed} & \textbf{\psed word}      & \textbf{\stg} & \textbf{\sto} & \textbf{\sctr} & \textbf{\setr} & \multicolumn{1}{c}{\textbf{Env./Gloss}} \\ \hline
\multirow{2}{*}{*a} & & a & a & a & a &  \\ \cline{2-7}
                    & *\textbf{a}l\textbf{a}ŋ & \textbf{a}l\textbf{a}ŋ & \textbf{a}l\textbf{a}ŋ & \textbf{a}l\textbf{a}ŋ & \textbf{a}l\textbf{a}ŋ & `village' \\ \hline
\multirow{2}{*}{*i} & & i & i & i & i &  \\ \cline{2-7}
                    & *\textbf{i}n\textbf{i} & \textbf{i}n\textbf{i} & \textbf{i}n\textbf{i} & \textbf{i}n\textbf{i} & \textbf{i}n\textbf{i} & `\textsc{neg}' \\ \hline
\multirow{2}{*}{*u} & & u & u & u & u &  \\ \cline{2-7}
                    & *r\textbf{u}d\textbf{u} & r\textbf{u}d\textbf{u} & r\textbf{u}d\textbf{u} & r\textbf{u}d\textbf{u} & r\textbf{u}d\textbf{u} & `nest' \\ \hline
\end{tabular}
\end{table}

As for \psedf *ə, it can only appears at non-final position. However, the penultimate schwa and the prepenultimate ones behave very differently among dialects. Therefore, they will be discussed separately. 

The penultimate *ə is regularly reflected as a mid front vowel \textit{e} in Tgdaya, but differently in other dialects. The penultimate *ə is preserved as \textit{ə} generally in \stof, \sctrf, and \setrf. However, if it is before \textit{y}, it will be assimilated as \textit{i}. Also, if there is no consonant between the *ə and the next vowel, the *ə will be assimilated as the same value of the next vowel. \stof tends to reflects *ə as \textit{u} when there is a *g (> \textit{w} in \stof) before or after it, but this does not occur in every case.  Table \ref{tab:pschwa} shows the reflexes of \psedf penultimate *ə. 

\begin{table}[!htbp]
\centering
\caption{Proto-Seediq *ə (penultimate)}
\label{tab:pschwa}
\begin{tabular}{c|c|cccc|l}
\textbf{\psed} & \textbf{\psed word}      & \textbf{\stg} & \textbf{\sto} & \textbf{\sctr} & \textbf{\setr} & \multicolumn{1}{c}{\textbf{Env./Gloss}} \\ \hline
\multirow{10}{*}{*ə} & & e                  & i                  & i                  & i                  & /\_yV(C)\#            \\  \cline{2-7}
                    & *b\textbf{ə}yax & b\textbf{e}yax & b\textbf{i}yax & b\textbf{i}yax & b\textbf{i}yax & `power' \\ \cline{2-7}
                    & & e                  & V\textsubscript{x} & V\textsubscript{x} & V\textsubscript{x} & /\_V\textsubscript{x}(C)\#        \\ \cline{2-7}
                    & *məh\textbf{ə}al & meh\textbf{e}an & məhal & məh\textbf{a}al & məh\textbf{a}al & `to carry on shoulder' \\
                    & *l\textbf{ə}iŋ & l\textbf{e}iŋ & tliŋ & l\textbf{i}iŋ & l\textbf{i}iŋ & `to hide' \\ \cline{2-7}
                    & & e                  & u                  & ə                  & ə                  & \makecell[l]{/g\_CV(C)\# (not all cases);\\/\_gV(C)\#}           \\ \cline{2-7}
                    & *g\textbf{ə}sak & g\textbf{e}sak & w\textbf{u}sak & g\textbf{i}sak & g\textbf{ə}sak & `k.o. bamboo tool' \\
                    & *m\textbf{ə}gay & m\textbf{e}ge & m\textbf{u}way & m\textbf{ə}gay & m\textbf{ə}gay & `to give.AV' \\ \cline{2-7}
                    & & e                  & ə                  & ə                  & ə                  & /elsewhere                        \\ \cline{2-7}
                    & *\textbf{ə}lug & \textbf{e}lu & \textbf{ə}lu & \textbf{ə}lug & \textbf{ə}lug & `road; path' \\ \hline
\end{tabular}
\end{table}

Unlike some Atayal dialects, none of the modern Seediq dialect still preserve phonemic distinctions of vowel in prepenultimate syllables. Generally speaking, it is better to reconstruct *ə for this prepenultimate neutralized (or reduced) vowel. It is reflected as \textit{ə} in \stof, \sctrf, and \setrf, and \textit{u} in \stgf if there is no other condition, as seen in Table \ref{tab:ppschwa}. [u] is more marked compare with [ə], and only one dialect reflects \textit{u}; thus, *ə is assigned because of the above reasons. 

As for the conditioned reflexes, again, since \stof reflects \textit{w} of \psedf *g, it effected the vowel quality of the prepenultimate *ə. Thus, the prepenultimate *ə is reflected as \textit{u} in \stof if there is a *g before or after it. 

In the \stgf dialect, the prepenultimate vowel will be assimilated by the following vowel if: (1) there is no consonant between two vowels, (2) the consonant between them is an \textit{h} [ħ]. 

\begin{table}[!htbp]
\centering
\caption{Proto-Seediq *ə (prepenultimate)}
\label{tab:ppschwa}
\begin{tabular}{c|c|cccc|l}
\textbf{\psed} & \textbf{\psed word}      & \textbf{\stg} & \textbf{\sto} & \textbf{\sctr} & \textbf{\setr} & \multicolumn{1}{c}{\textbf{Env./Gloss}} \\ \hline
\multirow{8}{*}{*ə} & & u                  & u                  & ə                  & ə                  & /g\_CVCV(C)\#; /\_gVCV(C)\#       \\ \cline{2-7}
                    & *g\textbf{ə}bian & b\textbf{u}bian & w\textbf{u}bian & g\textbf{ə}bian & g\textbf{ə}bian & `evening' \\ 
                    & *b\textbf{ə}gihur & b\textbf{u}gihun & b\textbf{u}wihur & b\textbf{ə}gihur & b\textbf{ə}gihur & `wind' \\ \cline{2-7}
                    & & V\textsubscript{x} & ə                  & ə                  & ə                  & /\_(h)V\textsubscript{x}CV(C)\#   \\ \cline{2-7}
                    & *m\textbf{ə}uyas & m\textbf{u}uyas & muyas & m\textbf{ə}uyas & m\textbf{ə}uyas & `to sing.AV' \\ 
                    & *c\textbf{ə}hədil & c\textbf{e}hedin & c\textbf{ə}hədil & s\textbf{ə}həɟil & s\textbf{ə}həɟil & `heavy' \\ \cline{2-7}
                    & & u                  & ə                  & ə                  & ə                  & /\_...σσ\# (``elsewhere'')                      \\ \cline{2-7}
& *d\textbf{ə}qəras & d\textbf{u}qeras & d\textbf{ə}qəras & d\textbf{ə}qəras & d\textbf{ə}qəras & `face' \\ \hline
                
\end{tabular}
\end{table}

In addition, \stof and some speakers of \sctrf tends to ``shorten'' a sequence of two identical vowels, as shown in Table \ref{tab:VxVx}. Note that in \stof, this is mandatory, but in \sctrf, it seems not very stable. 

\begin{table}[!htbp]
\centering
\caption{Proto-Seediq *V\textsubscript{x}V\textsubscript{x} sequence}
\label{tab:VxVx}
\begin{tabular}{c|c|cccc|l}
\textbf{\psed} & \textbf{\psed word}      & \textbf{\stg} & \textbf{\sto} & \textbf{\sctr} & \textbf{\setr} & \multicolumn{1}{c}{\textbf{Env./Gloss}} \\ \hline
\multirow{3}{*}{*V\textsubscript{x}V\textsubscript{x}} & & V\textsubscript{x}V\textsubscript{x} & V\textsubscript{x} & V\textsubscript{x}V\textsubscript{x} \~{ } V\textsubscript{x} & V\textsubscript{x}V\textsubscript{x} & \\ \cline{2-7}
 & *s\textbf{əə}diq & s\textbf{ee}diq & s\textbf{ə}diq & s\textbf{əə}ɟiq & s\textbf{əə}ɟiq & `human'      \\ 
 & *təl\textbf{əə}tu & tul\textbf{ee}tu & təl\textbf{ə}tu & təl\textbf{ə}tu & mətəl\textbf{əə}tu & `cold (thing)' \\ \hline

\end{tabular}
\end{table}

\paragraph{Diphthongs}

\psedf *ay and *aw are reflected as \textit{e} and \textit{o} respectively in \stgf in word-medial and final positions. On the contrary, they are preserved -\textit{ay} and -\textit{aw} word-finally but -\textit{ey}- and -\textit{ow}- word-medially in \stof, \sctrf, and \setrf. 

\psedf *uy can only appear at word-final position, and it is reflected as -\textit{uy} in all dialects. The correspondences of diphthongs are shown in Table \ref{tab:sedVG}.

\begin{table}[!htbp]
\centering
\caption{Proto-Seediq diphthongs (VG sequences)}
\label{tab:sedVG}
\begin{tabular}{c|c|cccc|l}
\textbf{\psed} & \textbf{\psed word}      & \textbf{\stg} & \textbf{\sto} & \textbf{\sctr} & \textbf{\setr} & \multicolumn{1}{c}{\textbf{Env./Gloss}} \\ \hline
\multirow{4}{*}{*ay} & & e  & ey & ey & ey & /\_σ        \\ \cline{2-7}
                     & *b\textbf{ay}luh & b\textbf{e}luh & b\textbf{ey}luh & b\textbf{ey}luh & b\textbf{ey}luh & `beans' \\ \cline{2-7}
                     & & e  & ay & ay & ay & /elsewhere \\ \cline{2-7}
                     & *bal\textbf{ay} & bal\textbf{e} & bal\textbf{ay} & bal\textbf{ay} & bal\textbf{ay} & `true' \\ \hline
\multirow{4}{*}{*aw} & & o  & ow & ow & ow & /\_σ        \\ \cline{2-7}
                     & *d\textbf{aw}riq & d\textbf{o}riq & d\textbf{ow}riq & d\textbf{ow}riq & d\textbf{ow}riq & `eye' \\ \cline{2-7}
                     & & o  & aw & aw & aw & /elsewhere \\ \cline{2-7}
                     & *bab\textbf{aw} & bob\textbf{o} & bab\textbf{aw} & bab\textbf{aw} & bab\textbf{aw} & `above; top' \\ \hline 
\multirow{2}{*}{*uy} & & uy & uy & uy & uy &            \\ \cline{2-7}
                     & *bab\textbf{uy} & bab\textbf{uy} & bab\textbf{uy} & bab\textbf{uy} & bab\textbf{uy} & `pig' \\ \hline 
\end{tabular}
\end{table}

Many readers may have noticed the inconsistency between the sound correspondences of *ay and *aw in word-medial position and word-final position. Based on the principles of the Comparative Method, these should be reconstructed as *-ey-/-ay and *-ow-/-aw to be more appropriate (while still being considered allophones of the same diphthongs). However, I manually ``adjusted'' them here for the following reasons:

Firstly, none of the modern dialects included in this study retain the *-ay- and *-aw- in word-medial position of \psedf. However, according to the investigation by \textcite{lee2012tawsa}, the Eastern Toda dialect (also known as Tawsa, Tawsay, Tuda, etc.)  retains the word-medial -\textit{ay}- and -\textit{aw}-. For example, \textit{m\textbf{ay}taq} `stab', and \textit{d\textbf{aw}riq} `eye.' Relevant examples are shown in Table \ref{tab:ctoeto}. Note that Lee uses <e> and <o> to transcribe the sounds in Central Toda, while this paper uses <ey> and <ow> since compared with \stgf, (Central) Toda has clear offglides such as [ej] and [ow]. 

\begin{table}[!htbp]
\centering
\caption{[e]-/ay/ and [o]-/aw/ correspondences in Central Toda and Eastern Toda (organized from \cite[95-96]{lee2012tawsa})}
\label{tab:ctoeto}
\begin{tabular}{llll}
\hline
                              & Central Toda & Eastern Toda & Gloss                                 \\ \hline
{[}e{]}-/ay/                  & metaq        & maytaq       & `\textsc{av}.stab'   \\ \hline
\multirow{5}{*}{{[}o{]}-/aw/} & soki         & sawki        & `sickle'                              \\
                              & doriq        & dawriq       & `eye'                                 \\
                              & tokan        & tawkan       & `man's basket'                        \\
                              & rodux        & rawdux       & `chicken'                             \\
                              & mosa         & mawsa/mosa   & `\textsc{av.irr}-go' \\ \hline
\end{tabular}
\end{table}

Nonetheless, due to the limited number of available lexical items, it was not included in the word list of this study and awaits more comprehensive vocabulary collection in the future. 

Based on the evidence from this modern dialect, reconstructing word-medial *-ay- and *-aw- in the \psedf stage poses no issues.

Secondly, some records from the Japanese colonial period indicate that various Seediq dialects retained the word-medial -\textit{ay}- and -\textit{aw}-. For example, the Musha/Paran dialect (related and similar to modern Tgdaya) in \textcite{ogawa2006voc} includes words like \textit{d\textbf{ao}ryak} (22), \textit{d\textbf{ao}deak} (23a) `eye', \textit{b\textbf{ai}roaχ} (23d) `bean', etc.\footnote{The code in the parentheses represents different speakers/villages/dialects in the book.} 

Similarly, \citeauthor{tashiro1900easterntw}'s (\citeyear{tashiro1900easterntw}) investigation of the Eastern Truku dialect (Taroko, 太魯閣) contains words such as \textit{t\textbf{au}kan} `net bag', \textit{b\textbf{ai}lo} `k.o. bean' that preserve -\textit{ay}- and -\textit{aw}- in their lexicon. 

However, there are also instances where speakers or villages documented in these sources have shifted to -\textit{e(y)}- and -\textit{o(w)}-. The raising of these two diphthongs or their further change to monophthongs appears to be independent sound changes that occurred over the past century across different dialects. Therefore, whether in the \psedf stage or at any possible subgroup's ancestral language level (see Section 5 for subgrouping and the Appendix for the word list), the diphthongs in the word-medial position should be reconstructed as *-ay- and *-aw-.



\subsection{Proto-Seediq phonotactics}

There are some restrictions of \psedf phonemes that I have already discussed in previous parts. For example, bilabial sounds including *p, *b, and *m are not able to occur in word-final position. However, if we consider the paradigm forms in \psedf, the alternation between bilabial sounds and velar sounds can be seen, as shown in Table \ref{tab:lab-vel}.

\begin{table}[!htbp]
\centering
\caption{Labial and velar alternations in paradigm forms of \psedf}
\label{tab:lab-vel}
\begin{tabular}{llll}
\hline
Alternation & AV form & Suffixed form & Gloss                  \\ \hline
p\~{ }k     & *<m>adu\textbf{k}  & *du\textbf{p}-un       & `to hunt (by chasing)' \\
b\~{ }k     & *<m>əlu\textbf{k}  & *lə\textbf{b}-an       & `to close'             \\
m\~{ }ŋ     & *<m>igi\textbf{ŋ}  & *gi\textbf{m}-an       & `to look for'          \\ \hline
\end{tabular}
\end{table}

The affricate *c could not appear word-finally as well. On the contrast, we cannot see any kinds of phoneme alternates with *c in terms of paradigm process. The voiced alveolar stop *d also cannot occur in the word-final position. It alternates with *t in the paradigm forms, as seen in Table \ref{tab:dandt}. Together with *b, it is clear that word-final *b and *d are devoiced. As for *g, it probably lenitioned as a fricative [ɣ] word-finally in \psedf, since it remains *-g in such a position. However, sequence *-ag cannot be found in Proto-Seediq word-finally, alternating with *-aw in paradigm forms, as seen in Table \ref{tab:agandaw}.  

\begin{table}[!htbp]
\centering
\caption{*d and *t alternations in paradigm forms of \psedf}
\label{tab:dandt}
\begin{tabular}{llll}
\hline
Alternation              & AV form  & Suffixed form & Gloss                   \\ \hline
\multirow{2}{*}{d\~{ }t} & *h<əm>aŋu\textbf{t} & *həŋə\textbf{d}-un     & `to cook'               \\
                         & *mə-haga\textbf{t} & *həga\textbf{d}-un     & `to pile (stone fence)' \\ \hline
\end{tabular}
\end{table}

\begin{table}[!htbp]
\centering
\caption{*ag and *aw alternations in paradigm forms of \psedf}
\label{tab:agandaw}
\begin{tabular}{llll}
\hline
Alternation              & AV form  & Suffixed form & Gloss                   \\ \hline
\multirow{2}{*}{ag\~{ }aw} & *r<əm>əŋ\textbf{aw} & *rəŋ\textbf{ag}-un     & `to speak'               \\
                         & *d<əm>ay\textbf{aw} & *dəy\textbf{ag}-un     & `to help' \\ \hline
\end{tabular}
\end{table}

The occurrence of voiceless velar fricative *x is also highly restricted. Word-initial *x can only be found in one word: *xiluy `iron', and medial ones are also rare, as mentioned in the sound correspondence subsection. 

The schwa *ə cannot occur in the final syllable, it also alternates with *u in such a case, as seen in Table \ref{tab:schwaandu}.

\begin{table}[!htbp]
\centering
\caption{*ə and *u alternations in paradigm forms of \psedf}
\label{tab:schwaandu}
\begin{tabular}{llll}
\hline
Alternation              & AV form    & Suffixed form & Gloss     \\ \hline
\multirow{2}{*}{ə\~{ }u} & *h<əm>aŋ\textbf{u}t & *həŋ\textbf{ə}d-un     & `to cook' \\
                         & *k<əm>ər\textbf{u}t & *kər\textbf{ə}t-un     & `to sow'   \\ \hline
\end{tabular}
\end{table}

There are several syllable structure types in \psedf, such as V, CV, VC, VG, CVG, and CVC. The structures of VC and CVC can only occur word-finally. For the examples of each syllable type, please see Table \ref{tab:syltype} for details. 

\begin{table}[!htbp]
\centering
\caption{\psedf syllable types}
\label{tab:syltype}
\begin{tabular}{llll}
\hline
Syllable type        & Position & Example   & Gloss                       \\ \hline
\multirow{2}{*}{V}   & Penultimate  & *\textbf{a}.laŋ    & `village'                   \\
                     & Final    & *ni.\textbf{i}     & `this'                      \\\hline  
\multirow{2}{*}{CV}  & Penultimate  & *\textbf{ru}.dan   & `old (person)'              \\
                     & Final    & *pa.\textbf{ru}    & `big'                       \\\hline  
\multirow{2}{*}{VC}  & Penultimate  & ---        & ---                         \\
                     & Final    & *mə.hə.\textbf{al} & `to carry on shoulder (AV)' \\\hline  
\multirow{2}{*}{VG}  & Penultimate  & *\textbf{aw}.raw        & `k.o. bamboo'                        \\
                     & Final    & *sa.\textbf{ay} & `to go (PV/LV.\textsc{hort})' \\\hline  
\multirow{2}{*}{CVG} & Penultimate  & *\textbf{daw}.riq  & `eye'                       \\
                     & Final    & *ru.\textbf{ŋay}   & `monkey'                    \\\hline  
\multirow{2}{*}{CVC} & Penultimate  & ---        & ---                          \\
                     & Final    & *ba.\textbf{raq}   & `lungs'         \\ \hline           
\end{tabular}
\end{table}

\section{Seediq subgrouping} \label{subgrouping}

In this section, I will propose a subgrouping hypothesis of Seediq dialects mainly based on shared phonological innovations. The historical development of \psedf *ə plays an important role in this subgrouping. Lexical evidence will be used as well, especially those contains sporadic sound change and those with replacement of the whole lexical item. 

\subsection{Subgrouping hypothesis}

The subgrouping of Seediq dialects based on the phonological and lexical evidence is presented in Figure \ref{fig:qhuni}. 

\begin{figure}[!htbp] 
\centering
\begin{forest}
for tree={l sep=5em, s sep=4em, inner sep=0, anchor=north, parent anchor=south, child anchor=north}
    [\psedf
        [\stgf, tier=word]
        [\ptotrf
            [\stof, tier=word]
            [\ptrf
                [\sctrf, tier=word]
                [\setrf, tier=word]
            ]
        ]
    ]
\end{forest}
\caption{Seediq subgrouping tree}\label{fig:qhuni}
\end{figure}

There is a subgroup of Toda-Truku, which is consisted of \stof, \sctrf, and \setrf. Moreover, \sctrf and \setrf can further form a subgroup under Toda-Truku. 

\stgf is thus divided as the other primary branch since there is no further evidence to link \stgf with the other dialects for now. On the other hand, \stgf itself has some phonological change which are very unique among Seediq dialects, especially the historical penultimate schwa *ə shift to /e/ in penultimate syllable and the pre-penultimate *ə shift to /u/ as the ``neutralized/reduced'' vowel. That is, Tgdaya is considered a primary offspring of \psedf. 

\subsection{Evidence for a Toda-Truku subgroup}

The \totrf subgroup is supported by two shared phonological changes, and several lexical changes. 

Two shared phonological changes are related to the historical development of schwa, and both are about vowel harmony/assimilation, but two changes occur in different conditions. 

One of their shared change the is that the penultimate schwa will be assimilated by the following vowel and become the same if there is no interluding consonant between them. The rule can be generalized as: \psed *ə > V\textsubscript{x} /\_V\textsubscript{x}(C)\#. The examples are shown in Table \ref{tab:totrsc1}, and the \setrf is the representative of modern dialects. 

\begin{table}[!htbp]
\centering
\caption{*ə harmony in \ptotrf}
\label{tab:totrsc1}
\begin{tabular}{llll}
\hline
\psed     & \ptotr    & \setrf   & Gloss                       \\ \hline
*məh\textbf{ə}al   & *məh\textbf{a}al   & məh\textbf{a}al   & `to carry on shoulder (AV)' \\
*məgəc\textbf{ə}as & *məgəc\textbf{a}as & məgəs\textbf{a}as & `brown-ish color'           \\
*cəm\textbf{ə}ax   & *cəm\textbf{a}ax   & səm\textbf{a}ax   & `to chop (AV)'              \\
*məs\textbf{ə}aŋ   & *məs\textbf{a}aŋ   & məs\textbf{a}aŋ   & `to be angry (AV)'          \\
*səməl\textbf{ə}an & *səməl\textbf{a}an & səməl\textbf{a}an & `to make (LV)'              \\
*ləm\textbf{ə}iŋ   & *ləm\textbf{i}iŋ   & ləm\textbf{i}iŋ   & `to hide (AV)'              \\
*qəhəd\textbf{ə}un & *qəhəd\textbf{u}un & qəhəd\textbf{u}un & `to end (PV)'               \\ \hline
\end{tabular}
\end{table}

The other shared change is also the harmony of schwa, \psedf *ə will be assimilated as high front vowel [i] by the homorganic glide \textit{y}. The rule can be written as: \psed *ə > *i /\_yV(C)\#. Table \ref{tab:totrsc2} shows the examples of this sound change. 

\begin{table}[!htbp]
\centering
\caption{*ə > *i in \ptotrf}
\label{tab:totrsc2}
\begin{tabular}{llll}
\hline
\psed    & \ptotr   & \setrf  & Gloss                 \\ \hline
*b\textbf{ə}yax   & *b\textbf{i}yax   & b\textbf{i}yax   & `power'               \\
*b\textbf{ə}yuq   & *b\textbf{i}yuq   & b\textbf{i}yuq   & `juice'               \\
*məh\textbf{ə}yug & *məh\textbf{i}yug & məh\textbf{i}yug & `to be standing (AV)' \\
*m\textbf{ə}yah   & *m\textbf{i}yah   & m\textbf{i}yah   & `to come (AV)'        \\
*kət\textbf{ə}yun & *kət\textbf{i}yun & kəc\textbf{i}yun & `to pluck (AV)'       \\ \hline
\end{tabular}
\end{table}

The lexical evidence with sporadic sound changes for the \totrf subgroup are shown in Table \ref{tab:totrlex}.  


\begin{table}[!htbp]
\centering
\caption{Lexical innovations with sporadic sound changes in \totrf}
\label{tab:totrlex}
%\begin{adjustwidth}{-0.885in}{-1in}
\begin{tabular}{lllllll}
\hline
\psed    & \ptotr   & \sto  & \sctr  & \setr       & Gloss                                                      \\ \hline
*səkadu  & *cəkacu  & cəkacu & səkasu  & səkasu       & `Formosan Michelia'  \\
*dəmahug & *cəmahug & cəmahu & səmahug & səmahug      & `to scoop up (AV)'   \\
*ŋayan   & *ŋahan   & haŋan  & haŋan   & haŋan; ŋahan & `name'                                               \\
*kəgig   & *kərig   & kəri   & kərig   & kərig        & `ramie'             \\ \hline          
\end{tabular}
%\end{adjustwidth}
\end{table}

Further sound changes from \ptotr to \stof is mainly the *g, *w merger as \textit{w}, this change also caused the \textit{ə} to be assimilated as \textit{u} (with counter example: \psed *gəlu > Toda \textit{wəlu}). Examples can be found in Table \ref{tab:todasc}.  

\begin{table}[!htbp]
\centering
\caption{*g, *w merger in Toda}
\label{tab:todasc}
\begin{tabular}{llll}
\hline
\psed    & \ptotr   & \ptr    & Gloss              \\ \hline
*\textbf{g}a\textbf{g}a    & *\textbf{g}a\textbf{g}a    & \textbf{w}a\textbf{w}a    & `that'             \\
*\textbf{w}a\textbf{w}a    & *\textbf{w}a\textbf{w}a    & \textbf{w}a\textbf{w}a    & `meat'             \\
*b\textbf{əg}ihur & *b\textbf{əg}ihur & b\textbf{uw}ihur & `wind'             \\
*\textbf{gə}sak   & *\textbf{gə}sak   & \textbf{wu}sak   & `k.o. bamboo tool' \\
*\textbf{gə}ciluŋ & \textbf{*gə}ciluŋ & \textbf{wu}ciluŋ & `lake; sea'        \\ \hline
\end{tabular}
\end{table}


\subsubsection{Evidence for a Truku subgroup}

The Truku subgroup is supported by two shared phonological changes, and several lexical changes as well.

First, \psedf and/or \ptotrf *c and *s merged as *s in \ptrf. The examples are given in Table \ref{tab:trcs}. 

\begin{table}[!htbp]
\centering
\caption{*c, *s merger in \ptrf}
\label{tab:trcs}
\begin{tabular}{llllll}
\hline
\psed      & \ptotr   & \ptr     & \sctr   & \setr   & Gloss               \\ \hline
(*səkadu)  & *\textbf{c}əka\textbf{c}u  & *\textbf{s}əka\textbf{s}u  & \textbf{s}əka\textbf{s}u  & \textbf{s}əka\textbf{s}u  & `Formosan Michelia' \\
(*dəmahug) & *\textbf{c}əmahug & *\textbf{s}əmahug & \textbf{s}əmahug & \textbf{s}əmahug & `to scoop up (AV)'       \\
*\textbf{c}amat     & *\textbf{c}amat   & *\textbf{s}amat   & \textbf{s}amac   & \textbf{s}amat   & `animal'            \\
*\textbf{c}aku\textbf{s}     & *\textbf{c}aku\textbf{s}   & *\textbf{s}aku\textbf{s}   & \textbf{s}aku\textbf{s}   & \textbf{s}aku\textbf{s}   & `camphor tree'      \\
*ma\textbf{c}u      & *ma\textbf{c}u    & *ma\textbf{s}u    & ma\textbf{s}u    & ma\textbf{s}u    & `Foxtail millet'    \\
*ma\textbf{s}ug     & *ma\textbf{s}ug   & *ma\textbf{s}ug   & ma\textbf{s}ug   & ma\textbf{s}ug   & `to divide (AV)'    \\ \hline
\end{tabular}
\end{table}

The second shared sound change is the palatalization of *t, *d as *c, *ɟ. Examples are shown in Table \ref{tab:trtdpal}.

\begin{table}[!htbp]
\centering
\caption{*t, *d palatalization in \ptrf}
\label{tab:trtdpal}
\begin{tabular}{llllll}
\hline
\psed      & \ptotr   & \ptr     & \sctr   & \setr   & Gloss               \\ \hline
*\textbf{t}imu  & *\textbf{t}imu  & *\textbf{c}imu  & \textbf{c}imu   & \textbf{c}imu   & `salt'          \\
*qu\textbf{t}i  & *qu\textbf{t}i  & *qu\textbf{c}i  & qu\textbf{c}i   & qu\textbf{c}i   & `faeces'        \\
*\textbf{d}iyan & *\textbf{d}iyan & *\textbf{ɟ}iyan & \textbf{ɟ}iyan  & \textbf{ɟ}iyan  & `daytime'       \\
*bu\textbf{d}i  & *bu\textbf{d}i  & *bu\textbf{ɟ}i  & bu\textbf{ɟ}i   & bu\textbf{ɟ}i   & `bow and arrow' \\ \hline
\end{tabular}
\end{table}

The lexical evidence for the Truku subgroup can be found in Table \ref{tab:trlex}. 

\begin{table}[!hbtp]
\centering
\caption{Lexical innovations in Truku}
\label{tab:trlex}
\begin{tabular}{llllll}
\hline
\psed      & \ptotr   & \ptr     & \sctr   & \setr   & Gloss               \\ \hline
*nunuh  & *nunuh  & *unuh    & unuh    & unuh    & `breast'        \\
*saya   & *saya   & *sayaŋ   & sayaŋ   & sayaŋ   & `today'         \\
*sawni  & *sawni  & *suni    & suni    & suni    & `just; now'     \\
*puru   & *puru   & *pupu    & pupu    & pupu    & `axe'           \\
*kəduruk& *kəduruk& *pəŋəlux & pəŋəlux & pəŋəlux & `forehead'      \\
*rawdux & *rawdux & *rudux   & rudux   & rudux   & `chicken'       \\
*-ruma  & *ruma   & *ruwan   & ruwan   & ruwan   & `inside'        \\
*ariŋ   & *ariŋ   & *qəlupas & qəlupas & qəlupas & `peach'         \\
*məluk  & *məluk  & *məduk   & məduk   & məduk   & `to close (AV)' \\
*ləban  & *ləban  & *dəpan   & dəpan   & dəpan   & `to close (LV)' \\ \hline
\end{tabular}
\end{table}

Two Truku dialects has nearly the same phonological system, except minor phonetic differences. The crucial evidence that distinguishes \setrf from \sctrf are lexical innovations. Note that these are mainly basic words, such as `woman', `fire', `many', etc, as shown in Table \ref{tab:etrlex}.

\begin{table}[!hbtp]
\centering
\caption{Lexical innovations in \setrf}
\label{tab:etrlex}
\begin{tabular}{llllll}
\hline
\psed      & \ptotr   & \ptr     & \sctr   & \setr   & Gloss               \\ \hline
*qəridil & *qəridil & *qəriɟil & qəriɟil & kuyuh   & `woman'         \\
*puniq   & *puniq   & *puniq   & puniq   & tahut   & `fire'          \\
*əgu     & *əgu     & *əgu     & əgu     & lala    & `many'          \\
*tuya    & *tuya    & *tuya    & tuya    & tuwil   & `eel'           \\
*cilu    & *cilu    & *silu    & silu    & təqətaq & `gecko'         \\ \hline
\end{tabular}
\end{table}


\section{Issues in Proto-Seediq lexical reconstruction}

In this section, two main topics will be discussed, respectively the special doublet phenomenon in \psedf and external evidence for \psedf lexical reconstruction. 

\begin{comment}
\subsection{Personal pronoun system in Proto-Seediq}

The personal pronoun system of \psedf is shown in Table \ref{tab:psedpron}. Just like many of Austronesian languages, Seediq distinguishes the differences of first person plural inclusive and exclusive. 

For the clitic (bound) forms, there are two sets of pronouns, some of the forms are the same between two cases. The oblique free forms are fully preserved in \stof, \sctrf, and \setrf. However in the \stgf dialect, there are only few remained, including \textit{kənan} `\textsc{1sg.obl}', \textit{sunan} `\textsc{2sg.obl}', and \textit{munan} `\textsc{1sg.obl}'. Note that \psedf *munan is for second person plural, but \textit{munan} in \stgf is for first person singular according to \textcite[62]{Sung2018Sedgrammar}. These oblique forms in \stgf are not usually used in daily speeches. The set of original nominative free form replace their function; the original nominative forms thus became neutral in \stgf (i.e. Nominative and oblique cases share the same forms).

\begin{table}[!htbp]
\centering
\caption{\psedf personal pronouns}
\label{tab:psedpron}
\begin{tabular}{lllll}
\hline
         & \multicolumn{2}{c}{Clitic} & \multicolumn{2}{c}{Free} \\ \hline
         & Nominative    & Genitive   & Nominative & Oblique     \\ \hline
\textsc{1sg}      & *=ku          & *=mu       & *yaku      & *kənan      \\
\textsc{2sg}      & *=su          & *=su       & *isu       & *sunan      \\
\textsc{3sg}      & ∅             & *=na; *=niya & *hiya      & *həyaan     \\
\textsc{1pl.incl} & *=ta          & *=ta       & *ita       & *tənan      \\
\textsc{1pl.excl} & *=nami        & *=nami     & *yami      & *mənan      \\
\textsc{2pl}      & *=namu        & *=namu     & *yamu      & *munan      \\
\textsc{3pl}      & ∅             & *=dəha     & *dəhəya    & *dəhəyaan   \\ \hline
\end{tabular}
\end{table}

Furthermore, \psedf also has three portmanteau pronouns, as shown in Table \ref{tab:por}.

\begin{table}[!htbp]
\centering
\caption{Portmanteau personal pronouns in \psedf}
\label{tab:por}
\begin{tabular}{ll}
\hline
Form   & Gloss \\ 
\hline
*=misu & `\textsc{1sg.gen+2sg.nom}' \\
*=saku & `\textsc{2sg.gen+1sg.nom}' \\
*=maku & `\textsc{1sg.gen+2pl.nom}' \\
\hline
\end{tabular}
\end{table}

In general, clitic pronouns are placed after the predicate in the sentence, except for possessive structure. The possessive structure in Seediq can be described as ``possessee=possessor.'' For example, \textit{tama=mu} `father=\textsc{1sg.gen}'. Regarding the detailed usage of more Seediq clitic pronouns in different dialects, please refer to \textcite{ochiai2009sedpron,kuondtrvpronoun,holmer2014clitic}.

Next, I will examine the reflexes of personal pronouns in each dialect set by set.

\subsubsection{Nominative clitic}

The five different personal pronouns in the nominative clitic form are reflected consistently across all dialects, as shown in Table \ref{tab:nomclitic}. 

\begin{table}[!htbp]
\centering
\caption{Reflexes of nominative clitic personal pronouns in Seediq dialects}
\label{tab:nomclitic}
\begin{tabular}{llllll}
\hline
       & =\textsc{1sg.nom} & =\textsc{2sg.nom} & =\textsc{1pl.nom.incl} & =\textsc{1pl.nom.excl}    & =\textsc{2pl.nom} \\ \hline
\psed & *=ku    & *=su    & *=ta         & *=nami          & *=namu  \\
\stg  & =ku     & =su     & =ta          & =nami; (=miyan) & =namu   \\
\sto  & =ku     & =su     & =ta          & =nami           & =namu   \\
\sctr & =ku     & =su     & =ta          & =nami           & =namu   \\
\setr & =ku     & =su     & =ta          & =nami           & =namu   \\ \hline
\end{tabular}
\end{table}

However, there is an additional form in \stgf, which is =\textit{miyan} `\textsc{1pl.nom.incl}'. It is speculated that this form might be borrowed from Atayal, where the corresponding form is =\textit{myan} meaning `\textsc{1pl.gen.incl}'. Furthermore, in Seediq, the first person plural nominative and genitive clitics have the same form, suggesting a possible further expansion in this loanword's function.

\subsubsection{Genitive clitic}

The genitive clitic forms of the five personal pronouns are also reflected consistently across the various dialects. Table \ref{tab:genclitic} shows the reflexes. 

\begin{table}[!htbp]
\centering
\caption{Reflexes of genitive clitic personal pronouns in Seediq dialects}
\label{tab:genclitic}
\begin{tabular}{lllll}
\hline
      & =\textsc{1sg.gen}      & =\textsc{2sg.gen}        & =\textsc{3sg.gen} & =\textsc{3sg.gen}       \\ \hline
\psed & *=mu          & *=su            & *=na     & *=niya         \\
\stg  & =mu           & =su             & =na      & ---            \\
\sto  & =mu           & =su             & =na      & =niya          \\
\sctr & =mu           & =su             & =na      & =niya          \\
\setr & =mu           & =su             & =na      & =niya          \\ \hline
      & =\textsc{1pl.gen.incl} & =\textsc{1pl.gen.excl}   & =\textsc{2pl.gen} & =\textsc{3pl.gen}       \\ \hline
\psed & *=ta          & *=nami          & *=namu   & *=dəha         \\
\stg  & =ta           & =nami; (=miyan) & =namu    & =daha          \\
\sto  & =ta           & =nami           & =namu    & =dəha          \\
\sctr & =ta           & =nami           & =namu    & =dəha          \\
\setr & =ta           & =nami           & =namu    & =dəha; (=nəha) \\ \hline
\end{tabular}
\end{table}

In addition to the previously mentioned borrowing from Atayal in \stgf with the form \textit{miyan}, there are two other notable points. First, `\textsc{3sg.gen}' seems to have two forms in \psedf stage, one short form *=na and one long form *=niya, although the latter is lost in \stgf. Another point is that according to \textcite{Lee2022TrukuPOS}, \setrf has =\textit{nəha} meaning `\textsc{3pl.gen}', but I has not found in other dialects, suggesting it might be borrowed from Atayal as well, where the form is \textit{nəhaʔ} `\textsc{3pl.gen}'.

\subsubsection{Nominative free}

Similarly, the nominative free forms show no significant variation or innovation across dialects. Table \ref{tab:nomfree} presents the forms in each dialect.

\begin{table}[!htbp]
\centering
\caption{Reflexes of nominative free personal pronouns in Seediq dialects}
\label{tab:nomfree}
\begin{tabular}{lllll}
\hline
      & \textsc{1sg.nom}      & \textsc{2sg.nom}      & \textsc{3sg.nom} &         \\ \hline
\psed & *yaku        & *isu         & *hiya   &         \\
\stg  & yaku         & isu          & heya    &         \\
\sto  & yaku         & isu          & hiya    &         \\
\sctr & yaku         & isu          & hiya    &         \\
\setr & yaku         & isu          & hiya    &         \\ \hline
      & \textsc{1pl.nom.incl} & \textsc{1pl.nom.excl} & \textsc{2pl.nom} & \textsc{3pl.nom} \\ \hline
\psed & *ita         & *yami        & *yamu   & *dəhəya \\
\stg  & ita          & yami         & yamu    & deheya  \\
\sto  & ita          & yami         & yamu    & dəhiya  \\
\sctr & ita          & yami         & yamu    & dəhiya  \\
\setr & ita          & yami         & yamu    & dəhiya  \\ \hline
\end{tabular}
\end{table}

However, it may be worth noting the origins of the \textsc{3sg.nom} and \textsc{3pl.nom} forms. Although they may appear similar, when reconstructing them, we need to consider external evidence and their etymology.

The \textsc{3sg.nom} form is derived from \pan *si ia, with its corresponding Atayal form being \textit{hiyaʔ}. As for the \textsc{3pl.nom} form *dəhəya, it may not have a clear \pan source, but it corresponds to Atayal forms such as \textit{ləhaʔ} and \textit{ləhəgaʔ}. The form \textit{ləhaʔ} in Atayal may have originated from *ləhəgaʔ > **ləhaʔaʔ > \textit{ləhaʔ}, as the deletion of /g/ would lead to further vowel coalescence. For discussions on /g/ and /ʔ/ in Atayal, please refer to \textcite{goderich2020phd,song2023Aicgprime}. 

\subsubsection{Oblique free}

Unlike the previous sets, the oblique forms in \stgf are nearly obsolete and are mostly replaced by the nominative forms, as mentioned earlier. The remaining forms, \textit{kenan}, \textit{sunan}, \textit{munan}, are only used in specific situations and vary from person to person (\cite{Sung2018Sedgrammar}).

\begin{table}[!htbp]
\centering
\caption{Reflexes of oblique free personal pronouns in Seediq dialects}
\label{tab:oblfree}
\begin{tabular}{lllll}
\hline
      & \textsc{1sg.obl}      & \textsc{2sg.obl}      & \textsc{3sg.obl} &           \\ \hline
\psed & *kənan       & *sunan       & *həyaan &           \\
\stg  & (kenan)      & (sunan)      & ---     &           \\
\sto  & kənan        & sunan        & həyaan  &           \\
\sctr & kənan        & sunan        & həyaan  &           \\
\setr & kənan        & sunan        & həyaan  &           \\ \hline
      & \textsc{1pl.obl.incl} & \textsc{1pl.obl.excl} & \textsc{2pl.obl} & \textsc{3pl.obl}   \\ \hline
\psed & *tənan       & *mənan       & *munan  & *dəhəyaan \\
\stg  & ---          & ---          & (munan) & ---       \\
\sto  & tənan        & mənan        & munan   & dəhəyaan  \\
\sctr & tənan        & mənan        & munan   & dəhəyaan  \\
\setr & tənan        & mənan        & munan   & dəhəyaan  \\ \hline
\end{tabular}
\end{table}

The oblique forms in \psedf are believed to have inherited from the reconstructed Acc(usative) set of personal pronouns proposed by \textcite{ross2006casepronoun} (Except for the third person forms), with the addition of the suffix *-an. A comparison with Paiwan and Ross's \pan Acc personal pronouns' reconstruction is shown in Table \ref{tab:panacc}

\begin{table}[!htbp]
\centering
\caption{A comparison of \pan \textsc{acc}, Paiwan \textsc{nom} (clitic), and \psedf \textsc{obl} personal pronouns}
\label{tab:panacc}
\begin{tabular}{lllllll}
\hline
       &     & \textsc{1sg}      & \textsc{2sg}        & \textsc{1pl.incl} & \textsc{1pl.incl} & \textsc{2pl}         \\\hline
\pan   & \textsc{acc} & *i-ak-ən & *iSu[qu]-n & *ita-ən  & *i-am-ən & *i-mu[qu]-n \\
Paiwan & \textsc{nom} & =[a]kən  & =[ə]sun    & =[i]cən  & =[a]mən  & =[ə]mun     \\
\psedf & \textsc{obl} & *kən-an  & *sun-an    & *tən-an  & *mən-an  & *mun-an    \\ \hline
\end{tabular}
\end{table}

\subsubsection{Portmanteau forms}

Due to the lack of data from \stof and \sctrf, this presentation will only include data from \stgf (from \cite[62]{Sung2018Sedgrammar}) and \setrf (from \cite[74]{Lee2018Trugrammar}), as shown in Table \ref{tab:porref}.

\begin{table}[!htbp]
\centering
\caption{Reflexes of portmanteau personal pronouns in Seediq dialects}
\label{tab:porref}
\begin{tabular}{llll}
\hline
                     & \textsc{1sg.gen+2sg.nom}                 & \textsc{2sg.gen+1sg.nom}                  & \textsc{1sg.gen+2pl.nom}                \\ 
\multicolumn{1}{c}{} & \multicolumn{1}{c}{(I to thee)} & \multicolumn{1}{c}{(Thou to me)} & \multicolumn{1}{c}{(I to you)} \\ \hline
\psedf               & *=misu                          & *=saku                           & *=maku                         \\
\stg                 & =misu                           & =saku                            & =maku                          \\
\setr                & =misu                           & =saku                            & =maku          \\ \hline                
\end{tabular}
\end{table}

Portmanteau pronouns, like other clitic pronouns, are enclitics placed after the predicate. They consist of two arguments and are found exclusively in (1) nominal predicate; and (2) non-agent voice (NAV) sentences, where they represent the agent or experiencer. Here are two examples from \textcite[74--75]{Lee2018Trugrammar}.

\begin{exe}

    \ex Nominal predicate sentence
    \gll \textit{anay=misu} \textit{ka} \textit{isu.} \\
    sister's.husband=\textsc{1sg.gen+2sg.nom} \textsc{nom} \textsc{2sg.nom}\\
    \trans `You are my sister's husband. (For me, you are my sister's husband.) '
    \ex NAV sentence
    \gll \textit{wada=misu} \textit{qta-an} \textit{da!} \\
    \textsc{perf}=\textsc{1sg.gen+2sg.nom} see-\textsc{lv} \textsc{ptcl}\\
    \trans `I have already seen you!'
\end{exe}

\end{comment}

\subsection{Special doublets in Proto-Seediq lexicon}

Across the modern dialects of the Seediq language, there are words with similar, but not identical forms, where their meanings are the same or related. Often the difference is just a few segments. Sometimes the two similar forms appear together in the same dialect, and sometimes they appear separately in different dialects without any conditions or rules. 

If we reconstruct these similar forms found in different dialects into \psedf, we can observe that these ``doublets'' in \psedf can be categorized into three types based on their characteristics: (1) the possible affixation (insertion of an infix <ra> or <na>, or suffixing) (Table \ref{tab:gender1}); (2) a difference of one segment (Table \ref{tab:gender2}); (3) a difference of one syllable (Table \ref{tab:gender3}). I refer to these phenomena as ``special doublets.''

Please note that some reconstructed forms are enclosed in parentheses. These forms may be difficult to reconstruct to the \psedf stage based on the current evidence, and are listed here only for comparison purposes. These lexical items could be innovations in a single dialect or within a specific subgroup. However, as seen in the examples in Table \ref{tab:gender1}, unaffixed forms, even if found in a single dialect, must be reconstructed back to \psedf, as these forms clearly represent a more original state. Some forms can find external evidence in \pataf or Atayal dialects, and therefore can be reconstructed back to \psedf. Relevant examples will be mentioned to present external evidence.

Note that there are two \pataf forms are reconstructed/revised by me, which are *quras `gray hair' and *ʔiɹil `left.' \textcite{goderich2020phd} did not reconstructed *quras. However, \textit{quras} can be found in the Matu'uwal (Mayrinax) dialect of Atayal as the female form (from \cite[6]{li1983gender}), so I reconstructed it here. As for *ʔiɹil, Goderich reconstructs *ʔil for this word; however, Squliq \textit{ʔəzil} and Skikun \textit{ʔiyil} imply that there should be a *ɹ or *y in the proto-form. Combine with the evidence from Seediq, the Proto-Atayal form should be revised as *ʔiɹil. As for the *ɹ-deletion in Plngawan \textit{ʔil} needs further investigation, since Plngawan usually preserves historical *ɹ as [ɹ].

\begin{table}[!htbp]
\centering
\caption{Special doublet phenomenon --- affixation}
\label{tab:gender1}
\begin{adjustwidth}{-0.3in}{-1in}
\begin{tabular}{lllllll}
\hline
\psed    &               & \stg                         & \sto    & \sctr & \setr         & Gloss                    \\ \hline
*babaw   &               & bobo                         & babaw   & babaw & babaw         & `above; on top of'       \\
*baraw   & < **bəba<ra>w & baro                         & baraw   & baraw & baraw         & `above; on'       \\
*rahut   &               & rahuc                        & tərahuc &       &               & `below' \\
*hunat   & < **rəhu<na>t & hunac                        & hunat `south' &       & hunat `south' & `below' \\
*ŋaŋut   &               & ŋaŋuc                        & ŋaŋuc   & ŋaŋuc & ŋaŋut         & `outside'                  \\
(*ŋərat)   & < **ŋəŋə<ra>t & ŋerac                        &         &       &               & `outside'                  \\
*ahiŋ    &               & ahiŋ                         &         &       &               & `shoulder'                 \\
*hiraŋ   & < **əhi<ra>ŋ  &                              & hiraŋ   & hiraŋ & hiraŋ         & `shoulder'      \\ 
*hapuy   &               &                              & məhapuy & məhapuy & məhapuy     & `to cook (AV)'     \\
(*həpuray) & < **həpu<ra>y & hupure                       &         &       &               & `to cook (AV) \\
*puniq   & < **həpu-niq  & puniq                        & puniq   & puniq &               & `fire' \\ \hline        
\end{tabular}
\end{adjustwidth}
\end{table}

\begin{table}[!htbp]
\centering
\caption{Special doublet phenomenon --- Segment substitution}
\label{tab:gender2}
\begin{tabular}{lllllll}
\hline
\pata & \psed    & \stg    & \sto    & \sctr   & \setr   & Gloss               \\ \hline
*gVlabaŋ & *ləlabaŋ &         & ləlabaŋ & ləlabaŋ & ləlabaŋ & `broad; wide'         \\
*gVlahaŋ & *gəlahaŋ & gulahaŋ &         &         &         & `broad; wide'         \\
 & (*daliŋ)   & daliŋ   &         &         &         & `close; near; nearby' \\
 & (*dalih)   &         & dalih   & dalih   & dalih   & `close; near; nearby' \\
 & *məridil & muridin & məridil & məriɟil & məriɟil & `crooked; diagonal'  \\
 & (*məridih) & muridih &         &         &         & `crooked; diagonal'   \\
*cəkaʔ & *cəka    & ceka    & cəka    & səka    & səka    & `half; to halve; to split; middle'     \\
 & *dəka    & deka    & dəka    & dəka    & dəka    & `half; to halve'     \\ \hline  
\end{tabular}
\end{table}

\begin{table}[!htbp]
\centering
\caption{Special doublet phenomenon --- Syllable substitution}
\label{tab:gender3}
\begin{tabular}{lllllll}
\hline
\pata & \psed   & \stg    & \sto    & \sctr  & \setr         & Gloss                 \\ \hline
*raŋiʔ & *daŋi  &  daŋi  & daŋi   & daŋi       & daŋi `lover'  & `friend'             \\
*rawin & *dawin   &    &         &        & \makecell[l]{dawin `partner;\\darling (to man)'}         & `friend?'             \\
*quras & *qudas  &         & qudas   &        &               & `gray hair'             \\
*quriʔ & *qudi   & qudi    &         &        & quɟi          & `gray hair'             \\
*ɹaŋaw & *raŋaw  & ruŋedi raŋo  & raŋaw   &        &          & `blowfly' \\
*ɹaŋəriʔ & *rəŋədi & ruŋedi  & rəŋədi  & rəŋəɟi & rəŋəɟi        & `housefly'             \\
 & *igi    &         & iwi     & igi    &               & `left'                  \\
*ʔiɹil & *iril   & irin    & iril    & iril   & iril          & `left'                  \\
*ʔukas & *uka    & uka     & uka     & uka    &               & `not exist'             \\
*ʔuŋat & *uŋat   &         &         & uŋac   & uŋat          & `not exist'             \\
 & (*dəgə-CVC)  & dugehiŋ & dəwəliŋ &        & dəgəril       & `narrow'                \\
 & (*təhə-CVC)  & teheruy & təhədil & təhəɟil & təhəɟil      & \makecell[l]{`to migrate;\\to move'}   \\       
\hline
\end{tabular}
\end{table}

In these three tables, we can observe that when a dialect has two forms, sometimes they appear to be synonymous, while at other times, there seems to be a slight difference in meaning. Occasionally, we can already see the difference in meaning in \psedf.

For example, the doublet \textit{bobo}/\textit{babaw}; \textit{baro}/\textit{baraw}. In Seediq dialects, the former expresses the meaning `on top of' with actual contact between objects, as seen in examples like \textit{bobo btunux} `on top of the stone' and \textit{bobo qsiya} `on the water' in the \stgf dialect. On the other hand, the latter indicates a more general sense of being above without necessarily requiring physical contact, as seen in examples like \textit{karac baro} `in the sky' and \textit{qhuni baro} `on the tree.' 

We can also notice that in fact, \textit{bobo} functions more like a preposition, while \textit{baro} behaves more like a locative noun or directional noun indicating a specific location or direction. Furthermore, \textit{bobo} has further grammaticalized into expressing `after (something); later' in a temporal sense. Although this doublet likely originated from the same etymology, language change and development has resulted in more distinct usage and functions.

However, in contrast, for the doublet \textit{ŋerac} and \textit{ŋaŋuc} `outside', it is currently difficult to find such significant differences. The relevant examples found in the dictionary of \textcite{ILRDFEdict} do not support the assumption that there must be semantic differences between two words whenever a doublet exists.

In Table \ref{tab:gender3}, we can observe many doublet combinations where one source comes from \panf, accompanied by another more innovative form. For example, \pan *qudaS > \psedf *qudas and *qud-i `gray hair'; \pan *laŋaw `housefly' > \psedf *raŋaw `blowfly' and *rəŋ-ədi `housefly'; \pan *wiRi > \psedf *igi and *i-ril `left'; \pan *uka > \psedf *uka and *u-ŋat `not exist.'

So far, my discussion has mainly focused on the form or meaning of these special doublets. But why does this phenomenon of special doublets occur, and what does it signify? How is the other form constructed? I have noticed that this phenomenon bears a striking resemblance to the Gender Register System observed in the Atayal language. A comparison with Atayal will be presented in Section 6.1.1.

\subsubsection{The possible relationship between Seediq's special doublet phenomenon and Atayal's Gender Register System}

Seediq is the sister language of Atayal, so I naturally thought of the relevance of the Gender Register System in Atayal to this phenomenon in Seediq.

According to \textcite{li1980gender,li1982gender,li1983gender} and \textcite{goderich2020phd}, Atayal has a Gender Register System in its lexicon. Nonetheless, among modern dialects, only Matu'uwal fully preserved the system, and only in some elderly speakers' speech. Meanwhile, the Gender Register System collapsed in other dialects, with only occasional remnants remaining.

Generally speaking, the female register is considered to be original, and the corresponding male form is an ``adjusted'' one. There are several strategies to derive words, such as affixation, deletion, substitution, etc. Note that the derivation from the female form to male form is irregular, and the strategy of derivation is also not predictable.

Come back to Seediq, \textcite[157]{goderich2020phd} states that there are relatively few traces of the register system in Seediq. However, I found that there are probably more cases than people used to realize. 

As mentioned earlier, two or more similar forms of the same or similar meaning can be found among Seediq dialects. Moreover, some of the lexical items are non-cognates between Atayal and Seediq; nevertheless, a pair of similar those forms can still be found in Seediq. This thus implies that at least in Proto-Seediq stage, the derivation methods was still productive and independent from Atayal.

There is a main differences between Seediq and Atayal. None of Seediq modern dialect preserved the distinction between male and female forms; therefore, sometimes we just cannot ensure which one should be the ``original'' word and which one is the derived one without external evidence. Thence I treat it like the ``legacy'' of the Gender Register System. 

In other words, the current evidence cannot definitively prove that \psedf still retains a clear distinction between male and female vocabulary. They may have preserved the morphological strategies in some sense, but whether it plays a certain societal function remains uncertain.

%Similar strategies are implemented in Seediq as well. Following tables show the example of each strategy, see Table \ref{tab:gender1} (Affixation), \ref{tab:gender2} (Segment substitution), and \ref{tab:gender3} (Syllable substitution) for details. 

Most common infixes are *<ra> and *<na>, which should be inserted before the final segment of words, as shown in Table \ref{tab:gender1}. Also, after the infixation, the prepenultimate syllables tends to be deleted. 

The medial *<ɹa> (corresponding to \psedf *<ra>) and *<na> can also be seen in Proto-Atayal or its dialects' reflexes. For example, in the Matu'uwal dialect, we have \textit{sumayug} (female, f) and \textit{sumayu<na>g} (male, m) `to change (AV)'.  In Proto-Atayal, we find *raqis (f) and *raqi<ɹa>s (m) `face' and so on. 

There is also a possible pair, \psedf *risaw `young man' and *rəs<ən>aw `man; husband', but the infix <ən> is inserted in a different position compare to other cases although their meaning is related. However, it is still possible since Atayal has a similar case: *guquh (f) and *guq<il>uh (m) `banana'.

In Table \ref{tab:gender2}, most of the examples are related to *h, probably indicating that the original consonants may have been replaced by *h. \textcite{li1983gender} also mentioned that in Matu'uwal (Mayrinax) Atayal, \textit{h} can replace other consonants such as \textit{l} or \textit{b}.

Regarding the majority of forms seen in Table \ref{tab:gender3}, as mentioned earlier, they mostly consist of very basic words with clear Proto-Austronesian origins. Moreover, these doublet forms correspond to some female and male forms in \pataf. Table \ref{tab:gender4} shows a comparison between \psedf and \pataf forms.

\begin{table}[!htbp]
\centering
\caption{A comparison of some doublet pairs between \psedf and \pataf}
\label{tab:gender4}
\begin{tabular}{llll}
\hline
\pan   & \psedf  & \pataf       & Gloss                 \\ \hline
*qudaS & *qudas  & *quras (f)   & `gray hair'           \\
       & *qudi   & *quriʔ (m)   & `gray hair'           \\
*laŋaw & *raŋaw  & *ɹaŋaw (f)   & `blowfly; robber fly' \\
       & *rəŋədi & *ɹaŋəriʔ (m) & `housefly'            \\
*wiRi  & *igi    & ---          & `left'                \\
       & *iril   & *ʔiɹil       & `left'                \\
*uka   & *uka    & *ʔukas        & `not exist'           \\
       & *uŋat   & *ʔuŋat       & `not exist'           \\ \hline
\end{tabular}
\end{table}

In addition, there are some more conservative forms can be found only in Seediq based on the current study, such as \psedf *igi `left'. This word is clearly from \pan *(w)iRi, and the other derived form in \psedf is *iril, which has a corresponding form: *ʔiɹil in Proto-Atayal. However, Proto-Atayal does not keep the female form *ʔigiʔ. 

By this evidence in Seediq we can fairly state that this kind of methods of deriving new words had been fully developed before the split of \psedf and \pataf. Indeed, after the split of the two languages, the development of this system in \pataf and Atayal dialects appear to be even more diverse and rich compared to the one in Seediq. 

However, we still cannot rule out the following possibilities: (1) the development of this system is a result of long-term cultural interaction between Seediq and Atayal at a later stage, and (2) \psedf may not have retained gender-specific vocabulary differences but only preserved the method of word formation. 

Nevertheless, we can establish at least some degree of relationship between the system in two languages through the comparison of vocabulary and word formation methods, which can also be related to their common development. Unfortunately, due to the lack of extensive evidence from modern languages, the strength of these inferences remains limited.

What have been found in \psedf provides some new evidence on a reconstruction of Proto-Atayalic lexicon and prehistory. This topic requires further future research to bolster its completeness.

\subsection{External evidence for lexical reconstructions}

I will provide examples of employing the top-down reconstruction method discussed in Section 1.2.2 for the reconstruction of some \psedf lexical items. The main external evidence includes the reconstructions mainly from \pataf, and only one from \pan. The examples are listed below, as in Table \ref{tab:external}. Note that glosses will only be added when the meanings of words from external evidence differ.

\begin{table}[!htbp]
\centering
\caption{External evidence for \psedf lexical reconstructions}
\label{tab:external}
\begin{tabular}{lllllll}
\hline
\psed    & \stg     & \sto         & \sctr   & \setr   & Gloss                                 & External evidence                                                   \\ \hline
*nanak   & nanaq    & \makecell[l]{nanak;\\nanaq} & nanaq   & nanak   & \makecell[l]{`alone; only;\\self'}                   & \pata *nanak                                                        \\
*sawki   &          & sowki        & sowki   & sowki   & `billhook'                            & \pata *sawki                                                        \\
*ləlabaŋ &          & ləlabaŋ      & ləlabaŋ & ləlabaŋ & `broad; wide'                         & \pata *gVlabaŋ (f)                                                  \\
*gəlahaŋ & gulahaŋ  &              &         &         & `broad; wide'                         & \pata *gVlahaŋ (m)                                                  \\
*ruluŋ   & (pulabu) & ruluŋ        & ruluŋ   & ruluŋ   & `cloud; fog'                          & \pata *ɹuluŋ                                                        \\
*diyax   & (ali)    & diyax        & ɟiyax   & ɟiyax   & `day; time'                           & \pata *riʔax                                                        \\
*qaqay   & (papak)  & qaqay        & qaqay   & qaqay   & `foot'                                & \pata *kakay                                                        \\
*dawin   &          &              &         & dawin   & `friend'                              & \makecell[l]{\pata *rawin (m)\\(borrowing?)}                        \\
*uŋat    &          &              & uŋat    & uŋat    & `not exist'                           & \pata *uŋat                                                         \\
*libu    & tibu     & tibu         & libu    & libu    & `pigsty'                              & \makecell[l]{\pan *Nibu `lair; den'\\\pata *libuʔ `coop; sty;\\pen'} \\
*buwax   & (beras)  & buwax        & buwax   & buwax   & \makecell[l]{`raw rice;\\rice seeds'} & \pata *buwax                                                        \\
*ramil   & (sapic)  & ramil        & ramil   & ramil   & `shoes'                               & \pata *ɹamil                                                        \\
*cəhəmu  & (rebu)   & cəhəmu       & səhəmu  & səhəmu  & `urine'                               & \pata *həmuq                                                       \\ \hline
\end{tabular}
\end{table}

In the Table \ref{tab:external}, it can be observed that some cognate words only appear in specific dialects, often within the same subgroup. However, based on evidence from \pataf or \pan, it is possible to reconstruct these lexical items back to the \psedf stage. Of course, we can never be fully certain that these are not borrowed words, but it is also challenging to confirm whether they are indeed loanwords.

\section{Proto-Seediq reflexes of Proto-Austronesian segments}

This section will provide the Proto-Seediq reflexes of Proto-Austronesian segments, including both regular and irregular examples. However, it must be noted that Atayalic languages, including Seediq and Atayal, have a relatively low amount of words derived from Proto-Austronesian. As a result, it may be difficult to find reflexes of certain Proto-Austronesian phonemes in specific positions, or there may be only one or two examples available.

\subsection{Regular reflexes}

I will first discuss the regular reflexes, while the irregular examples and their possible causes will be presented later in this section. The regular reflexes of \pan consonants in \psedf are shown in Table \ref{tab:refC}. The reflexes of vowels will be presented in Table \ref{tab:refV}.

\begin{spacing}{1.15}
\begin{longtable}[c]{lll}
\caption{\psedf reflexes of \panf consonants}
\label{tab:refC}\\
\hline
\pan & \psedf       & Examples                                                                                                                                                                                                       \\ \hline
\endfirsthead
%
\endhead
%
*p   & *p-/-p-/-k  & \begin{tabular}[c]{@{}l@{}}\pan *pitu > \psed *mə-pitu `seven'\\ \pan *Səpat > \psed *səpat `four'\\ \pan *Səyup > \psed *iyuk `to blow'\end{tabular}                                                         \\ \hline
*t   & *t          & \begin{tabular}[c]{@{}l@{}}\pan *təlu > \psed *təru `three'\\ \pan *ita > \psed *ita `\textsc{1pl.incl}'\\ \pan *waqit > \psed *waqit `canine tooth'\end{tabular}                                                      \\ \hline
*C   & *c-/-c-?/-t & \begin{tabular}[c]{@{}l@{}}\pan *Capaŋ > \psed *capaŋ `patch on clothes'\\ \pan *huRaC > \psed *urat `tendon; blood vessel'\end{tabular}                                                                      \\ \hline
*k   & *k          & \begin{tabular}[c]{@{}l@{}}\pan *kalih > \psed *kari `to dig'\\ \pan *baki > \psed *baki `grandfather'\\ \pan *buSuk > \psed *bəsuk-an `drunk'\end{tabular}                                                   \\ \hline
*q   & *q          & \begin{tabular}[c]{@{}l@{}}\pan *qabu > \psed *qəbulit `ashes'\\ \pan *waqit > \psed *waqit `canine tooth'\\ \pan *utaq > \psed *utaq `vomit'\end{tabular}                                                    \\ \hline
*b   & *b-/-b-/-k  & \begin{tabular}[c]{@{}l@{}}\pan *batu(x) > \psed *bətunux `stone'\\ \pan *babuy > \psed *babuy `pig'\\ \pan *Suab > \psed *suwak `yawn'\end{tabular}                                                          \\ \hline
*z   & *d          & \pan *ma-azaNih > \psed *dalih `near'                                                                                                                                                                         \\ \hline
*d   & *d-/-d-/-t  & \begin{tabular}[c]{@{}l@{}}\pan *daya > \psed *daya `uphill'\\ \pan *adaS > \psed *adas `to bring'\\ \pan *lahud > \psed *rahut `downhill'\end{tabular}                                                       \\ \hline
*j   & *-y-        & \pan *pajay > \psed *payay `riceplant'                                                                                                                                                                        \\ \hline
*g?  & *k?         & \begin{tabular}[c]{@{}l@{}}\pan *gaRaŋ > \psed *karaŋ `crab'\\ \pan *tagəRaŋ > \psed *təkəraŋ `ribs'\end{tabular}                                                                                             \\ \hline
*m   & *m-/-m-/-ŋ  & \begin{tabular}[c]{@{}l@{}}\pan *mujiŋ `face' > \psed *muhiŋ `nose'\\ \pan *ima > \psed *ima `who'\\ \pan *qaRəm > \psed *aruŋ `pangolin'\end{tabular}                                                        \\ \hline
*n   & *n          & \begin{tabular}[c]{@{}l@{}}\pan *naNaq > \psed *nalaq `pus'\\ \pan *tinun > \psed *tinun `to weave'\\ \pan *ŋajan > \psed *ŋayan `name'\end{tabular}                                                          \\ \hline
*ŋ   & *ŋ          & \begin{tabular}[c]{@{}l@{}}\pan *ŋajan > \psed *ŋayan `name'\\ \pan *Naŋuy > \psed *laŋuy `to swim'\\ \pan *Capaŋ > \psed *capaŋ `patch on clothes'\end{tabular}                                              \\ \hline
*s   & *h or *x?   & \begin{tabular}[c]{@{}l@{}}\pan *siRa > \psed *cə-higa `yesterday'\\ \pan *basuq > \psed *bahu `to wash clothes'\\ \pan *əsa `one' > \psed *t-əxa-l `once'\end{tabular}                                       \\ \hline
*S   & *s          & \begin{tabular}[c]{@{}l@{}}\pan *Sauni > \psed *sawni `just now'\\ \pan *uSa > \psed *usa `to go'\end{tabular}                                                                                                \\ \hline
*h   & *h or ∅     & \begin{tabular}[c]{@{}l@{}}\pan *huRaC > \psed *urat `tendon; blood vessel'\\ \pan *lahud > \psed *rahut `downhill'\\ \pan *baRah > \psed *bagah `charcoal'\\ \pan *kalih > \psed *kari `to dig'\end{tabular} \\ \hline
*N   & *l          & \begin{tabular}[c]{@{}l@{}}\pan *Naŋuy > \psed *laŋuy `to swim'\\ \pan *paNid > \psed *palit `wings'\\ \pan *SiSiN > \psed *sisil `Grey-cheeked fulvetta (omen bird)'\end{tabular}                            \\ \hline
*l   & *r          & \begin{tabular}[c]{@{}l@{}}\pan *lahud > \psed *rahut `downhill'\\ \pan *walay > \psed *waray `thread'\\ \pan *biqel > \psed *biqir `goiter'\end{tabular}                                                     \\ \hline
*R   & *g          & \begin{tabular}[c]{@{}l@{}}\pan *RamiS > \psed *gamil `root'\\ \pan *baRah > \psed *bagah `charcoal'\\ \pan *SipuR > \psed *səpug `to count'\end{tabular}                                                      \\ \hline
*w   & *w          & \begin{tabular}[c]{@{}l@{}}\pan *walay > \psed *waray `thread'\\ \pan *kawaS > \psed *kawas `year'\\ \pan *babaw > \psed *babaw `on; above'\end{tabular}                                                      \\ \hline
*y   & *y          & \begin{tabular}[c]{@{}l@{}}\pan *daya > \psed *daya `uphill'\\ \pan *pajay > \psed *payay `riceplant'\end{tabular}                                                                                            \\ \hline
\end{longtable}

\begin{longtable}[c]{ccccl}
\caption{\psedf reflexes of \panf vowels}
\label{tab:refV}\\
\hline
\multirow{2}{*}{\pan} & \multicolumn{3}{c}{\psedf}            & \multirow{2}{*}{Examples}                                                                                     \\ \cline{2-4}
                      & Prepenultimate & Penultimate & Final &                                                                                                               \\ \hline
\endfirsthead
%
\endhead
%
\hline
\endfoot
%
\endlastfoot
%
*a                    & *ə             & *a          & *a    & \pan *ama(x) > \psed *tama `father'                                                                           \\
*i                    & *ə             & *i          & *i    & \pan *wiRi > \psed *igi `left'                                                                                \\
*u                    & *ə             & *u          & *u    & \pan *Nutud > \psed *lutut `to join together'                                                          \\
*ə                    & *ə             & *ə          & *u    & \begin{tabular}[c]{@{}l@{}}\pan *təlu > \psed *təru `three'\\ \pan *luSəq > \psed *rusuq `tears'\end{tabular} \\ \hline
\end{longtable}
\end{spacing}

There are two points to mention regarding the restrictions of consonants in Seediq with respect to word-final position: (1) Voiced stops cannot appear in word-final positions, and (2) Labial stops cannot appear in word-final positions.

Therefore, \pan *p and *b merge into \psedf *k in word-final positions, while *t and *d are reflected as *t. Additionally, all labial sounds are reflected as their velar correspondents in such a position.

Additionally, the *j in \pan is regularly reflected as *y in \psedf, with some exceptions probably due to borrowing. According to \textcite{song2023Aicgprime}, it is noted that the reflex of *j in \psedf is *y, while in Pre-Proto-Atayal it is **g. Based on this correspondence, the *j in Proto-Atayalic can be reconstructed. The sound change *j > *y occurs from Proto-Atayalic to \psedf. Table \ref{tab:panjref} presents the reflex of \pan *j in \psedf. Note that only word-medial cases can be found, since word-initial *j- is banned in \pan, and word-final *-j is relatively rare. 

\begin{table}[!htbp]
\centering
\caption{\psedf reflexes of \pan *j}
\label{tab:panjref}
\begin{tabular}{ccl}
\hline
\pan              & \psedf                       & Gloss                 \\ \hline
*baRi\textbf{j}a  & *bəgi\textbf{y}a             & `reed of loom'        \\
*bu\textbf{j}əq   & *bə\textbf{y}uq (<M) `juice' & `foam; bubbles, etc.' \\
*ŋa\textbf{j}an   & *ŋa\textbf{y}an              & `name'                \\
*pa\textbf{j}ay   & *pa\textbf{y}ay              & `rice plant'          \\
*pa\textbf{j}iS   & *pa\textbf{y}is              & `enemy'               \\
*pi\textbf{j}a(x) & *pi\textbf{y}a               & `how many'            \\
*qa\textbf{j}iS   & (*a\textbf{y}us)             & `boundary'            \\
*Rimə\textbf{j}a  & *gəmə\textbf{y}a             & `sword grass'         \\
*Sua\textbf{j}i   & *səwa\textbf{y}i             & `younger sibling'     \\ \hline
\end{tabular}
\end{table}

The phoneme *g in \pan is highly controversial, but nonetheless, if we have to identify its reflex in \psedf, it would be *k. For example, \pan *gaRaŋ > \psed *karaŋ `crab,' and \pan *tagəRaŋ > \psed *təkəraŋ `ribs.'

Next, there are two cases that are difficult to determine that which reflex is regular, or they may represent some kind of free variation. The first set is \pan *s, which is reflected as \psedf *h or *x. The reflecting of \pan *s as *h is quite regular, but there is also a case where it is reflected as *x, as seen in the example of \pan root *əsa `one.'

\pan *əsa is reflected as a bound root *-(ə)xa- in \psed. For example, \psed *maxal `ten', and *təxal `once', which are both related to the meaning `one.' 

In fact, the origin of *x in Seediq or in the higher level of Proto-Atayalic is a mystery, also as mentioned by \textcite[263]{li1981paic}. \textcite[144]{goderich2020phd} argues that the Proto-Atayalic and Proto-Atayal word-final *-x reflects \pan *s, but he does not provide many examples. The only possible example mentioned is \pan *Sipəs > Proto-Atayal *hipux `cockroach.' However, the reflexes of this etymology in various Formosan languages often shows inconsistencies. Thus, the \pan reconstruction *Sipəs is quite questionable.

From the evidence we have from Atayal, it appears that there is a possibility that \pan *s could be reflected as *x in Atayal or Seediq. For example, \pan *asu is reflected as \pata *xu-ɹil (\psed *hu-liŋ), meaning `dog', and \pan *susu is reflected as Matu'uwal Atayal \textit{xuxuʔ}, meaning `breast.' In Atayal, there are also reflexes related to \pan *əsa, such as Squliq \textit{təxal}, meaning `again.'

Certainly, these examples are quite rare, but they are still related to the etyma from \pan and are difficult to explain as loanwords. Since other languages do not reflect \pan *s as /x/. In \psedf, there are indeed many instances of word-final *-x, but none of them can be directly linked to any phoneme in \pan. At the same time, this may be related to the development of the ``Gender Register System'' in Atayalic, as seen in the example of \pan *batu > \psed *bətu-nux `stone.'

It is currently difficult to explain why \pan *s is reflected as *h and *x in \psedf. It is possible that they represent as free variations or allophones at a certain stage. For now, I will consider both as regular reflexes.

Another case that is difficult to determine is the reflexes of \pan *h. In \psedf, we can observe two types of reflexes, either as ∅ or as *h. Table \ref{tab:panhref} shows the reflexes of \pan *h in \psedf. 

\begin{table}[!htbp]
\centering
\caption{\psedf reflexes of \pan *h}
\label{tab:panhref}
\begin{tabular}{ccl}
\hline
\pan     & \psedf         & Gloss                  \\ \hline
*huRaC   & *urat          & `tendon; blood vessel' \\
*buhət   & *bərihut       & `squirrel'             \\
*lahud   & *rahut; *hunat & `downhill'             \\
*Rahap   & *gəhak         & `seed'                 \\
*baqəRuh & *bəgurah       & `new'                  \\
*baRah   & *bagah         & `charcoal'             \\
*imah    & *imah          & `to drink'             \\
*nunuh   & *nunuh         & `breast'               \\
*kalih   & *kari          & `to dig'               \\ \hline
\end{tabular}
\end{table}

The only word-initial example is \pan *huRaC, which is reflected as \psedf *urat. However, this example does not rule out the possibility of borrowing. The three word-medial and the word-final examples, except for *kalih, all show \psedf *h. Therefore, we may consider *h to have a higher likelihood of being a regular reflex. 

\subsection{Irregular reflexes}

There are some \pan phonemes in \psedf that exhibit interesting and irregular reflexes, worthy of mention. I have named these instances of irregular reflexes as the following five ``problems'': (1) the ``DC'' problem; (2) the ``SH'' problem; (3) the ``CLS'' problem; (4) the ``LD'' problem; and (5) the ``GR'' problem. This naming approach is inspired by the names ``the RGH law'' and ``the RLD law'' in previous studies on the reflexes of \pan *R and *j among Austronesian languages. However, these cases in \psedf cannot be regarded as ``laws,'' so I have designated them as ``problems.''

Table \ref{tab:5problems} shows the regular and irregular reflexes, and I will introduce the five problems in 7.2.1 -- 7.2.5.

\begin{table}[!htbp]
\centering
\caption{The five ``problems''}
\label{tab:5problems}
\begin{tabular}{cccc}
\hline
Type                                 & \pan    & \psedf (regular) & \psedf (irregular)  \\ \hline
the ``DC'' problem                   & *d/(z?) & *d               & *c                  \\ \hline
the ``SH'' problem                   & *S      & *s               & *h                  \\ \hline
\multirow{2}{*}{the ``CLS'' problem} & *C      & *c               & \multirow{2}{*}{*l} \\ 
                                     & *S      & *s               &                     \\ \hline
the ``LD'' problem                   & *N      & *l               & *d                  \\ \hline
the ``GR'' problem                   & *R      & *g               & *r                  \\ \hline
\end{tabular}
\end{table}

\subsubsection{The ``DC'' problem}

The ``DC'' problem refers to the phenomenon in \psedf where the \pan *d is reflected as *c, in addition to the regular *d reflex. However, this phenomenon can be more reliably observed only in word-initial and word-medial positions, as these sounds merge as *t in word-final positions.

Table \ref{tab:dc} shows the examples of \pan *d > \psedf *c, with Proto-Atayal from \textcite{goderich2020phd} for comparison (\pan *z/d > \pata *r). 

\begin{table}[!htbp]
\centering
\caption{The examples of the ``DC'' problem}
\label{tab:dc}
\begin{tabular}{cccc}
\hline
\pan     & \psedf & Proto-Atayal & Gloss                      \\ \hline
*\textbf{d}akəS   & *\textbf{c}akus & *\textbf{r}akus       & `camphor tree'             \\
*\textbf{d}amaR   & *\textbf{c}amaw & ---          & `fire or light of candles' \\
*qaRi\textbf{d}aŋ & *gi\textbf{c}aŋ & *qagi\textbf{r}aŋ     & `k.o. bean'                \\ \hline
\end{tabular}
\end{table}

When encountering such irregular reflexes, it is natural to consider the possibility of borrowings. First, we might attempt to explain them using this approach. The Formosan languages that mainly reflect \pan *z/d as /ʦ/ are Tsou and Kanakanavu. 

For `camphor tree', Tsou has \textit{cʔósɨ}, Kanakanavu has \textit{cakɨ́sɨ}; for `bean', Tsou \textit{recŋi}, Kanakanavu \textit{ʔaricaŋ}. The reflexes for \pan *damaR cannot be found in these two languages. The other possibility is Siraya. At least one Siraya dialect reflected \pan *d as \textit{s}, which is possible that there is an intermediate stage **c between *d and \textit{s}.

However, I believe that the *d > *c change might be sporadic within the Seediq or Atayalic languages themselves, supported by the following evidence:

\begin{enumerate}
\setlength\itemsep{0em}
    \item The *cəka \~{} *dəka `half; to halve' doublet in \psedf. 
    \item A subdialect name of the Toda subgroup changed from *təuda > **tawca > \textit{tawsa(y)}, also retained as \textit{cow\textbf{c}a} in neighboring Atayal dialect (that subdialect has **c > \textit{s} sound change, \cite{lee2012tawsa}).
    \item  The irregular sound correspondences of some cognates between Seediq and Atayal (but without a \pan etymon, i.e. Atayalic lexical innovation).
\end{enumerate}

I believe that this correspondence may be related to the doublets mentioned in Section 6.1 of \psedf or possibly to the word-formation methods inherited from the ``Gender Register System.'' In the aforementioned evidence 1, we can find a set of synonyms that exhibit both the *d and *c forms. Furthermore, evidence 2 demonstrates that even proper nouns can undergo such changes, making it less likely to be borrowed from another language. Lastly, the cognates mentioned in evidence 3 are shown in Table \ref{tab:dc2}.

\begin{table}[!htbp]
\centering
\caption{The ``DC'' problem with in Atayalic languages}
\label{tab:dc2}
\begin{tabular}{ccc}
\hline
\psedf   & Proto-Atayal & Gloss                   \\ \hline
*\textbf{d}əŋusun & *\textbf{c}Vŋusan     & `target; goal; meaning' \\
*\textbf{c}əruhiŋ & *\textbf{r}aɹuhiŋ     & `nest fern'             \\
*u\textbf{c}ik    & *qu\textbf{r}ip       & `ginger'                \\ \hline
\end{tabular}
\end{table}

\subsubsection{The ``SH'' problem}

The ``SH'' problem refers to the \pan *S being reflected as *s or *h in \psedf, but please note that this phenomenon only occurs in word-initial position. Additionally, unlike the previous case, this is a more significant issue as it widely appears in different Formosan languages, possibly related to factors such as the reconstruction of \pan. Please refer to \textcite{tsuchida1976tsouic, ross2015coronals} for more details. Therefore, this paper will only list the different reflexes as in Table \ref{tab:sh} without attempting to explain higher-level issues.

\begin{table}[!htbp]
\centering
\caption{The examples of the ``SH'' problem}
\label{tab:sh}
\begin{tabular}{ccc}
\hline
\pan    & \psedf & Gloss           \\ \hline
*\textbf{S}auni  & *\textbf{s}awni & `just now'      \\
*\textbf{S}əpat  & *\textbf{s}əpat & `four'          \\
*\textbf{S}əpi   & *\textbf{s}əpi  & `dream'         \\
*\textbf{S}apuy  & *\textbf{h}apuy & `fire' > `cook' \\
*\textbf{S}əma   & *\textbf{h}əma  & `tongue'        \\
*\textbf{S}uRəNa & *\textbf{h}uda  & `snow'          \\ \hline
\end{tabular}
\end{table}

\subsubsection{The ``CLS'' problem}

The ``CLS'' problem refers to the expected reflexes of \pan *C and *S being *c and *s in \psedf, but sometimes *l (or /l/) is reflected instead in  \psedf or certain dialects of Seediq. This phenomenon can also be observed in Proto-Atayal and Atayal dialects (\cite[173]{goderich2020phd}). The examples are shown in Table \ref{tab:cls}.

\begin{table}[!htbp]
\centering
\caption{The examples of the ``CLS'' problem}
\label{tab:cls}
\begin{tabular}{ccccccc}
\hline
\pan           & \psedf    & \stg & \sto & \sctr & \setr & Gloss                \\ \hline
*\textbf{C}iŋaS         & *[s\textbf{l}]iŋas & siŋas & \textbf{l}iŋas & siŋas  & \textbf{l}iŋas  & `food between teeth' \\
*\textbf{C}uSuR         & *\textbf{l}ihug    & \textbf{l}ihu  & \textbf{l}ihu  & \textbf{l}ihug  & \textbf{l}ihug  & `to string'          \\
*\textbf{C}abu          & *\textbf{l}abu     & \textbf{l}abu  & \textbf{l}abu  & \textbf{l}abu   & \textbf{l}abu   & `to wrap'            \\
*\textbf{C}aŋis         & *\textbf{l}iŋis    & \textbf{l}iŋis & \textbf{l}iŋis & \textbf{l}iŋis  & \textbf{l}iŋis  & `to cry'             \\
*daki\textbf{S}         & *daki\textbf{l}    & daki\textbf{n} & daki\textbf{l} & daki\textbf{l}  & daki\textbf{l}  & `to climb'           \\
*Ramə\textbf{C}; *Rami\textbf{S} & *gami\textbf{l}    & gami\textbf{n} & gami\textbf{l} & gami\textbf{l}  & gami\textbf{l}  & `root'               \\ \hline
\end{tabular}
\end{table}

This phenomenon is difficult to explain through conventional ``sound change'' processes and appears more like a case of substitution that occurs randomly. Even interjections exhibit similar correspondences, such as \stgf \textit{qe\textbf{n}}, \stof \textit{qey\textbf{c}}, \sctrf \textit{qey\textbf{l}} `had it coming'.

Although features like ``random'' and ``substitution'' make it appear potentially related to Gender Register System in an earlier stage, there is currently no direct evidence to support this hypothesis. However, there is an intriguing external piece of evidence from a set of verb conjunction forms in Plngawan Atayal, which is \textit{maŋilis} \~{} \textit{caŋisan} `to cry AV\~{}LV'. The locative voice form \textit{caŋisan} still preserved the \textit{c} sound. However, no internal evidence can be found in Seediq. 

\subsubsection{The ``LD'' problem}

The ``LD'' problem refers to the regular reflex of \pan *N as *l in \psedf, but there are also many instances where it is reflected as *d. Additionally, we can observe cases where there is an l:d correspondence between dialects, while in other instances, it is consistently reflected as *d, as demonstrated in Table \ref{tab:ld}.

\begin{table}[!htbp]
\centering
\caption{The examples of the ``LD'' problem}
\label{tab:ld}
\begin{tabular}{ccccccc}
\hline
\pan   & \psedf    & \stg  & \sto & \sctr & \setr & Gloss                     \\ \hline
*ba\textbf{N}aR & *balaw & ba\textbf{d}o   & balaw & balaw  & balaw  & `a thorny vine'           \\
*\textbf{N}a    & *\textbf{d}a       & \textbf{d}a    & \textbf{d}a    & \textbf{d}a     & \textbf{d}a     & `already'                 \\
*\textbf{N}ayad & *layat & \textbf{d}ayac  & \textbf{d}ayac & layac  & layat  & `Sambucus formosana'      \\
*\textbf{N}usuŋ & *\textbf{d}uhuŋ    & \textbf{d}uhuŋ  & \textbf{d}uhuŋ & \textbf{d}uhuŋ  & \textbf{d}uhuŋ  & `mortar'                  \\
*qa\textbf{N}up & *a\textbf{d}uk     & a\textbf{d}uk   & a\textbf{d}uk  & a\textbf{d}uk   & a\textbf{d}uk   & `to hunt'                 \\
*qə\textbf{N}əb & *əluk  & əluk   & əluk  & ə\textbf{d}uk   & ə\textbf{d}uk   & `to close'                \\
*qi\textbf{N}aS & *i\textbf{d}as     & i\textbf{d}as   & i\textbf{d}as  & i\textbf{d}as   & i\textbf{d}as   & `moon'                    \\
*si\textbf{N}aR & *hi\textbf{d}aw    & hi\textbf{d}o   & hi\textbf{d}aw & hi\textbf{d}aw  & hi\textbf{d}aw  & `sun'                     \\
*sulə\textbf{N} & *huru\textbf{t}?   & huru\textbf{c}  & huru\textbf{c} & həri\textbf{c}  & huri\textbf{t}  & `to plug' > `to keep sth.' \\
*wana\textbf{N} & *na<ra>\textbf{t}? & nara\textbf{c}  & nara\textbf{c} & nara\textbf{c}  & nara\textbf{t}  & `right side'              \\ \hline
\end{tabular}
\end{table}

Similarly, even in languages like Taokas and Papora, where \pan *N is reflected as [d] or similar sounds have some chance to be the source of these words, the phenomenon is difficult to explain purely as a result of borrowings. This does not imply that these words are not borrowed from other languages, but rather that the change from **l to *d may have occurred within Seediq internally.

For example, the word for `orange/tangerine' in Kavalan is \textit{mulu}, and in Pazeh-Kaxabu it is \textit{amulu}. In \psedf, the corresponding term is *mudu, which suggests a borrowing scenario where one of these languages influenced Seediq, and the [l] sound in Seediq changed to [d].

Another point to consider is that in various Seediq dialects, there is mutual assimilation between [l] and [d]. For example, \psedf *təduliŋ `finger' becomes \textit{təluliŋ} in \stof and \setrf, where the \textit{l} assimilates \textit{d}. Similarly, \psedf *kənədalax `from' has variants of \textit{kənədalax} \~{} \textit{kənədadax} across almost all dialects, with the former \textit{d} assimilating the latter \textit{l}. This indirectly suggests an interaction between [l] and [d] within Seediq. 

Additionally, it is possible that there may be phonetic similarity between these two sounds, which could lead to some confusion in perception. However, conducting a more detailed phonetic experiment would be necessary to provide empirical evidence for these claims.

\subsubsection{The ``GR'' problem}

The ``GR'' problem refers to the regular reflex of \pan *R as *g and the irregular reflex as *r in \psedf. It is also considered the most confusing one among the five problems, as the two reflexes are relatively balanced in terms of frequency.

As described in Section 4.1, the correspondence between /g/ and /r/ in different Seediq dialects appears to be unstable. This phenomenon is also observed from \pan to \psedf. However, the \pan to \psedf situation is even more complex. In addition to sporadic changes such as those observed from \psedf to Seediq dialects, there may be a significant proportion of words that have been borrowed from other languages.

Table \ref{tab:gr} presents examples of the reflection of \pan *R as \psedf *r. In these examples, all dialects consistently reflect it as /r/. Therefore, based on a bottom-up approach, it is reconstructed as *r in the \psedf stage.

\begin{table}[!htbp]
\centering
\caption{The examples of the ``GR'' problem}
\label{tab:gr}
\begin{tabular}{ccc}
\hline
\pan      & \psedf   & Gloss                  \\ \hline
*\textbf{R}abi(qi) & *\textbf{r}abi    & `night (all night)'    \\
*\textbf{R}iNuk    & *\textbf{r}əhənuk & `k.o. berry'           \\
*\textbf{R}ubu     & *\textbf{r}udu    & `nest'                 \\
*\textbf{R}udaŋ    & *\textbf{r}udan   & `old man'              \\
*ba\textbf{R}aq    & *ba\textbf{r}aq   & `lungs'                \\
*da\textbf{R}a     & *da\textbf{r}a    & `maple tree'           \\
*da\textbf{R}aq    & *da\textbf{r}a    & `blood'                \\
*da\textbf{R}iŋ    & *da\textbf{r}iŋ   & `moan'                 \\
*ga\textbf{R}aŋ    & *ka\textbf{r}aŋ   & `crab'                 \\
*hu\textbf{R}aC    & *u\textbf{r}at    & `tendon; blood vessel' \\
*ka\textbf{R}i     & *ka\textbf{r}i    & `language'             \\
*kə\textbf{R}ət    & *kə\textbf{r}ut   & `to cut' > `to saw'    \\
*qa\textbf{R}əm    & *a\textbf{r}uŋ    & `pangolin'             \\
*tagə\textbf{R}aŋ  & *təkə\textbf{r}aŋ & `ribcage'              \\
*u\textbf{R}əŋ     & *u\textbf{r}uŋ    & `horn'                 \\ \hline
\end{tabular}
\end{table}

These words have a high likelihood of being borrowed from other Formosan languages into Seediq. Below, I will list the potential source languages and words for these words in Table \ref{tab:gr2}:\footnote{Data sources: (P-)Amis and Sakizaya (\cite{lin2022amisak, g0vAmismoedict}), (P-)Atayal (\cite{goderich2020phd}), Bunun, Hla'alua, Kanakanavu, and Thau (\cite{ILRDFEdict}), Northeast Formosan (Basay and Kavalan, \cite{tseng2022nef}), Pazeh (\cite{liandtsuchida2001paz}), (P-)Puyuma (\cite{ting1978puy, cauquelin2015puydict}), (P-)Rukai (\cite{li1977ruk}), Saisiyat (\cite{li1978say}), Yami (\cite{tsuchidaetal1987batanic}).}

\begingroup
\renewcommand\arraystretch{1}
\begin{longtable}[c]{llll}
\caption{Possible loanwords in the context of \pan *R > \psedf *r}
\label{tab:gr2}\\
\hline
Gloss         & \pan            & \psed                  & Potential source                                       \\ \hline
\endfirsthead
%
\endhead
%
`lungs'       & *baRaq          & *baraq                 & Saisiyat \textit{bæLæʔ}                                \\
              &                 &                        & P-Amis-Sakizaya *baɾaʡ                                 \\
`maple tree'  & *daRa           & *dara\textsubscript{1} & Saisiyat \textit{raLaʔ}                                \\
              &                 &                        & Bunun \textit{dala}                                    \\
`blood'       & *daRaq          & *dara\textsubscript{2} & Puyuma (Katripulr, Kasavakan, Likavung) \textit{daraɦ} \\
`moan'        & *daRiŋ          & *dariŋ                 & Bunun \textit{daliŋ}                                   \\
`crab'        & *gaRaŋ          & *karaŋ                 & Saisiyat \textit{kaLaŋ}                                \\
              &                 &                        & Bunun \textit{kalaŋ}                                   \\
              &                 &                        & P-Amis-Sakizaya *kaɾaŋ                                 \\
`vein'        & *huRaC          & *urat                  & Amis \textit{uɾat}                                     \\
              &                 &                        & Bunun \textit{ulat}                                    \\
              &                 &                        & P-Puyuma *uraʈ                                         \\
`language'    & *kaRi           & *kari                  & Saisiyat \textit{kaLiʔ}                                \\
              &                 &                        & Kanakanavu \textit{kari}                               \\
              &                 &                        & Hla'alua \textit{kari}                                 \\
`sew; to cut' & *kəRət `to cut' & *kərut                 & Amis \textit{kəɾət}                                    \\
              &                 &                        & Puyuma (Nanwang) \textit{kərət}                        \\
`pangolin'    & *qaRəm          & *aruŋ                  & Saisiyat \textit{ʔæLəm}                                \\
              &                 &                        & Puyuma (Nanwang) \textit{ʔarəm}                        \\
`(over)night' & *Rabi(qi)       & *rabi                  & P-Amis-Sakizaya *ɾabi                                  \\
              &                 &                        & Basay \textit{lavi, labe}                              \\
`nest'        & *Rubu           & *rudu                  & Kavalan \textit{rubu}                                  \\
              &                 &                        & P-Puyuma *rubu                                         \\
`old (man)'   & *Rudaŋ          & *rudan                 & P-Rukai *ma-roɖaŋə `old (man)'                         \\
`ribcage'     & *tagəRaŋ        & *təkəraŋ               & P-Puyuma *tagəraŋ `chest'                              \\
              &                 &                        & Pazeh \textit{takaxaŋ}                                 \\
`horn'        & *uRəŋ           & *uruŋ                  & Pazeh \textit{uxuŋ}                                    \\
              &                 &                        & Yami \textit{oroŋ}                                     \\
`Alocasia'    & *biRaq?         & *bərayaw               & Bunun \textit{baihal}                                  \\
              &                 &                        & Saisiyat \textit{biyaraL}                              \\
              &                 &                        & Thau \textit{farazay}                                  \\
              &                 &                        & P-Atayal *bagayag                                      \\ \hline
\end{longtable}
\endgroup

There are also some cases of messy sound correspondences among Seediq dialects, as shown in Table \ref{tab:gr3}.

\begin{table}[!htbp]
\centering
\caption{Messy corresponding cases}
\label{tab:gr3}
\begin{tabular}{cccccc}
\hline
\pan   &  \stg  & \sto   & \sctr    & \setr  & Gloss         \\ \hline
*Rahap &  \textbf{r}ehak & \textbf{r}əhak  & \textbf{g}əhak    & \textbf{g}əhak  & `seed'        \\
*kəRiw &  keguy & kə\textbf{r}i   & kə\textbf{r}ig    & kə\textbf{r}ig  & `ramie'       \\
*taRah &  ta\textbf{r}a  & tawa   & taga     & taga   & `to wait'     \\
*qauR  &  qo\textbf{r}an & qowwan & (qowwan) & qowgan & `k.o. bamboo' \\ \hline
\end{tabular}
\end{table}

These cases are currently difficult to explain and may be sporadic changes within each dialect. Alternatively, it is possible that /g/ and /r/ had some degree of phonetic similarity at a certain stage, leading to overcorrection by the speakers, but this cannot be verified.

\section{Conclusion}

\subsection{Summary}

This paper presents a reconstruction of \psedf phonology and lexicon, using four dialects: \stgf, \stof, \sctrf, and \setrf. Additionally, near 700 words of \psedf are reconstructed, and the wordlist is provided in the \hyperref[wordlist]{Appendix}. Also, based on my \psedf reconstruction, this paper provides a new view of Seediq subgrouping based on linguistic evidence.

Reconstruction of \psedf phoneme inventory including 18 consonants *p, *t, *k, *q, *b, *d, *g, *c, *s, *x, *h, *m, *n, *ŋ, *l, *r, *w, and *y; four monophthongs *a, *i, *u, *ə; and three diphthongs *aw, *ay, and *uy. Regarding subgrouping, \stof, \sctrf, and \setrf form a \totrf subgroup, being one of the primary branches with \stgf left alone.

The subgrouping hypothesis is highly based on the change of historical schwa *ə. Two *ə assimilation scenarios support the \totrf subgroup along with some significant lexical changes. On the other hand, \psedf prepenultimate *ə became \textit{u} while the penultimate *ə became \textit{e} in the \stgf dialect. 

I also point out the ``legacy'' of the Gender Register System in Seediq with a certain number of examples. The derivational strategies of words are similar to those in Atayal; thus, such a system had been well established before the split of Atayal and Seediq. Furthermore, the system was still productive in \psedf, since the phenomenon can be also seen in Seediq unique words. However, there is no explicit evidence to indicate that \psedf still retains a distinction between male and female vocabulary. 

Finally, this paper has compiled the reflexes from \panf to \psedf, including regular and irregular examples. Among them, I presented the so-called ``Five Problems,'' namely the ``DC'' problem, the ``SH'' problem, the ``CLS'' problem, the ``LD'' problem, and the ``GR'' problem. The reasons behind these problems are diverse, including language contact, regional irregular sound changes, possible phonetic perceptual issues, and some may be related to the Gender Register System. However, there are still some puzzling and mysterious examples that remain unexplained at the moment.

In conclusion, this paper provides a basis for future comparative studies between \psedf and its sister language, (Proto-)Atayal, which could lead to a reconstruction of their ancestral language, Proto-Atayalic. Through further comparisons of shared innovations and developments between Seediq and Atayal, more detailed evidence can be obtained to support the Atayalic subgroup, instead of relying solely on the term ``self-evident.'' 

\subsection{Remaining issues and future research directions}

This paper has some limitations and remaining issues in the research, including historical factors, data sources, and data quality, and so on. 

Regarding historical factors, events such as the Musha Incident (Japanese/Chinese: 霧社事件) and the construction of the Bandai and Musha Power Station (Japanese:萬大と霧社發電所; Chinese: 萬大發電廠/萬大水庫) led to forced relocations of various Seediq groups and tribes. As a result, the dialects of different tribes were mixed to some extent, and some sub-dialects might have been assimilated and disappeared, like the Paran dialect with [ð] sound. Therefore, linguistic data can only be sought from historical records.

The primary data sources for this paper are from the ILRDF wordlists, online dictionaries, and a Seediq dictionary database, which may have varying levels of data quality. Additionally, some data from certain sub-dialects are not included in the mentioned sources. For example, the Tausa (or Tuda, Tausay, Eastern Toda) dialect, which preserves -aw- and -ay- in word-medial position, or the Sadu dialect with a \psedf *q > [ʡ] shift. Collecting first-hand data in the future is necessary.

As for future research directions, collecting more data from various sub-dialects will enable a more comprehensive subgroup proposal and validate whether any corrections are needed for \psedf as reconstructed in this paper. Moreover, through vocabulary collection, further exploration of the ``Special doublet'' phenomenon can be conducted. As mentioned earlier, the reconstruction of \psedf can provide abundant material for future research on Atayalic languages, avoiding bias towards Atayal and ensuring equal attention to both languages. Language contact is also a significant factor in the development history of Seediq. All of these issues can serve as materials for future research.

\end{document}